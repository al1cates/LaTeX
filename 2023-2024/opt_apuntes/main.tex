\documentclass[11pt]{report}

%-------------------------------------------------------------------------------------------------%

% PAQUETES

\usepackage[a4paper, right = 0.8in, left = 0.8in, top = 0.8in, bottom = 0.8in]{geometry}
\usepackage[spanish, es-lcroman]{babel} % es-lcroman para romanos en minúscula
\usepackage{amsmath, amsfonts, amssymb, amsthm}
\usepackage{fouriernc} % Fuente
\usepackage{imakeidx} % Índice
\usepackage{setspace} % Espacio entre líneas del índice
\usepackage{mathtools} % Solo uso \underbracket
\usepackage[naturalnames]{hyperref} % Sin naturalnames el hiperenlace de los apéndices no funciona
\usepackage[table, svgnames, x11names]{xcolor}
\usepackage{enumitem}
\usepackage{parskip} % Cambia la sangría por espacio vertical
\usepackage{array}
\usepackage{aligned-overset}
\usepackage{tikz}
\usepackage{pgfplots}
\usepackage{titlesec}
\usepackage{emptypage} % Para que las páginas vacías de antes de un tema no tengan encabezado ni pie
\usepackage[labelfont=bf, labelsep=period]{caption} % Cambia «Figura 1:» por «Figura 1.» (en negrita)
\usepackage{float} % Para que el texto después de una gráfica no se ponga antes
\usepackage[framemethod=tikz]{mdframed}

\usepackage{graphicx}
\usepackage[utf8]{inputenc}
\usepackage{dirtytalk} % Comillas de apertura y cierre
\usepackage{fbox} % Cajas en las demostraciones de "si y solo si" y doble contención
\usepackage{multicol} % Dividir una lista en varias columnas
\usepackage{soul} % Cambiar de línea con palabras subrayadas, y para tachar
\usepackage{faktor} % Conjuntos cociente
\usepackage{centernot} % Negar símbolos como \implies
\usepackage{spalign}
\usepackage[outline]{contour}
\usepackage{ulem}
\usepackage{color}
\usepackage{fancyhdr}
\usepackage{bm}
\usepackage{eurosym} % Euros
\usepackage{booktabs}
\usepackage{hhline}
\usepackage{tikz-cd}
\usepackage{bbm} % Usar \mathbb con símbolos matemáticos
\usepackage{cancel}

%-------------------------------------------------------------------------------------------------%

% AJUSTES GENERALES

\decimalpoint

\setlist[enumerate]{label={\normalfont\textbf{(\textit{\roman*})}}} % Enumerar las listas con romanos en minúscula, cursiva y negrita
% normalfont para que los teoremas no italicen los paréntesis

\setlength{\columnsep}{0.8cm} % Recta que separa dos columnas (multicols)
\setlength{\columnseprule}{1.5pt}

\makeatletter % Para que no se ignore el espacio antes y después de los teoremas; también hace que la barra a la izquierda de definiciones, teoremas y todo eso aparezca correctamente
\def\thm@space@setup{%
  \thm@preskip=\parskip \thm@postskip=0pt
}
\makeatother

\makeatletter % Quitar el espacio adicional que el paquete parskip añade al principio y al final de una demostración
\renewenvironment{proof}[1][\proofname]{\par
  \pushQED{\qed}%
  \normalfont \topsep\z@skip % <---- changed here
  \trivlist
  \item[\hskip\labelsep
        \itshape
    #1\@addpunct{.}]\ignorespaces
}{%
  \popQED\endtrivlist\@endpefalse
}
\makeatother

\newcount\arrowcount % Flechas para el método de Gauss
\newcommand\arrows[1]{
        \global\arrowcount#1
        \ifnum\arrowcount>0
                \begin{matrix}[c]
                \expandafter\nextarrow
        \fi
}
\newcommand\nextarrow[1]{
        \global\advance\arrowcount-1
        \ifx\relax#1\relax\else \xrightarrow{#1}\fi
        \ifnum\arrowcount=0
                \end{matrix}
        \else
                \\
                \expandafter\nextarrow
        \fi
}
\makeatletter
\renewcommand*\env@matrix[1][*\c@MaxMatrixCols c]{%
  \hskip -\arraycolsep
  \let\@ifnextchar\new@ifnextchar
  \array{#1}}
\makeatother

\newcommand{\tikzmark}[2]{\tikz[overlay,remember picture,baseline] % Tachar en una tabla
\node [anchor=base] (#1) {$#2$};}
\newcommand{\DrawVLine}[3][]{%
  \begin{tikzpicture}[overlay,remember picture]
    \draw[shorten <=0.3ex, #1] (#2.north) -- (#3.south);
  \end{tikzpicture}
}
\newcommand{\DrawHLine}[3][]{%
  \begin{tikzpicture}[overlay,remember picture]
    \draw[shorten <=0.2em, #1] (#2.west) -- (#3.east);
  \end{tikzpicture}
}

\addto\captionsspanish{\renewcommand{\chaptername}{Tema}} % Cambiar el título de los temas
\addto\captionsspanish{\renewcommand{\contentsname}{Contenidos}} % Cambiar el título del índice

\pgfplotsset{compat=1.18}

\setcounter{MaxMatrixCols}{20} % Matrices más grandes

\titlespacing{\section}{0pt}{0.5\baselineskip}{0.5\baselineskip} % Espacio antes y después del título de una sección
\titlespacing{\subsection}{0pt}{0.5\baselineskip}{0.5\baselineskip} % Espacio antes y después del título de una subsección

\usepgfplotslibrary{fillbetween}

\usetikzlibrary{calc}
\usetikzlibrary{patterns}
\usetikzlibrary{shadows} % Sombras de los recuadros
\usetikzlibrary{cd} % Diagramas conmutativos
\usetikzlibrary{babel} % Para que tikz y babel no den problemas
\usetikzlibrary{positioning,shapes.multipart} % Para los árboles del método de ramificación y acotación
\usetikzlibrary{fit,shapes.geometric}

\counterwithout{figure}{chapter} % Para que la numeración de las figuras sea global y no por temas
\counterwithout{equation}{chapter} 

\DeclareMathAlphabet{\mathcal}{OMS}{zplm}{m}{n}

\graphicspath{{./images/}}

%-------------------------------------------------------------------------------------------------%

% ÁRBOLES DEL MÉTODO DE RAMIFICACIÓN Y ACOTACIÓN

\def\textA{
\scriptsize{$\begin{aligned}[t]
    x^0_{PR} &= (0.8,1.6)^t \\
    z^0_{PR} &= 8.8
\end{aligned}$}
\nodepart{two}
{\scriptsize{$9 \leq z^0_{PE}$}}
}

\def\textC{
\scriptsize{$\begin{aligned}[t]
    x^0_{P1} &= (0,4)^t \\
    z^0_{P1} &= 16
\end{aligned}$}
\nodepart{two}
{\scriptsize{$9 \leq z^0_{PE} \leq 16$}}
}

\def\textD{
\scriptsize{$\begin{aligned}[t]
    x^0_{P3} &= (1,1.5)^t \\
    z^0_{P3} &= 9
\end{aligned}$}
}

\def\textE{
\scriptsize{$\begin{aligned}[t]
    x^0_{P4} &= (2,1)^t \\
    z^0_{P4} &= 10
\end{aligned}$}
\nodepart{two}
{\scriptsize{$9 \leq z^0_{PE} \leq 10$}}
}

\def\textF{
\scriptsize{$\begin{aligned}[t]
    x^0_{P6} &= (1,2)^t \\
    z^0_{P6} &= 11
\end{aligned}$}
}

\def\textG{
\scriptsize{$\begin{aligned}[t]
    x^0_{P1} &= (2,1)^t \\
    z^0_{P1} &= 10
\end{aligned}$}
}

\def\textH{
\scriptsize{$\begin{aligned}[t]
    x^0_{P2} &= (0.66,2)^t \\
    z^0_{P2} &= 10
\end{aligned}$}
}

\def\textAA{
\scriptsize{$\begin{aligned}[t]
    x^0_{PR} &= (2.25,3.75)^t \\
    z^0_{PR} &= -41.25
\end{aligned}$}
\nodepart{two}
{\scriptsize{$-41 \leq z^0_{PE}$}}
}

\def\textBB{
\scriptsize{$\begin{aligned}[t]
    x^0_{P1} &= (3,3)^t \\
    z^0_{P1} &= -39
\end{aligned}$}
\nodepart{two}
{\scriptsize{$-41 \leq z^0_{PE} \leq -39$}}
}

\def\textCC{
\scriptsize{$\begin{aligned}[t]
    x^0_{P2} &= (1.8,4)^t \\
    z^0_{P2} &= -41
\end{aligned}$}
}

\def\textDD{
\scriptsize{$\begin{aligned}[t]
    x^0_{P3} &= (1,4.4)^t \\
    z^0_{P3} &= -40.55
\end{aligned}$}
\nodepart{two}
{\scriptsize{$-40 \leq z^0_{PE} \leq -39$}}
}

\def\textEE{
\scriptsize{\centering$\textup{\textit{Problema imposible}}$}
}

\def\textFF{
\scriptsize{$\begin{aligned}[t]
    x^0_{P5} &= (1,4)^t \\
    z^0_{P5} &= -37
\end{aligned}$}
}

\def\textGG{
\scriptsize{$\begin{aligned}[t]
    x^0_{P6} &= (0,5)^t \\
    z^0_{P6} &= -40
\end{aligned}$}
\nodepart{two}
{\scriptsize{$-40 \leq z^0_{PE} \leq -40$}}
}

\def\textAAA{
\scriptsize{$\begin{aligned}[t]
    x^0_{PR} &= (2.25,1.75)^t \\
    z^0_{PR} &= -5.75
\end{aligned}$}
\nodepart{two}
{\scriptsize{$-5 \leq z^0_{PE}$}}
}

\def\textBBB{
\scriptsize{$\begin{aligned}[t]
    x^0_{P1} &= (2,1.8)^t \\
    z^0_{P1} &= -5.66
\end{aligned}$}
}

\def\textCCC{
\scriptsize{\centering$\textup{\textit{Problema imposible}}$}
}

\def\textDDD{
\scriptsize{$\begin{aligned}[t]
    x^0_{P3} &= (2,1)^t \\
    z^0_{P3} &= -4
\end{aligned}$}
\nodepart{two}
{\scriptsize{$-5 \leq z^0_{PE} \leq -4$}}
}

\def\textEEE{
\scriptsize{$\begin{aligned}[t]
    x^0_{P4} &= (1.5,2)^t \\
    z^0_{P4} &= -5.5
\end{aligned}$}
}

\def\textFFF{
\scriptsize{$\begin{aligned}[t]
    x^0_{P5} &= (1,2.16)^t \\
    z^0_{P5} &= -5.33
\end{aligned}$}
}

\def\textGGG{
\scriptsize{\centering$\textup{\textit{Problema imposible}}$}
}

\def\textHHH{
\scriptsize{$\begin{aligned}[t]
    x^0_{P7} &= (1,2)^t \\
    z^0_{P7} &= -5
\end{aligned}$}
\nodepart{two}
{\scriptsize{$-5 \leq z^0_{PE} \leq -5$}}
}

\def\textIII{
\phantom{-}
\begin{center}\scriptsize{$\textup{(...)}$}\end{center}
\phantom{-}
}

\tikzset{>=stealth,parent node/.style={rectangle split, rectangle split parts=2,align=left,text width=2.6cm,draw,node distance=1cm and 1cm}}

%-------------------------------------------------------------------------------------------------%

% COLORES

\definecolor{c1}{HTML}{0065ff}
\definecolor{c2}{HTML}{ff5d00}

%-------------------------------------------------------------------------------------------------%

% PROPOSICIONES, COROLARIOS, TEOREMAS, DEFINICIONES Y EJEMPLOS

\newtheoremstyle{mydefinition}{}{}{}{}{\color{c1}\bfseries}{.}{ }{}
\newtheoremstyle{mytheorem}{}{}{\itshape}{}{\color{c2}\bfseries}{.}{ }{\thmname{#1}\thmnumber{ #2}\thmnote{ (#3)}} % El último paréntesis para que el título opcional esté en negrita
\newtheoremstyle{myexample}{}{}{}{}{\bfseries}{.}{ }{}

\theoremstyle{mytheorem}
\newtheorem{proposition}{Proposición}
\newtheorem{corollary}{Corolario} % [proposition] hace que siga la misma numeración que las proposiciones
\newtheorem{theorem}{Teorema}

\theoremstyle{mydefinition}
\newtheorem{definition}{Definición}

\theoremstyle{myexample}
\newtheorem*{example}{Ejemplo}
\newtheorem*{notation}{Notación}
\newtheorem*{obs}{Observación}

\addto\captionsspanish{ % Para que la palabra «Demostración» esté en negrita y sin cursiva
\let\oldproofname=\proofname
\renewcommand{\proofname}{\rm\bf{\oldproofname}}}

\newenvironment{cdefinition} % Definiciones con raya a la izquierda
  {\begin{mdframed}[
        linewidth=3pt,
        linecolor=c1,
        bottomline=false,
        topline=false,
        rightline=false,
        innerrightmargin=0pt,
        innertopmargin=0pt,
        innerbottommargin=0pt,
        innerleftmargin=1em,
        skipabove=\baselineskip]
    \begin{definition}}
  {\end{definition}\end{mdframed}}

\newenvironment{ctheorem} % Teoremas con raya a la izquierda
  {\begin{mdframed}[
        linewidth=3pt,
        linecolor=c2,
        bottomline=false,
        topline=false,
        rightline=false,
        innerrightmargin=0pt,
        innertopmargin=0pt,
        innerbottommargin=0pt,
        innerleftmargin=1em,
        skipabove=\baselineskip]
    \begin{theorem}}
  {\end{theorem}\end{mdframed}}

\newenvironment{cproposition} % Proposiciones con raya a la izquierda
  {\begin{mdframed}[
        linewidth=3pt,
        linecolor=c2,
        bottomline=false,
        topline=false,
        rightline=false,
        innerrightmargin=0pt,
        innertopmargin=0pt,
        innerbottommargin=0pt,
        innerleftmargin=1em,
        skipabove=\baselineskip]
    \begin{proposition}}
  {\end{proposition}\end{mdframed}}

\newenvironment{ccorollary} % Corolarios con raya a la izquierda
  {\begin{mdframed}[
        linewidth=3pt,
        linecolor=c2,
        bottomline=false,
        topline=false,
        rightline=false,
        innerrightmargin=0pt,
        innertopmargin=0pt,
        innerbottommargin=0pt,
        innerleftmargin=1em,
        skipabove=\baselineskip]
    \begin{corollary}}
  {\end{corollary}\end{mdframed}}

%-------------------------------------------------------------------------------------------------%

% ATAJOS

\newcommand{\R}{\mathbb R}
\newcommand{\N}{\mathbb N}
\newcommand{\Z}{\mathbb Z}
\newcommand{\Q}{\mathbb Q}
\newcommand{\C}{\mathbb C}

\newcommand{\pars}[1]{\left( #1 \right)}
\newcommand{\comment}[1]{}
\newcommand{\mybf}[1]{\boldmath\textbf{\color{c1}#1}\unboldmath} % Negrita de color c1

\newcommand*{\Scale}[2][4]{\scalebox{#1}{$#2$}}% % No recuerdo para qué sirve

%-------------------------------------------------------------------------------------------------%

% TÍTULOS DE CAPÍTULOS

\makeatletter
\def\thickhrulefill{\leavevmode \leaders \hrule height 1ex \hfill \kern \z@}
\def\@makechapterhead#1{%
  %\vspace*{50\p@}%
  \vspace*{10\p@}%
  {\parindent \z@ \centering \reset@font
        \thickhrulefill\quad
        {\scshape \@chapapp{} \thechapter}
        \quad \thickhrulefill
        \par\nobreak
        \vspace*{10\p@}%
        \interlinepenalty\@M
        \hrule
        \vspace*{0.5mm}%
        \Huge \bfseries #1\par\nobreak
        \par
        \vspace*{3mm}%
        \hrule
    %\vskip 40\p@
    \vskip 50\p@
  }}
\def\@makeschapterhead#1{%
  %\vspace*{50\p@}%
  \vspace*{10\p@}%
  {\parindent \z@ \centering \reset@font
        \thickhrulefill
        \par\nobreak
        \vspace*{10\p@}%
        \interlinepenalty\@M
        \hrule
        \vspace*{0.5mm}%
        \Huge \bfseries #1\par\nobreak
        \par
        \vspace*{3mm}%
        \hrule
    %\vskip 40\p@
    \vskip 50\p@
  }}

%-------------------------------------------------------------------------------------------------%

\begin{document}

%-------------------------------------------------------------------------------------------------%

% PÁGINA DEL TÍTULO

\begin{titlepage}

\vspace*{0.5cm}

\begin{tikzpicture}[remember picture,overlay]

  \fill[black] ($(current page.north west)-(0cm,4cm)$) rectangle ($(current page.south east)-(0cm,-22cm)$);

\end{tikzpicture}

\centering

\vspace{3\baselineskip}
	
{\fontsize{40pt}{0pt}\selectfont\textbf{\color{white}Optimización}}

\vspace{5\baselineskip}

{\color{black}\itshape\bfseries{

Universidad de Málaga

Grado en Matemáticas

Curso 2023-2024

}}

\end{titlepage}

\addtocounter{page}{1} % Para que la página del título cuente en la numeración

%-------------------------------------------------------------------------------------------------%

% PÁGINA DE LA TABLA DE CONTENIDOS

\doublespacing
\addtocontents{toc}{\protect\pagestyle{empty}}
\addtocontents{toc}{\protect\thispagestyle{empty}}
\tableofcontents
\thispagestyle{empty}
\singlespacing

%-------------------------------------------------------------------------------------------------%

\chapter{Programación lineal}

El asunto del que nos ocuparemos es el de minimizar o maximizar una función lineal sujeta a ciertas restricciones, también de carácter lineal. Este tipo de problemas serán planteados en la forma
\[
\begin{aligned}[t]
&\text{Min. } &&f(x) \\
& \, \text{s. a} &&\textit{(...)}
\end{aligned} \qquad \qquad 
\begin{aligned}[t]
&\text{Max. } &&f(x) \\
& \; \text{s. a} &&\textit{(...)}
\end{aligned}
\]
La función $f \colon \R^n \to \R$ es denominada {\textit{función objetivo}}, y "<\textit{(...)}"> son las restricciones en cuestión. "<Min."> y "<Max."> tienen significados evidentes y "<s. a"> significa "<sujeto a">. El conjunto $K$ al que se restringe el problema se llama {\textit{región factible}}, y si $f$ no es acotada superiormente o inferiormente en $K$ (dependiendo de si hay que maximizar o minimizar), se habla de {\textit{problema ilimitado}}.

\begin{example} \textit{Una empresa fabrica dos tipos de cinturones: tipo $A$ y tipo $B$. El beneficio neto por cinturón es de $2 \textup{\euro}$\ para los de tipo $A$ y $1,50 \textup{\euro}$ para los de tipo $B$. El tiempo para fabricar un cinturón de tipo $A$ es el doble que para el de $B$. Si todos fueran de tipo $B$, se podrían fabricar $1000$ cinturones diarios. El límite de cinturones fabricados diariamente de cualquier tipo es $800$. Se dispone de $400$ hebillas de tipo $A$ y $700$ de tipo $B$. ¿Cuántos cinturones se deben fabricar al día para maximiar el beneficio de la empresa?}

El planteamiento es el siguiente:
\[\begin{aligned}[t]
&\text{Min. } && 3x_1+\frac{3}{2}x_2 \\
& \, \text{s. a} &&
\left\{ \begin{alignedat}{6}
&  &x_1 &       &     & {}\leq{} & 400 \\
&  &    &       & x_2 & {}\leq{} & 700 \\
&  &x_1 & {}+{} & x_2 & {}\leq{} & 800 \\
& 2&x_1 & {}+{} & x_2 & {}\leq{} & 100 \\
& \mathrlap{x_1, x_2 \geq 0} & & & & &
\end{alignedat} \right.
\end{aligned}
\]
\end{example}

\section{Programación lineal en forma estándar}

Un problema de programación lineal se dice que se encuentra en {\textit{forma estándar}} si es de la forma
\[\begin{aligned}[t]
&\text{Min. } && c_1x_1+\mathellipsis+c_nx_n \\
& \, \text{s. a} && 
\left\{ \begin{alignedat}{10}
& & a_{11}&x_1 {}+{} \mathellipsis {}+{} & a_{1n}&x_n {}={} b_1 \\
& & \vdots & & & & \\
& & a_{m1}&x_1 {}+{} \mathellipsis {}+{} & a_{mn}&x_n {}={} b_m \\
& \mathrlap{x_1,\mathellipsis,x_n \geq 0,} & & & & &
\end{alignedat}
\right.
\end{aligned}\]
es decir, si la función objetivo ha de ser minimizada en lugar de maximizada y en las restricciones solo aparecen igualdades. 



El vector $b = (b_1,\mathellipsis,b_m)^t$ se denomina {\textit{vector de recursos}} o {\textit{vector de términos independientes}}, mientras que $c = (c_1,\mathellipsis, c_n)^t$ es el {\textit{vector de costes}}.



En este tipo de problemas se va a pedir siempre que $n \geq m$ y que la matriz de coeficientes $A$ tenga rango $m$, de forma que el sistema siempre sea compatible. En forma matricial, el problema sería
\[\begin{aligned}[t]
&\text{Min. } && c^tx \\
& \, \text{s. a} &&\begin{cases}
    Ax = b \\
    x \geq 0
\end{cases}
\end{aligned}\]
En forma vectorial,
\[\begin{aligned}[t]
&\text{Min. } && c^tx \\
& \, \text{s. a} &&\begin{cases}
    \begin{pmatrix}
        a_{11} \\
        \vdots \\
        a_{m1}
    \end{pmatrix} x_1 + \mathellipsis + \begin{pmatrix}
        a_{1n} \\
        \vdots \\
        a_{mn}
    \end{pmatrix} x_n = \begin{pmatrix}
        b_1 \\
        \vdots \\
        b_m
    \end{pmatrix} \\
    x \geq 0
\end{cases}
\end{aligned}\]
Evidentemente, por $x = (x_1,\mathellipsis,x_n)^t \geq 0$ se entiende que $x_i \geq 0$ para todo $i = 1,\mathellipsis,n$.



En la práctica lo más común es que el problema que se plantee no se encuentre en forma estándar, a causa de alguna de las siguientes situaciones:

\begin{enumerate}
    \item Que la función objetivo tenga que ser maximizada, es decir, que el problema sea de la forma
    \[
\begin{aligned}[t]
&\text{Max. } &&c^tx \\
& \; \text{s. a} &&\textit{(...)}
\end{aligned}
\]
Lo único que hay que hacer es escribir
\[
\begin{aligned}[t]
-&\text{Min. } &&-c^tx \\
& \, \text{s. a} &&\textit{(...)}
\end{aligned}
\]
y rezar para que las restricciones sean adecuadas.
    \item Que alguna de las variables $x_i$ no tenga restricción de signo. Lo que se hace en este caso es obviar la variable $x_i$ y añadir otras dos, $x_i^+$ y $x_i^-$, de forma que se verifique $x_i = x_i^+ - x_i^-$ y también $x_i^+,x_i^- \geq 0$.
    
    \item Que alguna de las restricciones sea de la forma
    \[a_{i1}x_1+\mathellipsis+a_{in}x_n \leq b_i\]
    Lo que se hace en este caso es añadir una nueva variable $s_i \geq 0$, denominada {\textit{variable de holgura}}, verificando
    \[a_{i1}x_1+\mathellipsis+a_{in}x_n +s_i = b_i\]
    Si en vez de un $\leq$ aparece un $\geq$, solo hay que cambiar $+ \, s_i$ por $-\, s_i$, transformándose la restricción en la siguiente igualdad:
    \[a_{i1}x_1+\mathellipsis+a_{in}x_n -s_i = b_i\]
\end{enumerate}

\section{Soluciones de un problema de programación lineal}

Dependiendo de varios factores, a la solución de un problema de programación lineal en forma estándar se le denomina de hasta siete maneras distintas.

\begin{enumerate}
    \item Se dice que $x \in \R^n$ es {\textit{solución factible}} del problema si $x \geq 0$ y $Ax = b$. 
    \begin{itemize}
        \item Si una solución factible $x$ verifica $c^tx \leq c^t\hat{x}$ para cualquier otra solución factible $\hat{x}$, se hablará de {\textit{solución óptima}}.
    \end{itemize}
    \item Supóngase que la matriz de coeficientes es reordenada de la forma $A = (B \ | \ N)$, siendo $B$ la matriz formada por las columnas independientes de $A$. Si se llama $x_B = B^{-1} b$, la {\textit{solución básica}} del problema es el vector $x = (x_B,0)^t \in \R^n$, donde $x_B \in \R^m, \, 0 \in \R^{n-m}$ y $m = \textup{rg}(A)$. \begin{itemize}
        \item Si una solución básica resulta ser una solución óptima, se denominará, para sorpresa de nadie, \textit{{solución óptima-básica}}.
    \end{itemize}
    \item Si la solución básica verifica $x_B \geq 0$, se hablará de {\textit{solución básica-factible}}. \begin{itemize}
        \item Si la solución básica verifica $x_B > 0$, se hablará de {\textit{solución básica-factible no degenerada}}.
        \item Si el vector $x_B$ de la solución básica tiene alguna componente nula, dicha solución se llamará {\textit{solución básica-factible degenerada}}.
    \end{itemize}
\end{enumerate}

\begin{example}
Se considera el problema
\[\begin{aligned}[t]
&\text{Min. } &&c_1x_1+c_2x_2 \\
& \, \text{s. a} && \left\{\begin{alignedat}{10}
& x_1 & {}+{} & x_2 {}\leq{} 6 \\
&     &       & x_2 {}\leq{} 3 \\
& \mathrlap{x_1, x_2 \geq 0} \\
\end{alignedat} \right.
\end{aligned}\]
Antes de nada hay que pasar a forma estándar introduciendo dos variables de holgura, $x_3$ y $x_4$:
\[\begin{aligned}[t]
&\text{Min. } &&c_1x_1+c_2x_2 \\
& \, \text{s. a} && \left\{\begin{alignedat}{10}
& x_1 & {}+{} & x_2 & {}+{} & x_3 &       &     & {}={} 6 \\
&     &       & x_2 &       &     & {}+{} & x_4 & {}={} 3 \\
& \mathrlap{x_1, x_2,x_3,x_4 \geq 0}
\end{alignedat} \right.
\end{aligned}\]
En este caso, $n = 4$ y $m = 2$, mientras que la matriz $A$ sería
\[A = \begin{pmatrix}
    1 & 1 & 1 & 0 \\
    0 & 1 & 0 & 1
\end{pmatrix}\]
Se va a hallar una solución básica escogiendo como matriz $B$ la siguiente:
\[B = \begin{pmatrix}
    1 & 1 \\
    1 & 0
\end{pmatrix}\]
En este caso,
\[x_B = \begin{pmatrix}
    1 & 1 \\
    1 & 0
\end{pmatrix}^{-1} \begin{pmatrix}
    6 \\
    3
\end{pmatrix} = \begin{pmatrix}
    3 \\
    3
\end{pmatrix}\]
Por tanto, una solución básica sería $x = (0,3,3,0)^t$. Al escoger otra de las posibles matrices $B$ se obtienen las distintas soluciones básicas del problema.
\end{example}

\section{El teorema fundamental}

\begin{ctheorem}[Teorema fundamental de la programación lineal]
Considérese un problema de programación lineal en forma estándar.
\begin{enumerate}
    \item Si existe una solución factible, entonces existe una solución básica-factible.
    \item Si existe una solución óptima, entonces existe una solución óptima-básica.
\end{enumerate}
\end{ctheorem}

\begin{proof}
Se comenzará por probar el primer apartado. Sea $x \in \R^n$ una solución factible del problema. Esto quiere decir que $Ax = b$ y $x_i \geq 0$ para todo $i = 1,\mathellipsis,n$. Escribiendo la matriz $A$ por columnas, $A = (a_1 \, | \, a_2 \, | \, \mathellipsis \, | \, a_n)$, se tendría $a_1x_1+\mathellipsis+a_nx_n = b$. Sin pérdida de generalidad, se supondrá que de tener $x$ alguna componente nula, esta se situará al final del vector. En otras palabras, se puede escribir $x = (x_1,\mathellipsis,x_p,0,\mathellipsis,0)^t$ con $p \leq n$ y $x_i > 0$ para todo $i = 1,\mathellipsis,p$. Así, $a_1x_1+\mathellipsis+a_px_p = b$. Ahora se presenta la siguiente disyuntiva:
\begin{itemize}
    \item[\textbf{(\textit{a})}] \textit{El conjunto de vectores $\{a_1,\mathellipsis,a_p\}$ es linealmente independiente.} En este caso, $x$ sería solución básica-factible y el primer apartado quedaría demostrado. Si fuese $p = m = \textup{rg}(A)$, la solución sería no degenerada, y si $p < m$, la solución degeneraría.
    \item[\textbf{(\textit{b})}] \textit{El conjunto de vectores $\{a_1,\mathellipsis,a_p\}$ es linealmente dependiente.} Así, existen $y_1,\mathellipsis,y_p \in \R$ tales que $ y_1a_1+\mathellipsis+y_pa_p=0$, siendo alguno de los escalares no nulo. Llamando $y = (y_1,\mathellipsis,y_p,0,\mathellipsis,0) \in \R^n$, se tendría $Ay = 0$. 

    Para llegar a una solución básica-factible, se define el vector $x_0 = x-\varepsilon y$ para algún $\varepsilon >0$ que habrá que determinar convenientemente. El propósito de $\varepsilon$ será hacer de $x_0$ una solución factible. Como se tiene que $Ax_0 = Ax-\varepsilon Ay = b$, solo hay que asegurarse de que $x_0 \geq 0$. De las desigualdades
    \[(x_0)_i \geq 0 \iff x_i-\varepsilon y_i \geq 0 \iff x_i \geq \varepsilon y_i\]
    se deduce que eligiendo
    \[\varepsilon = \min_{y_i > 0} \biggl\{\frac{x_i}{y_i}\biggr\}\]
    se tendría que, efectivamente, $(x_0)_i \geq 0$ para todo $i = 1,\mathellipsis,n$. Nótese que si algún $y_i$ fuese nulo o negativo, la desigualdad $x_i-\varepsilon y_i \geq 0$ sería cierta para cualquier $\varepsilon >0$. La única pega que se podría poner es que no existiese ningún $y_i$ positivo, pero esto no es un problema porque se podría cambiar el signo a todos los escalares $y_1,\mathellipsis,y_p$ y no pasaría absolutamente nada.

    

    Ya se sabe entonces que $x_0$ es solución factible, y que tiene, a lo sumo, $p-1$ componentes positivas. Ahora se presenta una nueva disyuntiva:
    \begin{itemize}
        \item \textit{Las columnas asociadas a las componentes positivas de $x_0$ forman un sistema de vectores linealmente independiente.} Hemos triunfado: $x_0$ es solución básica-factible.
        \item \textit{Las columnas asociadas a las componentes positivas de $x_0$ forman un sistema de vectores linealmente dependiente.} Solo habría que repetir el razonamiento anterior hasta que las componentes positivas de las sucesivas soluciones factibles se agoten. Este procedimiento siempre desemboca en una solución básica-factible, siendo el último caso posible el que se presenta cuando la solución factible tiene únicamente una componente positiva (la columna correspondiente es, trivialmente, un sistema de vectores linealmente independiente).
    \end{itemize}
\end{itemize}

El primer apartado puede darse por demostrado. 

Supóngase ahora que $x \in \R$ es solución óptima. Como $x$ es solución factible, se tiene que $Ax = b$ y $x \geq 0$. Una vez más, se puede suponer que $x = (x_1,\mathellipsis,x_p,0,\mathellipsis,0)^t$, con $x_i >0$ para todo $i = 1,\mathellipsis,p$ y $p \leq n$. Se distinguen dos casos:
\begin{itemize}
    \item[\textbf{(\textit{a})}] \textit{El sistema de vectores $\{a_1,\mathellipsis,a_p\}$ es linealmente independiente.} Se acabó lo que se daba: $x$ es solución básica (degenerada o no degenerada dependiendo de si $p <m$ o $p = m$) y el teorema está demostrado.
    \item[\textbf{(\textit{b})}] \textit{El sistema de vectores $\{a_1,\mathellipsis,a_p\}$ es linealmente dependiente.} La misma cantinela de antes: existen $y_1,\mathellipsis,y_p \in \R$ tales que $Ay = 0$, con $y = (y_1,\mathellipsis,y_p,0,\mathellipsis,0)^t \in \R^n$ y alguno de los escalares no nulos. Ahora se hace un chanchullo como el de antes: se define $x_1 = x-\varepsilon y$, con
    \[\varepsilon = \min_{y_i >0} \biggl\{\frac{x_i}{y_i}\biggr\}\] \normalsize
    Así, se garantiza que $x_1$ sea solución factible. A diferencia del primer apartado, ahora también hay que probar que $x_1$ es solución óptima. Por ser $x$ solución óptima, se tiene que $x_1$ también lo es si y solo si $c^tx = c^tx_1$, es decir, si y solo si $c^ty = 0$. Esta última igualdad será lo que se pruebe a continuación. Por reducción al absurdo, supóngase que $c^ty>0$. Sea $\delta$ un número real que verifique
    \[0 < \delta \leq \min_{y_i>0}\biggl\{\frac{x_i}{y_i}\biggr\}\] \normalsize
    y sea $x_2 = x-\delta y$, que por la elección de $\delta$ es solución factible. Por ser $x$ solución óptima, se tiene
    \[c^tx_2 = c^tx-\delta  c^t y \geq c^tx \iff \delta c^ty \leq 0 \iff c^ty \leq 0,\]
    que es una contradicción. Supóngase ahora $c^ty<0$, y sea $\lambda$ un número real que verifique
    \[0 < \lambda \leq \min_{y_i<0}\biggl\{-\frac{x_i}{y_i}\biggr\}\]\normalsize
    Sea $x_3 = x+\lambda y$, que por la elección de $\lambda$ es solución factible. Por ser $x$ solución óptima,
    \[c^tx_2 = c^tx+\lambda  c^t y \geq c^tx \iff \lambda c^ty \geq 0 \iff c^ty \geq 0,\]
    lo que también resulta ser contradictorio. Se obtiene entonces $c^ty = 0$. Queda demostrado que $x_1$ es solución óptima del problema con, a lo sumo, $p-1$ componentes positivas. Si las columnas asociadas a las componentes positivas constituyen un sistema linealmente independiente, hemos terminado; si no, se repite el razonamiento hasta que solo quede una componente positiva o hasta que se encuentre la solución óptima-básica.
\end{itemize}

\noindent Con esto concluye la demostración del apartado segundo.
\end{proof}

\section{Puntos extremos}

\begin{cproposition}
En un problema de programación lineal en forma estándar, la región factible es un conjunto convexo.
\end{cproposition}

\begin{proof}
Sean $x_1$ y $x_2$ puntos de la región factible $K$ y veamos que $\overline{x_1x_2} \subset K$, o lo que es lo mismo, que $\alpha x_1+(1-\alpha)x_2 \in K$ para todo $\alpha \in [0,1]$.


Sea $\alpha \in [0,1]$. Entonces se tiene que $1-\alpha \geq 0$, y como $x_1,x_2 \geq 0$ por ser soluciones factibles, entonces $\alpha x_1+(1-\alpha)x_2 \geq 0$. Además,
\[A(\alpha x_1+(1-\alpha)x_2) = \alpha Ax_1 + (1-\alpha)Ax_2 = \alpha b + b - \alpha b = b\]
luego $\alpha x_1+(1-\alpha)x_2 \in K$.
\end{proof}

\begin{cdefinition}
Dado un subconjunto $C$ de $\R^n$, se dice que $x \in C$ es {\mybf{punto extremo}} si no se puede escribir como combinación lineal convexa de otros dos puntos distintos de cero, es decir, si cada vez que se escriba $x = \alpha x_1 + (1-\alpha) x_2$ con $x_1, x_2 \in C$ no nulos y $\alpha \in (0,1)$, se tiene que $x_1 = x_2 = x$.
\end{cdefinition}

\begin{ctheorem}
En un problema de programación lineal en forma estándar, un punto de la región factible es punto extremo si y solo si es solución básica-factible.
\end{ctheorem}

\begin{proof}
Se razonará por reducción al absurdo para probar ambas implicaciones.


Supóngase primero que $x \in K$ es solución básica-factible. Puede escribirse entonces $x = (x_B,0)^t$ con $x_B = B^{-1}b \geq 0$, siendo $A = (B \, | \, N)$ y $B \in \mathcal{M}_m(\R)$ inversible. Supóngase ahora que existen $x_1,x_2 \in K$ no nulos y tales que $x = \alpha x_1+(1-\alpha)x_2$ para algún $\alpha \in (0,1)$. Si $x_1=(x_{11},x_{12})^t$ y $x_2 = (x_{21},x_{22})^t$, se tiene que
\[\begin{pmatrix}
    x_B \\
    0
\end{pmatrix} = \alpha \begin{pmatrix}
    x_{11} \\
    x_{12}
\end{pmatrix}+(1-\alpha)\begin{pmatrix}
    x_{21} \\
    x_{22}
\end{pmatrix}\]
Esto implica $0 = \alpha x_{12}+(1-\alpha)x_{22}$. Ahora bien, $\alpha$ y $1-\alpha$ son positivos, mientras que $x_{12},x_{22} \geq 0$, luego $x_{12} = x_{22} = 0$. Por tanto, $Ax_{11} = Bx_{11} = b$ y $Ax_{21} = Bx_{21} = b$, de donde se deduce que $x_{11} = x_{21} = B^{-1}b = x_B$. Conclusión: $x = x_1 = x_2$ y consecuentemente $x$ es punto extremo.



Supóngase ahora que $x \in K$ es punto extremo y, sin pérdida de generalidad, que $x = (x_1,\mathellipsis,x_p,0,\mathellipsis,0)^t$, con $p \leq n$ y $x_i >0$ para todo $i = 1,\mathellipsis,p$. Supóngase también que $x$ no es solución básica-factible. Como $x$ sí es solución factible, se tiene que $x$ no es solución básica, así que el sistema de vectores $\{a_1,\mathellipsis,a_p\}$ es linealmente dependiente. Por tanto, existen $y_1,\mathellipsis,y_p \in \R$ tales que $y_1a_1+\mathellipsis+y_pa_p = 0$, con alguno de los $y_i$ no nulo. Sea $y = (y_1,\mathellipsis,y_p,0,\mathellipsis,0)^t$. Se va a razonar como en la demostración del teorema fundamental: se construyen $x_1 = x+\varepsilon y$, $x_2 = x-\varepsilon y$, donde $\varepsilon$ es un número positivo lo suficientemente pequeño como para que $x_1,x_2 \geq 0$ (ya se vio que esta elección de $\varepsilon$ era posible). Como $Ax_1 = Ax_2 = b$, se tiene que $x_1,x_2 \in K$. Además,
\[x = \frac{1}{2} x_1+\frac{1}{2}x_2,\]
que es una combinación lineal convexa con $\alpha = \frac{1}{2}$. Esto contradice que $x$ sea punto extremo (pues $x \neq x_1$ y $x \neq x_2$), concluyéndose que $x$ tiene que ser solución básica-factible.
\end{proof}

\begin{ccorollary}
Considérese un problema de programación lineal en forma estándar con región factible $K$. Entonces
\begin{enumerate}
    \item $K$ tiene al menos un punto extremo.
    \item Si existe solución óptima, entonces existe un punto extremo que es solución óptima-básica.
    \item $K$ tiene, a lo sumo, $\binom{n}{m}$ puntos extremos.
\end{enumerate}
\end{ccorollary}

\begin{proof}
Consecuencia directa de todos los resultados anteriores.
\end{proof}

\chapter{Método del símplex}

\section{Pivotaje}
Una vez más, partimos de un problema de programación lineal en forma estándar, es decir, un problema del estilo
\[\begin{aligned}[t]
&\text{Min. } && c^tx \\
& \, \text{s. a} &&\begin{cases}
    Ax = b \\
    x \geq 0
\end{cases}
\end{aligned}\]
Se va a suponer además que la matriz $A$, de rango máximo, se encuentra descompuesta como $A = ( B \, | \, D)$, donde
\begin{itemize}
    \item $B \in \mathcal{M}_{m \times m}(\R)$ es la matriz identidad salvo en la $p$-ésima columna, para cierto $p \in \{1,\mathellipsis,m\}$. Es decir,
    \[B = \begin{pmatrix}
    1 & 0 & \mathellipsis & 0 & a_{1,p} & 0 & \mathellipsis & 0 \\
    0 & 1 & \mathellipsis & 0 & a_{2,p} & 0 & \mathellipsis & 0 \\
    \vdots & \vdots & \ddots & \vdots & \vdots & \vdots & \ddots & \vdots \\
    0 & 0 & \mathellipsis & 0 & a_{m,p} & 0 & \mathellipsis & 1
    \end{pmatrix}\]
    \item $D \in \mathcal{M}_{m \times (n-m)}(\R)$ tiene en su $q$-ésima columna la $p$-ésima columna de la matriz identidad, para cierto $q \in \{m+1,m+2,\mathellipsis,n\}$. Es decir,
    \[D = \begin{pmatrix}
    a_{1,m+1} & \mathellipsis & a_{1,q-1} & 0 & a_{1,q+1} & \mathellipsis & a_{1,n} \\
    a_{2,m+1} & \mathellipsis & a_{2,q-1} & 0 & a_{2,q+1} & \mathellipsis & a_{2,n} \\
    \vdots & \ddots & \vdots & \vdots & \vdots & \ddots & \vdots \\
    a_{p,m+1} & \mathellipsis & a_{p,q-1} & 1 & a_{p,q+1} & \mathellipsis & a_{p,n} \\
    \vdots & \ddots & \vdots & \vdots & \vdots & \ddots & \vdots \\
    a_{m,m+1} & \mathellipsis & a_{m,q-1} & 0 & a_{m,q+1} & \mathellipsis & a_{m,n}
    \end{pmatrix}\]
\end{itemize}

Una vez asimilada la matriz $A$, debe quedar claro que si se intercambiaran la $p$-ésima columna y la $q$-ésima columna mediante operaciones elementales de filas o columnas (obteniéndose un sistema de ecuaciones totalmente equivalente), se llegaría a una matriz de la forma $(I \, | \, C)$ para una cierta matriz $C$. Pues bien, haciendo las operaciones
\[f_p' = \frac{1}{a_{p,q}}f_p \qquad \textup{y} \qquad f_i' = f_i -\frac{a_{i,q}}{a_{p,q}}f_p \textup{ para cada } i \in \{1,\mathellipsis,m\} \textup{ con } i \neq p, \tag{$\ast$}\]
la matriz $A = (B \, | \, D)$ se transforma en $ B^{-1}A = (I \, | \, B^{-1}D)$, y el vector $b$, naturalmente, se transforma en $B^{-1}b$.

\begin{example}
\label{ex2.1.}
Se considera el siguiente problema de programación lineal en forma estándar:
\[\begin{aligned}[t]
&\text{Min. } && x_1+2x_2-x_4+x_5 \\
& \, \text{s. a} && \left\{\begin{alignedat}{10}
& x_1 & {}\phantom{+}{} &     &     &     & {}+{} &  &x_4 & {}+{} &  &x_5 & {}-{} & x_6 {}={} 5 \\
&     &     & x_2 &     &     & {}+{} & 2&x_4 & {}-{} & 3&x_5 & {}+{} & x_6 {}={} 3 \\
&     &     &     & {}\phantom{+}{}    & x_3 & {}-{} &  &x_4 & {}+{} & 2&x_5 & {}-{} & x_6 {}={} 1 \\
& \mathrlap{x_1, x_2,x_3,x_4,x_5,x_6 \geq 0} \\
\end{alignedat}\right.
\end{aligned}\]
Se tiene en este caso
\[ A= \begin{pmatrix*}[r]
    1 & 0 & 0 & 1 & 1 & -1 \\
    0 & 1 & 0 & 2 & -3 & 1 \\
    0 & 0 & 1 & -1 & 2 & -1
\end{pmatrix*}\qquad \qquad
b = \begin{pmatrix}
    5 \\
    3 \\
    1
\end{pmatrix}\] 
Tomando como base las columnas de $A$ que forman la identidad, se tiene que $x^0 = (5,3,1,0,0,0)^t$ es solución básica-factible del problema. La matriz $A$ no es exactamente como la que se ha planteado al principio del tema, pero el razonamiento va a ser análogo. Se escoge, por ejemplo, $p = 1$ y $q = 4$, es decir, la base considerada va a pasar de $\{a_1,a_2,a_3\}$ a $\{a_4,a_2,a_3\}$. Ahora se realizan las operaciones descritas en $(*)$:
\[
\begin{pmatrix}[rrrrrr|r]
     1 & 0 & 0 & 1 & 1 & -1 & 5 \\
     0 & 1 & 0 & 2 & -3 & 1 & 3 \\
     0 & 0 & 1 & -1 & 2 & -1 & 1 \\
\end{pmatrix} \begin{matrix}[c]
\arrows3{\scriptstyle{f_1' = f_1}}{\scriptstyle{f_2' = f_2 - 2f_1}}{\scriptstyle{f_3' = f_3+f_1}}
\end{matrix}
\begin{pmatrix}[rrrrrr|r]
     1 & 0 & 0 & 1 & 1 & -1 & 5 \\
     -2 & 1 & 0 & 0 & -5 & 3 & -7 \\
     1 & 0 & 1 & 0 & 3 & -2 & 6 \\
\end{pmatrix} \begin{matrix}[c]
\end{matrix}
\]
La gracia de esto es que se obtiene una nueva solución básica, $x^1 = (0,-7,6,5,0,0)^t$, y en la matriz de coeficientes sigue apareciendo la identidad, con lo cual puede repetirse el proceso unas cuantas veces para obtener soluciones básicas distintas de manera relativamente fácil. Esa va a ser, como se verá más adelante, la idea principal del método del símplex. Por completar el ejemplo, la matriz $B$ en este caso sería
\[B = \begin{pmatrix*}[r]
    1 & 0 & 0 \\
    2 & 1 & 0 \\
    -1 & 0 & 1
\end{pmatrix*}\]
Es decir, $B$ es la identidad salvo en la columna $p = 1$, donde se encuentra la columna $q = 4$ de $A$. Por otro lado, $D$ consta de las columnas de $A$ que no están en $B$:
\[D = \begin{pmatrix*}[r]
    1 & 1 & -1 \\
    0 & -3 & 1 \\
    0 & 2 & -1
\end{pmatrix*}\]
Haciendo los cálculos pertinentes se comprueba que
\[B^{-1}=\begin{pmatrix*}[r]
    1 &0 & 0 \\
    -2 & 1 & 0 \\
    1 & 0 & 1
\end{pmatrix*} \qquad B^{-1}D = \begin{pmatrix*}[r]
    1 & 1 & -1 \\
    -2 & 5 & 3 \\
    1 & 3 & -2
\end{pmatrix*} \qquad B^{-1}b = \begin{pmatrix*}[r]
    5 \\
    -7 \\
    6
\end{pmatrix*}\]
Como era de esperar, la matriz $B^{-1}D$ está formada por las columnas distintas de la identidad de la matriz que resulta de hacer las operaciones con filas en $A$. Se observa también que $B^{-1}b$ coincide con el nuevo vector de términos independientes, como tiene que ser.
\end{example}

\section{Costes reducidos}

\begin{cdefinition}
Sea  $A = (B \, | \, D)$ la matriz de un problema de programación lineal en forma estándar, con $B \in \mathcal{M}_{m \times m}(\R)$ inversible, y sea $c = (c_B \, | \ c_D)^t$ el vector de costes. Se conoce como {\mybf{vector de costes reducidos}} a la matriz fila $c^t-c_B^tA = (c_B^t-c_B^tB \, | \, c_D^t -c_B^tD)$.
\end{cdefinition}

Para cada $j \in \{1,\mathellipsis,n\}$, la $j$-ésima componente del vector de costes reducidos $c^t-c_B^tA$ suele ser denotada por $c_j-z_j$, donde $z_j = c_B^ta_j$ y $a_j$ es la $j$-ésima columna de $A$.



Obsérvese que si $B = I$, entonces las $m$ componentes $c_B^t-c_B^tB$ del vector de costes reducidos son todas nulas.



Por otro lado, cabe remarcar que en el desarrollo teórico la descomposición de $A$ siempre será $A = (B \, | \, D)$ por facilidad de escritura, pero en la práctica las matrices $B$ y $D$ estarán mezcladas entre las columnas de $A$.

\begin{example}
\label{ex2.2.}
Se calculará el vector de costes reducidos del ejemplo anterior. Se recuerdan los datos:
\[A= \begin{pmatrix*}[r]
    1 & 0 & 0 & 1 & 1 & -1 \\
    0 & 1 & 0 & 2 & -3 & 1 \\
    0 & 0 & 1 & -1 & 2 & -1
\end{pmatrix*} \qquad \qquad c = (1,2,0,-1,1,0)^t\]
La base considerada será $B = \{a_1,a_2,a_3\}$. Por tanto,
\[c^t-c_B^tA = \begin{pmatrix}
    1 & 2 & 0 & -1 & 1 & 0
\end{pmatrix} - \begin{pmatrix}
    1 & 2 & 0
\end{pmatrix} \begin{pmatrix*}[r]
    1 & 0 & 0 & 1 & 1 & -1 \\
    0 & 1 & 0 & 2 & -3 & 1 \\
    0 & 0 & 1 & -1 & 2 & -1
\end{pmatrix*} = \begin{pmatrix}
    0 & 0 & 0 & -6 & 6 & -1
\end{pmatrix}\]
Partiendo de la solución básica $x^0 =(5,3,1,0,0,0)^t$, los resultados obtenidos en el problema se suelen representar a través de una tabla como la que sigue:
\begin{center}
\begin{tabular}{|c|c|c|c|c|c|c|c|}
    \cline{3-8}
    \multicolumn{1}{c}{} & \multicolumn{1}{c|}{} & \multicolumn{1}{c}{1} & \multicolumn{1}{c}{2} & \multicolumn{1}{c}{0} & \multicolumn{1}{c}{-1} & \multicolumn{1}{c}{\phantom{-}1} & \multicolumn{1}{c|}{\phantom{-}0} \\ \hline
    1 & 5 & \multicolumn{1}{c}{1} & \multicolumn{1}{c}{0} & \multicolumn{1}{c}{0} & \multicolumn{1}{c}{\phantom{-}1} & \multicolumn{1}{c}{\phantom{-}1} & \multicolumn{1}{c|}{-1} \\
    2 & 3 & \multicolumn{1}{c}{0} & \multicolumn{1}{c}{1} & \multicolumn{1}{c}{0} & \multicolumn{1}{c}{\phantom{-}2} & \multicolumn{1}{c}{-3} & \multicolumn{1}{c|}{\phantom{-}1} \\
    0 & 1 & \multicolumn{1}{c}{0} & \multicolumn{1}{c}{0} & \multicolumn{1}{c}{1} & \multicolumn{1}{c}{-1} & \multicolumn{1}{c}{\phantom{-}2} & \multicolumn{1}{c|}{-1} \\ \hline
    \multicolumn{1}{c|}{} & 11 & \multicolumn{1}{c}{0} & \multicolumn{1}{c}{0} & \multicolumn{1}{c}{0} & \multicolumn{1}{c}{-6} & \multicolumn{1}{c}{\phantom{-}6} & \multicolumn{1}{c|}{-1} \\
    \cline{2-8}
\end{tabular}
\end{center}
Por si cupiese duda, en la primera columna de la tabla aparecen los costes básicos, mientras que el $11$ de la última fila se refiere al valor de la función objetivo en $x^0$. Recordando el ejemplo anterior, si se quiere pasar de la base $\{a_1,a_2,a_3\}$ a la base $\{a_4,a_2,a_3\}$, la nueva tabla sería
\begin{center}
\begin{tabular}{|c|c|c|c|c|c|c|c|}
    \cline{3-8}
    \multicolumn{1}{c}{} & \multicolumn{1}{c|}{} & \multicolumn{1}{c}{\phantom{-}1} & \multicolumn{1}{c}{2} & \multicolumn{1}{c}{0} & \multicolumn{1}{c}{-1} & \multicolumn{1}{c}{\phantom{-}1} & \multicolumn{1}{c|}{\phantom{-}0} \\ \hline
    
    -1 & \phantom{-}5 & \multicolumn{1}{c}{\phantom{-}1} & \multicolumn{1}{c}{0} & \multicolumn{1}{c}{0} & \multicolumn{1}{c}{\phantom{-}1} & \multicolumn{1}{c}{\phantom{-}1} & \multicolumn{1}{c|}{-1} \\
    
    \phantom{-}2 & -7 & \multicolumn{1}{c}{-2} & \multicolumn{1}{c}{1} & \multicolumn{1}{c}{0} & \multicolumn{1}{c}{\phantom{-}0} & \multicolumn{1}{c}{-5} & \multicolumn{1}{c|}{\phantom{-}3} \\
    
    \phantom{-}0 & \phantom{-}6 & \multicolumn{1}{c}{\phantom{-}1} & \multicolumn{1}{c}{0} & \multicolumn{1}{c}{1} & \multicolumn{1}{c}{\phantom{-}0} & \multicolumn{1}{c}{\phantom{-}3} & \multicolumn{1}{c|}{-2} \\ \hline
    
    \multicolumn{1}{c|}{} & -9 & \multicolumn{1}{c}{\phantom{-}5} & \multicolumn{1}{c}{0} & \multicolumn{1}{c}{0} & \multicolumn{1}{c}{\phantom{-}0} & \multicolumn{1}{c}{\phantom{-}12} & \multicolumn{1}{c|}{-7} \\
    \cline{2-8}
\end{tabular}
\end{center}
Obsérvese que ahora la solución básica es $x^1=(0,-7,6,5,0,0)^t$ y que el valor de la función objetivo en $x^1$ ha disminuido en comparación a la tabla anterior. Además, al cambiar los costes básicos, los costes reducidos también son diferentes.
\end{example}

En general, el procedimiento seguido en los dos ejemplos anteriores se escribe en forma de tabla tal que así:

\begin{center}
\begin{tabular}{|c|c|c|c|}
    \cline{3-4}
    \multicolumn{1}{c}{} & \multicolumn{1}{c|}{} & \multicolumn{1}{c}{$c_B^t$} & \multicolumn{1}{c|}{$c_D^t$} \\ \hline
    
    $c_B$ & $b$ & \multicolumn{1}{c}{$B$} & \multicolumn{1}{c|}{$D$} \\ \hline
    
    \multicolumn{1}{c|}{} & $c_B^tb$ & \multicolumn{1}{c}{$c_B^t-c_B^tB$} & \multicolumn{1}{c|}{$c_D^t-c_B^tD$} \\
    \cline{2-4}
\end{tabular} $\quad \rightsquigarrow \quad$ 
\begin{tabular}{|c|c|c|c|}
    \cline{3-4}
    \multicolumn{1}{c}{} & \multicolumn{1}{c|}{} & \multicolumn{1}{c}{$c_B^t$} & \multicolumn{1}{c|}{$c_D^t$} \\ \hline
    
    $c_B$ & $B^{-1}b$ & \multicolumn{1}{c}{$I$} & \multicolumn{1}{c|}{$B^{-1}D$} \\ \hline
    
    \multicolumn{1}{c|}{} & $c_B^tB^{-1}b$ & \multicolumn{1}{c}{$0$} & \multicolumn{1}{c|}{$c_D^t-c_B^tB^{-1}D$} \\
    \cline{2-4}
\end{tabular}
\end{center}
O, alternativamente,
\begin{center}
\begin{tabular}{|c|c|c|c|}
    \cline{3-4}
    \multicolumn{1}{c}{} & \multicolumn{1}{c|}{} & \multicolumn{1}{c}{$c_B^t$} & \multicolumn{1}{c|}{$c_D^t$} \\ \hline
    
    $c_B$ & $b$ & \multicolumn{1}{c}{$B$} & \multicolumn{1}{c|}{$D$} \\ \hline
    
    \multicolumn{1}{c|}{} & $c_B^tb$ & \multicolumn{1}{c}{$c_B^t-c_B^tB$} & \multicolumn{1}{c|}{$c_D^t-c_B^tD$} \\
    \hhline{~|=|=|=|}
    
    \multicolumn{1}{c|}{} & $B^{-1}b$ & \multicolumn{1}{c}{$I$} & \multicolumn{1}{c|}{$B^{-1}D$} \\ \cline{2-4}
    
    \multicolumn{1}{c|}{} & $c_B^tB^{-1}b$ & \multicolumn{1}{c}{$0$} & \multicolumn{1}{c|}{$c_D^t-c_B^tB^{-1}D$} \\
    \cline{2-4}
\end{tabular}
\end{center}

\section{Resultados preliminares}

\begin{ctheorem}
\label{teo2.1.}
Sea $x^0 = (x^0_1,\mathellipsis,x_m^0,0,\mathellipsis,0)^t$ una solución básica-factible no degenerada de un problema de programación lineal en forma estándar y sea $z_0 = c^tx^0$ el valor de la función objetivo en $x^0$. Si para algún $j \in \{1,\mathellipsis,n\}$ se tiene que $c_j-z_j < 0$, entonces existe $J \subset K$ tal que $c^tx < z_0$ para todo $x \in J$.
\end{ctheorem}

\begin{proof}
Sea $j \in \{1,\mathellipsis,n\}$ tal que $c_j-z_j<0$. Se va a suponer, sin pérdida de generalidad, que las primeras $m$ columnas de $A$ son la identidad, teniéndose por tanto que $j \in \{m+1,\mathellipsis,n\}$, pues si $j \in \{1,\mathellipsis,m\}$ sería $c_j-z_j=0$. 

Fijado un $\varepsilon > 0$ que habrá que determinar según sea conveniente, se construye el vector \[x_\varepsilon^0 = (x^0_1-\varepsilon a_{1,j},\mathellipsis,x^0_m-\varepsilon a_{m,j},0,\mathellipsis,0,\varepsilon,0,\mathellipsis,0)^t,\] ocupando $\varepsilon$ la componente $j$-ésima. Veamos primero que $x_\varepsilon^0$ es solución del problema:
\[
\begin{aligned}[t]
Ax_\varepsilon^0 &= A(x^0_1,\mathellipsis,x^0_m,0,\mathellipsis,0)^t-\varepsilon A(a_{1,j},\mathellipsis,a_{m,j},0,\mathellipsis,0,1,0,\mathellipsis,0)^t \\
&= Ax^0 - \varepsilon(a_{1,j}a_1+\mathellipsis+a_{m,j}a_m)+\varepsilon a_j \\
&= Ax^0 - \varepsilon a_j+\varepsilon a_j \\
&= b
\end{aligned}
\]
En la penúltima igualdad se recuerda que las primeras $m$ columnas de $A$ son los vectores de la base canónica de $\R^m$. Habrá que ver ahora si $x^0_\varepsilon$ es solución factible, distinguiéndose dos casos:
\begin{itemize}
    \item[\textbf{(\textit{a})}] \textit{Para todo $i \in \{1,\mathellipsis,m\}$ se tiene que $a_{i,j} \leq 0$}. Entonces $(x_\varepsilon^0)_i = x^0_i-\varepsilon a_{i,j} \geq 0$ para todo $i \in \{1,\mathellipsis,m\}$ y todo $\varepsilon > 0$, luego $x_\varepsilon^0 \geq 0$. Se verifica que
\[
\begin{aligned}[t]
c^tx_\varepsilon^0 &= c^t(x^0_1,\mathellipsis,x^0_m,0,\mathellipsis,0)^t-\varepsilon c^t(a_{1,j},\mathellipsis,a_{m,j},0,\mathellipsis,0,1,0,\mathellipsis,0)^t \\
&= c^tx^0 - \varepsilon(a_{1,j}c_1+\mathellipsis+a_{m,j}c_m)+\varepsilon c_j \\
&= z_0 + \varepsilon(c_j-z_j)
\end{aligned}
\]
Como por hipótesis $c_j-z_j<0$, puede afirmarse que $c^tx_\varepsilon^0 < z_0$ para todo $\varepsilon > 0$. El teorema queda demostrado y el subconjunto del enunciado sería
\[J = \{(x^0_1-\varepsilon a_{1,j},\mathellipsis,x^0_m-\varepsilon a_{m,j},0,\mathellipsis,0,\varepsilon,0,\mathellipsis,0)^t \in \R^n \colon \varepsilon > 0\} \subset K\]
    \item[\textbf{(\textit{b})}] \textit{Existe $i \in \{1,\mathellipsis,m\}$ tal que $a_{i,j} > 0$}. Tomando un $\varepsilon > 0$ que verifique
    \[0 < \varepsilon \leq \min_{a_{i,j} > 0} \biggl\{\frac{x_i^0}{a_{i,j}} \biggr\}\]
    se tendría que $x_k^0 -\varepsilon a_{k,j}> 0$ para todo $k \in \{1,\mathellipsis,m\}$, luego $x_\varepsilon^0 \geq 0$. Evidentemente, también se cumple
\[
c^tx_\varepsilon^0 = z_0 + \varepsilon(c_j-z_j) < z_0,
\]
luego el teorema también está probado, siendo en este caso
\[J = \{(x^0_1-\varepsilon a_{1,j},\mathellipsis,x^0_m-\varepsilon a_{m,j},0,\mathellipsis,\varepsilon,\mathellipsis,0)^t \in \R^n \colon 0 < \varepsilon \leq \min_{a_{i,j} > 0} \biggl\{\frac{x_i^0}{a_{i,j}} \biggr\} \} \subset K\]
\end{itemize}

Obsérvese que en el caso $(a)$ de la demostración se está ante un problema ilimitado, pues cuando $\varepsilon$ tiende a $\infty$ el valor de la función objetivo tiende a $-\infty$.
\end{proof}

\begin{ctheorem}
\label{teo2.2.}
Sea $x^0 = (x^0_1,\mathellipsis,x_m^0,0,\mathellipsis,0)^t$ una solución básica-factible de un problema de programación lineal en forma estándar. Si $c_j-z_j \geq 0$ para todo $j \in \{1,\mathellipsis,n\}$, entonces $x^0$ es solución óptima.
\end{ctheorem}

\begin{proof}
Una vez más, se va a suponer sin perder generalidad que las primeras $m$ columnas de $A$ son la identidad. Sea $y = (y_1,\mathellipsis,y_n)^t \in K$. Hay que demostrar que $c^tx^0 \leq c^ty$. Por hipótesis se tiene que $z_j \leq c_j$, y como $y_j \geq 0$, entonces $y_jz_j \leq y_jc_j$ para cada $j \in \{1,\mathellipsis,n\}$. Por tanto,
\[z_1y_1+\mathellipsis+z_ny_n \leq c_1y_1+\mathellipsis+c_ny_n = c^ty, \tag{$\ast$}\]
y si el término de la izquierda tuviese algo que ver con $c^tx^0$, la demostración podría terminarse. Por ser $x^0$ e $y$ soluciones del problema, se tiene que
\[b = x^0_1a_1+\mathellipsis+x^0_ma_m = y_1a_1+\mathellipsis+y_ma_m+\mathellipsis+y_na_n\]
Ahora bien, para cada $j \in \{1,\mathellipsis,n\}$ se verifica
\[a_j = \sum_{i=1}^m a_{i,j}a_i,\]
pues cabe recordar, una vez más, que $\{a_1,\mathellipsis,a_m\}$ es la base canónica de $\R^m$. Sustituyendo arriba,

\[
\begin{aligned}[t]
    x^0_1a_1+\mathellipsis+x^0_ma_m &= \biggl( \, \sum_{i=1}^m a_{i,1}a_i\biggr)y_1 + \mathellipsis+ \biggl( \, \sum_{i=1}^m a_{i,n}a_i \biggr)y_n \\
    &= \biggl( \, \sum_{i=1}^n y_ia_{1,i}\biggr)a_1+\mathellipsis+\biggl(\, \sum_{i=1}^n y_ia_{m,i}\biggr) a_m
\end{aligned}
\]
Tenemos dos expresiones distintas para un mismo vector en una base de $\R^m$, lo que obliga a que los coeficientes coincidan, es decir, para cada $j \in \{1,\mathellipsis,m\}$ debe cumplirse
\[x^0_j = \sum_{i=1}^n y_ia_{j,i}\]
Por otro lado,
\[
\begin{aligned}[t]
z_1y_1+\mathellipsis+z_ny_n &= \biggl(\, \sum_{i=1}^m c_ia_{i,1}\biggr)y_1+\mathellipsis+\biggl(\, \sum_{i=1}^m c_ia_{i,n}\biggr)y_n \\
&= \biggl(\, \sum_{i=1}^m y_ia_{1,i}\biggr)c_1+\mathellipsis+\biggl(\, \sum_{i=1}^m y_ia_{m,i}\biggr)c_m \\
&= x^0_1 c_1+\mathellipsis+x^0_mc_m \\
&= c^tx^0,
\end{aligned}
\]
y la desigualdad $(\ast)$ dice en realidad que $c^tx^0 \leq c^ty$, que es justo lo que quería probarse.
\end{proof}

Obsérvese que en el primer teorema era necesario pedir la no degeneración de la solución básica-factible de partida, mientras que en el segundo no hace falta.

\begin{ccorollary}
Sea $x^0 = (x^0_1,\mathellipsis,x^0_m,0,\mathellipsis,0)^t$ una solución básica-factible no degenerada de un problema de programación lineal en forma estándar. Supongamos que $c_j-z_j \geq 0$ para todo $j \in \{1,\mathellipsis,n\}$.

\begin{enumerate}
    \item Si $c_j-z_j = 0$ para algún $j \in \{m+1,\mathellipsis,n\}$, entonces el problema tiene infinitas soluciones óptimas.
    \item Si $c_j-z_j > 0$ para todo $j \in \{m+1,\mathellipsis,n\}$, entonces $x^0$ es la única solución óptima del problema.
\end{enumerate}

\end{ccorollary}

\begin{proof} 
Supóngase primero que $c_j-z_j = 0$ para algún $j \in \{m+1,\mathellipsis,n\}$. Por el teorema anterior, $x^0$ es solución óptima, luego $z_0 = c^tx^0$ es el valor mínimo de la función objetivo en la región factible. Siguiendo el razonamiento de la demostración del \hyperref[teo2.1.]{\color{gray}Teorema 3}, siempre existen infinitos $\varepsilon$ tales que 
\[x_\varepsilon^0 = (x^0_1-\varepsilon a_{1,j},\mathellipsis,x^0_m-\varepsilon a_{m,j},0,\mathellipsis,0,\varepsilon,0,\mathellipsis,0)^t\]
es solución básica-factible no degenerada. Concretamente, si fuese $a_{i,j} \leq 0$ para todo $i \in \{1,\mathellipsis,m\}$, los $\varepsilon$ pertenecen al conjunto
\[J = \{(x^0_1-\varepsilon a_{1,j},\mathellipsis,x^0_m-\varepsilon a_{m,j},0,\mathellipsis,0,\varepsilon,0,\mathellipsis,0)^t \in \R^n \colon \varepsilon > 0\} \subset K,\]
y si fuese $a_{i,j} > 0$ para cierto $i \in \{1,\mathellipsis,m\}$, entonces $\varepsilon$ varía en el conjunto
\[J = \{(x^0_1-\varepsilon a_{1,j},\mathellipsis,x^0_m-\varepsilon a_{m,j},0,\mathellipsis,\varepsilon,\mathellipsis,0)^t \in \R^n \colon 0 < \varepsilon \leq \min_{a_{i,j} > 0} \biggl\{\frac{x_i^0}{a_{i,j}} \biggr\} \} \subset K\]
Además, se tiene que
\[c^tx_\varepsilon^0 = z_0-\varepsilon (z_j-c_j) = z_0,\]
luego todos los $x_\varepsilon^0$ son soluciones óptimas del problema.



Para el segundo apartado, si $y = (y_1,\mathellipsis,y_n)^t$ es otra solución factible del problema, razonando como en la demostración el teorema anterior se llega a la desigualdad
\[z_1y_1+\mathellipsis+z_ny_n < c^ty\]
Nótese que la desigualdad $c_j-z_j >0$ sea estricta permite escribir desigualdades estrictas por doquier en la prueba del teorema anterior. También se probó que $z_1y_1+\mathellipsis+z_ny_n = c^tx^0$, así que $x^0$ es la única solución óptima del problema.
\end{proof}

\section{Método del símplex}

Supóngase que la matriz de un problema de programación lineal en forma estándar posee la base canónica de $\R^m$ en sus primeras $m$ columnas, y escójase una solución básica-factible no degenerada $x^0 = (x^0_1,\mathellipsis,x_m^0,0,\mathellipsis,0)^t$. Si la base canónica no se encontrase en las primeras $m$ columnas de $A$, se procede de la misma manera haciendo los cambios necesarios.



El método del símplex se basa en mover de columna los vectores de la base canónica de $A$ a través de las operaciones de filas adecuadas, obteniendo así distintas soluciones básica-factibles no degeneradas y con valor objetivo sucesivamente más pequeño. Sin más dilación, la receta es la siguiente:

\begin{itemize}
    \item[\textbf{(\textit{1})}]Se calculan los costes reducidos y se construye la correspondiente tablita que reúne todos los datos del problema.
    \item[\textbf{(\textit{2})}]Se distinguen los siguientes casos:
    \begin{itemize}
        \item[\textbf{(\textit{2.1})}]Si $c_j-z_j \geq 0$ para todo $j \in \{1,\mathellipsis,n\}$, entonces $x^0$ es solución óptima.
        \begin{itemize}
            \item[\textbf{(\textit{2.1.1})}]Si $c_j-z_j = 0$ para algún $j \in \{m+1,\mathellipsis,n\}$, entonces el problema tiene infinitas soluciones óptimas de la forma $x_\varepsilon^0 = (x^0_1-\varepsilon a_{1,j},\mathellipsis,x^0_m-\varepsilon a_{m,j},0,\mathellipsis,\varepsilon,\mathellipsis,0)^t$ para ciertos $\varepsilon > 0$ a determinar.
            \begin{itemize}
                \item[\textbf{(\textit{2.1.1.1})}]Si $a_{i,j} \leq 0$ para todo $i \in \{1,\mathellipsis,m\}$, cualquier $\varepsilon > 0$ es bueno. {\textit{La resolución se da por terminada}}.
                \item[\textbf{(\textit{2.1.1.2})}]Si existe $i \in \{1,\mathellipsis,m\}$ tal que $a_{i,j} > 0$, entonces debe ser
                \[0 < \varepsilon \leq \min_{a_{i,j} > 0} \biggl\{\frac{x_i^0}{a_{i,j}} \biggr\}\]
                lo que proporciona todo un segmento (también denominado \textit{rayo}) de soluciones óptimas. {\textit{La resolución se da por terminada}}.
            \end{itemize}
            \item[\textbf{(\textit{2.1.2})}]Si $c_j-z_j >0$ para todo $j \in \{m+1,\mathellipsis,n\}$, entonces $x^0$ es la única solución óptima del problema. {\textit{La resolución se da por terminada}}.
            \end{itemize}
        \item[\textbf{(\textit{2.2})}]Si existe $j \in \{1,\mathellipsis,n\}$ tal que $c_j-z_j < 0$, se escoge el coste reducido negativo de menor índice $q$. Esta elección de $q$ podría hacerse de otras formas, pero puede demostrarse que así se evita que el método entre en bucle.
        \begin{itemize}
            \item[\textbf{(\textit{2.2.1})}]Si $a_{i,q} \leq 0$ para todo $i \in \{1,\mathellipsis,m\}$, estamos ante un problema ilimitado. {\textit{La resolución se da por terminada}}.
            \item[\textbf{(\textit{2.2.2})}]Si existe $i \in \{1,\mathellipsis,m\}$ tal que $a_{i,q} > 0$, se toma
            \[\varepsilon = \frac{x^0_p}{a_{p,q}} = \min_{a_{i,q}>0} \biggl\{\frac{x^0_i}{a_{i,q}}\biggr\},\]
            escogiendo el de menor índice en caso de haber varios, y se realizan las operaciones de filas
            \[f_p' = \frac{1}{a_{p,q}}f_p \qquad \textup{y} \qquad f_i' = f_i -\frac{a_{i,q}}{a_{p,q}}f_p \textup{ para cada } i \in \{1,\mathellipsis,m\} \textup{ con } i \neq p,\]
            lo que resulta en el vector $a_p$ entrando en la base considerada y el vector $a_q$ saliendo. Al haberse obtenido una nueva solución básica-factible no degenerada, se vuelve al principio del paso \textbf{(\textit{2})} y se repite el procedimiento.
        \end{itemize}
    \end{itemize}
\end{itemize}

\begin{example}
Se considera el siguiente problema de programación lineal en forma estándar:
\[\begin{aligned}[t]
&\text{Min. } && -\frac{5}{2}x_1-x_2 \\
& \, \text{s. a} &&\left\{\begin{alignedat}{10}
& 3x_1 & {}+{} & 5x_2 & {}+{} & x_3 &       &     & {}={} 15 \\
& 5x_1 & {}+{} & 2x_2 &       &     & {}+{} & x_4 & {}={}10 \\
& \mathrlap{x_1, x_2,x_3,x_4 \geq 0} \\
\end{alignedat}\right.
\end{aligned}\]
La resolución mediante el método del símplex sería
\begin{center}
\begin{tabular}{|c|c|c|c|c|c|}
    \cline{3-6}
    \multicolumn{1}{c}{} & \multicolumn{1}{c|}{} & \multicolumn{1}{c}{-5/2} & \multicolumn{1}{c}{-1} & \multicolumn{1}{c}{\phantom{-}0} & \multicolumn{1}{c|}{\phantom{-}0} \\ \hline
    0 & \phantom{-}15 & \multicolumn{1}{c}{\phantom{-}3} & \multicolumn{1}{c}{\phantom{-}5} & \multicolumn{1}{c}{\phantom{-}1} & \multicolumn{1}{c|}{\phantom{-}0} \\
    0 & \phantom{-}10 & \multicolumn{1}{c}{\phantom{-}5} & \multicolumn{1}{c}{\phantom{-}2} & \multicolumn{1}{c}{\phantom{-}0} & \multicolumn{1}{c|}{\phantom{-}1} \\ \hline
    \multicolumn{1}{c|}{} & \phantom{-}0 & \multicolumn{1}{c}{-5/2} & \multicolumn{1}{c}{-1} & \multicolumn{1}{c}{\phantom{-}0} & \multicolumn{1}{c|}{\phantom{-}0}\\ \hhline{~|=|=|=|=|=|}
    \multicolumn{1}{c|}{} & \phantom{-}9 & \multicolumn{1}{c}{\phantom{-}0} & \multicolumn{1}{c}{\phantom{-}19/5} & \multicolumn{1}{c}{\phantom{-}1} & \multicolumn{1}{c|}{-3/5} \\
    \multicolumn{1}{c|}{} & \phantom{-}2 & \multicolumn{1}{c}{\phantom{-}1} & \multicolumn{1}{c}{\phantom{-}2/5} & \multicolumn{1}{c}{\phantom{-}0} & \multicolumn{1}{c|}{\phantom{-}1/5} \\ \cline{2-6}
    \multicolumn{1}{c|}{} & -5 & \multicolumn{1}{c}{\phantom{-}0} & \multicolumn{1}{c}{\phantom{-}0} & \multicolumn{1}{c}{\phantom{-}0} & \multicolumn{1}{c|}{\phantom{-}1/2} \\ \cline{2-6}
\end{tabular}
\end{center}
La solución básica de la que se parte es $x^0 = (0,0,15,10)^t$. Observando la cuarta fila de la tabla, hay que tomar $q = 1$. Como $a_{1,1} = 3$ y $a_{2,1} = 5$, se escoge
\[\varepsilon = \min \biggl\{\frac{15}{3},\frac{10}{5} \biggr\} = 2 = \frac{x^0_2}{a_{2,1}}\]
Como $x^0_2 = 10$ es en realidad la cuarta componente de $x^0$, se tiene que $p = 4$. Ahora hay que introducir $a_1$ en la base y sacar $a_4$. Una vez hecho esto, se observa que $c_j-z_j \geq 0$ para todo $j \in \{1,2,3,4\}$, lo que permite afirmar que $x^1 = (2,0,9,0)^t$ es solución óptima-básica no degenerada. Observamos además que el vector de costes reducidos tiene una componente nula (la segunda) que no se corresponde con las componentes no nulas de $x^1$, lo que indica que el problema tiene infinitas soluciones óptimas de la forma $x^1_\varepsilon = (x^1_1-\varepsilon a_{2,2},\varepsilon,x^1_2-\varepsilon a_{1,2},0)^t$. Sea
\[\varepsilon = \min_{a_{i,2} > 0} \biggl\{\frac{x^0_i}{a_{i,2}}\biggr\} = \min \biggl\{\frac{9}{19/5},\frac{2}{2/5}\biggr\} = \frac{9}{19/5}\]
Por tanto, $x^2 = (20/19,45/19,0,0)^t$ es solución óptima-básica,
y el segmento de extremos $x^1$ y $x^2$, es decir, el conjunto
\[\overline{x^1x^2} = \{\lambda x^1+(1-\lambda)x^2 \colon \lambda \in [0,1]\}\]
está lleno de soluciones óptimas. El método del símplex permite concluir que el problema tiene infinitas soluciones óptimas.
\end{example}

\section{Método de las dos fases}

En el método del símplex se necesita de una solución básica-factible para comenzar la resolución. Si la matriz $A$ tiene a la identidad entre sus columnas y las componentes de $b$ son no negativas, la solución con la que inicializar el algoritmo nos sale gratis. Si la identidad no estuviese disponible en la tabla inicial, podría resultar útil la introducción de nuevas variables. Considérese el problema
\[\begin{aligned}[t]
(P) \ \, &\text{Min. } && c^tx \\
& \, \text{s. a} &&\begin{cases}
    Ax = b \\
    x \geq 0
\end{cases}
\end{aligned}\]

Al introducir las variables $x_a \in \R^m$, denominadas {\textit{variables artificales}}, se obtendrá un nuevo problema:
\[\begin{aligned}[t]
(\tilde{P}) \ \,&\text{Min. } && d^tx \\
& \, \text{s. a} &&\begin{cases}
    Ax +x_a= b \\
    x,x_a \geq 0
\end{cases}
\end{aligned}\]

Ahora sí encontraremos la identidad en la primera tabla del método del símplex, así que el problema puede resolverse fácilmente. El quid de la cuestión será la obtención de soluciones de $(P)$ a partir de las de $(\tilde{P})$.

\begin{example}
Se considera el problema
\[\begin{aligned}[t]
&\text{Min. } && -x_1+2x_2 \\
& \, \text{s. a} && \left\{\begin{alignedat}{10}
&  &x_1 & {}+{} & x_2 & {}+{} & x_3 &       &     & {}={} 1 \\
& 2&x_1 & {}-{} & x_2 &       &     & {}-{} & x_4 & {}={} 3 \\
& \mathrlap{x_1,x_2,x_3,x_4 \geq 0}
\end{alignedat}\right.
\end{aligned}\]
La matriz de coeficientes es
\[A = \begin{pmatrix*}[r]
    1 & 1 & 1 & 0 \\
    2 & -1 & 0 & -1
\end{pmatrix*}\]
Si se añade una variable artifical $x_5$, las nuevas restricciones son
\[\left\{\begin{alignedat}{10}
&  &x_1 & {}+{} & x_2 & {}+{} & x_3 &       &     &       &     & {}={} 1 \\
& 2&x_1 & {}-{} & x_2 &       &     & {}-{} & x_4 & {}+{} & x_5 & {}={} 3 \\
& \mathrlap{x_1,x_2,x_3,x_4,x_5 \geq 0}
\end{alignedat}\right.\]
La nueva matriz de coeficientes quedaría
\[\tilde{A} = \begin{pmatrix*}[r]
    1 & 1 & 1 & 0 & 0 \\
    2 & -1 & 0 & -1 & 1
\end{pmatrix*}\]
Esta matriz tiene los vectores de la base canónica en las columnas tercera y quinta, observándose por tanto que $x^0 = (0,0,1,0,3)^t$ es solución básica-factible no degenerada.
\end{example}

Regresando al problema $(P)$, una vez introducidas las variables artificales, una forma de arreglar el vector de costes del nuevo problema $(\tilde{P})$ nos la brinda el {\textit{método de las dos fases}}. La receta es como sigue:
\begin{itemize}
    \item[\textbf{(\textit{F1})}] Se resuelve el problema
    \[\begin{aligned}[t]
    (P_{F1}) \ \,&\text{Min. } && 1^tx_a \\
    & \, \text{s. a} &&\begin{cases}
    Ax+x_a = b \\
    x,x_a \geq 0,
\end{cases}
\end{aligned}\]
donde $1 = (1,1,\mathellipsis,1)^t \in \R^m$. Obsérvese que dicho problema siempre tiene solución (basta tomar $\tilde{x} = (x,x_a)^t = (0,b)^t \in \R^{n+m}$) y no se trata de un problema ilimitado, pues $0$ es una cota inferior de $f$. Sea $(x^*,x_a^*)^t$ una solución óptima-básica de $(P_{F1})$. Se presenta la siguiente disyuntiva:
\begin{itemize}
    \item[\textbf{(\textit{F1.1})}] Supóngase que $x_a^* \neq 0$. Entonces el problema $(P)$ no tiene solución. En efecto, supóngase por reducción al absurdo que $x$ es una solución factible del problema. Entonces $(x,0)^t$ es solución factible de $(P_{F1})$. La funcion objetivo de $(P_{F1})$ vale $1^tx_a^* > 0$ en $(x^*,x_a^*)^t$ y $0$ en $(x,0)^t$, contradiciendo que $(x^*,x_a^*)^t$ sea solución óptima de $(P_{F1})$.
    \item[\textbf{(\textit{F1.2})}] Supóngase que $x_a^* = 0$. Entonces $x^*$ es solución factible de $(P)$ y se distinguen ahora dos casos:
    \begin{itemize}
        \item[\textbf{(\textit{F1.2.1})}] \textit{$x^*$ es solución básica-factible de $(P)$}. El método avanza a la siguiente fase.
        \item[\textbf{(\textit{F1.2.2})}] \textit{$x^*$ no es solución básica-factible de $(P)$}. Esto se debe a que alguna de las columnas de $A$ en la base considerada corresponde a una variable artificial. Por tanto, se toma una base que no tenga que ver con variables artificiales y se reescribe $x^*$ en este base. Ahora sí, $x^*$ es solución básica-factible de $(P)$ y el método avanza a la siguiente fase.
    \end{itemize}
\end{itemize}
    \item[\textbf{(\textit{F2})}] Se resuelve el problema $(P)$ mediante el método del símplex partiendo de la última tabla de la fase $1$ y eliminando las variables artificiales.
\end{itemize}

\begin{example}
Se considera el problema
\[\begin{aligned}[t]
&\text{Min. } && -3x_1+4x_2 \\
& \, \text{s. a} && \left\{\begin{alignedat}{10}
&  &x_1 & {}+{} &  &x_2 & {}+{} & x_3 &       &     & {}={} & \phantom{0}4 \\
& 2&x_1 & {}+{} & 3&x_2 &       &     & {}-{} & x_4 & {}={} & 18 \\
& \mathrlap{x_1,x_2,x_3,x_4 \geq 0}
\end{alignedat}\right.
\end{aligned}\]
La matriz del problema es
\[A= \begin{pmatrix*}[r]
    1 & 1 & 1 & 0 \\
    2 & 3 & 0 & -1
\end{pmatrix*}\]
Entre sus columnas no podemos encontrar los vectores de la base canónica, así que no es mala idea recurrir al método de las dos fases. Se considera el problema
\[\begin{aligned}[t]
&\text{Min. } && -3x_1+4x_2 \\
& \, \text{s. a} && \left\{\begin{alignedat}{10}
&  &x_1 & {}+{} &  &x_2 & {}+{} & x_3 &       &     &       &     & {}={} & \phantom{0}4 \\
& 2&x_1 & {}+{} & 3&x_2 &       &     & {}-{} & x_4 & {}+{} & x_5 & {}={} & 18 \\
& \mathrlap{x_1,x_2,x_3,x_4 \geq 0}
\end{alignedat}\right.
\end{aligned}\]
En este caso se tiene $x_a = (0,x_5)^t$. Para encontrar una solución óptima-básica de este problema, se recurre al método del símplex:
\begin{center}
\begin{tabular}{|c|c|c|c|c|c|c|}
    \cline{3-7}
    
    \multicolumn{1}{c}{} & \multicolumn{1}{c|}{} & \multicolumn{1}{c}{\phantom{-}0} & \multicolumn{1}{c}{\phantom{-}0} & \multicolumn{1}{c}{\phantom{-}0} & \multicolumn{1}{c}{\phantom{-}0} & \multicolumn{1}{c|}{1} \\ \hline
    
    0 & 4 & \multicolumn{1}{c}{\phantom{-}1} & \multicolumn{1}{c}{\phantom{-}1} & \multicolumn{1}{c}{\phantom{-}1} & \multicolumn{1}{c}{\phantom{-}0} & \multicolumn{1}{c|}{0} \\
    
    1 & 18 & \multicolumn{1}{c}{\phantom{-}2} & \multicolumn{1}{c}{\phantom{-}3} & \multicolumn{1}{c}{\phantom{-}0} & \multicolumn{1}{c}{-1} & \multicolumn{1}{c|}{1} \\ \hline
    
    \multicolumn{1}{c|}{} & 18 & \multicolumn{1}{c}{-2} & \multicolumn{1}{c}{-3} & \multicolumn{1}{c}{\phantom{-}0} & \multicolumn{1}{c}{\phantom{-}1} & \multicolumn{1}{c|}{0} \\ \hhline{~|=|=|=|=|=|=|}
    
    \multicolumn{1}{c|}{} & 4 & \multicolumn{1}{c}{\phantom{-}1} & \multicolumn{1}{c}{\phantom{-}1} & \multicolumn{1}{c}{\phantom{-}1} & \multicolumn{1}{c}{\phantom{-}1}& \multicolumn{1}{c|}{0}  \\
    
    \multicolumn{1}{c|}{} & 6 & \multicolumn{1}{c}{-1} & \multicolumn{1}{c}{\phantom{-}0} & \multicolumn{1}{c}{-3} & \multicolumn{1}{c}{-1}& \multicolumn{1}{c|}{1}  \\ \cline{2-7}
    
    \multicolumn{1}{c|}{} & 6 & \multicolumn{1}{c}{\phantom{-}1} & \multicolumn{1}{c}{\phantom{-}0} & \multicolumn{1}{c}{\phantom{-}3} & \multicolumn{1}{c}{\phantom{-}1} & \multicolumn{1}{c|}{0} \\ \cline{2-7}
\end{tabular}
\end{center}
La solución básica-factible de partida es $x^0 =(0,0,4,0,18)^t$. Se escogen $q = 2$ y
\[\varepsilon = \min_{a_{i,2}>0} \biggl\{\frac{x_i^0}{a_{i,2}}\biggr\} = \min \biggl\{\frac{4}{1},\frac{18}{3} \biggr\} = 4\]
Se tiene entonces que $p = 3$, así que $a_2$ entra en la base y sale $a_3$. Como se observa en la última fila de la tabla que $c_j-z_j \geq 0$ para todo $j \in \{1,2,3,4,5\}$, entonces $x^1 = (0,4,0,0,6)^t$ es solución óptima-básica del problema. Como $x_a^* = (0,x^1_5)^t = (0,6)^t \neq (0,0)^t$, el método de las dos fases permite concluir  que el problema de partida no tiene solución.
\end{example}

\chapter{Dualidad}

\section{Caso primal-dual asimétrico}

Considérese un problema de programación lineal en forma estándar, al que se llamará a partir de ahora {\textit{problema primal}}:

\[\begin{aligned}[t]
&\text{Min. } &&c_1x_1+c_2x_2+\mathellipsis+c_nx_n \\
& \, \text{s. a} && \left\{\begin{alignedat}{10}
& & a_{11}&x_1 & {}+{} & & a_{12}&x_2 {}+{} \mathellipsis {}+{} & a_{1n}&x_n {}={} b_1 \\
& & a_{21}&x_1 & {}+{} & & a_{22}&x_2 {}+{} \mathellipsis {}+{} & a_{2n}&x_n {}={} b_2 \\
& & \vdots \\
& & a_{m1}&x_1 & {}+{} & & a_{m2}&x_2 {}+{} \mathellipsis {}+{} & a_{mn}&x_n {}={} b_m \\
& \mathrlap{x_1, x_2,\mathellipsis,x_n \geq 0} \\
\end{alignedat}\right.
\end{aligned}\]
El {\textit{problema dual}} del problema anterior no es más que
\[\begin{aligned}[t]
&\text{Max. } &&b_1\omega_1+b_2\omega_2+\mathellipsis+b_m\omega_m \\
& \; \text{s. a} &&\left\{\begin{alignedat}{10}
& & a_{11}&\omega_1 & {}+{} & & a_{21}&\omega_2 {}+{} \mathellipsis {}+{} & a_{m1}&\omega_m {}={} c_1 \\
& & a_{12}&\omega_1 & {}+{} & & a_{22}&\omega_2 {}+{} \mathellipsis {}+{} & a_{m2}&\omega_m {}={} c_2 \\
& & \vdots \\
& & a_{1n}&\omega_1 & {}+{} & & a_{2n}&\omega_2 {}+{} \mathellipsis {}+{} & a_{nm}&\omega_m {}={} c_n
\end{alignedat}\right.
\end{aligned}\]
En estas circunstancias, se hablará de {\textit{caso primal-dual asimétrico}}. En forma reducida, los problemas anteriores serían
\[\begin{aligned}[t]
\textit{\textit{Primal: }} \ &\text{Min. } && c^tx \\
& \, \text{s. a} &&\begin{cases}
    Ax=b \\
    x \geq 0 \\
\end{cases}
\end{aligned} \qquad \qquad \begin{aligned}[t]
\textit{\textit{Dual: }} \ &\text{Max. } && \omega^t b \\
& \; \text{s. a} &&\begin{cases}
    \omega^t A \leq c^t
\end{cases}
\end{aligned}\]

\begin{example}
Se considera el problema
\[\begin{aligned}[t]
&\text{Min. } && x_2-3x_3+2x_5 \\
& \, \text{s. a} && \left\{\begin{alignedat}{10}
& x_1 & {}+{} & 3x_2 & {}-{} &  &x_3 &       &     & {}+{} & 2x_5 &       &     & {}={} \phantom{0}7 \\
&     & {}-{} & 2x_2 & {}+{} & 4&x_3 & {}+{} & x_4 &       &      &       &     & {}={} 12 \\
&     & {}-{} & 4x_2 & {}+{} & 3&x_3 &       &     & {}+{} & 8x_5 & {}+{} & x_6 & {}={} 10 \\
& \mathrlap{x_1,x_2,x_3,x_4,x_5,x_6 \geq 0} \\
\end{alignedat}\right.
\end{aligned}
\]
Su problema dual sería
\[\begin{aligned}[t]
&\text{Max. } && 7\omega_1+12\omega_2+10\omega_3 \\
& \; \text{s. a} &&\left\{\begin{alignedat}{10}
& \phantom{-}&  &\omega_1 &       &  &         &       &  &         & {}\leq{} &  &0 \\
& \phantom{-}& 3&\omega_1 & {}-{} & 2&\omega_2 & {}-{} & 4&\omega_3 & {}\leq{} &  &1 \\
& {}-{}      &  &\omega_1 & {}+{} & 4&\omega_2 & {}+{} & 3&\omega_3 & {}\leq{} & -&3 \\
&            &  &         &       &  &\omega_2 &       &  &         & {}\leq{} &  &0 \\
& \phantom{-}& 2&\omega_1 &       &  &         & {}+{} & 8&\omega_3 & {}\leq{} &  &2 \\
&            &  &         &       &  &         &       &  &\omega_3 & {}\leq{} &  &0
\end{alignedat}\right.
\end{aligned}\] 
\end{example}

\begin{cproposition}
En el caso primal-dual asimétrico, el problema dual del problema dual no es más que el problema primal.
\end{cproposition}

\begin{proof}
Considérese el problema dual
\[
\begin{aligned}[t]
&\text{Max. } && \omega^t b \\
& \; \text{s. a} &&\begin{cases}
    \omega^t A \leq c^t
\end{cases}
\end{aligned}\]
Lo primero será pasar este problema a forma estándar. Haciendo $\omega = \omega^* - \omega^{**} (\omega^*,\omega^{**} \geq 0)$ se arregla la ausencia de restricciones de signo:
\[
\begin{aligned}[t]
-&\text{Min. } && -(\omega^*)^tb+(\omega^{**})^tb \\
 &\, \text{s. a} &&\begin{cases}
    (\omega^*)^t A-(\omega^{**})^tA \leq c^t \\
    \omega^*,\omega^{**} \geq 0
\end{cases}
\end{aligned}\]
Introduciendo variables de holgura,
\[
\begin{aligned}[t]
-&\text{Min. } && -(\omega^*)^tb+(\omega^{**})^tb \\
 &\, \text{s. a} &&\begin{cases}
    (\omega^*)^t A-(\omega^{**})^tA+(\omega^{***})^t = c^t \\
    \omega^*,\omega^{**},\omega^{***} \geq 0
\end{cases}
\end{aligned}\]
Equivalentemente,
\[
\begin{aligned}[t]
-&\text{Min. } && (-b,b,0)^t(\omega^*,\omega^{**},\omega^{***}) \\
 &\, \text{s. a} &&\begin{cases}
    (A^t \, | \, -A^t \, | \, I)^t(\omega^*,\omega^{**},\omega^{***}) = c^t \\
    \omega^*,\omega^{**},\omega^{***} \geq 0
\end{cases}
\end{aligned}\]
Ahora que el problema está expresado en forma estándar, su dual sería
\[
\begin{aligned}[t]
-&\text{Max. } && -c^tx \\
 &\; \text{s. a} && \left\{\begin{alignedat}{1}
    -Ax &\leq -b \\
    \phantom{-}Ax &\leq \phantom{-}b \\
    -x &\leq \phantom{-}0,
\end{alignedat}\right.
\end{aligned}\]
o lo que es lo mismo,
\[
\begin{aligned}[t]
&\text{Min. } && c^tx \\
 &\, \text{s. a} &&\begin{cases}
    Ax = b \\
    x \geq 0,
\end{cases}
\end{aligned}\]
que es precisamente el problema primal.
\end{proof}

\pagebreak

\begin{cproposition}
\label{prop3.2.}
En el caso primal-dual asimétrico, si $x$ es una solución factible del problema primal y $\omega$ es una solución del problema dual, entonces $\omega^t b \leq c^tx$.
\end{cproposition}

\begin{proof}
En efecto, si $x$ resuelve el primal y $\omega$ el dual, entonces $Ax=b$, luego $\omega^t Ax=\omega^t b$. Además, $\omega^t A \leq c^t$, y como $x \geq 0$, entonces $\omega^t A x \leq c^tx$, concluyéndose que
$\omega^t b = \omega^t Ax \leq c^tx$.
\end{proof}

\begin{ccorollary}
\label{cor3.1.}
En el caso primal-dual asimétrico, si es $x^0$ una solución factible del primal, $\omega^0$ es una solución del dual y se verifica $c^tx^0 = (\omega^0)^tb$, entonces $x^0$ es solución óptima del primal y $\omega^0$ es solución óptima del dual.
\end{ccorollary}

\begin{proof}
Supóngase que $x^0$ es solución factible del primal y $\omega^0$ resuelve el dual, verificándose además que $c^tx^0 = (\omega^0)^tb$. Por la proposición anterior, se tiene que
\[c^tx^0 = (\omega^0)^tb \leq c^tx\]
para cualquier otra solución factible $x$ del primal, es decir, $x^0$ es solución óptima del primal. Además,
\[(\omega^0)^tb = c^tx^0 \geq \omega^t b\]
para cualquier otra solución $\omega$ del dual, lo que quiere decir que $\omega$ es solución óptima del dual.
\end{proof}

\begin{ctheorem}[Teorema de dualidad; caso primal-dual asimétrico]
Se consideran los problemas
\[\begin{aligned}[t]
\textit{\textit{Primal: }} \ &\textup{Min. } && c^tx \\
& \, \textup{s. a} &&\begin{cases}
    Ax=b \\
    x \geq 0 \\
\end{cases}
\end{aligned} \qquad \qquad \begin{aligned}[t]
\textit{\textit{Dual: }} \ &\textup{Max. } && \omega^t b \\
& \; \textup{s. a} &&\begin{cases}
    \omega^t A \leq c^t
\end{cases}
\end{aligned}\]
\begin{enumerate}
    \item Si uno de los problemas tiene solución óptima, entonces el otro también. Además, el valor de la función objetivo en ambas soluciones óptimas coincide.
    \item Si uno de los problemas es ilimitado, entonces el otro es imposible.
    \item Si uno de los problemas es imposible, entonces el otro es ilimitado o imposible.
\end{enumerate}
\end{ctheorem}

\begin{proof}
Para la prueba de $(i)$, supóngase en primer lugar que el problema primal tiene solución óptima. Por el teorema fundamental, existe una solución óptima-básica $x^0=(x_B^0,0)^t \in \R^n$, verificando 
\begin{itemize}
    \item[\textbf{(\textit{a})}] El vector de costes reducidos es no negativo, es decir, $c^t-c_B^tB^{-1}A \geq 0$ (si fuese negativo, el \hyperref[teo2.1.]{\color{gray}Teorema 3} permitiría encontrar soluciones con menor valor de la función objetivo).
    \item[\textbf{(\textit{b})}] $x^0_B = B^{-1}b$.
    \item[\textbf{(\textit{c})}] $c^tx^0 = c_B^tx_B^0$.
    \end{itemize}
    
Sea $\omega^0 \in \R^m$ tal que $(\omega^0)^t = c_B^tB^{-1}$ y veamos que $\omega^0$ es la solución óptima del dual buscada. En primer lugar, $\omega^0$ es solución del dual, pues
\[(\omega^0)^t A = c_B^tB^{-1}A \overset{(a)}{\leq} c^t\]
Además, el valor de la función objetivo en $\omega^0$ es el mismo que el de $x^0$: 
\[(\omega^0)^tb = c_B^tB^{-1}b \overset{(b)}{=} c_B^tx^0_B \overset{(c)}{=} c^tx^0\]

Evidentemente, esto implica que el valor de la función objetivo en $\omega^0$ es el mismo que el de cualquier solución óptima del problema primal. Veamos ahora que $\omega^0$ es solución óptima del dual. Por ser $x^0$ solución factible del primal (pues es solución óptima-básica), la \hyperref[prop3.2.]{\color{gray}Proposición 3} permite afirmar que
\[\omega^t b \leq c^tx^0 = (\omega^0)^tb\]
para toda solución $\omega$ del dual, lo que significa que $\omega^0$ es solución óptima del dual.



Supóngase ahora que el problema dual tiene solución óptima. Por lo ya probado se tendría que el dual del dual, que es precisamente el primal, tiene solución óptima, y el valor de la función objetivo en ambas soluciones óptimas es el mismo.



Con esto queda probado el apartado $(i)$. Para el segundo, supóngase que el problema primal es ilimitado y, por reducción al absurdo, sea $\omega$ una solución del dual. Al ser el primal ilimitado, para todo $K <0$ existe una solución factible $x$ con $c^tx \leq K$. Ahora bien, por la \hyperref[prop3.2.]{\color{gray}Proposición 3}, para todo $K <0$ se debería verificar que $\omega^t b \leq K$, lo cual es imposible porque $\R$ no está acotado inferiormente. 



Para terminar la demostración de $(ii)$, si el problema dual fuese ilimitado, entonces lo que se acaba de probar diría que el dual del dual, o sea, el primal, es un problema imposible.



Por último, si el problema primal fuese imposible, entonces el dual no puede tener solución óptima (si la tuviese entonces el primal también la tendría). Que el dual no tenga solución óptima quiere decir que o bien es imposible, o bien es ilimitado. Y si el dual fuese imposible, lo ya probado permitiría afirmar que el dual del dual (una vez más: el primal) sería imposible o ilimitado.
\end{proof}

\begin{example}
Se considera el problema
\[\begin{aligned}[t]
&\text{Min. } && 3x_1+4x_2+5x_3 \\
& \, \text{s. a} &&\left\{\begin{alignedat}{10}
&  &x_1 & {}+{} & 2&x_2 & {}+{} & 3&x_3 & {}-{} & x_4 &       &     & = 5 \\
& 2&x_1 & {}+{} & 2&x_2 & {}+{} &  &x_3 &       &     & {}-{} & x_5 & = 6 \\
&\mathrlap{x_1,x_2,x_3,x_4,x_5 \geq 0}
\end{alignedat}\right.
\end{aligned}\] 
Como la matriz identidad no se encuentra entre las columnas de la matriz de coeficientes, se recurre al método de las dos fases. Considérese el problema
\[\begin{aligned}[t]
&\text{Min. } && x_6+x_7 \\
& \, \text{s. a} &&\left\{\begin{alignedat}{10}
&  &x_1 & {}+{} & 2&x_2 & {}+{} & 3&x_3 & {}-{} & x_4 &       &     & {}+{} & x_6 &       &     & = 5 \\
& 2&x_1 & {}+{} & 2&x_2 & {}+{} &  &x_3 &       &     & {}-{} & x_5 &       &     & {}+{} & x_7 & = 6 \\
&\mathrlap{x_1,x_2,x_3,x_4,x_5,x_6,x_7 \geq 0}
\end{alignedat}\right.
\end{aligned}\] 
Se resuelve mediante el método del símplex:
\begin{center}
\begin{tabular}{|c|c|c|c|c|c|c|c|c|}
    \cline{3-9}
    
    \multicolumn{1}{c}{} & \multicolumn{1}{c|}{} & \multicolumn{1}{c}{\phantom{-}0} & \multicolumn{1}{c}{\phantom{-}0} & \multicolumn{1}{c}{\phantom{-}0} & \multicolumn{1}{c}{\phantom{-}\textcolor{Red}{0}} & \multicolumn{1}{c}{\phantom{-}\textcolor{Red}{0}} & \multicolumn{1}{c}{\phantom{-}1} & \multicolumn{1}{c|}{\phantom{-}1} \\ \hline
    
    1 & 5 & \multicolumn{1}{c}{\phantom{-}1} & \multicolumn{1}{c}{\phantom{-}2} & \multicolumn{1}{c}{\phantom{-}3} & \multicolumn{1}{c}{-1} & \multicolumn{1}{c}{\phantom{-}0} & \multicolumn{1}{c}{\phantom{-}1} & \multicolumn{1}{c|}{\phantom{-}0} \\
    
    1 & 6 & \multicolumn{1}{c}{\phantom{-}2} & \multicolumn{1}{c}{\phantom{-}2} & \multicolumn{1}{c}{\phantom{-}1} & \multicolumn{1}{c}{\phantom{-}0} & \multicolumn{1}{c}{-1} & \multicolumn{1}{c}{\phantom{-}0} & \multicolumn{1}{c|}{\phantom{-}1}\\ \hline
    
    \multicolumn{1}{c|}{} & 11 & \multicolumn{1}{c}{-3} & \multicolumn{1}{c}{-4} & \multicolumn{1}{c}{-4} & \multicolumn{1}{c}{\phantom{-}1} & \multicolumn{1}{c}{\phantom{-}1} & \multicolumn{1}{c}{\phantom{-}0} & \multicolumn{1}{c|}{\phantom{-}0} \\ \hhline{-|=|=|=|=|=|=|=|=|}
    
    1 & 2 & \multicolumn{1}{c}{\phantom{-}0} & \multicolumn{1}{c}{\phantom{-}1} & \multicolumn{1}{c}{\phantom{-}5/2} & \multicolumn{1}{c}{-1}& \multicolumn{1}{c}{\phantom{-}1/2} & \multicolumn{1}{c}{\phantom{-}1} & \multicolumn{1}{c|}{-1/2}  \\
    
    0 & 3 & \multicolumn{1}{c}{\phantom{-}1} & \multicolumn{1}{c}{\phantom{-}1} & \multicolumn{1}{c}{\phantom{-}1/2} & \multicolumn{1}{c}{\phantom{-}0}& \multicolumn{1}{c}{-1/2} & \multicolumn{1}{c}{\phantom{-}0} & \multicolumn{1}{c|}{\phantom{-}1/2} \\ \cline{1-9}
    
    \multicolumn{1}{c|}{} & 2 & \multicolumn{1}{c}{\phantom{-}0} & \multicolumn{1}{c}{-1} & \multicolumn{1}{c}{-5/2} & \multicolumn{1}{c}{\phantom{-}1} & \multicolumn{1}{c}{-1/2} & \multicolumn{1}{c}{\phantom{-}0} & \multicolumn{1}{c|}{\phantom{-}3/2} \\ \hhline{-|=|=|=|=|=|=|=|=|}

    0 & 2 & \multicolumn{1}{c}{\phantom{-}0} & \multicolumn{1}{c}{\phantom{-}1} & \multicolumn{1}{c}{\phantom{-}5/2} & \multicolumn{1}{c}{-1}& \multicolumn{1}{c}{\phantom{-}1/2} & \multicolumn{1}{c}{\phantom{-}1} & \multicolumn{1}{c|}{-1/2}  \\
    
    0 & 1 & \multicolumn{1}{c}{\phantom{-}1} & \multicolumn{1}{c}{\phantom{-}0} & \multicolumn{1}{c}{-2} & \multicolumn{1}{c}{\phantom{-}1}& \multicolumn{1}{c}{-1} & \multicolumn{1}{c}{-1} & \multicolumn{1}{c|}{\phantom{-}1} \\ \cline{1-9}
    
    \multicolumn{1}{c|}{} & 0 & \multicolumn{1}{c}{\phantom{-}0} & \multicolumn{1}{c}{\phantom{-}0} & \multicolumn{1}{c}{\phantom{-}0} & \multicolumn{1}{c}{\phantom{-}0} & \multicolumn{1}{c}{\phantom{-}0} & \multicolumn{1}{c}{\phantom{-}1} & \multicolumn{1}{c|}{\phantom{-}1} \\ \cline{2-9}
\end{tabular}
\end{center}
Se obtiene la solución óptima-básica $(x^*,x_a^*)^t = (1,2,0,0,0,0,0)^t$. Como $x_a^* = (0,0)^t$, entonces se tiene que $x^* = (1,2,0,0,0)^t$ es solución básica-factible del problema de partida. Ahora se inicializa el método del símplex para dicho problema con solución inicial $x^*$:
\begin{center}
\begin{tabular}{|c|c|c|c|c|c|c|}
    \cline{3-7}
    
    \multicolumn{1}{c}{} & \multicolumn{1}{c|}{} & \multicolumn{1}{c}{3} & \multicolumn{1}{c}{4} & \multicolumn{1}{c}{\phantom{-}5} & \multicolumn{1}{c}{\phantom{-}0} & \multicolumn{1}{c|}{\phantom{-}0} \\ \hline
    
    4 & 2 & \multicolumn{1}{c}{0} & \multicolumn{1}{c}{1} & \multicolumn{1}{c}{\phantom{-}5/2} & \multicolumn{1}{c}{-1} & \multicolumn{1}{c|}{\phantom{-}1/2} \\
    
    3 & 1 & \multicolumn{1}{c}{1} & \multicolumn{1}{c}{0} & \multicolumn{1}{c}{-2} & \multicolumn{1}{c}{\phantom{-}1} & \multicolumn{1}{c|}{-1} \\ \hline
    
    \multicolumn{1}{c|}{} & 11 & \multicolumn{1}{c}{0} & \multicolumn{1}{c}{0} & \multicolumn{1}{c}{\phantom{-}1} & \multicolumn{1}{c}{\phantom{-}\textcolor{Green}{1}} & \multicolumn{1}{c|}{\phantom{-}\textcolor{Green}{1}} \\ \cline{2-7}
\end{tabular}
\end{center}
De observar los costes reducidos se deduce que $x^*$ es la única solución óptima del problema. Por el teorema de dualidad, sabemos que el problema
\[\begin{aligned}[t]
&\text{Max. } && 5\omega_1+6\omega_2 \\
& \; \text{s. a} &&\left\{\begin{alignedat}{10}
& &\omega_1 & {}+{} & 2&\omega_2 & {}\leq{} 3 \\
&2&\omega_1 & {}+{} & 2&\omega_2 & {}\leq{} 4 \\
&3&\omega_1 & {}+{} &  &\omega_2 & {}\leq{} 5 \\
&-&\omega_1 &       &  &         & {}\leq{} 0 \\
& &         &       & -&\omega_2 & {}\leq{} 0 \\
\end{alignedat}\right.
\end{aligned}\] 
tiene solución óptima. Esta solución es $(\omega^0)^t = c_B^tB^{-1}$, donde $c_B = (4,3)^t$ y
\[B = (a_2 \, | \, a_1) = \begin{pmatrix}
    2 & 1 \\
    2 & 2 \\
\end{pmatrix}\]
Por tanto,
\[(\omega^0)^t = \begin{pmatrix}
    4 & 3
\end{pmatrix} \begin{pmatrix}
    \phantom{-}1 & -\frac{1}{2} \\
    -1 & \phantom{-}1
\end{pmatrix} = \begin{pmatrix}
    1 & 1
\end{pmatrix}\]
Obsérvese que en cualquier iteración del método del símplex los costes reducidos son
\[c_j-z_j = c_j-c_B^tB^{-1}a_j\]
Si en la columna $j$-ésima de $A$ se encontrase la columna $i$-ésima de la matriz identidad (o sea, el vector de la base canónica $e_i$), los costes reducidos de una iteración posterior serían
\[\overline{c_j-z_j} = c_j-c_B^tB^{-1}e_i = c_j-\omega^0_i,\]
luego $\omega^0_i = c_j -\overline{c_j-z_j}$, lo que permite hallar soluciones del problema dual en cada iteración del método del símplex mediante la mera contemplación de la tabla, sin realizar cálculos matriciales. En el ejemplo anterior, se tiene que $a_4 = -e_1$ y $a_5 = -e_2$, luego se tendría
\[\omega^0_1 = \textcolor{Green}{\overline{c_4-z_4}}-\textcolor{Red}{c_4} = 1-0=1 \qquad \textup{y} \qquad \omega^0_2 = \textcolor{Green}{\overline{c_5-z_5}}-\textcolor{Red}{c_5} =1-0=1\]
Los colores deberían indicar en qué rincones de la tabla hay que fijarse.
\end{example}

\section{Caso primal-dual simétrico}

Se va a tratar un caso de dualidad en el cual el problema primal no se encuentra en forma estándar. Considérense los problemas
\[\begin{aligned}[t]
\textit{\textit{Primal: }} \ &\text{Min. } && c^tx \\
& \, \text{s. a} &&\begin{cases}
    Ax\geq b \\
    x \geq 0 \\
\end{cases}
\end{aligned} \qquad \qquad \begin{aligned}[t]
\textit{\textit{Dual: }} \ &\text{Max. } && \omega^t b \\
& \; \text{s. a} &&\begin{cases}
    \omega^t A \leq c^t \\
    \omega \geq 0
\end{cases}
\end{aligned}\]
Por motivos evidentes, en esta situación se habla de {\textit{caso primal-dual simétrico}}. En realidad, este caso es un caso particular del asimétrico, pues pasando el problema primal a forma estándar se obtendría
\[\begin{aligned}[t]
\textit{\textit{Primal: }} \ &\text{Min. } && c^tx \\
& \, \text{s. a} &&\begin{cases}
    Ax-y = b \\
    x,y \geq 0 \\
\end{cases}
\end{aligned}\]
El dual de este problema (según el caso asimétrico) es
\[\begin{aligned}[t]
\textit{\textit{Dual: }} \ &\textup{Max. } && \omega^t b \\
& \; \textup{s. a} &&\begin{cases}
    \omega^t A \leq c^t \\
    -\omega \leq 0,
\end{cases}
\end{aligned}\]
obteniéndose el mismo dual de antes.

\begin{ctheorem}[Teorema de holgura complementaria]
Considérense los problemas
\[\begin{aligned}[t]
\textit{\textit{Primal: }} \ &\textup{Min. } && c^tx \\
& \, \textup{s. a} &&\begin{cases}
    Ax \geq b \\
    x \geq 0 \\
\end{cases}
\end{aligned} \qquad \qquad \begin{aligned}[t]
\textit{\textit{Dual: }} \ &\textup{Max. } && \omega^t b \\
& \; \textup{s. a} &&\begin{cases}
    \omega^t A \leq c^t \\
    \omega \geq 0
\end{cases}
\end{aligned}\]
En las soluciones óptimas de cada uno de los problemas, se verifica:
\begin{enumerate}
    \item Si la $k$-ésima variable de una de las dos soluciones es positiva, entonces la $k$-ésima restricción de su dual es una igualdad.
    \item Si la $k$-ésima restricción de uno de los dos problemas es una desigualdad estricta, entonces la $k$-ésima variable de la solución de su dual es nula.
\end{enumerate}
\end{ctheorem}

\begin{proof}
En primer lugar, se van a pasar ambos problemas a forma estándar. El primal sería
\[\begin{aligned}[t]
&\text{Min. } &&c_1x_1+c_2x_2+\mathellipsis+c_nx_n \\
& \, \text{s. a} && \left\{\begin{alignedat}{10}
& & a_{11}&x_1 & {}+{} & & a_{12}&x_2 {}+{} \mathellipsis {}+{} & a_{1n}&x_n {}-{} & x_{n+1} & {}={} b_1 \\
& & a_{21}&x_1 & {}+{} & & a_{22}&x_2 {}+{} \mathellipsis {}+{} & a_{2n}&x_n {}-{} & x_{n+2} & {}={} b_2 \\
& & \vdots \\
& & a_{m1}&x_1 & {}+{} & & a_{m2}&x_2 {}+{} \mathellipsis {}+{} & a_{mn}&x_n {}-{} & x_{n+m} & {}={} b_m \\
& \mathrlap{x_1, x_2,\mathellipsis,x_n,x_{n+1},x_{n+2},\mathellipsis,x_{n+m} \geq 0,} \\
\end{alignedat}\right.
\end{aligned}\]
y el dual,
\[\begin{aligned}[t]
&\text{Max. } &&b_1\omega_1+b_2\omega_2+\mathellipsis+b_m\omega_m \\
& \; \text{s. a} &&\left\{\begin{alignedat}{10}
& & a_{11}&\omega_1 & {}+{} & & a_{21}&\omega_2 {}+{} \mathellipsis {}+{} & a_{m1}& \omega_m {}+{} \omega_{m+1} & {}={} c_1 \\
& & a_{12}&\omega_1 & {}+{} & & a_{22}&\omega_2 {}+{} \mathellipsis {}+{} & a_{m2}& \omega_m {}+{} \omega_{m+2} & {}={} c_2 \\
& & \vdots \\
& & a_{1n}&\omega_1 & {}+{} & & a_{2n}&\omega_2 {}+{} \mathellipsis {}+{} & a_{nm}& \omega_m {}+{} \omega_{m+n} & {}={} c_n \\
& \mathrlap{\omega_1,\mathellipsis,\omega_m,\omega_{m+1}, \omega_{m+2},\mathellipsis,\omega_{m+n} \geq 0} \\
\end{alignedat}\right.
\end{aligned}\]
Sean $x^0 =(x_1^0,\mathellipsis,x_n^0,x_{n+1}^0,\mathellipsis,x_{n+m}^0)^t$ y $\omega^0 = (\omega_1^0,\mathellipsis,\omega_m^0,\omega_{m+1}^0,\mathellipsis,\omega_{m+n}^0)^t$ soluciones óptimas del primal y el dual, respectivamente. Entonces
\[
\begin{aligned}[t]
b_1\omega_1^0+\mathellipsis+b_m\omega_m^0 &= \begin{aligned}[t]
&(a_{11}x^0_1+a_{12}x_2^0+\mathellipsis+a_{1n}x_n^0-x_{n+1})\omega^0_1 {}+{} \mathellipsis {}+{} \\
&(a_{m1}x^0_1+a_{m2}x_2^0+\mathellipsis+a_{mn}x_n^0-x_{n+m})\omega^0_m
\end{aligned} \\
&= \begin{aligned}[t]
&(a_{11}\omega^0_1+\mathellipsis+a_{m1}\omega^0_m)x_1^0 {}+{} \\
&(a_{12}\omega^0_1+\mathellipsis+a_{m2}\omega^0_m)x_2^0 {}+{} \mathellipsis {}+{} \\
&(a_{1n}\omega^0_1+\mathellipsis+a_{mn}\omega^0_m)x_n^0 {}-{} x_{n+1}^0\omega^0_1-\mathellipsis-x^0_{n+m}\omega_m^0
\end{aligned} \\
&= (c_1-\omega^0_{m+1})x_1^0+\mathellipsis+ (c_n-\omega^0_{m+n})x_n^0 - x_{n+1}^0\omega^0_1-\mathellipsis-x^0_{n+m}\omega_m^0 \\
\end{aligned}\]
Por otro lado, por el teorema de dualidad,
\[b_1\omega_1^0+\mathellipsis+b_m\omega^0_m = c_1x_1^0+\mathellipsis+c_nx_n^0,\]
así que se tiene que
\[(c_1-\omega^0_{m+1})x_1^0+\mathellipsis+ (c_n-\omega^0_{m+n})x_n^0 - x_{n+1}^0\omega^0_1-\mathellipsis-x^0_{n+m}\omega_m^0 = c_1x_1^0+\mathellipsis+c_nx_n^0,\]
o lo que es lo mismo,
\[x_1^0\omega^0_{m+1}+\mathellipsis+x_n^0\omega^0_{m+n}+x_{n+1}^0\omega_1^0+\mathellipsis+x_{n+m}^0\omega^0_m = 0\]
Como todos los términos que intervienen son no negativos, entonces $x^0_i\omega^0_{m+i} = 0$ para todo $i \in \{1,\mathellipsis,n\}$ y $x^0_{n+i}\omega_i^0 = 0$ para todo $j \in \{1,\mathellipsis,m\}$. Por tanto,
\begin{itemize}
    \item[\textbf{(\textit{a})}] Para cada $k \in \{1,\mathellipsis,n\}$, si $x^0_k >0$, entonces $\omega^0_{m+k} =0$, y si $\omega^0_{m+k} > 0$, entonces $x^0_{k} = 0$.
    \item[\textbf{(\textit{b})}] Para cada $k \in \{1,\mathellipsis,m\}$, si $x^0_{n+k} >0$, entonces $\omega^0_{k} =0$, y si $\omega^0_{k} > 0$, entonces $x^0_{n+k} = 0$.
\end{itemize}
Esto es exactamente lo que dice el enunciado del teorema.
\end{proof}

\section{Otros casos de dualidad}

Se va a considerar un problema de programación lineal que sea diferente de los dos tipos de problemas primales vistos hasta ahora, por ejemplo
\[\begin{aligned}[t]
&\text{Min. } && c_1x_1+c_2x_2+c_3x_3 \\
& \, \text{s. a} &&\left\{\begin{alignedat}{10}
&a_{11}x_1+a_{12}x_2+a_{13}x_3 \geq b_1 \\
&a_{21}x_1+a_{22}x_2+a_{23}x_3 \leq b_2 \\
&a_{31}x_1+a_{32}x_2+a_{33}x_3 = b_3 \\
&\mathrlap{x_1 \geq 0} \\
&\mathrlap{x_2 \leq 0}
\end{alignedat}\right.
\end{aligned}\]
Pasando a forma estándar,
\[\begin{aligned}[t]
&\text{Min. } && c_1x_1-c_2x_2^*+c_3x_3^*-c_3x_3^{**} \\
& \, \text{s. a} &&\left\{\begin{alignedat}{10}
&a_{11}x_1-a_{12}x_2^*+a_{13}x_3^*-a_{13}x_3^{**}-x_4 & {}={} b_1 \\
&a_{21}x_1-a_{22}x_2^*+a_{23}x_3^*-a_{23}x_3^{**}+x_5 & {}={} b_2 \\
&a_{31}x_1-a_{32}x_2^*+a_{33}x_3^*-a_{33}x_3^{**}     & {}={} b_3 \\
&\mathrlap{x_1,x_2^*,x_3^*,x_3^{**},x_4,x_5\geq0} \\
\end{alignedat}\right.
\end{aligned}\]
En este problema sí queda claro quién es el dual:
\[\begin{aligned}[t]
&\text{Max. } && b_1\omega_1+b_2\omega_2+b_3\omega_3 \\
& \; \text{s. a} &&\left\{\begin{alignedat}{10}
& &a_{11}\omega_1 & {}+{} & a_{21}&\omega_2 & {}+{} & a_{31}\omega_3 & {}\leq{} &  &c_1 \\
&-&a_{12}\omega_1 & {}-{} & a_{22}&\omega_2 & {}-{} & a_{32}\omega_3 & {}\leq{} & -&c_2  \\
& &a_{13}\omega_1 & {}+{} & a_{23}&\omega_2 & {}+{} & a_{33}\omega_3 & {}={}    &  &c_3 \\
&-&\omega_1       &       &       &         &       &                & {}\leq{} &  &0 \\
& &               &       &       &\omega_2 &       &                & {}\leq{} &  &0
\end{alignedat}\right.
\end{aligned}\]
En general, cualquier transformación primal-dual o dual-primal se resume en la tabla siguiente:
\begin{center}
\begin{tabular}{|c|c|c|c|}
    \cline{2-3}

   \multicolumn{1}{c|}{}  & \text{Min.} & \text{Max.} & \multicolumn{1}{c}{} \\ \hline

                          & $\geq$ & $\geq 0$ &  \\ \cline{2-3}
    
    \text{Restricciones}  & $\leq$ & $\leq 0$ & \text{Variables} \\ \cline{2-3}
    
                          & $=$    & $\in \R$ &  \\ \hline

                          & $\geq 0$ & $\leq$ &  \\ \cline{2-3}
    
    \text{Variables}  & $\leq 0$ & $\geq$ & \text{Restricciones} \\ \cline{2-3}
    
                          & $\in \R$    & $=$ &  \\ \hline
\end{tabular}
\end{center}

\section{Método dual del símplex}

En esta sección se tratará de aplicar el algoritmo del símplex para obtener soluciones óptimas del problema dual. Se consideran los problemas
\[\begin{aligned}[t]
\textit{\textit{Primal: }} \ &\textup{Min. } && c^tx \\
& \, \textup{s. a} &&\begin{cases}
    Ax=b \\
    x \geq 0 \\
\end{cases}
\end{aligned} \qquad \qquad \begin{aligned}[t]
\textit{\textit{Dual: }} \ &\textup{Max. } && \omega^t b \\
& \; \textup{s. a} &&\begin{cases}
    \omega^t A \leq c^t
\end{cases}
\end{aligned}\]

Se recuerda que si los costes reducidos verifican $c_j-z_j = c_j-c_B^tB^{-1}a_j \geq 0$ para cada $j$, se puede asegurar que $\omega^t = c_B^tB^{-1}$ es solución del problema dual. Es por esto que en el algoritmo dual del símplex siempre se partirá de un vector de costes reducidos no negativos.



Es más, si fuese $b_i \geq 0$ para todo $i \in \{1,\mathellipsis,m\}$, entonces el método del símplex aplicado a la solución básica-factible inicial $(b,0)^t$ permitiría conseguir una solución óptima $x^0 = (x^0_B,0)^t$, y así obtendríamos una solución óptima del problema dual. Ahora bien, si algún $b_i$ es negativo, habrá que apañárselas de otra manera, pues no es posible inicializar el método del símplex con $(b,0)^t$. 



Supóngase entonces que existe $r \in \{1,\mathellipsis, m\}$ tal que $b_r < 0$, y supóngase también que existe $i \in \{1,\mathellipsis,n\}$ tal que $a_{r,i} <0$. Sea
    \[\varepsilon = \min_{a_{r,i}<0} \biggl\{-\frac{c_i-z_i}{a_{r,i}}\biggr\} = -\frac{c_k-z_k}{a_{r,k}}\]
    Entonces se puede hacer
    \[f_r' = \frac{1}{a_{r,k}}f_r \qquad \textup{y} \qquad f_i' = f_i -\frac{a_{i,k}}{a_{r,k}}f_r \textup{ para cada } i \in \{1,\mathellipsis,m\} \textup{ con } i \neq r,\]
    obteniéndose una nueva tabla en la que $\overline{b_r} = \frac{1}{a_{r,k}}b_r > 0$. Además, los nuevos costes reducidos son
    \[\overline{c_j-z_j} = c_j-z_j-\frac{a_{r,j}}{a_{r,k}}(c_k-z_k)\]
    En cuanto a la función objetivo, el nuevo valor sería
    \[\overline{z_0} = z_0-b_r\varepsilon = z_0+\frac{b_r}{a_{r,k}}(c_k-z_k) \tag{$*$}\]
    
    Evidentemente, conviene que sea $\overline{c_j-z_j} \geq 0$ (para obtener nuevas soluciones duales) y $\overline{z_0} \geq z_0$ (para que el valor de la función objetivo se acerque más al valor óptimo del problema dual). 
    \begin{enumerate}
    \item Veamos que $\overline{c_j-z_j} \geq 0$. Si $a_{r,j} \geq 0$, entonces
    \[\overline{c_j-z_j} = c_j-z_j-\frac{a_{r,j}}{a_{r,k}}(c_k-z_k) \geq 0\]
    por ser $c_j-z_j, c_k-z_k \geq 0$ y $a_{r,k} < 0$. Y si fuese $a_{r,j} < 0$, entonces se multiplica por $a_{r,j}$ en la desigualdad
    \[-\frac{c_j-z_j}{a_{r,j}} \geq -\frac{c_k-z_k}{a_{r,k}},\]
    obteniéndose
    \[c_j-z_j \geq a_{r,j} \frac{c_k-z_k}{a_{r,k}},\]
    de donde se deduce inmediatamente que $\overline{c_j-z_j} \geq 0$.
    \item Es claro que $\overline{z_0} \geq z_0$, pues se tiene que
    \[\frac{b_r}{a_{r,k}}(c_k-z_k) \geq 0,\]
    desigualdad que se verifica por ser $b_r, a_{r,k} <0$ y $c_k-z_k \geq 0$.
    \end{enumerate}
    
Por tanto, mediante este procedimiento puede asegurarse que el método del símplex sigue adelante y proporciona sucesivas soluciones duales con valores objetivos cada vez más grandes.


    
Falta por descartar el caso $a_{r,i} \geq 0$ para todo $i \in \{1,\mathellipsis,n\}$. En estas circunstancias, se tiene que el problema primal es imposible, pues si $x^*$ fuese solución factible del primal, entonces se tendría
\[a_{r,1}x_1^*+\mathellipsis+a_{r,n}x_n^* = b_r,\]
y como $a_{r,i} \geq 0$ y $x_i^* \geq 0$ para todo $i \in \{1,\mathellipsis,n\}$, debería ser $b_r \geq 0$, que contradice lo que se supuso anteriormente.



Por último, en $(*)$ se observa que si $b_r = 0$ para algún $r \in \{1,\mathellipsis,m\}$, entonces las sucesivas iteraciones del método darían soluciones óptimas con el mismo valor objetivo, es decir, el problema dual tendría múltiples soluciones óptimas. Supóngase entonces que se tiene una solución óptima $\omega^0$ del dual y que $b_r = 0$. Se distinguen dos casos:

\begin{enumerate}
    \item \textit{Existe $j \in \{1,\mathellipsis,n\}$ tal que $a_{r,j} < 0$}. Sea
    \[\varepsilon = \min_{a_{r,j}<0} \biggl\{-\frac{c_j-z_j}{a_{r,j}}\biggr\} = -\frac{c_k-z_k}{a_{r,k}}\]
    Pivotando sobre $a_{r,k}$ se obtiene una nueva solución del dual, $\omega_\varepsilon^0$, que verifica $\overline{z_0} = z_0$, es decir, $\omega_\varepsilon^0$ es otra solución óptima del dual. Por tanto, el segmento
    \[\overline{\omega^0\omega^0_\varepsilon} = \{\lambda \omega^0 +(1-\lambda)\omega^0_\varepsilon \colon \lambda \in [0,1]\}\]
    está lleno de soluciones óptimas del dual.
    \item \textit{Para todo $j \in \{1,\mathellipsis,n\}$ se tiene que $a_{r,j} \geq 0$}. En este caso, mediante un razonamiento similar al de antes, se prueba el problema primal es imposible, así que el teorema de dualidad diría que el dual es ilimitado o imposible. En realidad, el dual es ilimitado; no puede ser imposible porque se ha supuesto que los costes reducidos son positivos. Además, si la matriz $A$ es de la forma $A = (I \, | \, D)$, entonces, para todo $\varepsilon>0$, $\omega^0_\varepsilon = (\omega^0-\varepsilon a_{r,1},\mathellipsis,\omega^0-\varepsilon a_{r,m})^t$ es solución factible del dual.
\end{enumerate}

En resumidas cuentas, todos los razonamientos anteriores dan a luz a una receta, bautizada {\textit{método dual del símplex}}, que permite resolver el problema dual:

\begin{itemize}
    \item[\textbf{(\textit{1})}]Se parte de una tabla de costes reducidos positivos.
    \item[\textbf{(\textit{2})}]Se distinguen dos casos:
    \begin{itemize}
        \item[\textbf{(\textit{2.1})}]Supóngase que $b_i \geq 0$ para todo $i \in \{1,\mathellipsis,m\}$. Los $b_i$ son las componentes básicas de una solución óptima-básica del primal, pudiéndose obtener una solución óptima-básica del dual. Si $b_i >0$ para todo $i \in \{1,\mathellipsis,m\}$, el dual tiene una única solución óptima; si no, se observa el signo de los elementos de la fila $r$-ésima de $A$ para obtener múltiples soluciones del problema dual. {\textit{La resolución se da por terminada}}.
        \item[\textbf{(\textit{2.2})}]Supóngase que $b_r < 0$ para cierto $r \in \{1,\mathellipsis,m\}$. Si existen varios candidatos, se escoge el de menor índice. Se distinguen dos nuevos casos:
        \begin{itemize}
            \item[\textbf{(\textit{2.2.1})}]Supóngase que $a_{r,j}\geq 0$ para todo $j \in \{1,\mathellipsis,m\}$. Entonces el problema dual es ilimitado, y para todo $\varepsilon>0$, $\omega^0_\varepsilon = (\omega^0-\varepsilon a_{r,1},\mathellipsis,\omega^0-\varepsilon a_{r,m})^t$ es solución factible. {\textit{La resolución se da por terminada}}.
            \item[\textbf{(\textit{2.2.2})}]Supóngase que $a_{r,j} < 0$ para algún $j \in \{1,\mathellipsis,m\}$. Entonces se puede escoger
            \[\varepsilon = \min_{a_{r,j} <0} \biggl\{-\frac{c_j-z_j}{a_{r,j}}\biggr\} = -\frac{c_k-z_k}{a_{r,k}}\]
            Se pivota sobre $a_{r,k}$ mediante las conocidas operaciones de filas
            \[f_r' = \frac{1}{a_{r,k}}f_r \qquad \textup{y} \qquad f_i' = f_i -\frac{a_{i,k}}{a_{r,k}}f_r \textup{ para cada } i \in \{1,\mathellipsis,m\} \textup{ con } i \neq r,\]
            y se obtiene una nueva tabla con costes reducidos negativos y con mayor valor objetivo de antes. Marcha atrás: vuélvase al paso \textbf{(\textit{1})}.
        \end{itemize}
    \end{itemize}
\end{itemize}

\begin{example}
Considérese el problema de programación lineal siguiente:
\[\begin{aligned}[t]
&\text{Min. } && 2x_1+3x_2+4x_3 \\
& \, \text{s. a} &&\left\{\begin{alignedat}{10}
& &x_1 & {}+{} & 2&x_2 & {}+{} &  &x_3 & {}\geq{} & 3 \\
&2&x_1 & {}-{} &  &x_2 & {}+{} & 3&x_3 & {}\geq{} & 4  \\
&\mathrlap{x_1,x_2,x_3 \geq 0}
\end{alignedat}\right.
\end{aligned}\]
Se pasa a forma estándar:
\[\begin{aligned}[t]
(P_1)\ \,&\text{Min. } && 2x_1+3x_2+4x_3 \\
& \, \text{s. a} &&\left\{\begin{alignedat}{10}
& &x_1 & {}+{} & 2&x_2 & {}+{} &  &x_3 & {}-{} & x_4 &       &     & {}={} & 3 \\
&2&x_1 & {}-{} &  &x_2 & {}+{} & 3&x_3 &       &     & {}-{} & x_5 & {}={} & 4  \\
&\mathrlap{x_1,x_2,x_3,x_4,x_5 \geq 0}
\end{alignedat}\right.
\end{aligned}\]
Para tener la base canónica entre las columnas de $A$, se multiplican ambas ecuaciones por $-1$:
\[\begin{aligned}[t]
(P_2)\ \,&\text{Min. } && 2x_1+3x_2+4x_3 \\
& \, \text{s. a} &&\left\{\begin{alignedat}{10}
& -&x_1 & {}-{} & 2&x_2 & {}-{} &  &x_3 & {}+{} & x_4 &       &     & {}={} & -3 \\
&-2&x_1 & {}+{} &  &x_2 & {}-{} & 3&x_3 &       &     & {}+{} & x_5 & {}={} & -4  \\
&\mathrlap{x_1,x_2,x_3,x_4,x_5 \geq 0}
\end{alignedat}\right.
\end{aligned}\]
Cabe remarcar que este cambio de signo afectará al signo de las soluciones obtenidas al término de la resolución. Se aplica el algoritmo dual del símplex:
\begin{center}
\begin{tabular}{|c|c|c|c|c|c|c|}
    \cline{3-7}
    \multicolumn{1}{c}{} & \multicolumn{1}{c|}{} & \multicolumn{1}{c}{\phantom{-}2} & \multicolumn{1}{c}{\phantom{-}3} & \multicolumn{1}{c}{\phantom{-}4} & \multicolumn{1}{c}{\phantom{-}0} & \multicolumn{1}{c|}{\phantom{-}0} \\ \hline
    
    0 & -3 & \multicolumn{1}{c}{-1} & \multicolumn{1}{c}{-2} & \multicolumn{1}{c}{-1} & \multicolumn{1}{c}{\phantom{-}1} & \multicolumn{1}{c|}{\phantom{-}0} \\
    
    0 & -4 & \multicolumn{1}{c}{-2} & \multicolumn{1}{c}{\phantom{-}1} & \multicolumn{1}{c}{-3} & \multicolumn{1}{c}{\phantom{-}0} & \multicolumn{1}{c|}{\phantom{-}1} \\ \hline
    
    \multicolumn{1}{c|}{} & \phantom{-}0 & \multicolumn{1}{c}{\phantom{-}2} & \multicolumn{1}{c}{\phantom{-}3} & \multicolumn{1}{c}{\phantom{-}4} & \multicolumn{1}{c}{\phantom{-}0} & \multicolumn{1}{c|}{\phantom{-}0} \\ \hhline{-|=|=|=|=|=|=|}

    3 & \phantom{-}3/2 & \multicolumn{1}{c}{\phantom{-}1/2} & \multicolumn{1}{c}{\phantom{-}1} & \multicolumn{1}{c}{\phantom{-}1/2} & \multicolumn{1}{c}{-1/2} & \multicolumn{1}{c|}{\phantom{-}0} \\
    
    0 & -11/2 & \multicolumn{1}{c}{-5/2} & \multicolumn{1}{c}{\phantom{-}0} & \multicolumn{1}{c}{-7/2} & \multicolumn{1}{c}{\phantom{-}1/2} & \multicolumn{1}{c|}{\phantom{-}1} \\ \hline
    
    \multicolumn{1}{c|}{} & \phantom{-}9/2 & \multicolumn{1}{c}{\phantom{-}1/2} & \multicolumn{1}{c}{\phantom{-}0} & \multicolumn{1}{c}{\phantom{-}5/2} & \multicolumn{1}{c}{\phantom{-}3/2} & \multicolumn{1}{c|}{\phantom{-}0} \\ \hhline{-|=|=|=|=|=|=|}

    3 & \phantom{-}2/5 & \multicolumn{1}{c}{\phantom{-}0} & \multicolumn{1}{c}{\phantom{-}1} & \multicolumn{1}{c}{-1/5} & \multicolumn{1}{c}{-2/5} & \multicolumn{1}{c|}{\phantom{-}1/5} \\
    
    2 & \phantom{-}11/5 & \multicolumn{1}{c}{\phantom{-}1} & \multicolumn{1}{c}{\phantom{-}0} & \multicolumn{1}{c}{\phantom{-}7/5} & \multicolumn{1}{c}{-1/5} & \multicolumn{1}{c|}{-2/5} \\ \hline
    
    \multicolumn{1}{c|}{} & \phantom{-}28/5 & \multicolumn{1}{c}{\phantom{-}0} & \multicolumn{1}{c}{\phantom{-}0} & \multicolumn{1}{c}{\phantom{-}9/5} & \multicolumn{1}{c}{\phantom{-}8/5} & \multicolumn{1}{c|}{\phantom{-}1/5} \\
    
    \cline{2-7}
\end{tabular}
\end{center}
La solución del problema dual que se obtiene en la primera parte de la tabla es $\omega^0 = (0,0)^t$. Se escoge $b_1 = -3<0$ y 
\[\varepsilon = \min\biggl\{\frac{-2}{-1},\frac{-3}{-2},\frac{-4}{-1}\biggr\} = \frac{3}{2}\]
Hay que pivotar mediante las operaciones
\[f_1'=-\frac{1}{2}f_1 \qquad \textup{y} \qquad f_2' = f_2+\frac{1}{2}f_1,\]
obtieniéndose una nueva solución del dual: $\omega^1 = (-3/2,0)^t$. Ahora se tiene $b_2 = -5/2 <0$ y
\[\varepsilon = \min \biggl\{\frac{-1/2}{-5/2},\frac{-5/2}{-7/2}\biggr\} = \frac{1}{5},\]
así que ahora se hace
\[f_2'=-\frac{2}{5}f_2 \qquad \textup{y} \qquad f_1' = f_1+\frac{1}{5}f_2\]
La solución del dual obtenida sería $\omega^2=(-8/5,-1/5)^t$. Como en la última parte de la tabla ya se tiene $b_1\geq 0$ y $b_2\geq 0$ y además ninguno de los dos es nulo, puede concluirse que $\omega^2$ es la única solución óptima del problema dual de $(P_2)$, mientras que $\tilde{\omega}^2 = (8/5,1/5)^t$ sería la única solución óptima del dual de $(P_1)$.

\end{example}

\begin{example}
Considérese el problema de programación lineal siguiente:
\[\begin{aligned}[t]
&\text{Min. } && 4x_1+x_2 \\
& \, \text{s. a} &&\left\{\begin{alignedat}{10}
&4&x_1 & {}+{} & 3&x_2 & {}\leq{} & 1&2 \\
&2&x_1 & {}+{} &  &x_2 & {}\geq{} &  &4  \\
&\mathrlap{x_1,x_2 \geq 0}
\end{alignedat}\right.
\end{aligned}\]
Se pasa a forma estándar:
\[\begin{aligned}[t]
(P_1)\ \, &\text{Min. } && 4x_1+x_2 \\
& \, \text{s. a} &&\left\{\begin{alignedat}{10}
&4&x_1 & {}+{} & 3&x_2 & {}+{} & x_3 &       &     & {}={} & 1&2 \\
&2&x_1 & {}+{} &  &x_2 &       &     & {}-{} & x_4 & {}={} &  &4  \\
&\mathrlap{x_1,x_2,x_3,x_4 \geq 0}
\end{alignedat}\right.
\end{aligned}\]
Para tener la base canónica entre las columnas de $A$, se multiplica la segunda ecuación por $-1$:
\[\begin{aligned}[t]
(P_2)\ \, &\text{Min. } && 4x_1+x_2 \\
& \, \text{s. a} &&\left\{\begin{alignedat}{10}
& &4x_1 & {}+{} & 3&x_2 & {}+{} & x_3 &       &     & {}={} &  &12 \\
&-&2x_1 & {}-{} &  &x_2 &       &     & {}+{} & x_4 & {}={} & -&4  \\
&\mathrlap{x_1,x_2,x_3,x_4 \geq 0}
\end{alignedat}\right.
\end{aligned}\]
Se aplica el algoritmo dual del símplex:
\begin{center}
\begin{tabular}{|c|c|c|c|c|c|}
    \cline{3-6}
    \multicolumn{1}{c}{} & \multicolumn{1}{c|}{} & \multicolumn{1}{c}{\phantom{-}4} & \multicolumn{1}{c}{\phantom{-}1} & \multicolumn{1}{c}{\phantom{-}0}  & \multicolumn{1}{c|}{\phantom{-}0} \\ \hline
    
    0 & \phantom{-}12 & \multicolumn{1}{c}{\phantom{-}4} & \multicolumn{1}{c}{\phantom{-}3} & \multicolumn{1}{c}{\phantom{-}1} & \multicolumn{1}{c|}{\phantom{-}0} \\
    
    0 & -4 & \multicolumn{1}{c}{-2} & \multicolumn{1}{c}{-1} & \multicolumn{1}{c}{\phantom{-}0} & \multicolumn{1}{c|}{\phantom{-}1} \\ \hline
    
    \multicolumn{1}{c|}{} & \phantom{-}0 & \multicolumn{1}{c}{\phantom{-}4} & \multicolumn{1}{c}{\phantom{-}1} & \multicolumn{1}{c}{\phantom{-}0} & \multicolumn{1}{c|}{\phantom{-}0} \\ \hhline{-|=|=|=|=|=|}

    0 & \phantom{-}0 & \multicolumn{1}{c}{-2} & \multicolumn{1}{c}{\phantom{-}0} & \multicolumn{1}{c}{\phantom{-}1} & \multicolumn{1}{c|}{\phantom{-}3} \\
    
    1 & \phantom{-}4 & \multicolumn{1}{c}{\phantom{-}2} & \multicolumn{1}{c}{\phantom{-}1} & \multicolumn{1}{c}{\phantom{-}0} & \multicolumn{1}{c|}{-1} \\ \hline
    
    \multicolumn{1}{c|}{} & \phantom{-}4 & \multicolumn{1}{c}{\phantom{-}2} & \multicolumn{1}{c}{\phantom{-}0} & \multicolumn{1}{c}{\phantom{-}0} & \multicolumn{1}{c|}{\phantom{-}1} \\ \hhline{-|=|=|=|=|=|}

    4 & \phantom{-}0 & \multicolumn{1}{c}{\phantom{-}1} & \multicolumn{1}{c}{\phantom{-}0} & \multicolumn{1}{c}{-1/2} & \multicolumn{1}{c|}{-3/2} \\
    
    1 & \phantom{-}4 & \multicolumn{1}{c}{\phantom{-}0} & \multicolumn{1}{c}{\phantom{-}1} & \multicolumn{1}{c}{\phantom{-}1} & \multicolumn{1}{c|}{\phantom{-}2} \\ \hline
    
    \multicolumn{1}{c|}{} & \phantom{-}4 & \multicolumn{1}{c}{\phantom{-}0} & \multicolumn{1}{c}{\phantom{-}0} & \multicolumn{1}{c}{\phantom{-}1} & \multicolumn{1}{c|}{\phantom{-}4} \\
    
    \cline{2-6}
\end{tabular}
\end{center}
En la primera iteración se observa que $b_2 =-4<0$. Se tiene que
\[\varepsilon = \min \biggl\{\frac{-4}{-2},\frac{-1}{-1}\biggr\} = 1,\]
luego las operaciones de filas a realizar son
\[f_2' = -f_2 \qquad \textup{y} \qquad f_1'=f_1+3f_2\]
En la siguiente parte de la tabla, se observa que $b_1 =0$, con lo que el problema dual tiene múltiples soluciones óptimas. Se puede asegurar que $\omega^1 = (0,-1)^t$ es solución óptima del dual, y en la próxima iteración se va a hallar otra solución óptima. Sea
\[\varepsilon = \min \biggl\{\frac{-2}{-2}\biggr\} = 1\]
Hay que pivotar entonces sobre $a_{1,1}$, o sea, hay que hacer las operaciones
\[f_1'=-\frac{1}{2}f_1 \qquad \textup{y} \qquad f_2'= f_2+f_1\]
Se observa que $\omega^2 = (-1,-4)^t$ es también solución óptima del dual, concluyéndose que el segmento
\[\overline{\omega^1\omega^2} = \{\lambda\omega^1+(1-\lambda)\omega^2 \colon \lambda \in [0,1]\}\]
está lleno de soluciones óptimas del dual de $(P_2)$. Para el dual de $(P_1)$, la conclusión es la misma pero con $\tilde{\omega}^1 = (0,1)^t$ y $\tilde{\omega}^2 =(-1,4)^t$.
\end{example}

\begin{example}
Considérese el problema de programación lineal siguiente:
\[\begin{aligned}[t]
&\text{Min. } && 4x_1+3x_2 \\
& \, \text{s. a} &&\left\{\begin{alignedat}{10}
& &x_1 & {}+{} & x_2 & {}\leq{} & 1 \\
&2&x_1 & {}-{} & x_2 & {}\geq{} & 2  \\
&\mathrlap{x_1,x_2 \geq 0}
\end{alignedat}\right.
\end{aligned}\]
Se pasa a forma estándar:
\[\begin{aligned}[t]
(P_1)\ \, &\text{Min. } && 4x_1+3x_2 \\
& \, \text{s. a} &&\left\{\begin{alignedat}{10}
& &x_1 & {}+{} & x_2 & {}+{} & x_3 &       &     & {}={} & 1 \\
&2&x_1 & {}-{} & x_2 &       &     & {}-{} & x_4 & {}={} & 2 \\
&\mathrlap{x_1,x_2,x_3,x_4 \geq 0}
\end{alignedat}\right.
\end{aligned}\]
Para tener la base canónica entre las columnas de $A$, se multiplica la segunda ecuación por $-1$:
\[\begin{aligned}[t]
(P_2)\ \, &\text{Min. } && 4x_1+3x_2 \\
& \, \text{s. a} &&\left\{\begin{alignedat}{10}
&  & &x_1 & {}+{} & x_2 & {}+{} & x_3 &       &     & {}={} &  &1 \\
& -&2&x_1 & {}+{} & x_2 &       &     & {}+{} & x_4 & {}={} & -&2 \\
&\mathrlap{x_1,x_2,x_3,x_4 \geq 0}
\end{alignedat}\right.
\end{aligned}\]
Se aplica el algoritmo dual del símplex:
\begin{center}
\begin{tabular}{|c|c|c|c|c|c|}
    \cline{3-6}
    \multicolumn{1}{c}{} & \multicolumn{1}{c|}{} & \multicolumn{1}{c}{\phantom{-}4} & \multicolumn{1}{c}{\phantom{-}3} & \multicolumn{1}{c}{\phantom{-}0}  & \multicolumn{1}{c|}{\phantom{-}0} \\ \hline
    
    0 & \phantom{-}1 & \multicolumn{1}{c}{\phantom{-}1} & \multicolumn{1}{c}{\phantom{-}1} & \multicolumn{1}{c}{\phantom{-}1} & \multicolumn{1}{c|}{\phantom{-}0} \\
    
    0 & -2 & \multicolumn{1}{c}{-2} & \multicolumn{1}{c}{\phantom{-}1} & \multicolumn{1}{c}{\phantom{-}0} & \multicolumn{1}{c|}{\phantom{-}1} \\ \hline
    
    \multicolumn{1}{c|}{} & \phantom{-}0 & \multicolumn{1}{c}{\phantom{-}4} & \multicolumn{1}{c}{\phantom{-}3} & \multicolumn{1}{c}{\phantom{-}0} & \multicolumn{1}{c|}{\phantom{-}0} \\ \hhline{-|=|=|=|=|=|}

    0 & \phantom{-}0 & \multicolumn{1}{c}{\phantom{-}0} & \multicolumn{1}{c}{\phantom{-}3/2} & \multicolumn{1}{c}{\phantom{-}1} & \multicolumn{1}{c|}{\phantom{-}1/2} \\
    
    4 & \phantom{-}1 & \multicolumn{1}{c}{\phantom{-}1} & \multicolumn{1}{c}{-1/2} & \multicolumn{1}{c}{\phantom{-}0} & \multicolumn{1}{c|}{-1/2} \\ \hline
    
    \multicolumn{1}{c|}{} & \phantom{-}4 & \multicolumn{1}{c}{\phantom{-}0} & \multicolumn{1}{c}{\phantom{-}5} & \multicolumn{1}{c}{\phantom{-}0} & \multicolumn{1}{c|}{\phantom{-}2} \\
    
    \cline{2-6}
\end{tabular}
\end{center}
El pivotaje en la primera iteración se realiza sobre $a_{2,1}$, realizándose las operaciones de filas
\[f_2'=-\frac{1}{2}f_2 \qquad \textup{y} \qquad f_1'=f_1+\frac{1}{2}f_2\]
Se observa en la segunda parte de la tabla que $\omega^1 = (0,-2)^t$ es solución óptima del problema dual. Como además $b_1 = 0$ y $a_{1,j} \geq 0$ para todo $j \in \{1,2,3,4\}$, se puede concluir que para todo $\varepsilon >0$, 
\[\omega^0_\varepsilon = (0-\varepsilon \cdot 1,-2-\varepsilon \cdot \frac{1}{2})^t = (-\varepsilon,-2-\frac{\varepsilon}{2})^t\]
es solución óptima del dual.
\end{example}

\chapter{Análisis de la sensibilidad}

En este tema se estudiará cómo varían las soluciones de un problema de programación lineal cuando se alteran los datos del problema o se introducen datos nuevos. El ejemplo que sigue será utilizado en todas las secciones del capítulo.

\begin{example}
Considérese el problema de programación lineal siguiente:
\[\begin{aligned}[t]
&\text{Min. } && x_2-3x_3+2x_5 \\
& \, \text{s. a} &&\left\{\begin{alignedat}{10}
& &x_1 & {}+{} & 3x_2 & {}-{} &  &x_3 &       &     & {}+{} & 2x_5 &       &     & {}={} & 7 \\
& &    & {}-{} & 2x_2 & {}+{} & 4&x_3 & {}+{} & x_4 &       &      &       &     & {}={} & 12 \\
& &    & {}-{} & 4x_2 & {}+{} & 3&x_3 &       &     & {}+{} & 8x_5 & {}+{} & x_6 & {}={} & 10 \\
&\mathrlap{x_1,x_2,x_3,x_4,x_5,x_6 \geq 0}
\end{alignedat}\right.
\end{aligned}\]
La resolución del problema mediante el método del símplex es
\begin{center}
\begin{tabular}{|c|c|c|c|c|c|c|}
    \cline{2-7}
    
    \multicolumn{1}{c|}{} & \multicolumn{1}{c}{0} & \multicolumn{1}{c}{\phantom{-}1} & \multicolumn{1}{c}{-3} & \multicolumn{1}{c}{\phantom{-}0} & \multicolumn{1}{c}{2} & \multicolumn{1}{c|}{0} \\ \hline
    
    \phantom{-}7 & \multicolumn{1}{c}{1} & \multicolumn{1}{c}{\phantom{-}3} & \multicolumn{1}{c}{-1} & \multicolumn{1}{c}{\phantom{-}0} & \multicolumn{1}{c}{2} & \multicolumn{1}{c|}{0} \\

    \phantom{-}12 & \multicolumn{1}{c}{0} & \multicolumn{1}{c}{-2} & \multicolumn{1}{c}{\phantom{-}4} & \multicolumn{1}{c}{\phantom{-}1} & \multicolumn{1}{c}{0} & \multicolumn{1}{c|}{0} \\
    
    \phantom{-}10 & \multicolumn{1}{c}{0} & \multicolumn{1}{c}{-4} & \multicolumn{1}{c}{\phantom{-}3} & \multicolumn{1}{c}{\phantom{-}0} & \multicolumn{1}{c}{8} & \multicolumn{1}{c|}{1} \\ \hhline{|=|=|=|=|=|=|=|}

    \multicolumn{1}{|c|}{\phantom{-}(...)} & \multicolumn{6}{c|}{(...)} \\ \hhline{|=|=|=|=|=|=|=|}
    
    \phantom{-}4 & \multicolumn{1}{c}{2/5} & \multicolumn{1}{c}{\phantom{-}1} & \multicolumn{1}{c}{\phantom{-}0} & \multicolumn{1}{c}{\phantom{-}1/10}& \multicolumn{1}{c}{4/5} & \multicolumn{1}{c|}{0}  \\

    \phantom{-}5 & \multicolumn{1}{c}{1/5} & \multicolumn{1}{c}{\phantom{-}0} & \multicolumn{1}{c}{\phantom{-}1} & \multicolumn{1}{c}{\phantom{-}3/10}& \multicolumn{1}{c}{2/5} & \multicolumn{1}{c|}{0}  \\
    
    \phantom{-}11 & \multicolumn{1}{c}{1} & \multicolumn{1}{c}{\phantom{-}0} & \multicolumn{1}{c}{\phantom{-}0} & \multicolumn{1}{c}{-1/2}& \multicolumn{1}{c}{10} & \multicolumn{1}{c|}{1} \\ \cline{1-7}
    
    -11 & \multicolumn{1}{c}{1/5} & \multicolumn{1}{c}{\phantom{-}0} & \multicolumn{1}{c}{\phantom{-}0} & \multicolumn{1}{c}{\phantom{-}4/5} & \multicolumn{1}{c}{12/5} & \multicolumn{1}{c|}{0} \\ \cline{1-7}
\end{tabular}
\end{center}
Las etapas intermedias se han omitido por haberse realizado en múltiples ocasiones con problemas similares.
\end{example}

\section{Variación en el vector \textit{c}}

Supóngase que se está en la última tabla del método del símplex, disponiéndose de una solución óptima $x^0$, de componentes básicas $x^0_B = B^{-1}b$, y súmese una cantidad $\Delta c_k$ a la componente $c_k$ del vector de costes, siendo el nuevo vector de costes 
\[\overline{c} = (c_1,\mathellipsis,c_{k-1},c_k+\Delta c_k,c_{k+1},\mathellipsis,c_n)^t\]

Hay que investigar cuáles son los valores de $\Delta c_k$ que hacen que $x^0$ siga siendo solución óptima del problema, o sea, que hacen que los costes reducidos sigan siendo positivos. Se distinguen dos casos:

\begin{enumerate}
    \item Si $c_k$ no es un coste básico, entonces solo varía la componente $k$-ésima de los costes reducidos, que ahora es
    \[\overline{c_k-z_k} = c_k+\Delta c_k - c_B^tB^{-1}a_k = c_k-z_k+\Delta c_k\]
    En consecuencia, $x^0$ será una solución óptima del nuevo problema si y solo si $\overline{c_k-z_k} \geq 0$ , es decir, si y solo si $ c_k-z_k\geq -\Delta c_k$.
    \item Si $c_k$ es un coste básico, entonces varían todos los costes reducidos no básicos, que ahora son
    \[\overline{c_j-z_j} = c_j-\overline{c_B}^tB^{-1}a_j,\]
    así que $\Delta c_k$ tendrá que hacer que se tenga $\overline{c_j-z_j} \geq 0$. En lugar de dar los posibles valores de $\Delta c_k$ en el caso general, se va a razonar sobre el ejemplo del principio del tema.
\end{enumerate}

\begin{example}
Se tratará de estudiar qué variaciones pueden tener lugar en el vector de costes del ejemplo inicial para que $x^0 = (0,4,5,0,0,11)^t$ continúe siendo una solución óptima del problema.
\begin{enumerate}
    \item \textit{Costes no básicos}. Las variaciones que podrían hacerse son
    \[-\Delta c_1 \leq c_1-z_1, \qquad -\Delta c_4 \leq c_4-z_4 \qquad \textup{o} \qquad -\Delta c_5 \leq c_5-z_5,\]
    es decir,
    \[\Delta c_1 \geq -\frac{1}{5}, \qquad \Delta c_4 \geq -\frac{4}{5} \qquad \textup{o} \qquad \Delta c_5 \geq -\frac{12}{5}\]
    \item \textit{Costes básicos}. Supóngase que se realiza una variación $\Delta c_2$ en el coste básico $c_2$. Los nuevos costes reducidos no básicos son
    \[\begin{aligned}[t]
        \overline{c_1-z_1} &= c_1-\overline{c_B}^tB^{-1}a_1 = c_1-
            (c_2+\Delta c_2, c_3, c_6)^tB^{-1}a_1 = \frac{1}{5}-\frac{2}{5}\Delta c_2\\[5pt]
        \overline{c_4-z_4} &= c_4-\overline{c_B}^tB^{-1}a_4 = c_4-
            (c_2+\Delta c_2, c_3, c_6)^tB^{-1}a_4 = \frac{4}{5}-\frac{1}{10} \Delta c_2\\[5pt]
        \overline{c_5-z_5} &= c_5-\overline{c_B}^tB^{-1}a_5 = c_5-
            (c_2+\Delta c_2, c_3, c_6)^tB^{-1}a_5 = \frac{12}{5}-\frac{4}{5} \Delta c_2\\
    \end{aligned}\]
Obsérvese que no hay que realizar cálculo intermedio ninguno, pues todo lo que se necesita está en la tabla de la página anterior. Se han de resolver las inecuaciones
\[\left\{\begin{alignedat}{10}
& \frac{1}{5}  & {}-{} & &\frac{2}{5}\Delta c_2  & {}\geq{} & 0 \\[5pt]
& \frac{4}{5}  & {}-{} & &\frac{1}{10}\Delta c_2  & {}\geq{} & 0 \\[5pt]
& \frac{12}{5} & {}-{} & &\frac{4}{5}\Delta c_2 & {}\geq{} & 0 \\
\end{alignedat}\right.\]
La solución es
\[\Delta c_2 \leq \frac{1}{2}\]
Obsérvese que el sistema de inecuaciones obtenido es
\[\left\{\begin{alignedat}{10}
& (c_1-z_1) & {}-{} & &\overline{a}_{1,1}\Delta c_2  & {}\geq{} 0 \\[5pt]
& (c_4-z_4) & {}-{} & &\overline{a}_{1,4}\Delta c_2  & {}\geq{} 0 \\[5pt]
& (c_5-z_5) & {}-{} & &\overline{a}_{1,5}\Delta c_2  & {}\geq{} 0, \\
\end{alignedat}\right.\]
donde $B^{-1}A = (\overline{a}_{i,j})_{i,j=1}^n$.

Análogamente, si se realiza una variación $\Delta c_3$ en el coste básico $c_3$, habría que resolver
\[\left\{\begin{alignedat}{10}
& \frac{1}{5}  & {}-{} & &\frac{1}{5}\Delta c_3  & {}\geq{} & 0 \\[5pt]
& \frac{4}{5}  & {}-{} & &\frac{3}{10}\Delta c_3  & {}\geq{} & 0 \\[5pt]
& \frac{12}{5} & {}-{} & &\frac{2}{5}\Delta c_3 & {}\geq{} & 0 \\
\end{alignedat}\right.\]
Este sistema coincide con
\[\left\{\begin{alignedat}{10}
& (c_1-z_1) & {}-{} & &\overline{a}_{2,1}\Delta c_3  & {}\geq{} 0 \\[5pt]
& (c_4-z_4) & {}-{} & &\overline{a}_{2,4}\Delta c_3  & {}\geq{} 0 \\[5pt]
& (c_5-z_5) & {}-{} & &\overline{a}_{2,5}\Delta c_3  & {}\geq{} 0 \\
\end{alignedat}\right.\]
\end{enumerate}
\end{example}

\section{Variación en el vector \textit{b}}

Supóngase que se está en la última tabla del método del símplex, disponiéndose de una solución óptima $x^0$, de componentes básicas $x^0_B = B^{-1}b$, y súmese una cantidad $\Delta b_l$ a la componente $b_l$ del vector de recursos, siendo el nuevo vector de recursos 
\[\overline{b} = 
(b_1,\mathellipsis,b_{l-1},b_l+\Delta b_l,b_{l+1},\mathellipsis,b_m)^t\]

Hay que investigar qué valores de $\Delta b_l$ hacen que el vector de componentes básicas $\overline{x}^0_B = B^{-1}\overline{b}$ sea una solución óptima del nuevo problema. Como los costes reducidos no cambian, para que $\overline{x}^0$ sea solución óptima solo se necesita que sea solución factible. Por tanto, tendrá que cumplirse que
\[\overline{x}^0_B = B^{-1}\overline{b} = B^{-1}\begin{pmatrix}
    b_1 \\
    \vdots \\
    b_l+\Delta b_l \\
    \vdots \\
    b_m
\end{pmatrix} = B^{-1}\begin{pmatrix}
    b_1 \\
    \vdots \\
    b_l \\
    \vdots \\
    b_m
\end{pmatrix}+B^{-1}\begin{pmatrix}
    0 \\
    \vdots \\
    \Delta b_l \\
    \vdots \\
    0
\end{pmatrix} = x^0_B + B^{-1}\begin{pmatrix}
    0 \\
    \vdots \\
    \Delta b_l \\
    \vdots \\
    0
\end{pmatrix} \geq 0\]

\begin{example}
Se tratará de estudiar qué variaciones pueden tener lugar en el vector de recursos del ejemplo inicial para que el vector $\overline{x}^0$ de componentes básicas $\overline{x}^0_B = B^{-1}\overline{b}$ sea una solución óptima del nuevo problema. Se va a variar, por ejemplo, la componente $b_1$. Se tiene que
\[x^0_B+B^{-1}\begin{pmatrix}
    \Delta b_1 \\
    0 \\
    0
\end{pmatrix} \geq 0 \iff \begin{pmatrix}
    4 \\
    5 \\
    11
\end{pmatrix}+\begin{pmatrix}
    2/5 & \phantom{-}1/10 & 0 \\
    1/5 & \phantom{-}3/10 & 0 \\
    1 & -1/2 & 1
\end{pmatrix}\begin{pmatrix}
    \Delta b_1 \\
    0 \\
    0
\end{pmatrix} \geq 0 \iff \left\{\begin{alignedat}{10}
& 4  & {}-{} & &\frac{2}{5}\Delta b_1  & {}\geq{} & 0 \\[4pt]
& 5  & {}-{} & &\frac{1}{5}\Delta b_1  & {}\geq{} & 0 \\[4pt]
& 11 & {}-{} & &\phantom{\frac{1}{1}}\Delta b_1 & {}\geq{} & 0 \\
\end{alignedat}\right.\]
La solución es $\Delta b_1 \geq -10$. Para cualquier valor $\Delta b_1$ verificando tal desigualdad, se tiene que 
\[\overline{x}^0 = (0,4+\frac{2}{5}\Delta b_1,5+\frac{1}{5}\Delta b_1,0,0,11+\Delta b_1)^t\]
es una solución óptima del nuevo problema.
\end{example}

Obsérvese que variar $c$ podría causar que el problema primal tenga múltiples óptimos, mientras que variar $b$ podría hacer que el problema dual posea múltiples óptimos.

\section{Variación en la matriz \textit{A}}

Supóngase que se está en la última tabla del método del símplex, disponiéndose de una solución óptima $x^0$, de componentes básicas $x^0_B = B^{-1}b$, y súmese una cantidad $\Delta a_{l,k}$ a la entrada $a_{l,k}$ de la matriz inicial $A$. Se presenta la siguiente disyuntiva:

\begin{enumerate}
    \item Si $a_k$ no es un vector básico, entonces $x^0$ continúa siendo solución factible del problema, así que hay que preguntarse cuánto puede valer $\Delta a_{l,k}$ para que $x^0$ siga siendo solución óptima, o sea, para que los costes reducidos sigan siendo positivos. La variación que se ha introducido solo afecta al $k$-ésimo coste reducido, así que debe cumplirse
    \[
    \overline{c_k-z_k} = c_k-c_B^tB^{-1}\begin{pmatrix}
        a_{1,k} \\
        \vdots \\
        a_{l,k} \\
        \vdots \\
        a_{m,k}
    \end{pmatrix} - c_B^tB^{-1}\begin{pmatrix}
        0 \\
        \vdots \\
        \Delta a_{l,k} \\
        \vdots \\
        0
    \end{pmatrix}
    = c_k-z_k- c_B^tB^{-1}\begin{pmatrix}
        0 \\
        \vdots \\
        \Delta a_{l,k} \\
        \vdots \\
        0
    \end{pmatrix}  \geq 0
    \]
    Obsérvese que $c_B^tB^{-1}$ es una solución óptima del problema dual, que puede hallarse mediante la contemplación de la tabla.
    \item Si $a_k$ es un vector básico, $x^0$ podría dejar de ser solución factible, pues ahora varía la matriz $B$. Sea
    \[\overline{B} = B + \begin{pmatrix}
        0 & \mathellipsis & 0 & \mathellipsis & 0 \\
        \vdots & \ddots & \vdots & \ddots & \vdots \\
        0 & \mathellipsis & \Delta a_{l,k} & \mathellipsis & 0 \\
        \vdots & \ddots & \vdots & \ddots & \vdots \\
        0& \mathellipsis & 0 & \mathellipsis & 0 \\
    \end{pmatrix}\] El objetivo ahora es que $\overline{x}_B^0 = \overline{B}^{-1}b \geq 0$ y que  $\overline{c_j-z_j} = c_j-c_B^t\overline{B}^{-1}a_j \geq 0$ para todo índice no básico $j$. Obsérvese además que $\Delta a_{l,k}$ debe tomarse de forma que $\overline{B}$ siga siendo inversible. En lugar de realizarlos en el caso general, los cálculos se expondrán en el ejemplo de siempre. 
\end{enumerate}

\begin{example}
Supóngase que se le suma una cantidad $\Delta a_{1,2}$ al elemento $a_{1,2}$ de la matriz del ejemplo inicial. Al ser $a_2$ un vector básico, habrá que calcular la nueva matriz $\overline{B}$. Se tiene que
\[\overline{B} = \begin{pmatrix}
    3+\Delta a_{1,2} & -1 & 0 \\
    -2 & 4 & 0 \\
    -4 & 3 & 1
\end{pmatrix} = B + \begin{pmatrix}
    \Delta a_{1,2} & 0 & 0 \\
    0 & 0 & 0 \\
    0 & 0 & 0 \\
    \end{pmatrix} = B\begin{bmatrix}
        I+B^{-1}\Delta a_{1,2}\begin{pmatrix}
            1 & 0 & 0 \\
            0 & 0 & 0 \\
            0 & 0 & 0
        \end{pmatrix}
    \end{bmatrix}\]
    Por tanto,
    \[\overline{B}^{-1} =\begin{bmatrix}
        I+B^{-1}\Delta a_{1,2}\begin{pmatrix}
            1 & 0 & 0 \\
            0 & 0 & 0 \\
            0 & 0 & 0
        \end{pmatrix}
    \end{bmatrix}^{-1}B^{-1} \]
    Nótese que el cálculo de la primera matriz de la derecha es más simple que el de $\overline{B}^{-1}$. Se obtiene
\[\overline{B}^{-1} = \begin{pmatrix}
    \displaystyle\frac{2}{1+\frac{2}{5}\Delta a_{1,2}} & \displaystyle\frac{1}{10+4 \Delta a_{1,2}} & 0 \\[15pt]
    \displaystyle\frac{1}{1+\frac{2}{5}\Delta a_{1,2}} & \displaystyle\frac{3+\Delta a_{1,2}}{10+4 \Delta a_{1,2}} & 0 \\[15pt]
    \displaystyle\frac{5}{1+\frac{2}{5}\Delta a_{1,2}} & \displaystyle\frac{1}{10+4 \Delta a_{1,2}} & 1 \\
\end{pmatrix}\]
Para obtener una nueva solución factible, debe verificarse
\[\overline{x}^0_B = \overline{B}^{-1}b = \begin{pmatrix}
    \displaystyle\frac{20}{5+2\Delta a_{1,2}} \\[15pt]
    \displaystyle\frac{25+6\Delta a_{1,2}}{5+2\Delta a_{1,2}} \\[15pt]
    \displaystyle\frac{55+2\Delta a_{1,2}}{5+2\Delta a_{1,2}}
\end{pmatrix} \geq 0,\]
y para que esta solución factible sea óptima, se necesita que
\[\begin{aligned}[t]
\overline{c_1-z_1} &= c_1-c_B^t\overline{B}^{-1}a_1 = \frac{1}{5+2\Delta a_{1,2}} \geq 0 \\[5pt]
\overline{c_4-z_4} &= c_4-c_B^t\overline{B}^{-1}a_4 = \frac{8+3\Delta a_{1,2}}{10+4\Delta a_{1,2}} \geq 0 \\[5pt]
\overline{c_5-z_5} &= c_5-c_B^t\overline{B}^{-1}a_5 = \frac{12+4\Delta a_{1,2}}{5+2\Delta a_{1,2}} \geq 0 \\
\end{aligned}
\]
El sistema de inecuaciones a resolver es
\[\left\{\begin{alignedat}{10}
& 5  & {}+{} & &2\Delta a_{1,2} & {}>{} & 0& \\
& 25 & {}+{} & &6\Delta a_{1,2} & {}\geq{} & 0& \\
& 55 & {}+{} & &2\Delta a_{1,2} & {}\geq{} & 0& \\
& 8  & {}+{} & &3\Delta a_{1,2} & {}\geq{} & 0& \\
& 12 & {}+{} & &4\Delta a_{1,2} & {}\geq{} & 0& \\
\end{alignedat}\right.\]
La solución es \[\Delta a_{1,2} > -\frac{5}{2}\]
\end{example}

\section{Adición de una nueva variable}

Supóngase que se está en la última tabla del método del símplex y que se dispone de una solución óptima $x^0$, de componentes básicas $x^0_B = B^{-1}b$, e introdúzcase una nueva variable $x_{n+1}$ (o sea, una nueva columna $a_{n+1}$ en la matriz inicial) con coste $c_{n+1}$. Se distinguen dos casos:
\begin{enumerate}
    \item Si $c_{n+1}-z_{n+1} = c_{n+1}-c_B^tB^{-1}a_{n+1} \geq 0$, entonces $(x^0,0)^t$ es solución óptima del problema.
    \item Si no, hay que mejorar la solución $(x^0,0)^t$ mediante nuevas iteraciones del método del símplex.
\end{enumerate}

\begin{example}
Supóngase que se introduce en el problema una nueva columna $a_7 = (1,2,0)^t$ y un nuevo coste $c_7 = 3$. Se tiene que \[c_7-z_7 = c_7-c_B^tB^{-1}a_7 = 3- \begin{pmatrix}
    \displaystyle\frac{1}{5} & \displaystyle\frac{4}{5} & 0
\end{pmatrix}\begin{pmatrix}
    1 \\
    2 \\
    0
\end{pmatrix} = \frac{24}{5} \geq 0,\]
luego $(x^0,0)^t = (0,4,5,0,0,11,0)^t$ es solución óptima (la única) del nuevo problema.
\end{example}

\begin{example}
Supóngase que se introduce en el problema una nueva columna $a_7 = (1,1,0)^t$ y un nuevo coste $c_7 = -3$. El nuevo coste reducido es $c_7-z_7 = -2 \leq 0$, así que hay que trabajar un poco más que en el ejemplo de antes:

\begin{center}
\begin{tabular}{|c|c|c|c|c|c|c|c|}
    \cline{2-8}
    
    \multicolumn{1}{c|}{} & \multicolumn{1}{c}{\phantom{-}0} & \multicolumn{1}{c}{\phantom{-}1} & \multicolumn{1}{c}{-3} & \multicolumn{1}{c}{\phantom{-}0} & \multicolumn{1}{c}{\phantom{-}2} & \multicolumn{1}{c}{0} & \multicolumn{1}{c|}{-3} \\ \hline
    
    \phantom{-}7 & \multicolumn{1}{c}{\phantom{-}1} & \multicolumn{1}{c}{\phantom{-}3} & \multicolumn{1}{c}{-1} & \multicolumn{1}{c}{\phantom{-}0} & \multicolumn{1}{c}{\phantom{-}2} & \multicolumn{1}{c}{0} & \multicolumn{1}{c|}{\phantom{-}1} \\

    \phantom{-}12 & \multicolumn{1}{c}{\phantom{-}0} & \multicolumn{1}{c}{-2} & \multicolumn{1}{c}{\phantom{-}4} & \multicolumn{1}{c}{\phantom{-}1} & \multicolumn{1}{c}{\phantom{-}0} & \multicolumn{1}{c}{0} & \multicolumn{1}{c|}{\phantom{-}1} \\
    
    \phantom{-}10 & \multicolumn{1}{c}{\phantom{-}0} & \multicolumn{1}{c}{-4} & \multicolumn{1}{c}{\phantom{-}3} & \multicolumn{1}{c}{\phantom{-}0} & \multicolumn{1}{c}{\phantom{-}8} & \multicolumn{1}{c}{1} & \multicolumn{1}{c|}{\phantom{-}0} \\ \hhline{|=|=|=|=|=|=|=|=|}

    \multicolumn{1}{|c|}{\phantom{-}(...)} & \multicolumn{7}{c|}{(...)} \\ \hhline{|=|=|=|=|=|=|=|=|}
    
    \phantom{-}4 & \multicolumn{1}{c}{\phantom{-}2/5} & \multicolumn{1}{c}{\phantom{-}1} & \multicolumn{1}{c}{\phantom{-}0} & \multicolumn{1}{c}{\phantom{-}1/10}& \multicolumn{1}{c}{\phantom{-}4/5} & \multicolumn{1}{c}{0} & \multicolumn{1}{c|}{\phantom{-}1/2} \\

    \phantom{-}5 & \multicolumn{1}{c}{\phantom{-}1/5} & \multicolumn{1}{c}{\phantom{-}0} & \multicolumn{1}{c}{\phantom{-}1} & \multicolumn{1}{c}{\phantom{-}3/10}& \multicolumn{1}{c}{\phantom{-}2/5} & \multicolumn{1}{c}{0} & \multicolumn{1}{c|}{\phantom{-}1/2} \\
    
    \phantom{-}11 & \multicolumn{1}{c}{\phantom{-}1} & \multicolumn{1}{c}{\phantom{-}0} & \multicolumn{1}{c}{\phantom{-}0} & \multicolumn{1}{c}{-1/2}& \multicolumn{1}{c}{\phantom{-}10} & \multicolumn{1}{c}{1} & \multicolumn{1}{c|}{\phantom{-}1/2} \\ \cline{1-8}
    
    -11 & \multicolumn{1}{c}{\phantom{-}1/5} & \multicolumn{1}{c}{\phantom{-}0} & \multicolumn{1}{c}{\phantom{-}0} & \multicolumn{1}{c}{\phantom{-}4/5} & \multicolumn{1}{c}{\phantom{-}12/5} & \multicolumn{1}{c}{0} & \multicolumn{1}{c|}{-2} \\ \hhline{|=|=|=|=|=|=|=|=|}
    
    \phantom{-}8 & \multicolumn{1}{c}{\phantom{-}4/5} & \multicolumn{1}{c}{\phantom{-}2} & \multicolumn{1}{c}{\phantom{-}0} & \multicolumn{1}{c}{\phantom{-}1/5}& \multicolumn{1}{c}{\phantom{-}8/5} & \multicolumn{1}{c}{0} & \multicolumn{1}{c|}{\phantom{-}1} \\

    \phantom{-}1 & \multicolumn{1}{c}{-1/5} & \multicolumn{1}{c}{-1} & \multicolumn{1}{c}{\phantom{-}1} & \multicolumn{1}{c}{\phantom{-}1/5}& \multicolumn{1}{c}{-2/5} & \multicolumn{1}{c}{0} & \multicolumn{1}{c|}{\phantom{-}0} \\
    
    \phantom{-}7 & \multicolumn{1}{c}{\phantom{-}3/5} & \multicolumn{1}{c}{-1} & \multicolumn{1}{c}{\phantom{-}0} & \multicolumn{1}{c}{-3/5}& \multicolumn{1}{c}{\phantom{-}46/5} & \multicolumn{1}{c}{1} & \multicolumn{1}{c|}{\phantom{-}0} \\ \cline{1-8}
    
    -27 & \multicolumn{1}{c}{\phantom{-}9/5} & \multicolumn{1}{c}{\phantom{-}4} & \multicolumn{1}{c}{\phantom{-}0} & \multicolumn{1}{c}{\phantom{-}6/5} & \multicolumn{1}{c}{\phantom{-}28/5} & \multicolumn{1}{c}{0} & \multicolumn{1}{c|}{0} \\ \cline{1-8}
\end{tabular}
\end{center}
Se concluye que $\overline{x}^0 = (0,0,1,0,0,7,8)^t$ es la única solución del nuevo problema. Por si cupiese duda sobre la última columna de la segunda tabla, nótese que $(1/2,1/2,1/2)^t = B^{-1}(1,1,0)^t$.

\end{example}

\section{Adición de una nueva restricción}

Supóngase que se está en la última tabla del método del símplex, disponiéndose de una solución óptima $x^0$, de componentes básicas $x^0_B = B^{-1}b$, e introdúzcase una nueva restricción del tipo $\geq$, por ejemplo:
\[a_{m+1,1}x_1+a_{m+1,2}x_2+\mathellipsis+a_{m+1,n}x_n \geq b_{m+1}\]
Se distinguen dos casos:
\begin{enumerate}
    \item Si $x^0$ verifica la nueva restricción, entonces es solución óptima del nuevo problema.
    \item Si no, se continúa con el método del símplex introduciendo la nueva restricción hasta obtener una solución óptima.
\end{enumerate}

\begin{example}
Supóngase que al problema
\[\begin{aligned}[t]
&\text{Min. } && x_2-3x_3+2x_5 \\
& \, \text{s. a} &&\left\{\begin{alignedat}{10}
& &x_1 & {}+{} & 3x_2 & {}-{} &  &x_3 &       &     & {}+{} & 2x_5 &       &     & {}={} & 7 \\
& &    & {}-{} & 2x_2 & {}+{} & 4&x_3 & {}+{} & x_4 &       &      &       &     & {}={} & 12 \\
& &    & {}-{} & 4x_2 & {}+{} & 3&x_3 &       &     & {}+{} & 8x_5 & {}+{} & x_6 & {}={} & 10 \\
&\mathrlap{x_1,x_2,x_3,x_4,x_5,x_6 \geq 0}
\end{alignedat}\right.
\end{aligned}\]
se le añade la restricción $x_2+x_3-x_6 \leq 1$, quedando
\[\begin{aligned}[t]
&\text{Min. } && x_2-3x_3+2x_5 \\
& \, \text{s. a} &&\left\{\begin{alignedat}{10}
& &x_1 & {}+{} & 3&x_2 & {}-{} &  &x_3 &       &     & {}+{} & 2x_5 &       &     & {}={} & 7 \\
& &    & {}-{} & 2&x_2 & {}+{} & 4&x_3 & {}+{} & x_4 &       &      &       &     & {}={} & 12 \\
& &    & {}-{} & 4&x_2 & {}+{} & 3&x_3 &       &     & {}+{} & 8x_5 & {}+{} & x_6 & {}={} & 10 \\
& &    &       &  &x_2 & {}+{} &  &x_3 &       &     &       &      & {}-{} & x_6 & {}\leq{} & 1 \\
&\mathrlap{x_1,x_2,x_3,x_4,x_5,x_6 \geq 0}
\end{alignedat}\right.
\end{aligned}\]
Como la solución óptima $x^0 = (0,4,5,0,0,11)^t$ del problema original verifica $4+5-11=-2 \leq 1$, entonces $x^0$ es también la única solución óptima del nuevo problema.
\end{example}

\begin{example}
Supóngase que al problema
\[\begin{aligned}[t]
&\text{Min. } && x_2-3x_3+2x_5 \\
& \, \text{s. a} &&\left\{\begin{alignedat}{10}
& &x_1 & {}+{} & 3x_2 & {}-{} &  &x_3 &       &     & {}+{} & 2x_5 &       &     & {}={} & 7 \\
& &    & {}-{} & 2x_2 & {}+{} & 4&x_3 & {}+{} & x_4 &       &      &       &     & {}={} & 12 \\
& &    & {}-{} & 4x_2 & {}+{} & 3&x_3 &       &     & {}+{} & 8x_5 & {}+{} & x_6 & {}={} & 10 \\
&\mathrlap{x_1,x_2,x_3,x_4,x_5,x_6 \geq 0}
\end{alignedat}\right.
\end{aligned}\]
se le añade la restricción $-x_2+x_3 \leq 0$, quedando
\[\begin{aligned}[t]
&\text{Min. } && x_2-3x_3+2x_5 \\
& \, \text{s. a} &&\left\{\begin{alignedat}{10}
& &x_1 & {}+{} & 3&x_2 & {}-{} &  &x_3 &       &     & {}+{} & 2x_5 &       &     & {}={} & 7 \\
& &    & {}-{} & 2&x_2 & {}+{} & 4&x_3 & {}+{} & x_4 &       &      &       &     & {}={} & 12 \\
& &    & {}-{} & 4&x_2 & {}+{} & 3&x_3 &       &     & {}+{} & 8x_5 & {}+{} & x_6 & {}={} & 10 \\
& &    &       & -&x_2 & {}+{} &  &x_3 &       &     &       &      &       &     & {}\leq{} & 0 \\
&\mathrlap{x_1,x_2,x_3,x_4,x_5,x_6 \geq 0}
\end{alignedat}\right.
\end{aligned}\]
Como la solución óptima $x^0 = (0,4,5,0,0,11)^t$ del problema original verifica $-4+5=1 \not\leq 0$, entonces hay que trabajar un poco más que en el ejemplo anterior. Se introduce la nueva restricción (pasando antes a forma estándar, por supuesto) en la última tabla del método del símplex y se continúa aplicando el algoritmo. 

\begin{center}
\begin{tabular}{|c|c|c|c|c|c|c|c|}
    \cline{2-8}
    
    \multicolumn{1}{c|}{} & \multicolumn{1}{c}{\phantom{-}0} & \multicolumn{1}{c}{\phantom{-}1} & \multicolumn{1}{c}{-3} & \multicolumn{1}{c}{\phantom{-}0} & \multicolumn{1}{c}{\phantom{-}2} & \multicolumn{1}{c}{0} & \multicolumn{1}{c|}{\phantom{-}0} \\ \hline
    
    \phantom{-}4 & \multicolumn{1}{c}{\phantom{-}2/5} & \multicolumn{1}{c}{\phantom{-}1} & \multicolumn{1}{c}{\phantom{-}0} & \multicolumn{1}{c}{\phantom{-}1/10}& \multicolumn{1}{c}{\phantom{-}4/5} & \multicolumn{1}{c}{0} & \multicolumn{1}{c|}{\phantom{-}0} \\

    \phantom{-}5 & \multicolumn{1}{c}{\phantom{-}1/5} & \multicolumn{1}{c}{\phantom{-}0} & \multicolumn{1}{c}{\phantom{-}1} & \multicolumn{1}{c}{\phantom{-}3/10}& \multicolumn{1}{c}{\phantom{-}2/5} & \multicolumn{1}{c}{0} & \multicolumn{1}{c|}{\phantom{-}0} \\
    
    \phantom{-}11 & \multicolumn{1}{c}{\phantom{-}1} & \multicolumn{1}{c}{\phantom{-}0} & \multicolumn{1}{c}{\phantom{-}0} & \multicolumn{1}{c}{-1/2}& \multicolumn{1}{c}{\phantom{-}10} & \multicolumn{1}{c}{1} & \multicolumn{1}{c|}{\phantom{-}0} \\
    
    \phantom{-}0 & \multicolumn{1}{c}{\phantom{-}0} & \multicolumn{1}{c}{-1} & \multicolumn{1}{c}{\phantom{-}1} & \multicolumn{1}{c}{\phantom{-}0}& \multicolumn{1}{c}{\phantom{-}0} & \multicolumn{1}{c}{0} & \multicolumn{1}{c|}{\phantom{-}1} \\ \cline{1-8}
    
    -11 & \multicolumn{1}{c}{\phantom{-}1/5} & \multicolumn{1}{c}{\phantom{-}0} & \multicolumn{1}{c}{\phantom{-}0} & \multicolumn{1}{c}{\phantom{-}4/5} & \multicolumn{1}{c}{\phantom{-}12/5} & \multicolumn{1}{c}{0} & \multicolumn{1}{c|}{\phantom{-}0} \\ \hhline{|=|=|=|=|=|=|=|=|}

    \phantom{-}4 & \multicolumn{1}{c}{\phantom{-}2/5} & \multicolumn{1}{c}{\phantom{-}1} & \multicolumn{1}{c}{\phantom{-}0} & \multicolumn{1}{c}{\phantom{-}1/10}& \multicolumn{1}{c}{\phantom{-}4/5} & \multicolumn{1}{c}{0} & \multicolumn{1}{c|}{\phantom{-}0} \\

    \phantom{-}5 & \multicolumn{1}{c}{\phantom{-}1/5} & \multicolumn{1}{c}{\phantom{-}0} & \multicolumn{1}{c}{\phantom{-}1} & \multicolumn{1}{c}{\phantom{-}3/10}& \multicolumn{1}{c}{\phantom{-}2/5} & \multicolumn{1}{c}{0} & \multicolumn{1}{c|}{\phantom{-}0} \\
    
    \phantom{-}11 & \multicolumn{1}{c}{\phantom{-}1} & \multicolumn{1}{c}{\phantom{-}0} & \multicolumn{1}{c}{\phantom{-}0} & \multicolumn{1}{c}{-1/2}& \multicolumn{1}{c}{\phantom{-}10} & \multicolumn{1}{c}{1} & \multicolumn{1}{c|}{\phantom{-}0} \\
    
    \phantom{-}4 & \multicolumn{1}{c}{\phantom{-}2/5} & \multicolumn{1}{c}{\phantom{-}0} & \multicolumn{1}{c}{\phantom{-}1} & \multicolumn{1}{c}{\phantom{-}1/10} & \multicolumn{1}{c}{\phantom{-}4/5} & \multicolumn{1}{c}{0} & \multicolumn{1}{c|}{\phantom{-}1} \\ \cline{1-8}
    
    -11 & \multicolumn{1}{c}{\phantom{-}1/5} & \multicolumn{1}{c}{\phantom{-}0} & \multicolumn{1}{c}{\phantom{-}0} & \multicolumn{1}{c}{\phantom{-}4/5} & \multicolumn{1}{c}{\phantom{-}12/5} & \multicolumn{1}{c}{0} & \multicolumn{1}{c|}{\phantom{-}0} \\ \hhline{|=|=|=|=|=|=|=|=|}

    \phantom{-}4 & \multicolumn{1}{c}{\phantom{-}2/5} & \multicolumn{1}{c}{\phantom{-}1} & \multicolumn{1}{c}{\phantom{-}0} & \multicolumn{1}{c}{\phantom{-}1/10}& \multicolumn{1}{c}{\phantom{-}4/5} & \multicolumn{1}{c}{0} & \multicolumn{1}{c|}{\phantom{-}0} \\

    \phantom{-}5 & \multicolumn{1}{c}{\phantom{-}1/5} & \multicolumn{1}{c}{\phantom{-}0} & \multicolumn{1}{c}{\phantom{-}1} & \multicolumn{1}{c}{\phantom{-}3/10}& \multicolumn{1}{c}{\phantom{-}2/5} & \multicolumn{1}{c}{0} & \multicolumn{1}{c|}{\phantom{-}0} \\
    
    \phantom{-}11 & \multicolumn{1}{c}{\phantom{-}1} & \multicolumn{1}{c}{\phantom{-}0} & \multicolumn{1}{c}{\phantom{-}0} & \multicolumn{1}{c}{-1/2}& \multicolumn{1}{c}{\phantom{-}10} & \multicolumn{1}{c}{1} & \multicolumn{1}{c|}{\phantom{-}0} \\
    
    -1 & \multicolumn{1}{c}{\phantom{-}1/5} & \multicolumn{1}{c}{\phantom{-}0} & \multicolumn{1}{c}{\phantom{-}0} & \multicolumn{1}{c}{-1/5} & \multicolumn{1}{c}{\phantom{-}2/5} & \multicolumn{1}{c}{0} & \multicolumn{1}{c|}{\phantom{-}1} \\ \cline{1-8}
    
    -11 & \multicolumn{1}{c}{\phantom{-}1/5} & \multicolumn{1}{c}{\phantom{-}0} & \multicolumn{1}{c}{\phantom{-}0} & \multicolumn{1}{c}{\phantom{-}4/5} & \multicolumn{1}{c}{\phantom{-}12/5} & \multicolumn{1}{c}{0} & \multicolumn{1}{c|}{\phantom{-}0} \\ \hhline{|=|=|=|=|=|=|=|=|}
    
    \phantom{-}7/2 & \multicolumn{1}{c}{\phantom{-}1/2} & \multicolumn{1}{c}{\phantom{-}1} & \multicolumn{1}{c}{\phantom{-}0} & \multicolumn{1}{c}{\phantom{-}0}& \multicolumn{1}{c}{\phantom{-}1} & \multicolumn{1}{c}{0} & \multicolumn{1}{c|}{\phantom{-}1/2} \\

    \phantom{-}7/2 & \multicolumn{1}{c}{\phantom{-}1/2} & \multicolumn{1}{c}{0} & \multicolumn{1}{c}{\phantom{-}1} & \multicolumn{1}{c}{\phantom{-}0}& \multicolumn{1}{c}{\phantom{-}1} & \multicolumn{1}{c}{0} & \multicolumn{1}{c|}{\phantom{-}3/2} \\
    
    \phantom{-}27/2 & \multicolumn{1}{c}{\phantom{-}1/2} & \multicolumn{1}{c}{\phantom{-}0} & \multicolumn{1}{c}{\phantom{-}0} & \multicolumn{1}{c}{\phantom{-}0}& \multicolumn{1}{c}{\phantom{-}9} & \multicolumn{1}{c}{1} & \multicolumn{1}{c|}{-5/2} \\
    
    \phantom{-}5 & \multicolumn{1}{c}{-1} & \multicolumn{1}{c}{\phantom{-}\phantom{-}0} & \multicolumn{1}{c}{\phantom{-}0} & \multicolumn{1}{c}{\phantom{-}1} & \multicolumn{1}{c}{-2} & \multicolumn{1}{c}{0} & \multicolumn{1}{c|}{-5} \\ \cline{1-8}
    
    -8 & \multicolumn{1}{c}{\phantom{-}1} & \multicolumn{1}{c}{\phantom{-}0} & \multicolumn{1}{c}{\phantom{-}0} & \multicolumn{1}{c}{\phantom{-}0} & \multicolumn{1}{c}{\phantom{-}4} & \multicolumn{1}{c}{0} & \multicolumn{1}{c|}{\phantom{-}4} \\ \cline{1-8}
\end{tabular}
\end{center}

\noindent Obsérvese que al introducir la nueva restricción, la matriz identidad ya no se encuentra entre las columnas de $A$. Tras hacer las operaciones de filas necesarias para volver a tener la identidad en la tabla, lo que ha sucedido es que la última componente del vector $b$ es negativa y, por tanto, en ese paso se ha aplicado el algoritmo dual del símplex. Se concluye que la única solución del nuevo problema es $x^0 = (0,7/2,7/2,5,0,27/2,0)^t$.

\end{example}

\chapter{Programación lineal entera}

Un {\textit{problema de programación lineal entera}} (o {\textit{problema de programación lineal en enteros}}) no es más que un problema de la forma
\[\begin{aligned}[t]
(PE)\ \, &\text{Min. } && c^tx \\
& \, \text{s. a} &&\begin{cases}
Ax=b  \\
x \geq 0 \\
x_i \in \Z \textup{ para todo } i \in J,
\end{cases}
\end{aligned}\]
donde $J \subset \{1,\mathellipsis,n\}$. En estas circunstancias, el problema
\[\begin{aligned}[t]
(PR)\ \, &\text{Min. } && c^tx \\
& \, \text{s. a} &&\begin{cases}
Ax=b  \\
x \geq 0
\end{cases}
\end{aligned}\]
se denomina {\textit{problema relajado}}. Se va a suponer además que tanto $(PE)$ como $(PR)$ tienen una única solución óptima, y que tanto $A$ como $b$ tienen coeficientes enteros.



Resulta bastante natural resolver el problema $(PR)$ mediante las técnicas ya conocidas y rezar las oraciones que se sepan para que la solución satisfaga $x_i \in \Z$ para cada $i \in J$. Si esto no funciona, se podría tratar de redondear cada componente de la solución óptima de $(PR)$, y así se tendrá $x_i \in \Z$ para cada $i \in J$, pudiendo el resto de condiciones dejar de verificarse. Como mostrará el segundo de los ejemplos que siguen, este método no es eficaz.

\begin{example}
Considérese el problema
\[\begin{aligned}[t]
(PE)\ \,&\text{Min. } && -5x_1-4x_2 \\
& \; \text{s. a} &&\left\{\begin{alignedat}{10}
&   &x_1 & {}+{} &  &x_2 & {}\leq{} &5 \\
& 10&x_1 & {}+{} & 6&x_2 & {}\leq{} &45 \\
&\mathrlap{x_1,x_2 \geq 0} \\
&\mathrlap{x_1,x_2 \in \Z}
\end{alignedat}\right.
\end{aligned}\]
El problema relajado correspondiente es
\[\begin{aligned}[t]
(PR) \ \,&\text{Min. } && -5x_1-4x_2 \\
& \; \text{s. a} &&\left\{\begin{alignedat}{10}
&   &x_1 & {}+{} &  &x_2 & {}\leq{} &5 \\
& 10&x_1 & {}+{} & 6&x_2 & {}\leq{} &45 \\
&\mathrlap{x_1,x_2 \geq 0}
\end{alignedat}\right.
\end{aligned}\]
Resolviendo gráficamente, se observa que $x^0_{R} = (15/4,5/4)^t = (3.75,1.25)^t$ es la única solución óptima de $(PR)$, mientras que $x^0_E=(3,2)$ es la única solución óptima de $(PE)$. El redondeo de $x^0_R$ da lugar a $(4,1)$, $(4,2)$, $(3,1)$ y $(3,2)$, pero solo $(3,1)$ y $(3,2)$ están en la región factible. Comparando valores de la función objetivo, surge la tentación de asegurar que $(3,2)$ es la única solución óptima de $(PE)$.
\end{example}

\begin{example}
Considérese el problema
\[\begin{aligned}[t]
(PE)\ \,&\text{Min. } && 3x_1-4x_2 \\
& \; \text{s. a} &&\left\{\begin{alignedat}{10}
& 3&x_1 & {}-{} &  &x_2 & {}\leq{} &4 \\
&  &x_1 & {}-{} & 2&x_2 & {}\leq{} &4 \\
&\mathrlap{x_1,x_2 \geq 0} \\
&\mathrlap{x_1,x_2 \in \Z}
\end{alignedat}\right.
\end{aligned}\]
El problema relajado correspondiente es
\[\begin{aligned}[t]
(PR)\ \,&\text{Min. } && 3x_1-4x_2 \\
& \; \text{s. a} &&\left\{\begin{alignedat}{10}
& 3&x_1 & {}-{} &  &x_2 & {}\leq{} &4 \\
&  &x_1 & {}-{} & 2&x_2 & {}\leq{} &4 \\
&\mathrlap{x_1,x_2 \geq 0}
\end{alignedat}\right.
\end{aligned}\]
La única solución óptima de $(PR)$ es $x^0_R = (4/5,8/5)^t = (0.8,1.6)^t$, y la única solución óptima de $(PE)$ es $x^0_E = (2,1)$. Redondeando $x^0_R$, se puede obtener $(1,2)$, $(1,1)$, $(0,1)$ y $(0,2)$, siendo únicamente $(1,2)$ una solución factible de $(PE)$, que no coincide con la solución óptima del problema.
\end{example}

Consecuentemente, el redondeo no parece ser de gran utilidad y habrá que apañárselas de otra manera si se quiere resolver un problema como $(PE)$.

\section{Método del hiperplano de corte de Gomory}

Considérese un problema de programación lineal en enteros en el que $J = \{1,\mathellipsis,n\}$, esto es, un problema de la forma
\[\begin{aligned}[t]
(PE)\ \, &\text{Min. } && c^tx \\
& \, \text{s. a} &&\begin{cases}
Ax=b  \\
x \geq 0 \\
x \in \Z^n,
\end{cases}
\end{aligned}\]
y sea
\[\begin{aligned}[t]
(PR)\ \, &\text{Min. } && c^tx \\
& \, \text{s. a} &&\begin{cases}
Ax=b  \\
x \geq 0 \\
\end{cases}
\end{aligned}\]
el correspondiente problema relajado.



El algoritmo del hiperplano de corte de Gomory consiste en añadir restricciones lineales a $(PR)$ de manera que los problemas resultantes (que se saben resolver empleando las técnicas vistas en temas anteriores) tengan soluciones óptimas cada vez más cercanas a una solución óptima de $(PE)$.



Sea $x^0_R \in \R^n$ una solución óptima-básica de $(PR)$, hallada mediante el método del símplex. Sin perder generalidad, se va a suponer que la base de la última tabla del símplex se encuentran en las primeras $m$ columnas. La tabla sería entonces

\begin{center}
\setlength\extrarowheight{2.5pt}
\begin{tabular}{|c|c|c|c|c|c|c|c|}
    \cline{2-8}
    
    \multicolumn{1}{c|}{} & \multicolumn{1}{c}{$c_1$} & \multicolumn{1}{c}{$c_2$} & \multicolumn{1}{c}{$\mathellipsis$} & \multicolumn{1}{c}{$c_{m-1}$} & \multicolumn{1}{c}{$c_m$} & \multicolumn{1}{c}{$\mathellipsis$} & \multicolumn{1}{c|}{$c_n$} \\[2.5pt] \hline
    
    $\overline{b}_1$ & \multicolumn{1}{c}{1} & \multicolumn{1}{c}{0} & \multicolumn{1}{c}{$\mathellipsis$} & \multicolumn{1}{c}{0} & \multicolumn{1}{c}{$\alpha_{1,m+1}$} & \multicolumn{1}{c}{$\mathellipsis$} & \multicolumn{1}{c|}{$\alpha_{1,n}$} \\[2.5pt]

    $\overline{b}_2$ & \multicolumn{1}{c}{0} & \multicolumn{1}{c}{1} & \multicolumn{1}{c}{$\mathellipsis$} & \multicolumn{1}{c}{0} & \multicolumn{1}{c}{$\alpha_{2,m+1}$} & \multicolumn{1}{c}{$\mathellipsis$} & \multicolumn{1}{c|}{$\alpha_{2,n}$} \\[2.5pt]
    
    $\vdots$ & \multicolumn{1}{c}{$\vdots$} & \multicolumn{1}{c}{$\vdots$} & \multicolumn{1}{c}{$\ddots$} & \multicolumn{1}{c}{$\vdots$} & \multicolumn{1}{c}{$\vdots$} & \multicolumn{1}{c}{$\ddots$} & \multicolumn{1}{c|}{$\vdots$} \\[2.5pt]
    
    $\overline{b}_m$ & \multicolumn{1}{c}{0} & \multicolumn{1}{c}{0} & \multicolumn{1}{c}{$\mathellipsis$} & \multicolumn{1}{c}{1} & \multicolumn{1}{c}{$\alpha_{m,m+1}$} & \multicolumn{1}{c}{$\mathellipsis$} & \multicolumn{1}{c|}{$\alpha_{m,n}$} \\[2.5pt] \cline{1-8}
\end{tabular}
\end{center}
Por tanto, $x^0_R = (\overline{b}_1,\mathellipsis,\overline{b}_m,0,\mathellipsis,0)^t$. Supóngase que existe $r \in \{1,\mathellipsis,m\}$ de manera que $\overline{b}_r \in \R \setminus \Z$. Sean $I_r=E(\overline{b}_r), F_r = \overline{b}_r-I_r$, y sean $I_{r,j}=E(\alpha_{r,j}), F_{r,j} = \alpha_{r,j}-I_{r,j},$ para cada $j \in \{m+1,\mathellipsis,n\}$, donde $E$ denota la función parte entera. Considérese además una solución factible de $(PE)$, llámese $x = (x_1,\mathellipsis,x_n)$. Entonces se tiene
\[\overline{b}_r = x_r+\alpha_{r,m+1}x_{m+1}+\mathellipsis+\alpha_{r,n}x_n\]
Equivalentemente, usando que $\alpha_{r,j} = F_{r,j}+I_{r,j}$ y que $\overline{b}_r = F_r+I_r$,
\[x_r+\sum_{j=m+1}^nI_{r,j}x_j-I_r = F_r-\sum_{j=m+1}^nF_{r,j}x_j\]
Nótese que
\begin{enumerate}
    \item $\displaystyle{x_r+\sum_{j=m+1}^nI_{r,j}x_j-I_r \in \Z}$, pues $I_{r,j},I_r \in \Z$ y $x \in \Z^n$.
    \item $\displaystyle{F_r-\sum_{j=m+1}^nF_{r,j}x_j \leq F_r < 1}$, pues $x \geq 0, F_{r,j} \geq 0$ y $ 0 < F_r <1$.
\end{enumerate}
Los dos apartados anteriores implican que al añadir al problema $(PR)$ la restricción lineal
\[F_r-\sum_{j=m+1}^nF_{r,j}x_j \leq 0,\]
los coeficientes de la matriz y el término independiente seguirán siendo enteros, y además, como dicha restricción es verificada por todas las soluciones enteras de $(PR)$, entonces la solución óptima $(PE)$ sigue estando entre las soluciones del nuevo problema, que se puede resolver fácilmente por el método del símplex. 



La nueva restricción añadida al problema $(PR)$ es denominada {\textit{corte}}. Obsérvese que, al ser $(x^0_R)_j = 0$ para todo $j \in \{m+1,\mathellipsis,n\}$, la solución óptima-básica $x^0_R$ no verifica el corte, así que la región factible del nuevo problema es más pequeña que la anterior.



Lo que se consigue mediante este proceso es descartar soluciones factibles de $(PR)$ que no son soluciones factibles de $(PE)$. Puede probarse que este método converge hacia la solución de $(PE)$; por desgracia, dicha demostración escapa a los propósitos de esta asignatura. Este método, como anuncia el nombre de la sección, se conoce como {\textit{método del hiperplano de corte de Gomory}}.

\begin{example}
Considérese el problema
\[\begin{aligned}[t]
(PE)\ \,&\text{Min. } && 3x_1+4x_2 \\
& \; \text{s. a} &&\left\{\begin{alignedat}{10}
& 3&x_1 & {}+{} &  &x_2 & {}\geq{} &4 \\
&  &x_1 & {}+{} & 2&x_2 & {}\geq{} &4 \\
&\mathrlap{x_1,x_2 \geq 0} \\
&\mathrlap{x_1,x_2 \in \Z}
\end{alignedat}\right.
\end{aligned}\]
El problema relajado correspondiente es
\[\begin{aligned}[t]
(PR)\ \,&\text{Min. } && 3x_1+4x_2 \\
& \; \text{s. a} &&\left\{\begin{alignedat}{10}
& 3&x_1 & {}+{} &  &x_2 & {}\geq{} &4 \\
&  &x_1 & {}+{} & 2&x_2 & {}\geq{} &4 \\
&\mathrlap{x_1,x_2 \geq 0}
\end{alignedat}\right.
\end{aligned}\]
En forma estándar, sería
\[\begin{aligned}[t]
(\widetilde{PR})\ \,&\text{Min. } && 3x_1+4x_2 \\
& \; \text{s. a} &&\left\{\begin{alignedat}{10}
& 3&x_1 & {}+{} &  &x_2 & {}-{} & x_3 &       &     & {}={} &4 \\
&  &x_1 & {}+{} & 2&x_2 &       &     & {}-{} & x_4 & {}={} &4 \\
&\mathrlap{x_1,x_2,x_3,x_4 \geq 0}
\end{alignedat}\right.
\end{aligned}\]
Equivalentemente,
\[\begin{aligned}[t]
(\widetilde{PR})\ \,&\text{Min. } && 3x_1+4x_2 \\
& \; \text{s. a} &&\left\{\begin{alignedat}{10}
& -&3&x_1 & {}-{} &  &x_2 & {}+{} & x_3 &       &     & {}={} &-4 \\
& -& &x_1 & {}-{} & 2&x_2 &       &     & {}+{} & x_4 & {}={} &-4 \\
&\mathrlap{x_1,x_2,x_3,x_4 \geq 0}
\end{alignedat}\right.
\end{aligned}\]
Ahora se resuelve este último problema mediante el algoritmo del símplex dual. Se advierte que ciertos datos de la tabla van a ser recortados por aquello de ahorrar escritura.
\begin{center}
\begin{tabular}{|c|c|c|c|c|}
    \cline{1-5}

    -4 & \multicolumn{1}{c}{-3} & \multicolumn{1}{c}{-1} & \multicolumn{1}{c}{\phantom{-}1} & \multicolumn{1}{c|}{\phantom{-}0} \\

    -4 & \multicolumn{1}{c}{-1} & \multicolumn{1}{c}{-2} & \multicolumn{1}{c}{\phantom{-}0} & \multicolumn{1}{c|}{\phantom{-}1} \\ \cline{1-5}
    
    \multicolumn{1}{c|}{} & \multicolumn{1}{c}{\phantom{-}3} & \multicolumn{1}{c}{\phantom{-}4} & \multicolumn{1}{c}{\phantom{-}0} & \multicolumn{1}{c|}{\phantom{-}0} \\ \hhline{|=|=|=|=|=|}

    \phantom{-}4/3 & \multicolumn{1}{c}{\phantom{-}1} & \multicolumn{1}{c}{\phantom{-}1/3} & \multicolumn{1}{c}{-1/3} & \multicolumn{1}{c|}{\phantom{-}0} \\

    -8/3 & \multicolumn{1}{c}{\phantom{-}0} & \multicolumn{1}{c}{-5/3} & \multicolumn{1}{c}{-1/3} & \multicolumn{1}{c|}{\phantom{-}1} \\ \cline{1-5}
    
    \multicolumn{1}{c|}{} & \multicolumn{1}{c}{\phantom{-}0} & \multicolumn{1}{c}{\phantom{-}3} & \multicolumn{1}{c}{\phantom{-}1} & \multicolumn{1}{c|}{\phantom{-}0} \\ \hhline{|=|=|=|=|=|}

    \phantom{-}4/5 & \multicolumn{1}{c}{\phantom{-}1} & \multicolumn{1}{c}{\phantom{-}0} & \multicolumn{1}{c}{-2/5} & \multicolumn{1}{c|}{\phantom{-}1/5} \\

    \phantom{-}8/5 & \multicolumn{1}{c}{\phantom{-}0} & \multicolumn{1}{c}{\phantom{-}1} & \multicolumn{1}{c}{\phantom{-}1/5} & \multicolumn{1}{c|}{-3/5} \\ \cline{1-5}
    
    \multicolumn{1}{c|}{} & \multicolumn{1}{c}{\phantom{-}0} & \multicolumn{1}{c}{\phantom{-}0} & \multicolumn{1}{c}{\phantom{-}2/5} & \multicolumn{1}{c|}{\phantom{-}9/5} \\ \cline{2-5}
    
\end{tabular}
\end{center}
Se observa que $x^0_R  = (4/5,8/5,0,0)^t$ es la única solución óptima de $(\widetilde{PR})$. Ahora podría emplearse el método del hiperplano de corte de Gomory a la primera componente de $x^0_R$ o a la segunda. Se aplicará, por ejemplo, a la segunda. Sean
\[I_2 = E\biggl(\frac{8}{5}\biggr) = 1, \qquad I_{2,3} = E\biggl(\frac{1}{5}\biggr)=0 \qquad \textup{y} \qquad I_{2,4} = E\biggl(-\frac{3}{5}\biggr) = -1,\]
y sean
\[F_2 = \frac{8}{5}-1 = \frac{3}{5}, \qquad F_{2,3} = \frac{1}{5}-0=\frac{1}{5} \qquad \textup{y} \qquad F_{2,4} = -\frac{3}{5}+1 = \frac{2}{5}\]
El corte sería
\[\frac{3}{5}-\frac{1}{5}x_3-\frac{2}{5}x_4 \leq 0\]
Al introducir variables de holgura, quedaría
\[-\frac{1}{5}x_3-\frac{2}{5}x_4+x_5=-\frac{3}{5}\]
Ahora se regresa a la última tabla del método del símplex, se añade esta última restricción y se resuelve el problema obtenido:

\begin{center}
\begin{tabular}{|c|c|c|c|c|c|}
    \cline{1-6}

    \phantom{-}4/5 & \multicolumn{1}{c}{\phantom{-}1} & \multicolumn{1}{c}{\phantom{-}0} & \multicolumn{1}{c}{-2/5} & \multicolumn{1}{c}{\phantom{-}1/5} & \multicolumn{1}{c|}{\phantom{-}0} \\

    \phantom{-}8/5 & \multicolumn{1}{c}{\phantom{-}0} & \multicolumn{1}{c}{\phantom{-}1} & \multicolumn{1}{c}{\phantom{-}1/5} & \multicolumn{1}{c}{-3/5} & \multicolumn{1}{c|}{\phantom{-}0} \\

    -3/5 & \multicolumn{1}{c}{\phantom{-}0} & \multicolumn{1}{c}{\phantom{-}0} & \multicolumn{1}{c}{-1/5} & \multicolumn{1}{c}{-2/5} & \multicolumn{1}{c|}{\phantom{-}1} \\ \cline{1-6}
    
    \multicolumn{1}{c|}{} & \multicolumn{1}{c}{\phantom{-}0} & \multicolumn{1}{c}{\phantom{-}0} & \multicolumn{1}{c}{\phantom{-}2/5} & \multicolumn{1}{c}{\phantom{-}9/5} & \multicolumn{1}{c|}{\phantom{-}0} \\ \hhline{|=|=|=|=|=|=|}

    \phantom{-}2 & \multicolumn{1}{c}{\phantom{-}1} & \multicolumn{1}{c}{\phantom{-}0} & \multicolumn{1}{c}{\phantom{-}0} & \multicolumn{1}{c}{\phantom{-}1} & \multicolumn{1}{c|}{-2} \\

    \phantom{-}1 & \multicolumn{1}{c}{\phantom{-}0} & \multicolumn{1}{c}{\phantom{-}1} & \multicolumn{1}{c}{\phantom{-}0} & \multicolumn{1}{c}{-1} & \multicolumn{1}{c|}{\phantom{-}1} \\

    \phantom{-}3 & \multicolumn{1}{c}{\phantom{-}0} & \multicolumn{1}{c}{\phantom{-}0} & \multicolumn{1}{c}{\phantom{-}1} & \multicolumn{1}{c}{\phantom{-}2} & \multicolumn{1}{c|}{-5} \\ \cline{1-6}
    
    \multicolumn{1}{c|}{} & \multicolumn{1}{c}{\phantom{-}0} & \multicolumn{1}{c}{\phantom{-}0} & \multicolumn{1}{c}{\phantom{-}0} & \multicolumn{1}{c}{\phantom{-}1} & \multicolumn{1}{c|}{\phantom{-}2} \\ \cline{2-6}
\end{tabular}
\end{center}
La única solución óptima del nuevo problema es $x^1=(2,1,3,0,0)^t$. Como todas sus componentes son enteras, el método del hiperplano de Gomory permite concluir que $x^1$ es la única solución óptima de $(\widetilde{PE})$, y por tanto $x^*=(2,1)^t$ es la única solución óptima de $(PE)$.
\end{example}

\section{Método de ramificación y acotación}

Este método será expuesto sobre varios ejemplos, y, al igual que en la sección anterior, no se va a demostrar que el algoritmo en cuestión funciona.

\begin{example}
Considérese el problema
\[\begin{aligned}[t]
(PE)\ \,&\text{Min. } && 3x_1+4x_2 \\
& \; \text{s. a} &&\left\{\begin{alignedat}{10}
& 3&x_1 & {}+{} &  &x_2 & {}\geq{} &4 \\
&  &x_1 & {}+{} & 2&x_2 & {}\geq{} &4 \\
&\mathrlap{x_1,x_2 \geq 0} \\
&\mathrlap{x_1,x_2 \in \Z}
\end{alignedat}\right.
\end{aligned}\]
El problema relajado correspondiente es
\[\begin{aligned}[t]
(PR)\ \,&\text{Min. } && 3x_1+4x_2 \\
& \; \text{s. a} &&\left\{\begin{alignedat}{10}
& 3&x_1 & {}+{} &  &x_2 & {}\geq{} &4 \\
&  &x_1 & {}+{} & 2&x_2 & {}\geq{} &4 \\
&\mathrlap{x_1,x_2 \geq 0}
\end{alignedat}\right.
\end{aligned}\]
La única solución óptima de $(PR)$ es $x^0_{PR} = (0.8,1.6)^t$, con valor de la función objetivo $z^0_{PR} = 8.8$. Por tanto, la única solución de $(PE)$, bautizada $x^0_{PE}$, tendrá valor de la función objetivo $z^0_{PE} \geq 9$. Considérese la primera componente no entera de $x^0_{PR}$, que verifica $0 < 0.8 < 1$, y considérense los problemas
\[\begin{aligned}[t]
(P1)\ \,&\text{Min. } && 3x_1+4x_2 \\
& \; \text{s. a} &&\left\{\begin{alignedat}{10}
& 3&x_1 & {}+{} &  &x_2 & {}\geq{} &4 \\
&  &x_1 & {}+{} & 2&x_2 & {}\geq{} &4 \\
&\mathrlap{x_1,x_2 \geq 0} \\
&\mathrlap{x_1 \leq 0} \\
\end{alignedat}\right.
\end{aligned} \quad \begin{aligned}[t]
(P2)\ \,&\text{Min. } && 3x_1+4x_2 \\
& \; \text{s. a} &&\left\{\begin{alignedat}{10}
& 3&x_1 & {}+{} &  &x_2 & {}\geq{} &4 \\
&  &x_1 & {}+{} & 2&x_2 & {}\geq{} &4 \\
&\mathrlap{x_1,x_2 \geq 0} \\
&\mathrlap{0 < x_1 < 1} \\
\end{alignedat}\right.
\end{aligned} \quad \begin{aligned}[t]
(P3)\ \,&\text{Min. } && 3x_1+4x_2 \\
& \; \text{s. a} &&\left\{\begin{alignedat}{10}
& 3&x_1 & {}+{} &  &x_2 & {}\geq{} &4 \\
&  &x_1 & {}+{} & 2&x_2 & {}\geq{} &4 \\
&\mathrlap{x_1,x_2 \geq 0} \\
&\mathrlap{x_1 \geq 1} \\
\end{alignedat}\right.
\end{aligned}\]
El problema $(P2)$ queda descartado, pues no tiene a $x^0_{PE}$ como solución factible. Se resuelven entonces $(P1)$ y $(P3)$. La única solución de $(P1)$ y su valor en la función objetivo son, respectivamente,
\[x^0_{P1} = (0,4)^t \qquad \textup{y} \qquad z^0_{P1} = 16\]
Nótese que $z^0_{P1}= 16$ es una cota superior de $z^0_{PE}$, que tendrá que verificar entonces $9 \leq z^0_{PE} \leq 16$. Por otro lado, en cuanto a $(P3)$, se tiene
\[x^0_{P3} = (1,1.5)^t \qquad \textup{y} \qquad z^0_{P3} = 9\]
Se observa que la única componente no entera de las dos soluciones anteriores es $1.5$, la segunda de $x^0_{P3}$. Se procede entonces de la misma manera: considérense los problemas
\[\begin{aligned}[t]
(P4)\ \,&\text{Min. } && 3x_1+4x_2 \\
& \; \text{s. a} &&\left\{\begin{alignedat}{10}
& 3&x_1 & {}+{} &  &x_2 & {}\geq{} &4 \\
&  &x_1 & {}+{} & 2&x_2 & {}\geq{} &4 \\
&\mathrlap{x_1,x_2 \geq 0} \\
&\mathrlap{x_1 \geq 1} \\
&\mathrlap{x_2 \leq 1}
\end{alignedat}\right.
\end{aligned} \quad \begin{aligned}[t]
(P5)\ \,&\text{Min. } && 3x_1+4x_2 \\
& \; \text{s. a} &&\left\{\begin{alignedat}{10}
& 3&x_1 & {}+{} &  &x_2 & {}\geq{} &4 \\
&  &x_1 & {}+{} & 2&x_2 & {}\geq{} &4 \\
&\mathrlap{x_1,x_2 \geq 0} \\
&\mathrlap{x_1 \geq 1} \\
&\mathrlap{1 < x_2 < 2} \\
\end{alignedat}\right.
\end{aligned} \quad \begin{aligned}[t]
(P6)\ \,&\text{Min. } && 3x_1+4x_2 \\
& \; \text{s. a} &&\left\{\begin{alignedat}{10}
& 3&x_1 & {}+{} &  &x_2 & {}\geq{} &4 \\
&  &x_1 & {}+{} & 2&x_2 & {}\geq{} &4 \\
&\mathrlap{x_1,x_2 \geq 0} \\
&\mathrlap{x_1 \geq 1} \\
&\mathrlap{2 \leq x_2} \\
\end{alignedat}\right.
\end{aligned}\]
Se descarta $(P5)$. En $(P4)$ se tiene
\[x^0_{P4} = (2,1)^t \qquad \textup{y} \qquad z^0_{P4} = 10\]
Se deduce entonces $9 \leq x^0_{PE} \leq 10$. Respecto a $(P6)$,
\[x^0_{P6} = (1,2)^t \qquad \textup{y} \qquad z^0_{P6} = 11\]
Al no quedar decimales en las dos soluciones, el método de ramificación y acotación permite concluir que $x^0_{P4}$ es la única solución óptima de $(PE)$. Este procedimiento se suele representar como

\hfill

\begin{center}
\begin{tikzpicture}
\node[parent node](A){\textA};
\node[parent node,below left =of A](C){\textC};
\node[parent node,below right =of A, rectangle split parts=1](D){\textD};
\node[parent node,node distance=1 and 0.3,below left =of D](E){\textE};
\node[parent node,node distance=1 and 0.3,below right =of D, rectangle split parts=1](F){\textF};

\draw[->](A.south)--+(0,-0.5)-|(D)node[right,near end]{\scriptsize $x_1\geq 1$};
\draw[->](A.south)--+(0,-0.5)-|(C)node[left,near end]{\scriptsize $x_1\leq 0$};
\draw[->](D.south)--+(0,-0.5)-|(E)node[left,near end]{\scriptsize $x_2\leq 1$};
\draw[->](D.south)--+(0,-0.5)-|(F)node[right,near end]{\scriptsize $x_2\geq 2$};
\end{tikzpicture}
\end{center}

\noindent ¿Qué habría ocurrido si al inicio se hubiese escogido la segunda componente de $x^0_{PR}$? La respuesta la ofrece el siguiente diagrama:

\hfill

\begin{center}
\begin{tikzpicture}
\node[parent node](A){\textA};
\node[parent node,below left =of A, rectangle split parts=1](C){\textG};
\node[parent node,below right =of A, rectangle split parts=1](D){\textH};

\draw[->](A.south)--+(0,-0.5)-|(D)node[right,near end]{\scriptsize $x_2\geq 2$};
\draw[->](A.south)--+(0,-0.5)-|(C)node[left,near end]{\scriptsize $x_2\leq 1$};
\end{tikzpicture}
\end{center}

Nótese que continuar la ramificación en el problema $(P2)$ no tiene ningún sentido, pues las soluciones óptimas que van a obtenerse tendrán valor de la función objetivo mayor o igual que $10$, y ya se conoce una solución factible de $(PE)$ con dicho valor de la función objetivo.
\end{example}

\begin{example}
Considérese el problema
\[\begin{aligned}[t]
(PE)\ \,&\text{Min. } && -2x_1-2x_2 \\
& \; \text{s. a} &&\left\{\begin{alignedat}{10}
&  &2&x_1 & {}+{} & 6&x_2 & {}\leq{} &15 \\
& 2&8&x_1 & {}+{} & 8&x_2 & {}\leq{} &77 \\
&\mathrlap{x_1,x_2 \geq 0} \\
&\mathrlap{x_1,x_2 \in \Z}
\end{alignedat}\right.
\end{aligned}\]
Se va a aplicar el método de ramificación y acotación a la primera componente de la solución óptima del problema relajado. Nótese que resolver $(P8)$ es perder el tiempo, pues ya se sabe que $x^0_{P7}$ es la única solución óptima de $(PE)$.
\begin{center}
\begin{tikzpicture}
\node[parent node](A){\textAAA};
\node[parent node,below left =of A, rectangle split parts=1](B){\textBBB};
\node[parent node,below right =of A, rectangle split parts=1](C){\textCCC};
\node[parent node,node distance=1 and 0.3,below left =of B](D){\textDDD};
\node[parent node,node distance=1 and 0.3,below right =of B, rectangle split parts=1](E){\textEEE};
\node[parent node,node distance=1 and 0.3,below left =of E, rectangle split parts=1](F){\textFFF};
\node[parent node,node distance=1 and 0.3,below right =of E, rectangle split parts=1](G){\textGGG};
\node[parent node,node distance=1 and 0.3,below left =of F](H){\textHHH};
\node[parent node,node distance=1 and 0.3,below right =of F, rectangle split parts=1](I){\textIII};

\draw[->](A.south)--+(0,-0.5)-|(B)node[left,near end]{\scriptsize $x_1\leq 2$};
\draw[->](A.south)--+(0,-0.5)-|(C)node[right,near end]{\scriptsize $x_1\geq 3$};
\draw[->](B.south)--+(0,-0.5)-|(D)node[left,near end]{\scriptsize $x_2\leq 1$};
\draw[->](B.south)--+(0,-0.5)-|(E)node[right,near end]{\scriptsize $x_2\geq 2$};
\draw[->](E.south)--+(0,-0.5)-|(F)node[left,near end]{\scriptsize $x_1\leq 1$};
\draw[->](E.south)--+(0,-0.5)-|(G)node[right,near end]{\scriptsize $x_1\geq 2$};
\draw[->](F.south)--+(0,-0.5)-|(H)node[left,near end]{\scriptsize $x_2\leq 2$};
\draw[->](F.south)--+(0,-0.5)-|(I)node[right,near end]{\scriptsize $x_2\geq 3$};
\end{tikzpicture}
\end{center}
\end{example}

\begin{example}
Considérese el problema

\vspace*{-4mm}

\[\begin{aligned}[t]
(PE)\ \,&\text{Min. } && -5x_1-8x_2 \\
& \; \text{s. a} &&\left\{\begin{alignedat}{10}
&  &x_1 & {}+{} &  &x_2 & {}\leq{} &6 \\
& 5&x_1 & {}+{} & 9&x_2 & {}\leq{} &45 \\
&\mathrlap{x_1,x_2 \geq 0} \\
&\mathrlap{x_1,x_2 \in \Z}
\end{alignedat}\right.
\end{aligned}\]

\vspace*{-2mm}

Aplicamos el método de ramificación y acotación a la segunda componente de $x^0_{PR}$:

\begin{center}
\begin{tikzpicture}
\node[parent node](A){\textAA};
\node[parent node,below left =of A](B){\textBB};
\node[parent node,below right =of A, rectangle split parts=1](C){\textCC};
\node[parent node,node distance=1 and 0.3,below left =of C](D){\textDD};
\node[parent node,node distance=1 and 0.3,below right =of C, rectangle split parts=1](E){\textEE};
\node[parent node,node distance=1 and 0.3,below left =of D, rectangle split parts=1](F){\textFF};
\node[parent node,node distance=1 and 0.3,below right =of D](G){\textGG};

\draw[->](A.south)--+(0,-0.5)-|(C)node[right,near end]{\scriptsize $x_2\geq 4$};
\draw[->](A.south)--+(0,-0.5)-|(B)node[left,near end]{\scriptsize $x_2\leq 3$};
\draw[->](C.south)--+(0,-0.5)-|(D)node[left,near end]{\scriptsize $x_1\leq 1$};
\draw[->](C.south)--+(0,-0.5)-|(E)node[right,near end]{\scriptsize $x_1\geq 2$};
\draw[->](D.south)--+(0,-0.5)-|(F)node[left,near end]{\scriptsize $x_2\leq 4$};
\draw[->](D.south)--+(0,-0.5)-|(G)node[right,near end]{\scriptsize $x_2\geq 5$};
\end{tikzpicture}
\end{center}
\end{example}

\chapter{Problema del transporte}

\section{Introducción}

En este tema se tratarán un tipo específico de problema de programación lineal, denominado {\textit{problema del transporte}}. La casuística es la siguiente:



\textit{Un cierto producto debe ser transportado desde $m$ orígenes $\theta_1,\mathellipsis,\theta_m$ hasta $n$ destinos $d_1,\mathellipsis,d_n$. Las cantidades de producto disponibles en cada uno de los orígenes son $a_1,\mathellipsis,a_m$, y las necesidades de producto en cada uno de los destinos son $b_1,\mathellipsis,b_n$. Cuesta $c_{i,j}$ transportar una unidad de producto desde $\theta_i$ hasta $d_j$. El número de unidades de producto que se transportan desde $\theta_i$ hasta $d_j$ es $x_{i,j}$.}



La resolución de uno de estos problemas consiste en determinar las unidades de producto que deben ser transportadas desde cada origen hasta cada destino de manera que de cada origen salga todo lo disponible, a cada destino llegue todo lo que necesita y el coste total sea mínimo. Traduciendo todo esto al cristiano, se trata de resolver un problema de programación lineal de la forma
\[\begin{aligned}[t]
(PT) \ \, &\text{Min. } && \sum_{i=1}^m \sum_{j=1}^n c_{i,j}x_{i,j} \\
& \, \text{s. a} &&\begin{cases}
    \displaystyle{\sum_{j=1}^n x_{i,j}} = a_i \textup{ para todo } i \in \{1,\mathellipsis,m\} \\[14pt]
    \displaystyle{\sum_{i=1}^m x_{i,j}} = b_j \textup{ para todo } j \in \{1,\mathellipsis,n\} \\[14pt]
    x_{i,j} \geq 0 \textup{ para todo } i \in \{1,\mathellipsis,m\} \textup{ y todo } j \in \{1,\mathellipsis,n\} \phantom{\sum}
\end{cases}
\end{aligned}\]

El problema a resolver tiene $nm$ incógnitas y $n+m$ restricciones. Además, de las restricciones del problema se deduce que
\[\sum_{i=1}^ma_i = \sum_{j=1}^nb_j\]

Normalmente, los datos de un problema de transporte serán expuestos en una tabla como la siguiente:
\begin{center}
\setlength\extrarowheight{1.5pt}
\begin{tabular}{ccccc}

     \multicolumn{1}{c}{} & \multicolumn{1}{c}{$d_1$} & \multicolumn{1}{c}{$\mathellipsis$} & \multicolumn{1}{c}{$d_n$} & \multicolumn{1}{c}{} \\[1.5pt] \cline{2-4}

    \multicolumn{1}{c|}{$\theta_1$} & \multicolumn{1}{c}{} & \multicolumn{1}{c}{} & \multicolumn{1}{c|}{} & \multicolumn{1}{c}{$a_1$} \\

    \multicolumn{1}{c|}{$\vdots$} & \multicolumn{1}{c}{} & \multicolumn{1}{c}{$c_{i,j}$} & \multicolumn{1}{c|}{} & \multicolumn{1}{c}{$\vdots$} \\

    \multicolumn{1}{c|}{$\theta_m$} & \multicolumn{1}{c}{} & \multicolumn{1}{c}{} & \multicolumn{1}{c|}{} & \multicolumn{1}{c}{$a_m$} \\[4pt] \cline{2-4}

     \multicolumn{1}{c}{} & \multicolumn{1}{c}{$b_1$} & \multicolumn{1}{c}{$\mathellipsis$} & \multicolumn{1}{c}{$b_n$} & \multicolumn{1}{c}{} \\
\end{tabular}
\end{center}

Si se trata de escribir en forma matricial, el problema del transporte enunciado anteriormente no es más que
\[\begin{aligned}[t]
(PT) \ \, &\text{Min. } && c^tx \\
& \, \text{s. a} &&\begin{cases}
    Ax = b \\
    x \geq 0,
\end{cases}
\end{aligned}\]
donde
\begin{enumerate}
    \item El vector solución es
    \[x = (x_{11},x_{12},\mathellipsis,x_{1n},x_{21},x_{22},\mathellipsis,x_{2n},\mathellipsis,x_{m1},x_{m2},\mathellipsis,x_{mn})^t \in \R^{nm}\]
    \item El vector de costes es \[c = (c_{11},c_{12},\mathellipsis,c_{1n},c_{21},c_{22},\mathellipsis,c_{2n},\mathellipsis,c_{m1},c_{m2},\mathellipsis,c_{mn})^t \in \R^{nm}\]
    \item El vector de términos independientes es
    \[b = (a_1,\mathellipsis,a_m,b_1,\mathellipsis,b_n)^t \in \R^{n+m}\]
    \item La matriz de coeficientes es, escrita por bloques de tamaño $m \times n$, \[A =\left( \begin{array}{cccc|cccc|c|cccc}
    1 & 1 & \mathellipsis & 1 & 0 & 0 & \mathellipsis & 0 & \mathellipsis & 0 & 0 & \mathellipsis & 0 \\
    0 & 0 & \mathellipsis & 0 & 1 & 1 & \mathellipsis & 1 & \mathellipsis & 0 & 0 & \mathellipsis & 0 \\
    \vdots & \vdots & \ddots & \vdots & \vdots & \vdots & \ddots & \vdots & \ddots & \vdots & \vdots & \ddots & \vdots \\
    0 & 0 & \mathellipsis & 0 & 0 & 0 & \mathellipsis & 0 & \mathellipsis & 1 & 1 & \mathellipsis & 1 \\ \hline
    1 & 0 & \mathellipsis & 0 & 1 & 0 & \mathellipsis & 0 & \mathellipsis & 1 & 0 & \mathellipsis & 0 \\
    0 & 1 & \mathellipsis & 0 & 0 & 1 & \mathellipsis & 0 & \mathellipsis & 0 & 1 & \mathellipsis & 0 \\
    \vdots & \vdots & \ddots & \vdots & \vdots & \vdots & \ddots & \vdots & \ddots & \vdots & \vdots & \ddots & \vdots \\
    0 & 0 & \mathellipsis & 1 & 0 & 0 & \mathellipsis & 1 & \mathellipsis & 0 & 0 & \mathellipsis & 1 \\
    \end{array}\right) \in \mathcal{M}_{(m+n)\times nm}(\R)\]
\end{enumerate}

\begin{ctheorem}
    El problema del transporte tiene solución óptima.
\end{ctheorem}

\begin{proof}
Obsérvese que el problema $(PT)$ no es ilimitado, pues se tiene que $x_{i,j} \leq \min_{i,j}\{a_i,b_j\}$ para cualquier solución factible de componentes $x_{i,j}$. Tampoco es imposible, pues el vector de $\R^{nm}$ de componentes
\[x_{i,j} = \frac{a_ib_j}{\sum_{i=1}^m a_i}\]
es solución factible de $(PT)$, como se comprueba fácilmente. Se concluye así  que $(PT)$ tiene solución óptima.
\end{proof}

\begin{ctheorem}
$\textup{rg}(A) = n+m-1$.
\end{ctheorem}

\begin{proof}
En primer lugar, obsérvese que por ser $\sum_{i=1}^ma_i = \sum_{j=1}^nb_j$ se tiene que $\textup{rg}(A) \leq n+m-1$. 



Se va a dar una submatriz regular de $A$ de orden $n+m-1$, construida como sigue: se elimina de $A$ la fila $m+1$-ésima, y se escogen las columnas de lugares $1$, $n+1,2n+1,\mathellipsis,(m-2)n+1$ y $(m-1)n+1$, y las columnas de lugares $2,3,\mathellipsis,n-1$ y $n$. Así, la matriz obtenida sería
\[B = \left(\begin{array}{cccc|cccc}
     1 & 0 & \mathellipsis & 0 & 1 & 1 & \mathellipsis & 1 \\
     0 & 1 & \mathellipsis & 0 & 0 & 0 & \mathellipsis & 0 \\
     \vdots & \vdots & \ddots & \vdots & \vdots & \vdots & \ddots & \vdots \\
     0 & 0 & \mathellipsis & 1 & 0 & 0 & \mathellipsis & 0 \\ \hline
     0 & 0 & \mathellipsis & 0 & 1 & 0 & \mathellipsis & 0 \\
     0 & 0 & \mathellipsis & 0 & 0 & 1 & \mathellipsis & 0 \\
     \vdots & \vdots & \ddots & \vdots & \vdots & \vdots & \ddots & \vdots \\
     0 & 0 & \mathellipsis & 0 & 0 & 0 & \mathellipsis & 1 \\
\end{array}\right) = \left(\begin{array}{c|c}
    I_m & M \\ \hline
    0 & I_{n-1} 
\end{array}\right),\]
donde $M$ es la matriz de orden $m$ que tiene la primera fila llena de unos. Al ser $B$ una matriz triangular, su determinante es el producto de los elementos diagonales, que resulta ser no nulo.
\end{proof}

\begin{ccorollary}
Las soluciones básicas de $(PT)$ tienen, a lo sumo, $n+m-1$ componentes positivas.
\end{ccorollary}

\begin{proof}
Es más que evidente.
\end{proof}

\begin{ctheorem}
Toda submatriz de $A$ tiene determinante $1$, $-1$ o $\, 0$.
\end{ctheorem}

\begin{proof}
No va a hacerse.
\end{proof}

\begin{ctheorem}
    Cualquier base formada por columnas de la matriz $A$ se puede escribir como una matriz triangular, cambiando el orden de filas o columnas si fuera necesario.
\end{ctheorem}

\begin{proof}
Esta tampoco.
\end{proof}

\section{Cálculo de soluciones básica-factibles}

\begin{example}
Considérese el problema de transporte dado por la tabla siguiente:
\begin{center}
\setlength\extrarowheight{2pt}
\begin{tabular}{ccccccc}
    \multicolumn{1}{c}{} & \multicolumn{1}{c}{1} & \multicolumn{1}{c}{2} & \multicolumn{1}{c}{3} & \multicolumn{1}{c}{4} & \multicolumn{1}{c}{5} & \multicolumn{1}{c}{} \\ \cline{2-6}

    \multicolumn{1}{c|}{A} & \multicolumn{1}{c}{55} & \multicolumn{1}{c}{30} & \multicolumn{1}{c}{40} &  \multicolumn{1}{c}{50} & \multicolumn{1}{c|}{40} & \multicolumn{1}{c}{40} \\

    \multicolumn{1}{c|}{B} & \multicolumn{1}{c}{35} & \multicolumn{1}{c}{30} & \multicolumn{1}{c}{100} &  \multicolumn{1}{c}{45} & \multicolumn{1}{c|}{60} & \multicolumn{1}{c}{20} \\
    
    \multicolumn{1}{c|}{C} & \multicolumn{1}{c}{40} & \multicolumn{1}{c}{60} & \multicolumn{1}{c}{95} &  \multicolumn{1}{c}{35} & \multicolumn{1}{c|}{30} & \multicolumn{1}{c}{40} \\[2pt] \cline{2-6}

    \multicolumn{1}{c}{} & \multicolumn{1}{c}{25} & \multicolumn{1}{c}{10} & \multicolumn{1}{c}{20} & \multicolumn{1}{c}{30} & \multicolumn{1}{c}{15} & \multicolumn{1}{c}{} \\
\end{tabular}
\end{center}
Las soluciones básica-factibles del problema se escribirán sobre la misma tabla, poniendo la componente $x_{i,j}$ en la posición del coste $c_{i,j}$. Existen múltiples métodos de obtención de soluciones básica-factibles, todos ellos siguiendo la dinámica de partir de la tabla

\begin{center}
\setlength\extrarowheight{2pt}
\begin{tabular}{ccccccc}
    \multicolumn{1}{c}{} & \multicolumn{1}{c}{1} & \multicolumn{1}{c}{2} & \multicolumn{1}{c}{3} & \multicolumn{1}{c}{4} & \multicolumn{1}{c}{5} & \multicolumn{1}{c}{} \\ \cline{2-6}

    \multicolumn{1}{c|}{A} & \multicolumn{1}{c}{} & \multicolumn{1}{c}{} & \multicolumn{1}{c}{} &  \multicolumn{1}{c}{} & \multicolumn{1}{c|}{} & \multicolumn{1}{c}{40} \\

    \multicolumn{1}{c|}{B} & \multicolumn{1}{c}{} & \multicolumn{1}{c}{} & \multicolumn{1}{c}{} &  \multicolumn{1}{c}{} & \multicolumn{1}{c|}{} & \multicolumn{1}{c}{20} \\
    
    \multicolumn{1}{c|}{C} & \multicolumn{1}{c}{} & \multicolumn{1}{c}{} & \multicolumn{1}{c}{} &  \multicolumn{1}{c}{} & \multicolumn{1}{c|}{} & \multicolumn{1}{c}{40} \\[2pt] \cline{2-6}

    \multicolumn{1}{c}{} & \multicolumn{1}{c}{25} & \multicolumn{1}{c}{10} & \multicolumn{1}{c}{20} & \multicolumn{1}{c}{30} & \multicolumn{1}{c}{15} & \multicolumn{1}{c}{} \\
\end{tabular}
\end{center}
y rellenar las casillas atendiendo a ciertos criterios.

\begin{enumerate}
\item {\textit{Método de la esquina noroeste}}. Como el nombre indica, se empieza a rellenar la tabla por la esquina noroeste, escogiendo en este caso $25 = \min\{25,40\}$.
\begin{center}
\setlength\extrarowheight{2pt}
\begin{tabular}{cccccccc}
    \multicolumn{1}{c}{} & \multicolumn{1}{c}{1} & \multicolumn{1}{c}{2} & \multicolumn{1}{c}{3} & \multicolumn{1}{c}{4} & \multicolumn{1}{c}{5} & \multicolumn{1}{c}{} & \multicolumn{1}{c}{}\\ \cline{2-6}

    \multicolumn{1}{c|}{A} & \multicolumn{1}{c}{25} & \multicolumn{1}{c}{} & \multicolumn{1}{c}{} &  \multicolumn{1}{c}{} & \multicolumn{1}{c|}{} & \multicolumn{1}{c}{\cancel{40}} & \multicolumn{1}{c}{15}\\

    \multicolumn{1}{c|}{B} & \multicolumn{1}{c}{} & \multicolumn{1}{c}{} & \multicolumn{1}{c}{} &  \multicolumn{1}{c}{} & \multicolumn{1}{c|}{} & \multicolumn{1}{c}{20} & \multicolumn{1}{c}{}\\
    
    \multicolumn{1}{c|}{C} & \multicolumn{1}{c}{} & \multicolumn{1}{c}{} & \multicolumn{1}{c}{} &  \multicolumn{1}{c}{} & \multicolumn{1}{c|}{} & \multicolumn{1}{c}{40} & \multicolumn{1}{c}{}\\[2pt] \cline{2-6}

    \multicolumn{1}{c}{} & \multicolumn{1}{c}{\cancel{25}} & \multicolumn{1}{c}{10} & \multicolumn{1}{c}{20} & \multicolumn{1}{c}{30} & \multicolumn{1}{c}{15} & \multicolumn{1}{c}{} & \multicolumn{1}{c}{}\\
\end{tabular}
\end{center}
    
    Como las necesidades en el primer destino se han agotado, continuamos con la siguiente columna, escogiéndose ahora $10 = \min\{10,15\}$.

\begin{center}
\setlength\extrarowheight{2pt}
\begin{tabular}{ccccccccc}
    \multicolumn{1}{c}{} & \multicolumn{1}{c}{1} & \multicolumn{1}{c}{2} & \multicolumn{1}{c}{3} & \multicolumn{1}{c}{4} & \multicolumn{1}{c}{5} & \multicolumn{1}{c}{} & \multicolumn{1}{c}{} & \multicolumn{1}{c}{}\\ \cline{2-6}

    \multicolumn{1}{c|}{A} & \multicolumn{1}{c}{25} & \multicolumn{1}{c}{10} & \multicolumn{1}{c}{} &  \multicolumn{1}{c}{} & \multicolumn{1}{c|}{} & \multicolumn{1}{c}{\cancel{40}} & \multicolumn{1}{c}{\cancel{15}} & \multicolumn{1}{c}{5}\\

    \multicolumn{1}{c|}{B} & \multicolumn{1}{c}{} & \multicolumn{1}{c}{} & \multicolumn{1}{c}{} &  \multicolumn{1}{c}{} & \multicolumn{1}{c|}{} & \multicolumn{1}{c}{20} & \multicolumn{1}{c}{} & \multicolumn{1}{c}{}\\
    
    \multicolumn{1}{c|}{C} & \multicolumn{1}{c}{} & \multicolumn{1}{c}{} & \multicolumn{1}{c}{} &  \multicolumn{1}{c}{} & \multicolumn{1}{c|}{} & \multicolumn{1}{c}{40} & \multicolumn{1}{c}{} & \multicolumn{1}{c}{}\\[2pt] \cline{2-6}

    \multicolumn{1}{c}{} & \multicolumn{1}{c}{\cancel{25}} & \multicolumn{1}{c}{\cancel{10}} & \multicolumn{1}{c}{20} & \multicolumn{1}{c}{30} & \multicolumn{1}{c}{15} & \multicolumn{1}{c}{} & \multicolumn{1}{c}{} & \multicolumn{1}{c}{} \\
\end{tabular}
\end{center}
Repitiendo la misma cantinela un par de veces más, se llega a

\begin{center}
\setlength\extrarowheight{2pt}
\begin{tabular}{ccccccccc}
    \multicolumn{1}{c}{} & \multicolumn{1}{c}{1} & \multicolumn{1}{c}{2} & \multicolumn{1}{c}{3} & \multicolumn{1}{c}{4} & \multicolumn{1}{c}{5} & \multicolumn{1}{c}{} & \multicolumn{1}{c}{} & \multicolumn{1}{c}{}\\ \cline{2-6}

    \multicolumn{1}{c|}{A} & \multicolumn{1}{c}{25} & \multicolumn{1}{c}{10} & \multicolumn{1}{c}{5} &  \multicolumn{1}{c}{} & \multicolumn{1}{c|}{} & \multicolumn{1}{c}{\cancel{40}} & \multicolumn{1}{c}{\cancel{15}} & \multicolumn{1}{c}{\cancel{5}}\\

    \multicolumn{1}{c|}{B} & \multicolumn{1}{c}{} & \multicolumn{1}{c}{} & \multicolumn{1}{c}{15} &  \multicolumn{1}{c}{5} & \multicolumn{1}{c|}{} & \multicolumn{1}{c}{\cancel{20}} & \multicolumn{1}{c}{\cancel{5}} & \multicolumn{1}{c}{}\\
    
    \multicolumn{1}{c|}{C} & \multicolumn{1}{c}{} & \multicolumn{1}{c}{} & \multicolumn{1}{c}{} &  \multicolumn{1}{c}{25} & \multicolumn{1}{c|}{15} & \multicolumn{1}{c}{\cancel{40}} & \multicolumn{1}{c}{\cancel{15}} & \multicolumn{1}{c}{}\\[2pt] \cline{2-6}

    \multicolumn{1}{c}{} & \multicolumn{1}{c}{\cancel{25}} & \multicolumn{1}{c}{\cancel{10}} & \multicolumn{1}{c}{\cancel{20}} & \multicolumn{1}{c}{\cancel{30}} & \multicolumn{1}{c}{\cancel{15}} & \multicolumn{1}{c}{} & \multicolumn{1}{c}{} & \multicolumn{1}{c}{} \\
    
    \multicolumn{1}{c}{} & \multicolumn{1}{c}{} & \multicolumn{1}{c}{} & \multicolumn{1}{c}{\cancel{15}} & \multicolumn{1}{c}{\cancel{25}} & \multicolumn{1}{c}{} & \multicolumn{1}{c}{} & \multicolumn{1}{c}{} & \multicolumn{1}{c}{} \\
\end{tabular}
\end{center}
\item {\textit{Método de mínimo-filas}}. Se comienza por el elemento de la primera fila más pequeño en la tabla original; en nuestro caso, 30. Después se escogería el 40, y así hasta agotar la fila. Al realizar este proceso en el resto de filas, queda determinada una solución básica-factible.
\begin{center}
\setlength\extrarowheight{2pt}
\begin{tabular}{ccccccccc}
    \multicolumn{1}{c}{} & \multicolumn{1}{c}{1} & \multicolumn{1}{c}{2} & \multicolumn{1}{c}{3} & \multicolumn{1}{c}{4} & \multicolumn{1}{c}{5} & \multicolumn{1}{c}{} & \multicolumn{1}{c}{} & \multicolumn{1}{c}{} \\ \cline{2-6}

    \multicolumn{1}{c|}{A} & \multicolumn{1}{c}{} & \multicolumn{1}{c}{10} & \multicolumn{1}{c}{20} &  \multicolumn{1}{c}{} & \multicolumn{1}{c|}{10} & \multicolumn{1}{c}{\cancel{40}} & \multicolumn{1}{c}{\cancel{30}} & \multicolumn{1}{c}{\cancel{10}} \\

    \multicolumn{1}{c|}{B} & \multicolumn{1}{c}{20} & \multicolumn{1}{c}{} & \multicolumn{1}{c}{} &  \multicolumn{1}{c}{} & \multicolumn{1}{c|}{} & \multicolumn{1}{c}{\cancel{20}} & \multicolumn{1}{c}{} & \multicolumn{1}{c}{} \\
    
    \multicolumn{1}{c|}{C} & \multicolumn{1}{c}{5} & \multicolumn{1}{c}{} & \multicolumn{1}{c}{} &  \multicolumn{1}{c}{30} & \multicolumn{1}{c|}{5} & \multicolumn{1}{c}{\cancel{40}} & \multicolumn{1}{c}{\cancel{35}} & \multicolumn{1}{c}{\cancel{5}} \\[2pt] \cline{2-6}

    \multicolumn{1}{c}{} & \multicolumn{1}{c}{\cancel{25}} & \multicolumn{1}{c}{\cancel{10}} & \multicolumn{1}{c}{\cancel{20}} & \multicolumn{1}{c}{\cancel{30}} & \multicolumn{1}{c}{\cancel{15}} & \multicolumn{1}{c}{} & \multicolumn{1}{c}{} & \multicolumn{1}{c}{} \\
    
    \multicolumn{1}{c}{} & \multicolumn{1}{c}{\cancel{5}} & \multicolumn{1}{c}{} & \multicolumn{1}{c}{} & \multicolumn{1}{c}{} & \multicolumn{1}{c}{\cancel{5}} & \multicolumn{1}{c}{} & \multicolumn{1}{c}{} & \multicolumn{1}{c}{} \\
\end{tabular}
\end{center}
\item {\textit{Método de mínimo-columnas}}. Sobran palabras. El método anterior, pero por columnas. Es pura casualidad que este método y el anterior desemboquen en la misma solución.
\begin{center}
\setlength\extrarowheight{2pt}
\begin{tabular}{ccccccccc}
    \multicolumn{1}{c}{} & \multicolumn{1}{c}{1} & \multicolumn{1}{c}{2} & \multicolumn{1}{c}{3} & \multicolumn{1}{c}{4} & \multicolumn{1}{c}{5} & \multicolumn{1}{c}{} & \multicolumn{1}{c}{} & \multicolumn{1}{c}{} \\ \cline{2-6}

    \multicolumn{1}{c|}{A} & \multicolumn{1}{c}{} & \multicolumn{1}{c}{10} & \multicolumn{1}{c}{20} &  \multicolumn{1}{c}{} & \multicolumn{1}{c|}{10} & \multicolumn{1}{c}{\cancel{40}} & \multicolumn{1}{c}{\cancel{30}} & \multicolumn{1}{c}{\cancel{10}} \\

    \multicolumn{1}{c|}{B} & \multicolumn{1}{c}{20} & \multicolumn{1}{c}{} & \multicolumn{1}{c}{} &  \multicolumn{1}{c}{} & \multicolumn{1}{c|}{} & \multicolumn{1}{c}{\cancel{20}} & \multicolumn{1}{c}{} & \multicolumn{1}{c}{} \\
    
    \multicolumn{1}{c|}{C} & \multicolumn{1}{c}{5} & \multicolumn{1}{c}{} & \multicolumn{1}{c}{} &  \multicolumn{1}{c}{30} & \multicolumn{1}{c|}{5} & \multicolumn{1}{c}{\cancel{40}} & \multicolumn{1}{c}{\cancel{35}} & \multicolumn{1}{c}{\cancel{5}} \\[2pt] \cline{2-6}

    \multicolumn{1}{c}{} & \multicolumn{1}{c}{\cancel{25}} & \multicolumn{1}{c}{\cancel{10}} & \multicolumn{1}{c}{\cancel{20}} & \multicolumn{1}{c}{\cancel{30}} & \multicolumn{1}{c}{\cancel{15}} & \multicolumn{1}{c}{} & \multicolumn{1}{c}{} & \multicolumn{1}{c}{} \\
    
    \multicolumn{1}{c}{} & \multicolumn{1}{c}{\cancel{5}} & \multicolumn{1}{c}{} & \multicolumn{1}{c}{} & \multicolumn{1}{c}{} & \multicolumn{1}{c}{\cancel{10}} & \multicolumn{1}{c}{} & \multicolumn{1}{c}{} & \multicolumn{1}{c}{} \\
\end{tabular}
\end{center}
\item {\textit{Método de mínimo matriz}}. El nombre también habla por sí solo.
\begin{center}
\setlength\extrarowheight{2pt}
\begin{tabular}{cccccccc}
    \multicolumn{1}{c}{} & \multicolumn{1}{c}{1} & \multicolumn{1}{c}{2} & \multicolumn{1}{c}{3} & \multicolumn{1}{c}{4} & \multicolumn{1}{c}{5} & \multicolumn{1}{c}{} & \multicolumn{1}{c}{} \\ \cline{2-6}

    \multicolumn{1}{c|}{A} & \multicolumn{1}{c}{5} & \multicolumn{1}{c}{10} & \multicolumn{1}{c}{20} &  \multicolumn{1}{c}{5} & \multicolumn{1}{c|}{} & \multicolumn{1}{c}{\cancel{40}} & \multicolumn{1}{c}{\cancel{30}} \\

    \multicolumn{1}{c|}{B} & \multicolumn{1}{c}{20} & \multicolumn{1}{c}{} & \multicolumn{1}{c}{} &  \multicolumn{1}{c}{} & \multicolumn{1}{c|}{} & \multicolumn{1}{c}{\cancel{20}} & \multicolumn{1}{c}{}\\
    
    \multicolumn{1}{c|}{C} & \multicolumn{1}{c}{} & \multicolumn{1}{c}{} & \multicolumn{1}{c}{} &  \multicolumn{1}{c}{25} & \multicolumn{1}{c|}{15} & \multicolumn{1}{c}{\cancel{40}} & \multicolumn{1}{c}{\cancel{25}} \\[2pt] \cline{2-6}

    \multicolumn{1}{c}{} & \multicolumn{1}{c}{\cancel{25}} & \multicolumn{1}{c}{\cancel{10}} & \multicolumn{1}{c}{\cancel{20}} & \multicolumn{1}{c}{\cancel{30}} & \multicolumn{1}{c}{\cancel{15}} & \multicolumn{1}{c}{} & \multicolumn{1}{c}{} \\
    
    \multicolumn{1}{c}{} & \multicolumn{1}{c}{\cancel{5}} & \multicolumn{1}{c}{} & \multicolumn{1}{c}{} & \multicolumn{1}{c}{\cancel{5}} & \multicolumn{1}{c}{} & \multicolumn{1}{c}{} & \multicolumn{1}{c}{} \\
\end{tabular}
\end{center}
\item {\textit{Método de Vogel o de la diferencia máxima}}. Por cada fila y cada columna de la tabla de los costes, hágase la diferencia en valor absoluto de los dos costos más pequeños. La mayor de las diferencias es 55, correspondiente a la tercera columna. En esta columna, el menor coste es 40, así que en la tabla vacía se rellena la casilla $(1,3)$. Esto se resume en las siguientes tablas:
\begin{center}
\setlength\extrarowheight{2pt}
\begin{tabular}{cccccc}
    \cline{1-5}

    \multicolumn{1}{|c}{55} & \multicolumn{1}{c}{30} & \multicolumn{1}{c}{40} &  \multicolumn{1}{c}{50} & \multicolumn{1}{c|}{40} & \multicolumn{1}{c}{10} \\

    \multicolumn{1}{|c}{35} & \multicolumn{1}{c}{30} & \multicolumn{1}{c}{100} &  \multicolumn{1}{c}{45} & \multicolumn{1}{c|}{60} & \multicolumn{1}{c}{5} \\
    
    \multicolumn{1}{|c}{40} & \multicolumn{1}{c}{60} & \multicolumn{1}{c}{95} &  \multicolumn{1}{c}{35} & \multicolumn{1}{c|}{30} & \multicolumn{1}{c}{5} \\[2pt] \cline{1-5}

    \multicolumn{1}{c}{5} & \multicolumn{1}{c}{0} & \multicolumn{1}{c}{55} & \multicolumn{1}{c}{10} & \multicolumn{1}{c}{10} & \multicolumn{1}{c}{} \\
\end{tabular}
\qquad
\setlength\extrarowheight{2pt}
\begin{tabular}{ccccccc}
    \cline{1-5}

    \multicolumn{1}{|c}{} & \multicolumn{1}{c}{} & \multicolumn{1}{c}{20} &  \multicolumn{1}{c}{} & \multicolumn{1}{c|}{} & \multicolumn{1}{c}{\cancel{40}} & \multicolumn{1}{c}{20}\\

    \multicolumn{1}{|c}{} & \multicolumn{1}{c}{} & \multicolumn{1}{c}{} &  \multicolumn{1}{c}{} & \multicolumn{1}{c|}{} & \multicolumn{1}{c}{20} & \multicolumn{1}{c}{}\\
    
    \multicolumn{1}{|c}{} & \multicolumn{1}{c}{} & \multicolumn{1}{c}{} &  \multicolumn{1}{c}{} & \multicolumn{1}{c|}{} & \multicolumn{1}{c}{40} & \multicolumn{1}{c}{}\\[2pt] \cline{1-5}

     \multicolumn{1}{c}{25} & \multicolumn{1}{c}{10} & \multicolumn{1}{c}{\cancel{20}} & \multicolumn{1}{c}{30} & \multicolumn{1}{c}{15} & \multicolumn{1}{c}{} & \multicolumn{1}{c}{}\\
\end{tabular}
\end{center}
Como la columna tercera ya está agotada, se tacha esta columna de la tabla de las diferencias, y se vuelven a calcular dichas diferencias. La nueva diferencia máxima es 10, apareciendo en varios lugares. Se escoge, por ejemplo, el de la fila primera. Como el menor coste de la primera fila es 30, se rellena la posición $(1,2)$.

\begin{center}
\setlength\extrarowheight{2pt}
\begin{tabular}{cccccc}
    \cline{1-5}

    \multicolumn{1}{|c}{55} & \multicolumn{1}{c}{30} & \multicolumn{1}{c}{\tikzmark{A}{40}} &  \multicolumn{1}{c}{50} & \multicolumn{1}{c|}{40} & \multicolumn{1}{c}{10} \\

    \multicolumn{1}{|c}{35} & \multicolumn{1}{c}{30} & \multicolumn{1}{c}{100} &  \multicolumn{1}{c}{45} & \multicolumn{1}{c|}{60} & \multicolumn{1}{c}{5} \\
    
    \multicolumn{1}{|c}{40} & \multicolumn{1}{c}{60} & \multicolumn{1}{c}{95} &  \multicolumn{1}{c}{35} & \multicolumn{1}{c|}{30} & \multicolumn{1}{c}{5} \\[2pt] \cline{1-5}

    \multicolumn{1}{c}{5} & \multicolumn{1}{c}{0} & \multicolumn{1}{c}{\tikzmark{B}{55}} & \multicolumn{1}{c}{10} & \multicolumn{1}{c}{10} & \multicolumn{1}{c}{}
\end{tabular}
\DrawVLine[black, thick, opacity=1]{A}{B}
\qquad
\setlength\extrarowheight{2pt}
\begin{tabular}{cccccccc}
    \cline{1-5}

    \multicolumn{1}{|c}{} & \multicolumn{1}{c}{10} & \multicolumn{1}{c}{20} &  \multicolumn{1}{c}{} & \multicolumn{1}{c|}{} & \multicolumn{1}{c}{\cancel{40}} & \multicolumn{1}{c}{\cancel{20}} & \multicolumn{1}{c}{10}\\

    \multicolumn{1}{|c}{} & \multicolumn{1}{c}{} & \multicolumn{1}{c}{} &  \multicolumn{1}{c}{} & \multicolumn{1}{c|}{} & \multicolumn{1}{c}{20} & \multicolumn{1}{c}{} & \multicolumn{1}{c}{}\\
    
    \multicolumn{1}{|c}{} & \multicolumn{1}{c}{} & \multicolumn{1}{c}{} &  \multicolumn{1}{c}{} & \multicolumn{1}{c|}{} & \multicolumn{1}{c}{40} & \multicolumn{1}{c}{} & \multicolumn{1}{c}{} \\[2pt] \cline{1-5}

     \multicolumn{1}{c}{25} & \multicolumn{1}{c}{\cancel{10}} & \multicolumn{1}{c}{\cancel{20}} & \multicolumn{1}{c}{30} & \multicolumn{1}{c}{15} & \multicolumn{1}{c}{} & \multicolumn{1}{c}{} & \multicolumn{1}{c}{} \\
\end{tabular}
\end{center}
Se actualiza la tabla de las diferencias, tachando la columna segunda y cambiando las diferencias cuando sea necesario. Se vuelve a escoger el 10 de la fila primera. Ahora hay que fijarse en el coste 40 de la posición $(1,5)$, y rellenar dicha posición en la tabla de la derecha:
\begin{center}
\setlength\extrarowheight{2pt}
\begin{tabular}{ccccccc}
    \cline{1-5}

    \multicolumn{1}{|c}{55} & \multicolumn{1}{c}{\tikzmark{C}{30}} & \multicolumn{1}{c}{\tikzmark{A}{40}} &  \multicolumn{1}{c}{50} & \multicolumn{1}{c|}{40} & \multicolumn{1}{c}{10} & \multicolumn{1}{c}{} \\

    \multicolumn{1}{|c}{35} & \multicolumn{1}{c}{30} & \multicolumn{1}{c}{100} &  \multicolumn{1}{c}{45} & \multicolumn{1}{c|}{60} & \multicolumn{1}{c}{\cancel{5}} & \multicolumn{1}{c}{10} \\
    
    \multicolumn{1}{|c}{40} & \multicolumn{1}{c}{60} & \multicolumn{1}{c}{95} &  \multicolumn{1}{c}{35} & \multicolumn{1}{c|}{30} & \multicolumn{1}{c}{5} & \multicolumn{1}{c}{} \\[2pt] \cline{1-5}

    \multicolumn{1}{c}{5} & \multicolumn{1}{c}{\tikzmark{D}{0}} & \multicolumn{1}{c}{\tikzmark{B}{55}} & \multicolumn{1}{c}{10} & \multicolumn{1}{c}{10} & \multicolumn{1}{c}{} & \multicolumn{1}{c}{} \\

    \multicolumn{1}{c}{} & \multicolumn{1}{c}{} & \multicolumn{1}{c}{} & \multicolumn{1}{c}{} & \multicolumn{1}{c}{} & \multicolumn{1}{c}{} & \multicolumn{1}{c}{}
\end{tabular}
\DrawVLine[black, thick, opacity=1]{A}{B}
\DrawVLine[black, thick, opacity=1]{C}{D}
\qquad
\setlength\extrarowheight{2pt}
\begin{tabular}{cccccccc}
    \cline{1-5}

    \multicolumn{1}{|c}{} & \multicolumn{1}{c}{10} & \multicolumn{1}{c}{20} &  \multicolumn{1}{c}{} & \multicolumn{1}{c|}{10} & \multicolumn{1}{c}{\cancel{40}} & \multicolumn{1}{c}{\cancel{20}} & \multicolumn{1}{c}{\cancel{10}}\\

    \multicolumn{1}{|c}{} & \multicolumn{1}{c}{} & \multicolumn{1}{c}{} &  \multicolumn{1}{c}{} & \multicolumn{1}{c|}{} & \multicolumn{1}{c}{20} & \multicolumn{1}{c}{} & \multicolumn{1}{c}{}\\
    
    \multicolumn{1}{|c}{} & \multicolumn{1}{c}{} & \multicolumn{1}{c}{} &  \multicolumn{1}{c}{} & \multicolumn{1}{c|}{} & \multicolumn{1}{c}{40} & \multicolumn{1}{c}{} & \multicolumn{1}{c}{} \\[2pt] \cline{1-5}

     \multicolumn{1}{c}{25} & \multicolumn{1}{c}{\cancel{10}} & \multicolumn{1}{c}{\cancel{20}} & \multicolumn{1}{c}{30} & \multicolumn{1}{c}{\cancel{15}} & \multicolumn{1}{c}{} & \multicolumn{1}{c}{} & \multicolumn{1}{c}{} \\
     
     \multicolumn{1}{c}{} & \multicolumn{1}{c}{} & \multicolumn{1}{c}{} & \multicolumn{1}{c}{} & \multicolumn{1}{c}{5} & \multicolumn{1}{c}{} & \multicolumn{1}{c}{} & \multicolumn{1}{c}{}
\end{tabular}
\end{center}
Si se continúa este proceso hasta el final, las tablas serían
\begin{center}
\setlength\extrarowheight{2pt}
\begin{tabular}{ccccccc}
    \cline{1-5}

    \multicolumn{1}{|c}{\tikzmark{E}{55}} & \multicolumn{1}{c}{\tikzmark{C}{30}} & \multicolumn{1}{c}{\tikzmark{A}{40}} &  \multicolumn{1}{c}{50} & \multicolumn{1}{c|}{\tikzmark{G}{40}} & \multicolumn{1}{c}{\tikzmark{F}{10}} & \multicolumn{1}{c}{} \\

    \multicolumn{1}{|c}{35} & \multicolumn{1}{c}{30} & \multicolumn{1}{c}{100} &  \multicolumn{1}{c}{45} & \multicolumn{1}{c|}{60} & \multicolumn{1}{c}{\cancel{5}} & \multicolumn{1}{c}{10} \\
    
    \multicolumn{1}{|c}{40} & \multicolumn{1}{c}{60} & \multicolumn{1}{c}{95} &  \multicolumn{1}{c}{35} & \multicolumn{1}{c|}{30} & \multicolumn{1}{c}{5} & \multicolumn{1}{c}{} \\[2pt] \cline{1-5}

    \multicolumn{1}{c}{5} & \multicolumn{1}{c}{\tikzmark{D}{0}} & \multicolumn{1}{c}{\tikzmark{B}{55}} & \multicolumn{1}{c}{10} & \multicolumn{1}{c}{\cancel{10}} & \multicolumn{1}{c}{} & \multicolumn{1}{c}{} \\

    \multicolumn{1}{c}{} & \multicolumn{1}{c}{} & \multicolumn{1}{c}{} & \multicolumn{1}{c}{} & \multicolumn{1}{c}{\tikzmark{H}{30}} & \multicolumn{1}{c}{} & \multicolumn{1}{c}{}
\end{tabular}
\DrawVLine[black, thick, opacity=1]{A}{B}
\DrawVLine[black, thick, opacity=1]{C}{D}
\DrawHLine[black, thick, opacity=1]{E}{F}
\DrawVLine[black, thick, opacity=1]{G}{H}
\qquad
\setlength\extrarowheight{2pt}
\begin{tabular}{cccccccc}
    \cline{1-5}

    \multicolumn{1}{|c}{} & \multicolumn{1}{c}{10} & \multicolumn{1}{c}{20} &  \multicolumn{1}{c}{} & \multicolumn{1}{c|}{10} & \multicolumn{1}{c}{\cancel{40}} & \multicolumn{1}{c}{\cancel{20}} & \multicolumn{1}{c}{\cancel{10}}\\

    \multicolumn{1}{|c}{20} & \multicolumn{1}{c}{} & \multicolumn{1}{c}{} &  \multicolumn{1}{c}{} & \multicolumn{1}{c|}{} & \multicolumn{1}{c}{\cancel{20}} & \multicolumn{1}{c}{} & \multicolumn{1}{c}{}\\
    
    \multicolumn{1}{|c}{5} & \multicolumn{1}{c}{} & \multicolumn{1}{c}{} &  \multicolumn{1}{c}{30} & \multicolumn{1}{c|}{5} & \multicolumn{1}{c}{\cancel{40}} & \multicolumn{1}{c}{\cancel{35}} & \multicolumn{1}{c}{} \\[2pt] \cline{1-5}

     \multicolumn{1}{c}{\cancel{25}} & \multicolumn{1}{c}{\cancel{10}} & \multicolumn{1}{c}{\cancel{20}} & \multicolumn{1}{c}{\cancel{30}} & \multicolumn{1}{c}{\cancel{15}} & \multicolumn{1}{c}{} & \multicolumn{1}{c}{} & \multicolumn{1}{c}{} \\
     
     \multicolumn{1}{c}{\cancel{5}} & \multicolumn{1}{c}{} & \multicolumn{1}{c}{} & \multicolumn{1}{c}{} & \multicolumn{1}{c}{\cancel{5}} & \multicolumn{1}{c}{} & \multicolumn{1}{c}{} & \multicolumn{1}{c}{}
\end{tabular}
\end{center}
\end{enumerate}
\end{example}

De entre los cinco métodos de cálculo de soluciones básica-factibles vistos en el ejemplo anterior, el que proporciona mejores soluciones es el de Vogel, entendiéndose por \textit{mejores soluciones} aquellas con valor de la función objetivo más próximo al óptimo.

\begin{example}
Considérese el problema de transporte dado por
\begin{center}
\setlength\extrarowheight{2pt}
\begin{tabular}{ccccc}
    \multicolumn{1}{c}{} & \multicolumn{1}{c}{A} &  \multicolumn{1}{c}{B} &  \multicolumn{1}{c}{C} &  \multicolumn{1}{c}{} \\ \cline{2-4}

    \multicolumn{1}{c|}{1} & \multicolumn{1}{c}{7} & \multicolumn{1}{c}{8} & \multicolumn{1}{c|}{4} & \multicolumn{1}{c}{35} \\

    \multicolumn{1}{c|}{2} & \multicolumn{1}{c}{1} & \multicolumn{1}{c}{2} & \multicolumn{1}{c|}{6} & \multicolumn{1}{c}{30} \\

    \multicolumn{1}{c|}{3} & \multicolumn{1}{c}{3} & \multicolumn{1}{c}{4} & \multicolumn{1}{c|}{7} & \multicolumn{1}{c}{10} \\[2pt] \cline{2-4}

    \multicolumn{1}{c}{} & \multicolumn{1}{c}{15} & \multicolumn{1}{c}{20} & \multicolumn{1}{c}{10} & \multicolumn{1}{c}{}
\end{tabular}
\end{center}
Se va a encontrar una solución básica-factible del problema mediante el método de Vogel. Lo primero que hay que observar es que la suma de las necesidades ($15+20+10=45$) no coincide con la suma de las disponibilidades ($35+30+10=75$). Esto se soluciona añadiendo un destino ficticio al problema de la siguiente manera:
\begin{center}
\setlength\extrarowheight{2pt}
\begin{tabular}{cccccc}
    \multicolumn{1}{c}{} & \multicolumn{1}{c}{A} &  \multicolumn{1}{c}{B} &  \multicolumn{1}{c}{C} &  \multicolumn{1}{c}{D} & \multicolumn{1}{c}{} \\ \cline{2-5}

    \multicolumn{1}{c|}{1} & \multicolumn{1}{c}{7} & \multicolumn{1}{c}{8} & \multicolumn{1}{c}{4} & \multicolumn{1}{c|}{0} & \multicolumn{1}{c}{35} \\

    \multicolumn{1}{c|}{2} & \multicolumn{1}{c}{1} & \multicolumn{1}{c}{2} & \multicolumn{1}{c}{6} & \multicolumn{1}{c|}{0} & \multicolumn{1}{c}{30} \\

    \multicolumn{1}{c|}{3} & \multicolumn{1}{c}{3} & \multicolumn{1}{c}{4} & \multicolumn{1}{c}{7} & \multicolumn{1}{c|}{0} & \multicolumn{1}{c}{10} \\[2pt] \cline{2-5}

    \multicolumn{1}{c}{} & \multicolumn{1}{c}{15} & \multicolumn{1}{c}{20} & \multicolumn{1}{c}{10} & \multicolumn{1}{c}{30} & \multicolumn{1}{c}{}
\end{tabular}
\end{center}
Ahora sí: la resolución mediante el método de Vogel desemboca en las tablas que siguen.
\begin{center}
\setlength\extrarowheight{2pt}
\begin{tabular}{ccccccc}
    \cline{1-4}

    \multicolumn{1}{|c}{\tikzmark{A}{7}} & \multicolumn{1}{c}{8} & \multicolumn{1}{c}{4} &  \multicolumn{1}{c|}{\tikzmark{D}{0}} & \multicolumn{1}{c}{\cancel{4}} & \multicolumn{1}{c}{\tikzmark{B}{3}} & \multicolumn{1}{c}{} \\

    \multicolumn{1}{|c}{1} & \multicolumn{1}{c}{2} & \multicolumn{1}{c}{6} &  \multicolumn{1}{c|}{0} & \multicolumn{1}{c}{\cancel{1}} & \multicolumn{1}{c}{4} & \multicolumn{1}{c}{} \\
    
    \multicolumn{1}{|c}{3} & \multicolumn{1}{c}{4} & \multicolumn{1}{c}{7} &  \multicolumn{1}{c|}{0} & \multicolumn{1}{c}{\cancel{3}} & \multicolumn{1}{c}{\cancel{1}} & \multicolumn{1}{c}{3} \\[2pt] \cline{1-4}

    \multicolumn{1}{c}{\tikzmark{C}{\cancel{2}}} & \multicolumn{1}{c}{2} & \multicolumn{1}{c}{\cancel{2}} & \multicolumn{1}{c}{\tikzmark{E}{0}} & \multicolumn{1}{c}{} & \multicolumn{1}{c}{} & \multicolumn{1}{c}{} \\

    \multicolumn{1}{c}{} & \multicolumn{1}{c}{} & \multicolumn{1}{c}{1} & \multicolumn{1}{c}{} & \multicolumn{1}{c}{} & \multicolumn{1}{c}{} & \multicolumn{1}{c}{}
\end{tabular}
\DrawHLine[black, thick, opacity=1]{A}{B}
\DrawVLine[black, thick, opacity=1]{A}{C}
\DrawVLine[black, thick, opacity=1]{D}{E}
\qquad
\setlength\extrarowheight{2pt}
\begin{tabular}{ccccccc}
    \cline{1-4}

    \multicolumn{1}{|c}{} & \multicolumn{1}{c}{} & \multicolumn{1}{c}{5} &  \multicolumn{1}{c|}{30} & \multicolumn{1}{c}{\cancel{35}} & \multicolumn{1}{c}{\cancel{5}} & \multicolumn{1}{c}{} \\

    \multicolumn{1}{|c}{15} & \multicolumn{1}{c}{15} & \multicolumn{1}{c}{} &  \multicolumn{1}{c|}{} & \multicolumn{1}{c}{\cancel{30}} & \multicolumn{1}{c}{\cancel{15}} & \multicolumn{1}{c}{} \\
    
    \multicolumn{1}{|c}{} & \multicolumn{1}{c}{5} & \multicolumn{1}{c}{5} &  \multicolumn{1}{c|}{} & \multicolumn{1}{c}{\cancel{10}} & \multicolumn{1}{c}{} & \multicolumn{1}{c}{} \\[2pt] \cline{1-4}

    \multicolumn{1}{c}{\cancel{15}} & \multicolumn{1}{c}{\cancel{20}} & \multicolumn{1}{c}{\cancel{10}} & \multicolumn{1}{c}{\cancel{30}} & \multicolumn{1}{c}{} & \multicolumn{1}{c}{} & \multicolumn{1}{c}{} \\

    \multicolumn{1}{c}{} & \multicolumn{1}{c}{\cancel{5}} & \multicolumn{1}{c}{\cancel{5}} & \multicolumn{1}{c}{} & \multicolumn{1}{c}{} & \multicolumn{1}{c}{} & \multicolumn{1}{c}{}
\end{tabular}
\end{center}
\end{example}

\section{Dual del problema del transporte}

Al ser $(PT)$ un problema de $nm$ variables y $n+m$ restricciones, el dual tendrá $nm$ restricciones y $n+m$ variables, que serán escritas como $(u_1,\mathellipsis,u_m,v_1,\mathellipsis,v_n)^t \in \R^{n+m}$. Sin más dilación, como el problema primal es

\[\begin{aligned}[t]
(PT) \ \, &\text{Min. } && \sum_{i=1}^m \sum_{j=1}^n c_{i,j}x_{i,j} \\
& \, \text{s. a} &&\begin{cases}
    \displaystyle{\sum_{j=1}^n x_{i,j}} = a_i \ \forall \ i \in \{1,\mathellipsis,m\} \\[14pt]
    \displaystyle{\sum_{i=1}^m x_{i,j}} = b_j \ \forall\  j \in \{1,\mathellipsis,n\} \\[14pt]
    x_{i,j} \geq 0 \ \forall\  i \in \{1,\mathellipsis,m\},j \in \{1,\mathellipsis,n\}, \phantom{\sum}
\end{cases}
\end{aligned}\]
entonces el dual será
\[\begin{aligned}[t]
(PTD) \ \, &\text{Min. } && \sum_{i=1}^ma_iu_i+\sum_{j=1}^nb_jv_j \\
& \, \text{s. a} &&\begin{cases}
    u_i+v_j \leq c_{i,j} \ \forall \ i \in \{1,\mathellipsis,m\},j\in \{1,\mathellipsis,n\}
\end{cases}
\end{aligned}\]

\section{Resolución del problema del transporte}

\begin{ctheorem}[Teorema del transporte]
Sea $x^0$ una solución básica-factible no degenerada de $(PT)$ de componentes $x^0_{ij}$. Sea
\[I = \{(i,j) \in \N \times \N \colon x^0_{i,j} \textup{ es variable básica}\}\]
Para cada $i \in \{1,\mathellipsis,m\}$ y cada $j \in \{1,\mathellipsis,n\}$, sean $u^0_i,v^0_j \in \R$ tales que
\begin{enumerate}
    \item $u_i^0+v_j^0=c_{ij}$ para todo $(i,j) \in I$.
    \item $u^0_i+v_j^0 \leq c_{ij}$ para todo $(i,j) \in I^c$.
\end{enumerate}
Entonces $x^0$ es solución óptima de $(PT)$.
\end{ctheorem}

\begin{proof}
Nótese que, en las condiciones del enunciado, el vector $\omega^0 = (u_1^0,\mathellipsis,u_m^0,v_1^0,\mathellipsis,v_n^0)^t$ es solución de $(PTD)$, el dual del problema del transporte. Se va a probar que el valor de la función objetivo del primal en $x^0$ coincide con el valor de la función objetivo del dual en $\omega^0$, y el \hyperref[cor3.1.]{\color{gray}{Corolario 3}} terminará la demostración. Hay que probar entonces que
\[\sum_{i=1}^m\sum_{j=1}^nc_{i,j}x_{i,j}^0 = \sum_{i=1}^m a_iu_i^0+\sum_{j=1}^nb_jv_j^0\]
Obsérvese que
\[\sum_{i=1}^m a_iu_i^0+\sum_{j=1}^nb_jv_j^0 = \sum_{i=1}^m \sum_{j=1}^nx_{i,j}^0u_i^0+\sum_{j=1}^n\sum_{i=1}^mx_{i,j}^0v_j^0 = \sum_{i=1}^m\sum_{j=1}^n(u_i^0+v_j^0)x^0_{i,j},\]
donde en la penúltima igualdad se ha usado que
\[\sum_{j=1}^n x_{i,j}^0 = a_i \qquad \textup{y} \qquad \sum_{i=1}^m x_{i,j}^0 = b_j\]
por ser $x^0$ solución factible de $(PT)$. Total, hay que demostrar que
\[\sum_{i= 1}^m\sum_{j=1}^n(c_{i,j}-u^0_i-v^0_j)x^0_{i,j} = 0\]
Ahora bien, por ser $\omega^0$ solución factible de $(PTD)$, se tiene que
\[c_{i,j}-u^0_i-v^0_j \geq 0 \textup{ para todos } i \in \{1,\mathellipsis,m\}, j \in \{1,\mathellipsis,n\},\]
y por ser $x^0$ solución factible de $(PT)$, puede afirmarse que
\[x^0_{i,j} \geq 0 \textup{ para todos } i \in \{1,\mathellipsis,m\},j \in \{1,\mathellipsis,n\},\]
Por tanto,
\[\sum_{i= 1}^m\sum_{j=1}^n(c_{i,j}-u^0_i-v^0_j)x^0_{i,j} = 0 \iff (c_{i,j}-u^0_i-v^0_j)x^0_{i,j} =0\textup{ para todos } i \in \{1,\mathellipsis,m\}, j \in \{1,\mathellipsis,n\}\]
Sea $i \in \{1,\mathellipsis,m\}$ y sea $j \in \{1,\mathellipsis,n\}$. Si $(i,j) \in I$, entonces
\[c_{i,j}-u^0_i-v^0_j = 0,\]
y si $(i,j) \in I^c$, entonces $x^0_{i,j}$ no es variable básica, es decir,
\[x^0_{i,j} = 0\]
Como en cualquier caso se tiene $(c_{i,j}-u^0_i-v^0_j)x^0_{i,j} = 0$, puede concluirse que $x^0$ es solución óptima del problema del transporte.
\end{proof}

\begin{example}
Considérese el problema de transporte dado por la tabla siguiente:
\begin{center}
\setlength\extrarowheight{2pt}
\begin{tabular}{ccccccc}
    \multicolumn{1}{c}{} & \multicolumn{1}{c}{1} & \multicolumn{1}{c}{2} & \multicolumn{1}{c}{3} & \multicolumn{1}{c}{4} & \multicolumn{1}{c}{5} & \multicolumn{1}{c}{} \\ \cline{2-6}

    \multicolumn{1}{c|}{A} & \multicolumn{1}{c}{55} & \multicolumn{1}{c}{30} & \multicolumn{1}{c}{40} &  \multicolumn{1}{c}{50} & \multicolumn{1}{c|}{40} & \multicolumn{1}{c}{40} \\

    \multicolumn{1}{c|}{B} & \multicolumn{1}{c}{35} & \multicolumn{1}{c}{30} & \multicolumn{1}{c}{100} &  \multicolumn{1}{c}{45} & \multicolumn{1}{c|}{60} & \multicolumn{1}{c}{20} \\
    
    \multicolumn{1}{c|}{C} & \multicolumn{1}{c}{40} & \multicolumn{1}{c}{60} & \multicolumn{1}{c}{95} &  \multicolumn{1}{c}{35} & \multicolumn{1}{c|}{30} & \multicolumn{1}{c}{40} \\[2pt] \cline{2-6}

    \multicolumn{1}{c}{} & \multicolumn{1}{c}{25} & \multicolumn{1}{c}{10} & \multicolumn{1}{c}{20} & \multicolumn{1}{c}{30} & \multicolumn{1}{c}{15} & \multicolumn{1}{c}{} \\
\end{tabular}
\end{center}
Recuérdese la solución básica-factible no degenerada que se obtuvo mediante el método de Vogel:
\begin{center}
\setlength\extrarowheight{2pt}
\begin{tabular}{ccccccc}
    \multicolumn{1}{c}{} & \multicolumn{1}{c}{1} & \multicolumn{1}{c}{2} & \multicolumn{1}{c}{3} & \multicolumn{1}{c}{4} & \multicolumn{1}{c}{5} & \multicolumn{1}{c}{} \\ \cline{2-6}

    \multicolumn{1}{c|}{A} & \multicolumn{1}{c}{} & \multicolumn{1}{c}{10} & \multicolumn{1}{c}{20} &  \multicolumn{1}{c}{} & \multicolumn{1}{c|}{10} & \multicolumn{1}{c}{40} \\

    \multicolumn{1}{c|}{B} & \multicolumn{1}{c}{20} & \multicolumn{1}{c}{} & \multicolumn{1}{c}{} &  \multicolumn{1}{c}{} & \multicolumn{1}{c|}{} & \multicolumn{1}{c}{20} \\
    
    \multicolumn{1}{c|}{C} & \multicolumn{1}{c}{5} & \multicolumn{1}{c}{} & \multicolumn{1}{c}{} &  \multicolumn{1}{c}{30} & \multicolumn{1}{c|}{5} & \multicolumn{1}{c}{40} \\[2pt] \cline{2-6}

    \multicolumn{1}{c}{} & \multicolumn{1}{c}{25} & \multicolumn{1}{c}{10} & \multicolumn{1}{c}{20} & \multicolumn{1}{c}{30} & \multicolumn{1}{c}{15} & \multicolumn{1}{c}{} \\
\end{tabular}
\end{center}
Se va a tratar de utilizar el teorema anterior para comprobar si esta solución es óptima. Se tiene que
\[I = \{(1,2),(1,3),(1,5),(2,1),(3,1),(3,4),(3,5)\}\]
Hay que encontrar $u^0_i$ y $v^0_j$ tales que
\[u^0_1+v^0_2=30,\qquad u^0_1+v^0_3=40,\qquad u^0_1+v^0_5=40,\qquad u^0_2+v^0_1=35,\]
\[\qquad u^0_3+v^0_1=40,\qquad u^0_3+v^0_4=35\qquad \textup{y} \qquad u^0_3+v^0_5=30,\]
lo que se traduce en resolver un sistema de $7$ ecuaciones con $8$ incógnitas. Si se le da un valor a arbitrario a una de las incógnitas, por ejemplo $u^0_1=0$, se obtiene
\[u^0_2=-15,\qquad u^0_3=-10, \qquad v^0_1=50,\qquad v^0_2=30, \qquad v^0_3=40, \qquad v^0_4=45 \qquad \textup{y} \qquad v^0_5=40\]
Ahora se calculan las diferencias $c_{i,j}-u^0_i-v^0_j$:
\begin{center}
\setlength\extrarowheight{2pt}
\begin{tabular}{ccccc}
    \multicolumn{1}{c}{} & \multicolumn{1}{c}{} & \multicolumn{1}{c}{} &  \multicolumn{1}{c}{} & \multicolumn{1}{c}{} \\
    \cline{1-5}

    \multicolumn{1}{|c}{55} & \multicolumn{1}{c}{30} & \multicolumn{1}{c}{40} &  \multicolumn{1}{c}{50} & \multicolumn{1}{c|}{40} \\

    \multicolumn{1}{|c}{35} & \multicolumn{1}{c}{30} & \multicolumn{1}{c}{100} &  \multicolumn{1}{c}{45} & \multicolumn{1}{c|}{60} \\
    
    \multicolumn{1}{|c}{40} & \multicolumn{1}{c}{60} & \multicolumn{1}{c}{95} &  \multicolumn{1}{c}{35} & \multicolumn{1}{c|}{30} \\[2pt] \cline{1-5}
\end{tabular}
\begin{tabular}{c}
    \multicolumn{1}{c}{} \\ 
    \multicolumn{1}{c}{} \\ 
    \multicolumn{1}{c}{$\ -$} \\ 
    \multicolumn{1}{c}{} \\ 
\end{tabular}
\setlength\extrarowheight{2pt}
\begin{tabular}{cccccc}
    \multicolumn{1}{c}{50} & \multicolumn{1}{c}{30} & \multicolumn{1}{c}{40} &  \multicolumn{1}{c}{45} & \multicolumn{1}{c}{40} & \multicolumn{1}{c}{} \\ \cline{1-5}

    \multicolumn{1}{|c}{50} & \multicolumn{1}{c}{30} & \multicolumn{1}{c}{40} &  \multicolumn{1}{c}{45} & \multicolumn{1}{c|}{40} & \multicolumn{1}{c}{0} \\

    \multicolumn{1}{|c}{35} & \multicolumn{1}{c}{15} & \multicolumn{1}{c}{25} &  \multicolumn{1}{c}{30} & \multicolumn{1}{c|}{25} & \multicolumn{1}{c}{-15} \\
    
    \multicolumn{1}{|c}{40} & \multicolumn{1}{c}{20} & \multicolumn{1}{c}{30} &  \multicolumn{1}{c}{35} & \multicolumn{1}{c|}{30} & \multicolumn{1}{c}{-10} \\[2pt] \cline{1-5}
\end{tabular}
\begin{tabular}{c}
    \multicolumn{1}{c}{} \\ 
    \multicolumn{1}{c}{} \\ 
    \multicolumn{1}{c}{$= \ $} \\ 
    \multicolumn{1}{c}{} \\ 
\end{tabular}
\begin{tabular}{ccccc}
    \multicolumn{1}{c}{} & \multicolumn{1}{c}{} & \multicolumn{1}{c}{} &  \multicolumn{1}{c}{} & \multicolumn{1}{c}{} \\
    \cline{1-5}

    \multicolumn{1}{|c}{5} & \multicolumn{1}{c}{0} & \multicolumn{1}{c}{0} &  \multicolumn{1}{c}{5} & \multicolumn{1}{c|}{0} \\

    \multicolumn{1}{|c}{0} & \multicolumn{1}{c}{15} & \multicolumn{1}{c}{75} &  \multicolumn{1}{c}{15} & \multicolumn{1}{c|}{35} \\
    
    \multicolumn{1}{|c}{0} & \multicolumn{1}{c}{40} & \multicolumn{1}{c}{65} &  \multicolumn{1}{c}{0} & \multicolumn{1}{c|}{0} \\[2pt] \cline{1-5}
\end{tabular}
\end{center}
Como todos los elementos de la última tabla son no negativos, puede afirmarse que $x^0$ es solución óptima del problema. Se recuerda la solución que se obtuvo mediante el método de la esquina noroeste:
\begin{center}
\setlength\extrarowheight{2pt}
\begin{tabular}{ccccc}
    \cline{1-5}

    \multicolumn{1}{|c}{25} & \multicolumn{1}{c}{10} & \multicolumn{1}{c}{5} &  \multicolumn{1}{c}{} & \multicolumn{1}{c|}{} \\

    \multicolumn{1}{|c}{} & \multicolumn{1}{c}{} & \multicolumn{1}{c}{15} &  \multicolumn{1}{c}{5} & \multicolumn{1}{c|}{} \\
    
    \multicolumn{1}{|c}{} & \multicolumn{1}{c}{} & \multicolumn{1}{c}{} &  \multicolumn{1}{c}{25} & \multicolumn{1}{c|}{15} \\[2pt] \cline{1-5}
\end{tabular}
\end{center}
Si se hubiese partido de esta solución básica-factible, la tabla de las diferencias obtenida sería
\begin{center}
\setlength\extrarowheight{2pt}
\begin{tabular}{ccccc}
    \cline{1-5}

    \multicolumn{1}{|c}{0} & \multicolumn{1}{c}{0} & \multicolumn{1}{c}{0} &  \multicolumn{1}{c}{$+$} & \multicolumn{1}{c|}{$+$} \\

    \multicolumn{1}{|c}{-80} & \multicolumn{1}{c}{-60} & \multicolumn{1}{c}{0} &  \multicolumn{1}{c}{0} & \multicolumn{1}{c|}{+} \\
    
    \multicolumn{1}{|c}{-60} & \multicolumn{1}{c}{-20} & \multicolumn{1}{c}{+} &  \multicolumn{1}{c}{0} & \multicolumn{1}{c|}{0} \\[2pt] \cline{1-5}
\end{tabular}
\end{center}
Por tanto, la solución de la esquina noroeste no es óptima. Los elementos positivos no se han escrito porque su valor es irrelevante; solo importa que son positivos. A partir de la solución dada, se va a tratar de hallar otra solución básica-factible. Como si de los costes reducidos del método del símplex se tratase, se escoge el elemento de la tabla más pequeño (en este caso -80) y se trata de obtener una solución básica-factible que tenga a $(2,1)$ como componente básica:
\begin{center}
\setlength\extrarowheight{2pt}
\begin{tabular}{ccccc}
    \cline{1-5}

    \multicolumn{1}{|c}{25$-\theta$} & \multicolumn{1}{c}{10} & \multicolumn{1}{c}{5$+\theta$} &  \multicolumn{1}{c}{} & \multicolumn{1}{c|}{} \\

    \multicolumn{1}{|c}{$\theta$} & \multicolumn{1}{c}{} & \multicolumn{1}{c}{15$-\theta$} &  \multicolumn{1}{c}{5} & \multicolumn{1}{c|}{} \\
    
    \multicolumn{1}{|c}{} & \multicolumn{1}{c}{} & \multicolumn{1}{c}{} &  \multicolumn{1}{c}{25} & \multicolumn{1}{c|}{15} \\[2pt] \cline{1-5}
\end{tabular}
\end{center}
Nótese que plantar un $\theta$ en la casilla $(2,1)$ afecta al resto de la tabla, ya que las disponibilidades y necesidades tienen que seguir cuadrando. Como es menester que todos los elementos de la tabla anterior sean positivos (pues la nueva solución ha de ser factible) y una de las componentes básicas anteriores debe anularse (pues la nueva solución ha de ser básica), tiene que ser $\theta = 15$. Por tanto, la nueva solución básica-factible sería
\begin{center}
\setlength\extrarowheight{2pt}
\begin{tabular}{ccccc}
    \cline{1-5}

    \multicolumn{1}{|c}{10} & \multicolumn{1}{c}{10} & \multicolumn{1}{c}{20} &  \multicolumn{1}{c}{} & \multicolumn{1}{c|}{} \\

    \multicolumn{1}{|c}{15} & \multicolumn{1}{c}{} & \multicolumn{1}{c}{5} &  \multicolumn{1}{c}{5} & \multicolumn{1}{c|}{} \\
    
    \multicolumn{1}{|c}{} & \multicolumn{1}{c}{} & \multicolumn{1}{c}{} &  \multicolumn{1}{c}{25} & \multicolumn{1}{c|}{15} \\[2pt] \cline{1-5}
\end{tabular}
\end{center}
Por tanto, ahora puede repetirse el proceso de calcular los $u^0_i$ y $v^0_j$ para comprobar si esta solución es óptima.
\end{example}



Cabe remarcar que el ejemplo anterior está lleno de afirmaciones que deberían ser razonadas e incluso demostradas, pero como la asignatura no da para más, no se van a detallar más justificaciones teóricas.

\end{document}