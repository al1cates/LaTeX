\documentclass[a4paper, 12pt, oneside]{book}

\usepackage{preamble}

\begin{document}

\pagenumbering{roman}

%-----------------------------------------------------------------------------------------------------%

% PÁGINA DEL TÍTULO

\begin{titlepage}

\includegraphics[scale = 0.18]{logo}

\vskip2truecm

\begin{center}

{\huge\bfseries Título del tfg en español  \\[0.4cm]}

\vskip1truecm

{\huge\bfseries Título del tfg en inglés \\[0.4cm]}

\vskip1truecm

{\Large Trabajo Fin de Grado en Matemáticas} \\[\baselineskip]

{\Large Universidad de Málaga} \\

\end{center}

\vskip2truecm

\noindent\rule{\textwidth}{0.4mm}

\vskip.2truecm

{\bf Autor:} {(Nombre y apellidos)} \\

{\bf Área de conocimiento y/o departamento:} \\

{\bf Fecha de presentación: (mes y año)} \\

{\bf Tema:} \\

{\bf Tipo:} {(trabajo de revisión bibliográfica, de iniciación a la investigación,...)} \\

{\bf Modalidad:} {(individual o grupal)} \\

{\bf Número de páginas (sin incluir introducción, bibliografía ni anexos):} \\

\end{titlepage}

%------------------------------------------------------------------------------------------------------%

\newpage
\mbox{}
\thispagestyle{empty}

%------------------------------------------------------------------------------------------------------%

% DECLARACIÓN DE ORIGINALIDAD

\begin{titlepage}

\begin{center}

\textsc{\Large DECLARACIÓN DE ORIGINALIDAD DEL TFG}\\[0.5cm]
\bigskip

\vskip2truecm

\end{center}

D./Dña. \textit{(nombre del autor)}, con DNI (NIE o pasaporte) \textit{(DNI, NIE o pasaporte)}, estudiante del Grado en \textit{(titulación)} de la Facultad de Ciencias de la Universidad de Málaga,\\
\textbf{DECLARO:}\\

Que he realizado el Trabajo Fin de Grado titulado ``\textit{(Título)}'' y que lo presento para su evaluación. Dicho trabajo es original y todas las fuentes bibliográficas utilizadas para su realización han sido debidamente citadas en el mismo.\\
\medskip

De no cumplir con este compromiso, soy consciente de que, de acuerdo con la normativa reguladora de los procesos de evaluación de los aprendizajes del estudiantado de la Universidad de Málaga de 23 de julio de 2019, esto podrá conllevar la calificación de suspenso en la asignatura, sin perjuicio de las responsabilidades disciplinarias en las que pudiera incurrir en caso de plagio.
\bigskip

Para que así conste, firmo la presente en Málaga, el \textit{(fecha)}\\

\vskip.3truecm

\qquad\qquad\qquad {Fdo:..............................................................}

\end{titlepage}

%------------------------------------------------------------------------------------------------------%

\tableofcontents

\addcontentsline{toc}{chapter}{Resumen} 
\addcontentsline{toc}{chapter}{Abstract} 
\addcontentsline{toc}{chapter}{Introducción}

\pagebreak

%------------------------------------------------------------------------------------------------------%

% RESUMEN

{\let\clearpage\relax

{\Large \textbf{El Título aquí}}\\

\chapter*{Resumen}}

Texto.

\vfill

\textbf{Palabras clave:}\\

\textsc{poner aquí las palabras clave.}

\pagebreak

%------------------------------------------------------------------------------------------------------%

% ABSTRACT

{\let\clearpage\relax

{\Large \textbf{El Título (en inglés) aquí}}\\

\chapter*{Abstract}
}

Text. 

\vfill

\textbf{Key words:}\\

\textsc{key words.}

%------------------------------------------------------------------------------------------------------%

% INTRODUCCIÓN

\chapter*{Introducción}

%------------------------------------------------------------------------------------------------------%

% CAPÍTULO 1

\chapter{Preliminares}

\pagenumbering{arabic}
\setcounter{page}{1} 

Este capítulo recoge definiciones y resultados básicos estudiados en las asignaturas de Análisis Real, Análisis Funcional y Ecuaciones en Derivadas Parciales y Análisis de Fourier.

\section[Espacios \texorpdfstring{$L^p$}{Lp}]{Espacios \texorpdfstring{\boldmath$L^p$}{Lp}}

En esta sección introducimos brevemente los espacios $L^p(\T)$ y $l^p$. En primer lugar, si $1 \leq p < \infty$, se define
\[\mathcal{L}^p(\T) = \Bigl\{f \colon \R \to \C \mid f \textup{ es medible, } 2\pi\textup{-periódica y tal que } \integral{-\pi}{\pi}{|f(t)|^p} < \infty \Bigr\}.\]
En este conjunto, se define la relación $\sim$ como sigue: si $f,g \in \mathcal{L}^p(\T)$, $f \sim g$ si y solo si $f = g$ en casi todo punto. Esta relación resulta ser de equivalencia, y el conjunto cociente se denota
\[L^p(\T) = \faktor{\mathcal{L}^p(\T)}{\sim}.\]
Junto con la suma y el producto por escalares habituales, $L^p(\T)$ es un espacio vectorial sobre $\R$ y sobre $\C$. Además, la aplicación $\|\cdot\|_p \colon L^p(\T) \to \R$ dada por
\[\|f\|_{p} = \Bigl(\frac{1}{2\pi}\integral{-\pi}{\pi}{|f(t)|^p}\Bigr)^{\frac{1}{p}}\]
es una norma sobre $L^p(\T)$, y el espacio normado $(L^p(\T),\|\cdot\|_p)$ es de Banach. En particular, para $p=2$, la norma anterior proviene del producto escalar $\langle \cdot,\cdot\rangle \colon L^2(\T)\times L^2(\T)\to \C$ dado por
\[\langle f,g\rangle = \Bigl(\frac{1}{2\pi}\integral{-\pi}{\pi}{f(t)\overline{g(t)}}\Bigr)^{\frac{1}{2}}.\]
Por otro lado, definimos
\[\mathcal{L}^\infty(T) = \Bigl\{f \colon \R \to \C \mid f \textup{ es medible, } 2\pi\textup{-periódica y tal que } \supes_{x \in \R} |f(x)| < \infty\Bigr\},\]
y se denota
\[L^\infty(\T) = \faktor{\mathcal{L}^\infty(T)}{\sim},\]
donde $\sim$ es la relación definida como antes: si $f,g \in \mathcal{L}^\infty(T)$, $f \sim g$ si y solo si $f=g$ en casi todo punto. Esta relación también es de equivalencia, y junto con la suma y el producto por escalares habituales, $L^\infty(\T)$ es un espacio vectorial sobre $\R$ y sobre $\C$. Por último, la aplicación $\|\cdot\|_\infty \colon L^\infty(\T) \to \R$ dada por
\[\|f\|_{\infty} = \supes_{x \in \R} |f(x)|\]
es una norma sobre $L^\infty(\T)$, y el espacio normado $(L^\infty(\T),\|\cdot\|_\infty)$ también es de Banach.

Los espacios $L^p(\T)$ son espacios de medida finita, y por tanto se tiene la cadena de contenciones
\[L^\infty(\T) \subset \mathellipsis \subset L^q(T) \subset \mathellipsis \subset L^p(\T) \subset \mathellipsis \subset L^1(\T),\]
siendo $1 \leq p < q \leq \infty$.
Además, los conjuntos
\begin{align*}
    \mathcal{C}(\T) &= \{f \colon \R \to \C \mid f \textup{ es continua y } 2\pi \textup{-periódica}\}, \\
    \mathcal{C}^k(\T) &= \{f \colon \R \to \C \mid f \textup{ es de clase } k \textup{ y } 2\pi \textup{-periódica}\},
    \\
    \mathcal{C}^\infty(\T) &= \{f \colon \R \to \C \mid f \textup{ es de clase } \infty \textup{ y } 2\pi \textup{-periódica}\},
\end{align*}
son densos en $L^p(\T)$, con $1 \leq p \leq \infty$.

\begin{theorem}[Desigualdad de Hölder]
    Sean $p$ y $p'$ exponentes conjugados con $1 \leq p,p' \leq \infty$. Si $f \in L^p(\T)$ y $g \in L^{p'}(\T)$, entonces $fg \in L^1(\T)$ y
    \[\|fg\|_1 \leq \|f\|_p\|g\|_{p'}.\]
\end{theorem}



Introducimos ahora los espacios $l^p$. Todas las sucesiones consideradas en estos espacios tendrán índices en $\Z$. En primer lugar, si $1\leq p<\infty$, se define
\[l^p=\Bigl\{\{a_k\}_{k\in \Z} \mid a_k \in \C \textup{ para todo } k \in \Z \textup{ y } \sum_{k\in\Z}|a_k|^p < \infty\Bigr\}.\]
Resulta que $l^p$ es un espacio vectorial sobre $\R$ y sobre $\C$, y la aplicación $\|\cdot\|_{l^p} \colon l^p \to \R$ dada por
\[\|\{a_k\}_{k\in\Z}\|_{l^p} = \Bigl(\sum_{k\in\Z}|a_k|^p\Bigr)^{\frac{1}{p}}\]
es una norma sobre $l^p$. En particular, para $p = 2$, esta norma proviene del producto escalar $\langle \cdot, \cdot \rangle_{l^2} \colon l^2 \times l^2 \to \C$ definido por
\[\langle \{a_k\}_{k\in\Z}, \{b_k\}_{k\in\Z} \rangle_{l^2} = \Bigl(\sum_{k\in\Z}a_k\overline{b_k}\Bigr)^{\frac{1}{2}}.\] 

Por otro lado, definimos $l^\infty$ como el conjunto de las sucesiones de números complejos que son acotadas. También se tiene que $l^\infty$ es un espacio vectorial sobre $\R$ y sobre $\C$, y que la aplicación $\|\cdot\|_{l^\infty} \colon l^\infty \to \R$ dada por
\[\|\{a_k\}_{k\in\Z}\|_{l^\infty} = \sup_{k\in\Z} |a_k|\]
es una norma sobre $l^\infty$.

\section{Series de Fourier}

Estudiamos a continuación algunas nociones básicas sobre coeficientes de Fourier y series de Fourier.

\begin{definition}
    Si $f \in L^1(\T)$, se definen los \emph{coeficientes de Fourier de $f$} como
    \[c_k(f) = \frac{1}{2\pi}\integral{-\pi}{\pi}{f(t)e^{-ikt}}, \qquad k \in \Z.\]
    La \emph{serie de Fourier de $f$} es la serie formal
    \[Sf(x) = \sum_{ k \in \Z} c_k(f)e^{ikx}.\]
\end{definition}

Si $f \in L^1(\T)$, la suma parcial $n$-ésima de la serie de Fourier de $f$ se va a denotar por $S_nf$. Al tratarse de una serie con índices enteros, esta suma parcial es
\[S_nf(x) = \sum_{k=-n}^n c_k(f)e^{ikx}.\]
Llamando $a_0(f) = 2c_0(f)$, $a_k(f) = c_k(f)+c_{-k}(f)$ y $b_k(f) = i(c_k(f)-c_{-k}(f))$, la serie de Fourier de $f$ también puede escribirse como
\[Sf(x) = \frac{a_0(f)}{2}+\sum_{k=1}^\infty (a_k(f)\cos(kx)+b_k(f)\sen(kx)),\]
y la suma parcial $n$-ésima, como
\[S_nf(x) = \frac{a_0(f)}{2}+\sum_{k=1}^{n}(a_k(f)\cos(kx)+b_k(f)\sen(kx)).\]
Se demuestra que
\begin{align*}
    a_k(f) &= \frac{1}{\pi}\integral{-\pi}{\pi}{f(t)\cos(kt)}, \qquad k \in \N\cup\{0\}, \\
    b_k(f) &= \frac{1}{\pi}\integral{-\pi}{\pi}{f(t)\sen(kt)}, \qquad k \in \N.
\end{align*}
En general, una función $F \colon \R \to \C$ de la forma
\[F(x) = \sum_{k=-n}^n c_ke^{ikx},\]
con $c_n \neq 0$ o $c_{-n}\neq 0$, se dice que es un \emph{polinomio trigonométrico de grado $n$}. Denotamos por $\mathcal{P}$ al conjunto de todos los polinomios trigonométricos.

\begin{definition}
    Dado $n \in \N \cup \{0\}$, a la función $D_n \colon \R \to \C$ definida por
    \[D_n(t) = \sum_{k=-n}^n e^{ikt}\]
    se le denomina \emph{núcleo de Dirichlet de orden $n$}.
\end{definition}

Se sabe que los núcleos de Dirichlet son funciones pares, continuas y $2\pi$-periódicas, así que están en $L^1(\T)$. Además, toman valores reales y para todo $t \in [-\pi,\pi]$ se verifica
\[D_n(t) = 1+2\sum_{k=1}^n \cos(kt) = \begin{cases}
    2n+1 & $ si $ t = 0, \\[10pt]
    \displaystyle\frac{\sen((n+\frac{1}{2})t)}{\sen(\frac{t}{2})} & $ si $ t \neq 0.
\end{cases}\]

Haciendo uso de los núcleos de Dirichlet, la suma parcial $n$-ésima de la serie de Fourier de $f \in L^1(\T)$ se puede escribir como
\begin{equation}
    S_nf(x) = \frac{1}{2\pi}\integral{0}{\pi}{(f(x+t)+f(x-t))D_n(t)},
\end{equation}
o también como
\begin{equation}\label{1.2.4}
    S_nf(x) = \frac{1}{2\pi}\integral{-\pi}{\pi}{f(t)D_n(x-t)},
\end{equation}
es decir, $S_nf = f \ast D_n$.

Respecto a la convolución de funciones de $L^1(\T)$, se sabe que si $f \in L^1(\T)$ y $g \in L^p(\T)$, con $1 \leq p \leq \infty$, entonces $f \ast g$ está definida en casi todo punto de $\R$. Tras extenderla a todo $\R$, se tiene que $f \ast g \in L^1(\T)$ y que
\[\|f \ast g\|_1 \leq \|f\|_1\|g\|_p.\]
También se sabe que si $f \in L^p(\T)$ y $g \in \mathcal{C}^k(\T)$, entonces $f \ast g \in \mathcal{C}^k(\T)$. Por tanto, si $f \in L^p(\T)$ y $g \in \mathcal{C}^\infty(\T)$, se tiene que $f \ast g \in \mathcal{C}^\infty(\T)$.

\begin{definition}\label{1.2.5}
    Dado $n \in \N \cup \{0\}$, a la función $K_n \colon \R \to \C$ definida por
    \[K_n(t) = \frac{1}{n+1}\sum_{k=0}^n D_k(t)\]
    se le denomina \emph{núcleo de Fejér de orden $n$}.
\end{definition}

Los núcleos de Fejér tienen propiedades similares a los de Dirichlet. Son funciones pares, continuas y $2\pi$-periódicas que toman valores reales no negativos y que verifican, para $t \in [-\pi,\pi]$,
\[K_n(t) = \begin{cases}
    n+1 & $ si $ t = 0, \\[10pt]
    \displaystyle\frac{1}{n+1}\frac{\sen^2(\frac{n+1}{2}t)}{\sen^2(\frac{t}{2})} & $ si $ t \neq 0. 
\end{cases}\]

Como los núcleos de Dirichlet son polinomios trigonométricos de término constante $1$, los núcleos de Fejér también. Además,
\[\|K_n\|_1 = \frac{1}{2\pi}\integral{-\pi}{\pi}{|K_n(t)|} =  \frac{1}{2\pi}\integral{-\pi}{\pi}{K_n(t)} = 1.\]

Tomemos $f \in L^1(\T)$ y consideremos la sucesión de medias de Cesàro de $f$, $\{\sigma_nf\}_{n=0}^\infty$, donde
\[\sigma_nf(x) = \frac{1}{n+1}\sum_{k=0}^nS_nf(x).\]
Entonces
\[\sigma_nf(x) = f \ast K_n(x) = \frac{1}{2\pi}\integral{-\pi}{\pi}{f(t)K_n(x-t)},\]
y se tiene el resultado siguiente.

\begin{theorem}\label{1.2.6}
    Sea $p \in \overline{\R}$ con $1 \leq p \leq \infty$ y sea $f \in L^p(\T)$.
    \begin{enumerate}
        \item Si $1 \leq p < \infty$, entonces
        \[\lim_{n \to \infty} \|\sigma_nf - f\|_p = 0.\]
        \item Si $p = \infty$ y $f \in \mathcal{C}(\T)$, entonces
        \[\lim_{n \to \infty} \|\sigma_nf - f\|_\infty = 0.\]
    \end{enumerate}
\end{theorem}

Si $f,g \in L^1(\T)$ son tales que $c_k(f) = c_k(g)$ para todo $k \in \Z$, entonces $\sigma_nf(x)=\sigma_ng(x)$ para todo $n \in \N \cup \{0\}$ y todo $x \in \R$, y de la unicidad del límite en $L^1(\T)$ se deduce el resultado que sigue.

\begin{corollary}\label{1.1.7}
    Si $f,g \in L^1(\T)$ son tales que $c_k(f) = c_k(g)$ para todo $k \in \Z$, entonces $f = g$ en casi todo punto.
\end{corollary}

Enunciamos a continuación algunos resultados relacionados con las series de Fourier en $L^2(\T)$.

\begin{theorem}[Teorema de Riesz-Fischer]
    La aplicación
    \begin{align*}
        \Phi \colon L^2(\T) &\longrightarrow l^2 \\
        f &\longmapsto \Phi(f) = \{c_k(f)\}_{k\in\Z}
    \end{align*}
    es un isomorfismo isométrico.
\end{theorem}

Por tanto, si $f,g \in L^2(\T)$, como los isomorfismos isométricos preservan el producto escalar, tenemos
\[\langle f,g \rangle = \langle \{c_k(f)\}_{k\in\Z}, \{c_k(g)\}_{k\in\Z} \rangle,\]
es decir,
\[\Bigl(\frac{1}{2\pi}\integral{-\pi}{\pi}{f(t)\overline{g(t)}}\Bigr)^{\frac{1}{2}} = \Bigl(\sum_{k\in\Z} c_k(f)\overline{c_k(g)}\Bigr)^{\frac{1}{2}},\]
luego
\begin{equation}\label{1.2.9}
    \frac{1}{2\pi}\integral{-\pi}{\pi}{f(t)\overline{g(t)}} = \sum_{k\in\Z} c_k(f)\overline{c_k(g)}.
\end{equation}

Más adelante se demostrará que si $1<p<\infty$ y $f\in L^p(\T)$, entonces la serie de Fourier de $f$ converge a $f$ en $L^p(\T)$. Para $p=2$, este resultado ya lo conocemos.
\begin{theorem}
    Si $f \in L^2(\T)$, entonces $\{S_nf\}_{n=1}^\infty$ converge a $f$ en $L^2(\T)$.
\end{theorem}

Recordamos por último un criterio básico sobre la convergencia puntual de series de Fourier.

\begin{theorem}[Criterio de Dini]
    Sea $f \in L^1(\T)$ y sea $x \in \R$. Si existen $\delta > 0$ y $A \in \C$ tales que
    \[\integral{0}{\delta}{\Bigl|\frac{f(x+t)+f(x-t)}{2}-A\Bigr|\frac{1}{t}} < \infty,\]
    entonces $Sf(x) = A$, es decir, la serie de Fourier de $f$ converge puntualmente a $A$ en $x$.
\end{theorem}



\section{Resultados de Análisis Funcional}

Si $(X,\|\cdot\|_X)$ e $(Y,\|\cdot\|_Y)$ son dos espacios normados y $T \colon X \to Y$ es una aplicación lineal, se tiene que $T$ es continua si y solo si existe $C > 0$ tal que
\[\|T(x)\|_Y \leq C\|x\|_X\]
para todo $x \in X$. En ese caso, la \emph{norma de $T$} se define como
\[\|T\| = \inf\{C > 0 \colon \|T(x)\|_Y \leq C\|x\|_X \textup{ para todo } x \in X\}.\]
Esta norma admite algunas expresiones alternativas.
\[\|T\| = \sup_{x \neq 0} \frac{\|T(x)\|_Y}{\|x\|_X} = \sup_{\|x\|_X = 1} \|T(x)\|_Y = \sup_{\|x\|_X \leq 1} \|T(x)\|_Y.\]
También se sabe que $\|\cdot\|$ es una norma sobre
\[B(X,Y) = \{T \colon X \to Y \mid T \textup{ es lineal y continua}\},\]
que es un espacio vectorial sobre $\R$ o sobre $\C$ con las operaciones usuales de suma de funciones y producto de un escalar por una función.

Si en lugar de un espacio normado cualquiera $(Y,\|\cdot\|_Y)$ consideramos $(\K,|\cdot|)$, donde $\K$ denota a $\R$ o a $\C$, llamamos $X^* = B(X,\K)$ y decimos que $(X^*,\|\cdot\|)$ es el \emph{espacio dual de $(X,\|\cdot\|_X)$}. Así,
\[X^* = \{f \colon X \to \K \mid f \textup{ es lineal y continua}\},\]
y para toda $f \in X^*$,
\[\|f\| = \sup_{\|x\|_X=1}|f(x)|.\]

Otros resultados de Análisis Funcional de los que haremos uso son el teorema de la acotación uniforme y un resultado básico sobre extensión de aplicaciones lineales y continuas.

\begin{theorem}[Teorema de la acotación uniforme]\label{1.2.2}
    Sean $(X, \|\cdot\|_X)$ e $(Y,\|\cdot\|_Y)$ dos espacios normados y sea $\{T_j\}_{j \in I}$ una familia de aplicaciones lineales y continuas de $X$ en $Y$. Supongamos que:
    \begin{enumerate}
        \item $(X,\|\cdot\|_X)$ es de Banach.
        \item Para cada $x \in X$, el conjunto $\{T_j(x) \colon j \in I\}$ es acotado en $Y$.
    \end{enumerate}
    Entonces el conjunto $\{\|T_j\| \colon j \in I\}$ es acotado en $\R$.
\end{theorem}

\begin{theorem}\label{1.3.3}
    Sean $(X,\|\cdot\|_X)$ e $(Y,\|\cdot\|_Y)$ dos espacios normados. Supongamos que $(Y,\|\cdot\|_Y)$ es de Banach. Sea $M$ un subespacio vectorial denso en $X$ y sea $f \colon M \to Y$ una aplicación lineal y continua. Consideremos la aplicación $F \colon X \to Y$ definida por 
    \[F(x) = \lim_{n\to\infty} f(x_n),\]
    siendo $\{x_n\}_{n=1}^\infty$ una sucesión cualquiera en $M$ con $\lim_{n\to\infty} \|x-x_n\|_X = 0$. Entonces $F$ está bien definida, es lineal, es continua y verifica $F\! \mid_M \ = f$ y $\|F\| = \|f\|$.
\end{theorem}

\chapter{Convergencia puntual}

En este capítulo se exponen resultados relacionados con la convergencia puntual de las series de Fourier. Principalmente, probaremos que existe una función continua cuya serie de Fourier diverge en un punto, y que existe una función de $L^1(\T)$ cuya serie de Fourier diverge en todo punto.

\section{Una función continua cuya serie de Fourier no converge en un punto}

Se sabe que si $f \in L^1(\T)$ y $x \in \R$ son tales que $f(x^+)$ y $f(x^-)$ existen, entonces el único $A \in \C$ que puede verificar las hipótesis del criterio de Dini es $A = \frac{1}{2}(f(x^+)+f(x^-))$. En particular, si $f$ es continua en $x$, el único $A \in \C$ posible es $A = f(x)$. Sin embargo, esto no garantiza que la serie de Fourier de $f$ converja puntualmente a $f$ en $x$.

En esta sección se demuestra que existe una función continua con serie de Fourier divergente en un punto. Para ello, se va a seguir el mismo camino que en \cite{fierros}, utilizando el \hyperref[1.2.2]{\color{blue}teorema de la acotación uniforme} y el lema siguiente.

\begin{lemma}\label{2.1.1}
    Los núcleos de Dirichlet verifican \[\lim_{n \to \infty} \|D_n\|_1 = \infty.\]
\end{lemma}

\begin{proof}
En efecto, para todo $n \in \N$,
\begin{align*}
    \|D_n\|_1 &= \frac{1}{2\pi}\integral{-\pi}{\pi}{|D_n(t)|} = \frac{1}{2\pi}\integral{-\pi}{\pi}{\frac{|\sen((n+\frac{1}{2})t)|}{|\sen(\frac{t}{2})|}} \\
    \overset{(\asts{1})}&{\geq} \frac{1}{2\pi}\integral{-\pi}{\pi}{\frac{|\sen((n+\frac{1}{2})t)|}{|\frac{t}{2}|}} = \frac{1}{\pi}\integral{-\pi}{\pi}{\frac{|\sen((n+\frac{1}{2})t)|}{|t|}} \\
    &\geq \frac{1}{\pi}\integral{0}{\pi}{\frac{|\sen((n+\frac{1}{2})t)|}{t}} \overset{(\asts{2})}{=} \frac{1}{\pi}\integral[s]{0}{(n+\frac{1}{2})\pi}{\frac{|\sen(s)|}{s}} \\
    &\geq \frac{1}{\pi}\integral[s]{0}{n\pi}{\frac{|\sen(s)|}{s}} = \frac{1}{\pi}\sum_{k=1}^n \integral[s]{(k-1)\pi}{k\pi}{\frac{|\sen(s)|}{s}} \\
    &\geq \frac{1}{\pi}\sum_{k=1}^n \integral[s]{(k-1)\pi}{k\pi}{\frac{|\sen(s)|}{k\pi}} = \frac{1}{\pi^2}\sum_{k=1}^n \frac{1}{k}\integral[s]{(k-1)\pi}{k\pi}{|\sen(s)|} \overset{(\asts{3})}{=} \frac{2}{\pi^2}\sum_{k=1}^n \frac{1}{k},
\end{align*}
y como $\lim_{n\to\infty} \sum_{k=1}^n \frac{1}{k} = \infty$, entonces $\lim_{n \to \infty} \|D_n\|_1 = \infty$. Algunas aclaraciones:
\begin{itemize}
    \item[(\asts{1})] Se ha usado que $|\sen(x)| \leq |x|$ para todo $x \in \R$.
    \item[(\asts{2})] Se ha realizado el cambio de variable $(n+\frac{1}{2})t = s$, $dt = \frac{1}{n+\frac{1}{2}}\,ds$.
    \item[(\asts{3})] Se usa que
    \[\integral[s]{(k-1)\pi}{k\pi}{|\sen(s)|} = \integral[s]{0}{\pi}{|\sen(s)|} = \integral[s]{0}{\pi}{\sen(s)} =2,\]
    teniendo en cuenta en la primera igualdad que la función $|\sen|$ es $\pi$-periódica. \qedhere
\end{itemize}
\end{proof}

\begin{theorem}\label{2.1.2}
    Existe una función continua cuya serie de Fourier diverge en un punto.
\end{theorem}

\begin{proof}
    En el espacio de Banach $(\mathcal{C}([-\pi,\pi]),\|\cdot\|_\infty)$, consideremos la familia de aplicaciones $\{T_n\}_{n\in \N}$, donde $T_n \colon \mathcal{C}([-\pi,\pi]) \to \R$, $T_n(g) = S_ng(0)$. Utilizando \hyperref[1.2.4]{\color{blue}(1.2.4)} y teniendo en cuenta que $D_n$ es par,
    \[T_n(g) = \frac{1}{2\pi}\integral{-\pi}{\pi}{g(t)D_n(-t)} = \frac{1}{2\pi}\integral{-\pi}{\pi}{g(t)D_n(t)}.\]
    Sea $n \in \N$. La linealidad de $T_n$ se deduce inmediatamente de la linealidad de la integral. Si $g \in \mathcal{C}([-\pi,\pi])$,
    \begin{align*}
        |T_n(g)|\leq \frac{1}{2\pi}\integral{-\pi}{\pi}{|g(t)||D_n(t)|} \leq \frac{1}{2\pi}\integral{-\pi}{\pi}{\|g\|_\infty |D_n(t)|} = \|D_n\|_1\|g\|_\infty,
    \end{align*}
    luego $T_n$ es continua y $\|T_n\| \leq \|D_n\|_1$. Veamos que esta desigualdad es en realidad una igualdad. Sea $f_n \colon [-\pi,\pi] \to \R$ la función dada por
    \[f_n(t) = \begin{cases}
        \phantom{-}1 & $ si $ D_n(t) \geq 0, \\
        -1 & $ si $ D_n(t) < 0.
    \end{cases}\]
    Para todo $t \neq 0$ se tiene que $D_n(t) = \frac{\sen((n+\frac{1}{2})t)}{\sen(\frac{t}{2})}$, luego
    \[D_n(t) = 0 \iff \bigl(n+\frac{1}{2}\bigr)t = k \pi, \ k \in \Z \iff t = \frac{k\pi}{n+\frac{1}{2}}, \ k \in \Z.\]
    De esto se deduce que $D_n$ se anula un número finito de veces en el intervalo $[-\pi,\pi]$, así que $f_n$ tiene un número finito de discontinuidades. Modificándola en un entorno de cada discontinuidad, puede obtenerse una sucesión $\{g_{n,m}\}_{m=1}^\infty$ de funciones continuas en $[-\pi,\pi]$ tales que $\lim_{m \to \infty}g_{n,m}(t) = f_n(t)$ para todo $t \in [-\pi,\pi]$ y $\|g_{n,m}\|_\infty = 1$ para todo $m \in \N$. Se tiene que
    \begin{align*}
        \lim_{m \to \infty} T_n(g_{n,m}) &= \lim_{m \to \infty} \frac{1}{2\pi}\integral{-\pi}{\pi}{g_{n,m}(t)D_n(t)} =\frac{1}{2\pi}\integral{-\pi}{\pi}{\lim_{m \to \infty} g_{n,m}(t) D_n(t)} \\
        &= \frac{1}{2\pi}\integral{-\pi}{\pi}{f_n(t) D_n(t)} = \frac{1}{2\pi}\integral{-\pi}{\pi}{|D_n(t)|} = \|D_n\|_1,
    \end{align*}
    donde en la segunda igualdad el intercambio del límite con la integral está justificado por el teorema de la convergencia dominada: si $m \in \N$ y $t \in [-\pi,\pi]$,
    \[|g_{n,m}(t) D_n(t)| \leq |D_n(t)|,\]
    y $\integral{-\pi}{\pi}{|D_n(t)|} < \infty$. Tenemos entonces que para todo $\varepsilon >0$ existe $m_0 \in \N$ tal que para todo $m \geq m_0$ se verifica
    \[\|D_n\|_1 - |T_n(g_{n,m})| \leq  |T_n(g_{n,m}) - \|D_n\|_1| < \varepsilon,\]
    luego, usando que $\|g_{n,m}\|_\infty = 1$,
    \[\|D_n\|_1 \leq |T_n(g_{n,m})| + \varepsilon \leq \sup_{\|g\|_\infty = 1} |T_n(g)| + \varepsilon = \|T_n\| + \varepsilon.\]
    Como esto es cierto para todo $\varepsilon > 0$, entonces $\|D_n\|_1 \leq \|T_n\|$ y queda probado que $\|D_n\|_1 = \|T_n\|$. Por tanto, $\lim_{n \to \infty} \|T_n\| = \lim_{n \to \infty} \|D_n\|_1 = \infty$. Como el conjunto de números reales $\{\|T_n\| \colon n \in \N\}$ no es acotado, el contrarrecíproco del \hyperref[1.2.2]{\color{blue}teorema de la acotación uniforme} permite afirmar que existe $g \in \mathcal{C}([-\pi,\pi])$ tal que el conjunto $\{T_n(g) \colon n \in \N\}$ no es acotado en $\R$, es decir,
    \[\sup_{n \in \N} |T_n(g)| = \sup_{n \in \N} |S_ng(0)|= \infty.\]
    Se concluye que existe una función continua, $g$, cuya serie de Fourier en $0$ no converge.
\end{proof}

\section[Una función de \texorpdfstring{$L^1$}{L1} cuya serie de Fourier diverge en todo punto]{Una función de \texorpdfstring{\boldmath$L^1$}{L1} cuya serie de Fourier diverge en todo punto}

El teorema que protagoniza esta sección fue probado originalmente por Kolmogorov. La demostración aquí expuesta se basa en \cite{zygmund}, y para realizarla se necesita demostrar previamente varios lemas. También se ha seguido \cite{schmidt} para probar el \hyperref[]{\color{blue}Lema 2.2.3}.

\begin{lemma}\label{2.2.1}
    Sea $f \in L^1(\T)$. Para todo $n \in \N$ y todo $x \in \R$,
    \[|S_nf(x)| \leq (4n+1)\|f\|_{1}.\]
\end{lemma}

\begin{proof}
    Para todo $k \in \N \cup \{0\}$,
    \[|a_k(f)| = \Bigl|\frac{1}{\pi}\integral{-\pi}{\pi}{f(t)\cos(kt)}\Bigr| \leq \frac{1}{\pi}\integral{-\pi}{\pi}{|f(t)|} = 2\|f\|_{1}.\]
    Análogamente, para todo $k \in \N$,
    \[|b_k(f)| = \Bigl|\frac{1}{\pi}\integral{-\pi}{\pi}{f(t)\sen(kt)}\Bigr| \leq \frac{1}{\pi}\integral{-\pi}{\pi}{|f(t)|} = 2\|f\|_{1}.\]
    Por tanto,
    \begin{align*}
        |S_nf(x)| &= \Bigl|\frac{a_0(f)}{2}+\sum_{k=1}^n (a_k(f)\cos(kx)+b_k(f)\sen(kx))\Bigr| \\
        &\leq \frac{|a_0(f)|}{2}+\sum_{k=1}^n|a_k(f)|+\sum_{k=1}^n|b_k(f)| \\
        &\leq \|f\|_{1}+\sum_{k=1}^n2\|f\|_{1}+\sum_{k=1}^n 2\|f\|_{1} \\
        &=(4n+1)\|f\|_{1}. \qedhere
    \end{align*}
\end{proof}

Antes de enunciar el próximo lema, introducimos la parte fraccionaria de un número real.

\begin{definition}
    Sea $\alpha \in \R$.
    \begin{itemize}
        \item Si $\alpha \geq 0$, se define la \emph{parte fraccionaria de $\alpha$} como $\angles{\alpha} = \alpha - E(\alpha)$, donde $E(\alpha)$ es el único número entero que satisface $E(\alpha) \leq \alpha < E(\alpha)+1$.
        \item Si $\alpha < 0$, se define la \emph{parte fraccionaria de $\alpha$} como $\angles{\alpha} = \angles{-\alpha}$.
    \end{itemize} 
\end{definition}

Estudiemos algunas propiedades elementales de la parte fraccionaria que se usarán en resultados posteriores.

\begin{enumerate}
    \item $\angles{\alpha} \in [0,1)$ para todo $\alpha\in\R$. Esto es inmediato usando que $E(\alpha) \leq \alpha < E(\alpha)+1$.
    \item Si $|\alpha| < 1$, entonces $\angles{\alpha}=|\alpha| $. En efecto, si $0 \leq \alpha < 1$, entonces $E(\alpha) = 0$ y por tanto $\angles{\alpha} = \alpha - E(\alpha) = \alpha = |\alpha|$. Y si $-1 < \alpha \leq 0$, entonces $0\leq -\alpha < 1$ y usando lo que se acaba de probar obtenemos $\angles{\alpha} = \angles{-\alpha} = |\!-\alpha| = |\alpha|$.
    \item Si $\alpha\in\R$ y $n \in \Z$, entonces $\angles{\alpha+n} = \angles{\alpha}$. Veámoslo. Supongamos primero que $\alpha \geq 0$. Se distinguen dos casos:
    \begin{itemize}
        \item Si $\alpha+n \geq 0$, entonces
        \[\angles{\alpha+n} = \alpha+n-E(\alpha+n) = \alpha+n-(E(\alpha)+n) = \alpha-E(\alpha) = \angles{\alpha}.\]
        \item Si $\alpha+n < 0$, entonces
        \[
        \begin{aligned}[t]
            \angles{\alpha+n} &= \angles{-\alpha-n} = -\alpha-n-E(-\alpha-n) = -\alpha-n-(E(-\alpha)-n) \\ &= -\alpha-E(-\alpha) = \angles{-\alpha} = \angles{\alpha}.
        \end{aligned}
        \]
    \end{itemize}
    Si $\alpha < 0$, aplicando lo que se acaba de probar obtenemos que para todo $n \in \Z$ se cumple $\angles{-\alpha} = \angles{-\alpha-n}$ y por tanto $\angles{\alpha} = \angles{\alpha+n}$.
    \item Si $\alpha,\beta > 0$, entonces $\angles{\alpha+\beta} = \angles{\angles{\alpha}+\angles{\beta}}$. En efecto, por definición de parte fraccionaria,
    \[\angles{\alpha}+\angles{\beta} = \alpha-E(\alpha)+\beta-E(\beta),\]
    luego
    \[\angles{\angles{\alpha}+\angles{\beta}} = \angles{\alpha-E(\alpha)+\beta-E(\beta)} \overset{(c)}{=} \angles{\alpha+\beta}.\]
\end{enumerate}

\begin{lemma}
    Dado $\alpha \in \R \setminus \Q$, el conjunto $\{\angles{k\alpha} \mid k \in \N\}$ es denso en $[0,1)$.
\end{lemma}

\begin{proof}
    Como $\angles{k\alpha} = \angles{-k\alpha}$ para todo $k \in \N$, basta probar el resultado para $\alpha > 0$. Hay que demostrar que 
    \begin{equation}\label{2.2.3}
    \textup{ para todo } x \in [0,1) \textup{ y todo } \varepsilon > 0 \textup{, existe } k \in \N \textup{ con } |\angles{k\alpha}-x|<\varepsilon.
    \end{equation}
    Antes de probar esto, fijemos $m \in \N$ con $m > 1$ y consideremos los intervalos
    \begin{equation}\label{2.2.4}
        I_n = \Bigl[\frac{n}{m},\frac{n+1}{m}\Bigr), \qquad n \in \{0,1,\mathellipsis,m-1\}.
    \end{equation}
    Es claro que $\bigcup_{n=0}^{m-1}I_n = [0,1)$ y que estos intervalos son disjuntos. Se consideran los números reales
    \[\angles{n\alpha}, \qquad n \in \{0,1,\mathellipsis,m\}.\]
    Estos números son todos distintos, pues si existiesen $n_1,n_2\in\{0,1,\mathellipsis,m\}$ con $n_1 \neq n_2$ y $\angles{n_1\alpha} = \angles{n_2\alpha}$, se tendría $n_1\alpha-E(n_1\alpha) = n_2\alpha-E(n_2\alpha)$ y por tanto
    \[\alpha = \frac{E(n_1\alpha)-E(n_2\alpha)}{n_1-n_2} \in \Q,\] que es una contradicción.
    
    Como además $\angles{n\alpha}\in[0,1)$ para todo $n \in \{0,1,\mathellipsis,m\}$, entonces alguno de los $m$ intervalos dados en \hyperref[2.2.4]{\color{blue}(2.2.4)} contiene a al menos dos de los $m+1$ números anteriores. Es decir, existen  $n_1,n_2 \in \{0,1,\mathellipsis,m\}$ tales que $n_1 \neq n_2$ y $\angles{n_1\alpha},\angles{n_2\alpha} \in I_n$ para algún $n \in \{0,1,\mathellipsis,m-1\}$. Sin pérdida de generalidad, supongamos que $n_1 > n_2$. Como $I_n$ es un intervalo de longitud $\frac{1}{m}$, entonces
    \[|\angles{n_1\alpha}-\angles{n_2\alpha}| = |(n_1-n_2)\alpha -(E(n_1\alpha)-E(n_2\alpha))| \leq \frac{1}{m}.\]
    Nótese que $|\angles{n_1\alpha}-\angles{n_2\alpha}| < 1$ porque $\angles{n_1\alpha},\angles{n_2\alpha} \in [0,1)$. Usando las propiedades de la parte fraccionaria mencionadas antes del lema,
    \[|(n_1-n_2)\alpha -(E(n_1\alpha)-E(n_2\alpha))| \overset{(\textup{b})}{=} \angles{(n_1-n_2)\alpha -(E(n_1\alpha)-E(n_2\alpha))}  \overset{(\textup{c})}{=} \angles{(n_1-n_2)\alpha}.\]

    Llamando $k = n_1-n_2 \in \N$ (recordamos que $n_1>n_2$), se ha obtenido que para todo $m \in \N$ con $m>1$ existe $k \in \N$ tal que
    \[\angles{k\alpha} \leq \frac{1}{m}.\]
    
    Ya estamos en condiciones de demostrar \hyperref[2.2.3]{\color{blue}(2.2.3)}. Sea $x \in [0,1)$ y sea $\varepsilon > 0$. Si $x = 0$, tomamos $m \in \N$ con $\frac{1}{m}<\varepsilon$ y por lo probado anteriormente, existe $k \in \N$ tal que $|\angles{k\alpha}-x| = \angles{k\alpha}  \leq \frac{1}{m}< \varepsilon$.
    
    Supongamos entonces que $x > 0$. Sea $m \in \N$ tal que $\frac{1}{m} < \min\{\varepsilon,x\}$. Por lo probado anteriormente, existe $k \in \N$ tal que
    \[\angles{k\alpha} \leq \frac{1}{m}< \varepsilon.\]
    Sea $n$ el mayor número natural tal que $n \angles{k\alpha} \leq x$ (existe porque $\angles{k\alpha} \leq \frac{1}{m} < x$). Entonces $n \angles{k\alpha} \leq x \leq (n+1)\angles{k\alpha}$, y como
    \[(n+1)\angles{k\alpha} - n\angles{k\alpha} = \angles{k\alpha} < \varepsilon,\]
    entonces
    \[|n\angles{k\alpha}-x| < \varepsilon.\]
    Veamos que $n\angles{k\alpha} = \angles{nk\alpha}$, lo que finalizará la prueba de \hyperref[2.2.3]{\color{blue}(2.2.3)}. Por definición de la parte fraccionaria,
    \[n\angles{k\alpha} = nk\alpha-nE( k\alpha ),\]
    luego 
    \[\angles{n\angles{k\alpha}} = \angles{nk\alpha-nE( k\alpha)}.\]
    Se tiene que
    \begin{itemize}
        \item $\angles{n\angles{k\alpha}} = n\angles{k\alpha}$ por la propiedad (b) de la parte fraccionaria, teniendo en cuenta que $0 \leq n\angles{k\alpha} \leq x < 1$.
        \item $\angles{nk\alpha-nE( k\alpha )} = \angles{nk\alpha}$ por la propiedad (c) de la parte fraccionaria.
    \end{itemize}
    Por tanto, $n\angles{k\alpha} = \angles{nk\alpha}$ y concluimos que $|n\angles{k\alpha}-x| = |\angles{nk\alpha}-x|<\varepsilon$.
\end{proof}

Para ahorrar escritura en los próximos resultados, será conveniente emplear la notación siguiente:
\[\mathbb{P} = \{2n \colon n \in \N\}, \qquad \I = \{2n-1 \colon n \in \N\}.\]

\begin{lemma}
    Dado $\alpha \in \R \setminus \Q$, el conjunto $\{\angles{k\alpha} \mid k \in \I\}$ es denso en $[0,1)$.
\end{lemma}

\begin{proof}
    Al igual que en el lema anterior, basta probar el resultado para $\alpha > 0$. Sea $\varepsilon > 0$, sea $x \in [0,1)$ y veamos que existe $k \in \I$ tal que $|\angles{k\alpha}-x|<\varepsilon$. Para todo $n \in \N$,
    \begin{align}\label{2.2.7}
        |\angles{(2n+1)\alpha} - x| &= |\angles{2n\alpha+\alpha}-x| º\overset{(\asts{1})}{=} |\angles{\angles{2n\alpha} + \angles{\alpha}} - x| \notag\\
        &= |\angles{2n\alpha}+\angles{\alpha}-E(\angles{2n\alpha}+\angles{\alpha}) - x| \notag\\
        &\leq |\angles{2n\alpha}-(x-\angles{\alpha})|+|E(\angles{2n\alpha}+\angles{\alpha})|,
    \end{align}
    donde en (\asts{1}) se ha usado la propiedad (d) de la parte fraccionaria. Como $2\alpha \in \R\setminus\Q$, entonces $\{\angles{2n\alpha}\mid n \in \N\}$ es denso en $[0,1]$, y como $x-\angles{\alpha} \in [0,1)$, existe $n_0 \in \N$ tal que
    \[|\angles{2n_0\alpha} - (x-\angles{\alpha})| < \widetilde{\varepsilon},\]
    donde $\widetilde{\varepsilon} = \min\{\varepsilon,1-x\} > 0$. Tenemos entonces $\angles{2n_0\alpha} - (x-\angles{\alpha}) <\widetilde{\varepsilon} \leq 1-x$, luego $0 \leq \angles{2n_0\alpha}+\angles{\alpha} < 1$ y por tanto $E(\angles{2n_0\alpha}+\angles{\alpha}) = 0$. Llevando esto a \hyperref[2.2.7]{\color{blue}(2.2.7)}, concluimos
    \[|\angles{(2n_0+1)\alpha} - x| \leq |\angles{2n_0\alpha}-(x-\angles{\alpha})| < \widetilde{\varepsilon} \leq \varepsilon. \qedhere\]
\end{proof}

\begin{lemma}
    Sea $\rho \in \N$ y sea $\theta \in (0,1)$ con $\theta\neq\frac{1}{2}$. Entonces existe $\rho_\theta \in \I$ con $\rho_\theta\geq \rho$ y tal que
    \[\sen(2\pi\rho_\theta\theta) > \frac{1}{2}.\]
\end{lemma}

\begin{proof}
    Sea $\rho\in \N$ y sea $\theta \in (0,1)$ con $\theta\neq\frac{1}{2}$. Si $s \in \Z$, se tiene que $\sen(x) > \frac{1}{2}$ para todo $x \in (2\pi s+\frac{\pi}{6},2\pi s+\frac{5\pi}{6})$. Basta probar que existe $\rho_\theta \in \I$ con $\rho_\theta \geq \rho$ y tal que para algún $s \in \Z$ se tiene $2\pi\rho_\theta\theta \in (2\pi s +\frac{\pi}{6},2\pi s +\frac{5\pi}{6})$, es decir, $\rho_\theta\theta \in (s+\frac{1}{12},s+\frac{5}{12})$. Sea $J=(\frac{1}{12},\frac{5}{12})$.

    Supongamos primero que $\theta \in \Q$, de manera que existen $p,q \in \N$ con $\textup{mcd}(p,q) = 1$ y tales que $\theta = \frac{p}{q}$. Se distinguen dos casos.
    \begin{itemize}
        \item Supongamos que $q \in \I$. Nótese que $q > 1$ porque $\theta \not\in\N$. Como $\textup{mcd}(2,q) = 1$, entonces $\Z_q = 2\Z_q$, luego
        \[\Z_q = \{\overline{0},\overline{1},\overline{2},\mathellipsis,\overline{q-1}\} = \{\overline{0},\overline{2},\overline{4},\mathellipsis,\overline{2q-2}\}.\]
        Al sumar $\rho$, las clases de equivalencia no cambian:
        \[\Z_q= \{\overline{\rho},\overline{\rho+2},\overline{\rho+4},\mathellipsis,\overline{\rho+2q-2}\}.\]
        Y como $\Z_q = p\Z_q$ por ser $\textup{mcd}(p,q) = 1$,
        \[\Z_q = \{\overline{\rho p},\overline{(\rho+2)p},\mathellipsis,\overline{(\rho+2q-2)p}\}.\]
        Esto permite afirmar que los números 
        \[(\rho+2k)p, \qquad k \in \{0,1,\mathellipsis,q-1\},\]
        son todos distintos módulo $q$, así que al dividirlos entre $q$, los restos obtenidos son $0,1,\mathellipsis,q-1$. Si $q > 3$, como $J$ es un intervalo de longitud $\frac{1}{3}$ y la distancia entre los números $0,\frac{1}{q},\mathellipsis,\frac{q-1}{q}$ es $\frac{1}{q} < \frac{1}{3}$, existe $r \in \{0,1,\mathellipsis,q-1\}$ tal que $\frac{r}{q} \in J$. Para $q = 3$, esto último sigue siendo cierto porque $\frac{1}{3} \in J$. Sea $k \in \{0,1,\mathellipsis,q-1\}$ tal que $r$ es el resto de dividir $(\rho+2k)p$ entre $q$. Entonces existe $s \in \N$ tal que
        \[(\rho+2k)p = sq + r,\]
        es decir,
        \[\frac{(\rho+2k)p}{q} = (\rho+2k)\theta = s + \frac{r}{q}.\]
        Llamando $\rho_\theta = \rho+2k$, se tiene que $\rho_\theta \in \I$, que $\rho_\theta \geq \rho$ y que $\rho_\theta\theta \in (s+\frac{1}{2},s+\frac{5}{12})$.
        \item Supongamos que $q \in \mathbb{P}$, es decir, que existe $q'\in\N$ tal que $q = 2q'$. Nótese que $q > 2$; no puede tenerse $\theta = \frac{p}{2}$ porque $\theta \in (0,1)$ y $\theta \neq \frac{1}{2}$. Además, $p \in \I$; no puede ser par porque $\textup{mcd}(p,q) = 1$. Razonando como antes,
        \begin{align*}
            \Z_q &= \{\overline{0},\overline{1},\overline{2},\mathellipsis,\overline{2q'-1}\} \\
        &= \{\overline{\rho},\overline{\rho+1},\overline{\rho+2},\mathellipsis,\overline{\rho+2q'-1}\} \\
        &= \{\overline{\rho p},\overline{(\rho+1)p},\overline{(\rho+2)p},\mathellipsis,\overline{(\rho+2q'-1)p}\}.
        \end{align*}
        Consideramos los representantes impares de estas clases de equivalencia,
        \[(\rho+2k)p, \qquad k \in \{0,1,\mathellipsis,q'-1\}.\]
        Estos números son todos distintos módulo $q$, y al dividirlos entre $q$ se obtienen todos los restos impares, es decir, $1,3,\mathellipsis,2q'-1$. Si $q > 6$, como $J$ es un intervalo de longitud $\frac{1}{3}$ y la distancia entre los números $\frac{1}{q},\frac{3}{q},\mathellipsis,\frac{2q'-1}{q}$ es $\frac{2}{q} < \frac{2}{6}=\frac{1}{3}$, existe $r \in \{1,3,\mathellipsis,2q'-1\}$ tal que $\frac{r}{q} \in J$. Si $q = 4$ o $q = 6$, esto último sigue siendo cierto porque $\frac{1}{4},\frac{1}{6} \in J$. Sea $k \in \{0,1,\mathellipsis,q'-1\}$ tal que $r$ es el resto de dividir $(\rho+2k)p$ entre $q$. Entonces existe $s \in \N$ tal que
        \[(\rho+2k)p = sq+r,\]
        es decir,
        \[\frac{(\rho+2k)p}{q} = (\rho+2k)\theta = s + \frac{r}{q}.\]
        Llamando $\rho_\theta = \rho+2k$, se tiene que $\rho_\theta \in \I$, que $\rho_\theta \geq \rho$ y que $\rho_\theta\theta \in (s+\frac{1}{2},s+\frac{5}{12})$.
    \end{itemize}

    Supongamos ahora que $\theta \in \R\setminus\Q$. Por el lema anterior, el conjunto $\{\angles{k\theta} \mid k \in \I\}$ es denso en $[0,1)$, y como $J \subset [0,1)$, existen infinitos $k \in \I$ tales que $\angles{k\theta} \in J$. Por tanto, existe $\rho_\theta \in \I$ con $\rho_\theta \geq \rho$ y tal que $\angles{\rho_\theta\theta} \in J$. Llamando $s = E( \rho_\theta\theta)$, concluimos que $s \in \Z$ y que $\rho_\theta\theta = s + \angles{\rho_\theta\theta} \in (s +\frac{1}{12},s +\frac{5}{12})$.
\end{proof}

\begin{lemma}
    Sea $n \in \N$ y sea $\delta \in (0,\frac{\pi}{2n+1})$. Para cada $j \in \{0,1,\mathellipsis,2n\}$, sea
    \[x_j = \frac{2\pi j}{2n+1},\]
    y para cada $j \in \{0,1,\mathellipsis,2n-1\}$, sea \[I_j = (x_j+\delta,x_{j+1}-\delta).\]
    Entonces existen $m_0,m_1,\mathellipsis,m_n \in \N$ con $m_0 < m_1 < \mathellipsis < m_n$ y tales que para todo $j \in \{0,1,\mathellipsis,n-1\}$ y todo $x \in I_{2j} \cup I_{2j+1}$, existe $k_x \in \N$ verificando las siguientes propiedades:
    \begin{enumerate}
        \item $2k_x+1$ es múltiplo de $2n+1$.
        \item $m_{j} \leq k_x < \frac{1}{2}m_{j+1}$.
        \item $\sen((k_x+\frac{1}{2})x) < -\frac{1}{2}$.
    \end{enumerate}
\end{lemma}

\begin{proof}
    En primer lugar, nótese que los intervalos $I_j$ tienen sentido porque
    \[x_{j+1}-\delta -(x_j+\delta) = \frac{2\pi(j+1)}{2n+1}-\frac{2\pi j}{2n+1} -2\delta = \frac{2\pi}{2n+1}-2\delta > \frac{2\pi}{2n+1}-\frac{2\pi}{2n+1} = 0,\]
    utilizándose en la desigualdad que $\delta < \frac{\pi}{2n+1}$.

    La definición de los números naturales $m_0,m_1,\mathellipsis,m_n$ se realizará más adelante. Por ahora, fijemos $j \in \{0,1,\mathellipsis,n-1\}$ y $N \in \N$. Para cada $\rho \in \I$, sea \[k_\rho = \frac{\rho(2n+1)-1}{2},\]
    y para cada $x \in I_{2j} \cup I_{2j+1}$ sea
    \[\theta_x = \frac{2n+1}{4\pi}(x_{2j+2}-x).\]
    Nótese que $\rho(2n+1) \in \I$ por ser producto de números impares, luego $\rho(2n+1)-1 \in \mathbb{P}$ y por tanto $k_\rho \in \N$. Además, $2k_\rho+1=\rho(2n+1)$, luego $2k_\rho+1$ es múltiplo de $2n+1$. Por otra parte,
    \[\sen\Bigl(\Bigl(k_\rho+\frac{1}{2}\Bigr)x\Bigr) = \sen\Bigl(\Bigl(\frac{\rho(2n+1)-1}{2}+\frac{1}{2}\Bigr)x\Bigr) = \sen\Bigl(\frac{\rho(2n+1)}{2}x\Bigr).\]
    Como $(2n+1)x_{2j+2} = 2\pi(2j+2)$ y $\frac{\rho}{2} \in \N$, entonces $\frac{\rho(2n+1)}{2}x_{2j+2}$ es un múltiplo de $2\pi$, y por tanto, usando que el seno es una función impar y $2\pi$-periódica,
    \begin{align}\label{2.2.5}
        -\sen\Bigl(\Bigl(k_\rho+\frac{1}{2}\Bigr)x\Bigr) &= \sen\Bigl(-\frac{\rho(2n+1)}{2}x\Bigr) = \sen\Bigl(\frac{\rho(2n+1)}{2}(x_{2j+2}-x)\Bigr) \notag \\ 
        &= \sen(2\pi \rho\theta_x).
    \end{align}
    
    Acotemos $\theta_x$ usando que $x \in I_{2j}\cup I_{2j+1}= (x_{2j}+\delta,x_{2j+1}-\delta) \cup (x_{2j+1}+\delta,x_{2j+2}-\delta)$. Sea $\eta = \frac{2n+1}{4\pi}\delta$. Nótese que $\eta > 0$ y que, por ser $\delta < \frac{\pi}{2n+1}$, se tiene que $\eta < \frac{1}{4}$.
    \begin{itemize}
        \item Como $x > x_{2j}+\delta$, entonces
        \[x_{2j+2}-x < x_{2j+2}-x_{2j}-\delta = \frac{2\pi}{2n+1}(2j+2-2j)-\delta = \frac{4\pi}{2n+1}-\delta,\]
        luego 
        \[\theta_x = \frac{2n+1}{4\pi}(x_{2j+2}-x) < 1-\frac{2n+1}{4\pi}\delta = 1-\eta.\]
        \item Como $x < x_{2j+2}-\delta$, entonces
        \[x_{2j+2}-x > x_{2j+2}-x_{2j+2}+\delta = \delta,\]
        luego
        \[\theta_x = \frac{2n+1}{4\pi}(x_{2j+2}-x) > \frac{2n+1}{4\pi}\delta = \eta.\]
        \item Se da una de las dos situaciones siguientes: $x < x_{2j+1}-\delta$ o $x > x_{2j+1}+\delta$. En el primer caso, se tendría
        \[x_{2j+2}-x > x_{2j+2}-x_{2j+1}+\delta = \frac{2\pi}{2n+1}(2j+2-2j-1)+\delta = \frac{2\pi}{2n+1}+\delta,\]
        luego
        \[\theta_x = \frac{2n+1}{4\pi}(x_{2j+2}-x) > \frac{1}{2}+\frac{2n+1}{4\pi}\delta = \frac{1}{2}+\eta.\]
        En el segundo caso, se tendría
        \[x_{2j+2}-x < x_{2j+2}-x_{2j+1}-\delta = \frac{2\pi}{2n+1}(2j+2-2j-1)-\delta = \frac{2\pi}{2n+1}- \delta,\]
        luego
        \[\theta_x = \frac{2n+1}{4\pi}(x_{2j+2}-x) < \frac{1}{2}-\frac{2n+1}{4\pi}\delta = \frac{1}{2}-\eta.\]
    \end{itemize}

    Sea $S = (\eta,\frac{1}{2}-\eta) \cup (\frac{1}{2}+\eta,1-\eta)$, que por ser $\eta \in (0,\frac{1}{4})$ verifica $S \subset (0,1)$ y $\frac{1}{2}\not\in S$. Los razonamientos anteriores prueban que
    \[x \in I_{2j} \cup I_{2j+1} \iff \theta_x \in S.\]
    Para cada $\theta \in S$, sea $\rho_\theta$ el menor número natural impar con $\rho_\theta \geq N$ y tal que
    \[\sen(2\pi\rho_\theta\theta) > \frac{1}{2}.\]
    Este $\rho_\theta$ existe por el lema anterior. De hecho, como $\overline{S} = [\eta,\frac{1}{2}-\eta] \cup [\frac{1}{2}+\eta,1-\eta] \subset (0,1)$ y $\frac{1}{2}\not\in\overline{S}$, esto también tiene sentido para $\theta \in \overline{S}$. Por \hyperref[2.2.5]{\color{blue}(2.2.5)}, para todo $x \in I_{2j} \cup I_{2j+1}$ se tiene
    \[-\sen\Bigl(\Bigl(k_{\rho_{\theta_x}}+\frac{1}{2}\Bigr)x\Bigr) > \frac{1}{2}.\]
    Para cada $x \in I_{2j} \cup I_{2j+1}$, sea
    \[k_x = k_{\rho_{\theta_x}} = \frac{\rho_{\theta_x}(2n+1)-1}{2}.\] Como $\rho_{\theta_x} \geq N$, entonces $k_x \geq \frac{N(2n+1)-1}{2}$. Además, $k_x$ verifica (a) y (c). 

    Recapitulando, hemos probado que si $j \in \{0,1,\mathellipsis,n-1\}$ y $N \in \N$, entonces para todo $x \in I_{2j}\cup I_{2j+1}$ existe $k_x \in \N$ con $k_x \geq \frac{N(2n+1)-1}{2}$ y tal que se verifican (a) y (c).
    
    Finalmente, definamos $m_0,m_1,\mathellipsis,m_n\in \N$. Tomemos como $m_0$ cualquier número natural y definamos $m_1$. Aplicando lo que se acaba de probar con $j = 0$ y cualquier $N \in \N$ verificando $N \geq \frac{2m_0+1}{2n+1}$, obtenemos que para todo $x \in I_0 \cup I_1$ existe $k_x \in \N$ con $k_x \geq \frac{N(2n+1)-1}{2} \geq \frac{2m_0+1-1}{2}= m_0$ y verificando (a) y (c). Veamos que existe $m_1 \in \N$ con $m_0 < m_1$ y tal que para todo $x \in I_0\cup I_1$ se tiene que $k_x < \frac{1}{2} m_1$.

    Sea $\theta \in \overline{S}$. Como $\sen(2\pi\rho_\theta\theta) > \frac{1}{2}$ y el seno es una función continua, existe $\delta_\theta > 0$ tal que para todo ${\xi} \in (\theta-\delta_\theta,\theta+\delta_\theta)$ se tiene que $\sen(2\pi\rho_{\theta}{\xi}) > \frac{1}{2}$. Como $\rho_\theta \in \I$, $\rho_\theta \geq N$ y $\sen(2\pi\rho_\theta\xi)> \frac{1}{2}$, entonces $\rho_\xi \leq \rho_\theta$, ya que $\rho_\xi$ es el menor número natural impar con $\rho_\xi \geq N$ y $\sen(2\pi \rho_\xi\xi) > \frac{1}{2}$. Y como
    \[\overline{S} \subset \bigcup_{\theta \in \overline{S}}(\theta - \delta_\theta, \theta + \delta_\theta)\]
    y $\overline{S}$ es compacto, existen $\theta_1,\theta_2,\mathellipsis,\theta_k \in \overline{S}$ tales que
    \[S \subset \overline{S} \subset \bigcup_{i=1}^k (\theta_i-\delta_{\theta_i},\theta_i+\delta_{\theta_i}).\]
    Por tanto, para todo $\theta \in S$ se tiene que $\rho_\theta \leq \max\{\rho_{\theta_1},\rho_{\theta_2},\mathellipsis,\rho_{\theta_n}\}$. Tomando
    \[m_0' > \max \{\rho_{\theta_1},\rho_{\theta_2},\mathellipsis,\rho_{\theta_n}\},\]
    tenemos que $\rho_\theta < m_0'$ para todo $\theta \in S$. Así,, para todo $x \in I_0 \cup I_1$ se tiene
    \[k_x = \frac{\rho_{\theta_x}(2n+1)-1}{2} < \frac{m_0'(2n+1)-1}{2}.\]
    Llamando $m_1 = m_0'(2n+1)-1$, tenemos que $m_0 \leq k_x < \frac{1}{2}m_1$ para todo $x \in I_0 \cup I_1$. Además, $m_0 < \frac{1}{2}m_1 < m_1$.

    Razonando de forma análoga, hallamos $m_2 \in \N$ con $m_0 < m_1 < m_2$ y tal que para todo $x \in I_{2} \cup I_3$ existe $k_x \in \N$ satisfaciendo (a) y (c) y tal que $m_1 \leq k_x < \frac{1}{2}m_2$. 
    
    Reiterando este proceso, encontramos $m_0,m_1,\mathellipsis,m_n\in\N$ con $m_0 < m_1 < \mathellipsis < m_n$ y tales que para todo $j \in \{0,1,\mathellipsis,n-1\}$ y todo $x \in I_{2j}\cup I_{2j+1}$, existe $k_x \in \N$ cumpliendo las propiedades (a), (b) y (c).
\end{proof}

\begin{lemma}\label{2.2.8}
    Para cada $n \in \N$, sea $H_n = \sum_{i=1}^n \frac{1}{i}$. Entonces
    \[\lim_{n \to \infty} \frac{H_n}{\log(n)} = 1.\]
\end{lemma}

\begin{proof}
    La sucesión $\{\log(n)\}_{n=1}^\infty$ es estrictamente creciente y $\lim_{n\to\infty} \log(n) = \infty$. Como además
    \begin{align*}
        \lim_{n\to\infty}\frac{H_{n+1}-H_n}{\log(n+1)-\log(n)} &= \lim_{n\to\infty} \frac{\frac{1}{n+1}}{\log(\frac{n+1}{n})} = \lim_{n\to\infty}\frac{1}{n\log(1+\frac{1}{n})+\log(1+\frac{1}{n})} 
        \\ &= \lim_{n\to\infty}\frac{1}{\log((1+\frac{1}{n})^n)+\log(1+\frac{1}{n})} = \frac{1}{\log(e)+\log(1)} = 1,
    \end{align*}
    el criterio de Stolz permite afirmar que
    \[\lim_{n\to\infty}\frac{H_n}{\log(n)} = \lim_{n\to\infty}\frac{H_{n+1}-H_n}{\log(n+1)-\log(n)} = 1.\qedhere\]
\end{proof}

\begin{lemma}
    Existen una sucesión de polinomios trigonométricos $\{F_n\}_{n=n_0}^\infty$, una sucesión de números reales positivos $\{A_n\}_{n=n_0}^\infty$, una sucesión de intervalos $\{E_n\}_{n=n_0}^\infty$ y una sucesión de números naturales $\{\lambda_n\}_{n=n_0}^\infty$ verificando las siguientes propiedades:
    \begin{enumerate}
        \item Para cada $n \geq n_0$, $F_n$ es un polinomio trigonométrico no negativo de la forma
        \[F_n(t)=1+\sum_{j=1}^{\nu_n}(a_j(n)\cos(jt)+b_j(n)\sen(jt)), \qquad t \in \R.\]
        \item $\lim_{n \to \infty} A_n = \infty$.
        \item $E_n \subset [0,2\pi]$ para cada $n \geq n_0$.
        \item $E_n \subset E_{n+1}$ para cada $n \geq n_0$.
        \item $\bigcup_{n=n_0}^\infty E_n = [0,2\pi)$.
        \item $\lim_{n\to\infty} \lambda_n = \infty$.
        \item Para cada $n \geq n_0$ y cada $x \in E_n$, existe $k \in \N$ con $S_kF_n(x) > A_n$ y tal que $\lambda_n \leq k \leq \nu_n$, donde $\nu_n$ es el grado del polinomio trigonométrico $F_n$.
    \end{enumerate}
\end{lemma}

\begin{proof}
    Vamos a definir $F_n$, $A_n$, $E_n$ y $\lambda_n$ para $n$ tan grande como se necesite, de ahí que las sucesiones del enunciado comiencen en $n_0\in\N$.
    
    Sea $n \in \N$ suficientemente grande y sea $\delta > 0$ suficientemente pequeño (se concretará más adelante). Para cada $j \in \{0,1,\mathellipsis,2n\}$, sean
    \[x_j = \frac{2\pi j}{2n+1}, \qquad I_j' = [x_j-\delta,x_j+\delta],\]
    y para cada $j \in \{0,1,\mathellipsis,2n-1\}$, sea
    \[I_j = (x_j+\delta, x_{j+1}-\delta).\]
    Para que el intervalo $I_j$ tenga sentido, tomamos $\delta < \frac{\pi}{2n+1}$, tal y como se vio en el \hyperref[2.2.4]{\color{blue}Lema 2.2.4}. Se observa que
    \[\Bigl(\bigcup_{j=0}^{2n}I_j'\Bigr) \cup\Bigl(\bigcup_{j=0}^{2n-1}I_j\Bigr) = [x_0-\delta,x_{2n}+\delta].\]
    Definimos
    \[F_n = \phi_n+f_n,\]
    donde:
    \begin{itemize}
        \item $\phi_n(x) = \frac{1}{2}K_{m_0}((2n+1)x)$, siendo $m_0 \in \N$ tal que $K_{m_0}(0) > 2n$ (se recuerda que $K_{m_0}$ es el \hyperref[1.2.5]{\color{blue}núcleo de Fejér de orden $m_0$}). Esta elección de $m_0$ es posible porque \[\lim_{m \to \infty} K_m(0) = \lim_{m \to \infty} (m+1) = \infty.\]
        Nótese que $K_{m_0}(0)=2m_0+1 > 2n$ implica $m_0 > \frac{2n-1}{2}$. 
        
        Si $j \in \{0,1,\mathellipsis,2n\}$, entonces
        \[\phi_n(x_j) = \frac{1}{2}K_{m_0}((2n+1)x_j) = \frac{1}{2}K_{m_0}(2\pi j) = \frac{1}{2}K_{m_0}(0) > \frac{1}{2}2n = n,\] 
        utilizándose en la tercera igualdad que $K_{m_0}$ es $2\pi$-periódica. Como $\phi_n$ es continua (pues $K_{m_0}$ lo es), existe $\delta' > 0$ tal que para todo $x \in (x_j-\delta',x_j+\delta')$ se verifica $\phi_n(x) \geq n$. Tomando $\delta < \min\{\delta',\frac{\pi}{2n+1}\}$, se tiene que
        \begin{equation}\label{2.2.10}
            \phi_n(x) \geq n \textup{ para todo } x \in [x_j-\delta,x_j+\delta] = I_j'.
        \end{equation}
        Por ser $K_{m_0}$ un polinomio trigonométrico no negativo de término constante $1$ y grado $m_0$, se tiene que $\phi_n$ es un polinomio trigonométrico no negativo de término constante $\frac{1}{2}$ y grado $m_0$. 
        
        Por otra parte, como $D_{m_0}(0) = 2m_0+1 > 0$ y $D_{m_0}$ es continua, existe $\delta'' > 0$ tal que $D_{m_0}(x) \geq 0$ para todo $x \in (-\delta'',\delta'')$. Tomando $\delta < \min\{\delta',\delta'',\frac{\pi}{2n+1}\}$, se tiene que
        \begin{equation}\label{2.2.9}
            D_{m_0}(x) \geq 0 \textup{ para todo } x \in [-\delta,\delta] = I_0'.
        \end{equation}
        \item $f_n(x) = \frac{1}{2n}\sum_{i=1}^n g_i(x)$, donde $g_i(x)= K_{m_i}(x-x_{2i})$ y $m_1,m_2,\mathellipsis,m_n$ son los números naturales proporcionados por el \hyperref[2.2.6]{\color{blue}Lema 2.2.6} (en la demostración de dicho lema se vio que puede tomarse como $m_0$ cualquier número natural; en este caso, se escoge como $m_0$ el grado de $\phi_n$). Como $g_i$ es un polinomio trigonométrico no negativo de término constante 1, entonces $f_n$ es un polinomio trigonométrico no negativo de término constante $\frac{1}{2}$.
    \end{itemize}

    De esta manera, tenemos que $F_n$ es un polinomio trigonométrico no negativo de término constante 1, pues $f_n$ y $\phi_n$ son polinomios trigonométricos no negativos de término constante $\frac{1}{2}$.

    Antes de definir las otras tres sucesiones que aparecen en el enunciado, se van a probar las tres afirmaciones que siguen:

    \begin{itemize}
        \item \textit{Si $j \in \{0,1,\mathellipsis,n-1\}$ y $k \in \N$ son tales que $m_j \leq k < \frac{1}{2}m_{j+1}$, entonces
        \begin{equation}\label{2.2.12}
            S_kf_n(x) \geq \frac{1}{2n}\sum_{i=j+1}^n \frac{m_i-k}{m_i+1}D_k(x-x_{2i})
        \end{equation}
        para todo $x \in \bigcup_{l=0}^{n-1}I_{2l}\cup I_{2l+1}$.}

        Sea $x \in \bigcup_{l=0}^{n-1}I_{2l}\cup I_{2l+1}$ y supongamos que existen $j \in \{0,1,\mathellipsis,n-1\}$ y $k \in \N$ tales que $m_j \leq k < \frac{1}{2}m_{j+1}$. Para todo $i \in \{1,2,\mathellipsis,n\}$,
        \begin{align*}
            g_i(x) &= K_{m_i}(x-x_{2i}) = \frac{1}{m_i+1}\sum_{l=0}^{m_i}D_l(x-x_{2i}) = \frac{1}{m_i+1}\Bigl(1+\sum_{l=1}^{m_i}D_l(x-x_{2i})\Bigr)  \\
            &= \frac{1}{m_i+1}\Bigl(1+\sum_{l=1}^{m_i}\Bigl(1+2\sum_{m=1}^l \cos(m(x-x_{2i}))\Bigr)\Bigr) \\
            &= \frac{1}{m_i+1}\Bigl(1+m_i+2\sum_{l=1}^{m_i}\sum_{m=1}^l \cos(m(x-x_{2i}))\Bigr) \\
            &= \frac{1}{m_i+1}\Bigl(1+m_i+2\sum_{m=1}^{m_i}(m_i-m+1)\cos(m(x-x_{2i}))\Bigr) \\
            &= 1+2\sum_{m=1}^{m_i}\frac{m_i-m+1}{m_i+1}\cos(m(x-x_{2i})).
        \end{align*}
        Si $j+1 \leq i \leq n$, entonces $k < \frac{1}{2}m_{j+1} \leq m_{j+1} \leq m_i$, y como $g_i$ es un polinomio trigonométrico de grado $m_i$,
        \[S_{k}g_i(x) = 1+2\sum_{m=1}^{k}\frac{m_i-m+1}{m_i+1}\cos(m(x-x_{2i})),\]
        mientras que si $i \leq j$, entonces $m_i \leq m_{j} \leq k$ y $S_{k}g_i(x) = g_i(x) = K_{m_i}(x-x_{2i})$. De todo esto se obtiene que
        \begin{align*}
            S_{k}f_n(x) &= \frac{1}{2n}\sum_{i=1}^n S_{k}g_i(x) =  \frac{1}{2n}\sum_{i=1}^j S_{k}g_i(x)+ \frac{1}{2n}\sum_{i=j+1}^n S_{k}g_i(x)\\
            &= \frac{1}{2n}\sum_{i=1}^jK_{m_i}(x-x_{2i})+\frac{1}{2n}\sum_{i=j+1}^n \Bigl(1+2\sum_{m=1}^{k}\frac{m_i-m+1}{m_i+1}\cos(m(x-x_{2i}))\Bigr).
        \end{align*}
        Como los núcleos de Fejér son no negativos,
        \begin{align*}
            S_{k}f_n(x) \geq \frac{1}{2n}\sum_{i=j+1}^n \Bigl(1+2\sum_{m=1}^{k}\frac{m_i-m+1}{m_i+1}\cos(m(x-x_{2i}))\Bigr).
        \end{align*}
        Y como $m_i-m+1 = (m_i-k)+(k-m+1) \geq m_i-k$,
        \begin{align*}
            S_{k}f_n(x) &\geq \frac{1}{2n}\sum_{i=j+1}^n \Bigl(1+2\sum_{m=1}^{k}\frac{m_i-k}{m_i+1}\cos(m(x-x_{2i}))\Bigr).
        \end{align*}
        Usando ahora que $1 \geq \frac{m_i-k}{m_i+1}$ (pues $-k \leq 1$),
        \begin{align*}
            S_{k}f_n(x) &\geq \frac{1}{2n}\sum_{i=j+1}^n \Bigl(\frac{m_i-k}{m_i+1}+2\sum_{m=1}^{k}\frac{m_i-k}{m_i+1}\cos(m(x-x_{2i}))\Bigr) \\
            &= \frac{1}{2n}\sum_{i=j+1}^n \frac{m_i-k}{m_i+1}\Bigl(1+2\sum_{m=1}^{k} \cos(m(x-x_{2i}))\Bigr) \\
            &= \frac{1}{2n}\sum_{i=j+1}^n \frac{m_i-k}{m_i+1}D_{k}(x-x_{2i}).
        \end{align*}
        Esto completa la prueba de \hyperref[2.2.12]{\color{blue}(2.2.12)}.
        \item \textit{Existe $C > 0$ tal que para todo $j \in \{0,1,\mathellipsis,E(n-\sqrt{n})\}$ y todo $x \in I_{2j}\cup I_{2j+1}$, se verifica
        \begin{equation}\label{2.2.11}
            S_{k_x}F_n(x) \geq C\log(n)
        \end{equation}
        para algún $k_x \in \N$ con $m_j \leq k_x < \frac{1}{2}m_{j+1}$.
        }

        Sea $j \in \{0,1,\mathellipsis,E(n-\sqrt{n})\}$ y sea $x \in I_{2j} \cup I_{2j+1}$. El \hyperref[2.2.6]{\color{blue}Lema 2.2.6} permite afirmar que existe $k_x \in \N$ tal que $2k_x+1$ es múltiplo de $2n+1$, $m_{j} \leq k_x < \frac{1}{2}m_{j+1}$ y $\sen((k_x+\frac{1}{2})x) < -\frac{1}{2}$.

        Sea $\alpha \in \N$ tal que $2k_x+1 = \alpha(2n+1)$. Como $x \in I_{2j} \cup I_{2j+1}$, entonces $x < x_{2j+2}$. De esto se obtiene que si $i \geq j+1$, entonces $x \neq x_{2i}$, luego
        \begin{align*}
            D_{k_x}(x-x_{2i}) &= \frac{\sen((k_x+\frac{1}{2})(x-x_{2i}))}{\sen(\frac{1}{2}(x-x_{2i}))} = \frac{\sen((k_x+\frac{1}{2})x-(k_x+\frac{1}{2})x_{2i})}{\sen(\frac{1}{2}(x-x_{2i}))} \\
            &= \frac{\sen((k_x+\frac{1}{2})x-\frac{\alpha}{2}(2n+1)x_{2i})}{\sen(\frac{1}{2}(x-x_{2i}))} = \frac{\sen((k_x+\frac{1}{2})x-\frac{\alpha}{2}(2n+1)\frac{4\pi i}{2n+1})}{\sen(\frac{1}{2}(x-x_{2i}))} \\
            &= \frac{\sen((k_x+\frac{1}{2})x-2\pi\alpha i)}{\sen(\frac{1}{2}(x-x_{2i}))} = \frac{\sen((k_x+\frac{1}{2})x)}{\sen(\frac{1}{2}(x-x_{2i}))}.
        \end{align*}
        Usando esto y \hyperref[2.2.12]{\color{blue}(2.2.12)},
        \begin{align*}
            S_{k_x}f_n(x) &\geq \frac{1}{2n}\sum_{i=j+1}^n \frac{m_i-k_x}{m_i+1}D_{k_x}(x-x_{2i}) \\
            &= \frac{\sen((k_x+\frac{1}{2})x)}{2n}\sum_{i=j+1}^n \frac{m_i-k_x}{m_i+1}\frac{1}{\sen(\frac{1}{2}(x-x_{2i}))},
        \end{align*}
        y usando que el seno es una función impar,
        \[S_{k_x}f_n(x) \geq -\frac{\sen((k_x+\frac{1}{2})x)}{2n}\sum_{i=j+1}^n \frac{m_i-k_x}{m_i+1}\frac{1}{\sen(\frac{1}{2}(x_{2i}-x))}.\]
        Como $\sen((k_x+\frac{1}{2})x) < -\frac{1}{2}$,
        \begin{align*}
            S_{k_x}f_n(x) &\geq \frac{1}{4n}\sum_{i=j+1}^n \frac{m_i-k_x}{m_i+1}\frac{1}{\sen(\frac{1}{2}(x_{2i}-x))}.
        \end{align*}
       Si $j+1 \leq i \leq n$, entonces $k_x < \frac{1}{2}m_{j+1} \leq \frac{1}{2}m_i$, luego $\frac{k_x}{m_i} \leq \frac{1}{2}$ y por tanto
            \[\frac{m_i-k_x}{m_i+1} \geq \frac{m_i-k_x}{m_i+m_i} = \frac{1}{2}-\frac{k_x}{2m_i} > \frac{1}{2}-\frac{1}{4} = \frac{1}{4},\]
        así que
        \[S_{k_x}f_n(x) \geq \frac{1}{4n}\sum_{i=j+1}^n \frac{1}{4}\frac{1}{\sen(\frac{1}{2}(x_{2i}-x))}.\]
        Como $x \in I_{2j} \cup I_{2j+1} \subset (x_{2j}+\delta,x_{2j+2}-\delta)$ y $x_{2j+2}\leq x_{2i}$ siempre que $j+1 \leq i \leq n$, entonces $x < x_{2j+2} \leq x_{2i}$, luego
        \[\sen\Bigl(\frac{1}{2}(x_{2i}-x)\Bigr) \leq \Bigl|\sen\Bigl(\frac{1}{2}(x_{2i}-x)\Bigr)\Bigr| \leq \frac{1}{2}|x_{2i}-x| = \frac{1}{2}(x_{2i}-x).\]
        Llevando esto a la desigualdad anterior,
        \[S_{k_x}f_n(x) \geq \frac{1}{4n}\sum_{i=j+1}^n\frac{1}{4} \frac{1}{\frac{1}{2}(x_{2i}-x)} = \frac{1}{16n}\sum_{i=j+1}^n \frac{1}{x_{2i}-x}.\]
        Usando de nuevo que $x \in (x_{2j}+\delta,x_{2j+2}-\delta)$,
        \[x_{2i}-x < x_{2i}-x_{2j} = \frac{4\pi i}{2n+1}-\frac{4\pi j}{2n+1} = \frac{4\pi}{2n+1}(i-j),\]
        y por tanto
        \[\begin{aligned}[t]
            S_{k_x}f_n(x) &\geq \frac{2n+1}{4\pi} \frac{1}{16n}\sum_{i=j+1}^n \frac{1}{i-j} = \frac{2n+1}{64\pi n}\sum_{i=1}^{n-j} \frac{1}{i} =  \frac{2n+1}{64\pi n}H_{n-j} \\ &= \frac{2n+1}{64\pi n}\log(n-j)\frac{H_{n-j}}{\log(n-j)}.
        \end{aligned}\]
        Como $j \leq n-\sqrt{n}$, entonces $n-j \geq \sqrt{n}$ y por tanto $\lim_{n\to\infty}(n-j) = \infty$. Usando esto y el \hyperref[2.2.8]{\color{blue}Lema 2.2.8}, 
        \[\lim_{n\to\infty} \frac{2n+1}{64\pi n}\frac{H_{n-j}}{\log(n-j)} = \frac{1}{32\pi} > 0,\]
        luego existe $C>0$ tal que $\frac{2n+1}{64\pi n}\frac{H_{n-j}}{\log(n-j)} > 2C$ para $n$ suficientemente grande. En consecuencia,
        \[S_{k_x}f_n(x)  \geq 2C\log(n-j)\]
        Como $n-j \geq \sqrt{n}$ y el logaritmo natural es estrictamente creciente,
        \[S_{k_x}f_n(x)  \geq 2C\log(\sqrt{n}) = C\log(n).\]
        Por tanto,
        \[
            S_{k_x}F_n(x) = S_{k_x}f_n(x) + S_{k_x}\phi_n(x) = S_{k_x}f_n(x) + \phi_n(x)\geq S_{k_x}f_n(x) \geq C\log(n),
        \]
        utilizándose que $\phi_n$ es un polinomio trigonométrico no negativo de grado $m_0 \leq k_x$. Con esto queda probado \hyperref[2.2.11]{\color{blue}(2.2.11)}.
        \item \textit{Si $n$ es suficientemente grande, para todo $x \in \bigcup_{j=0}^{2n}I_j'$ se verifica \begin{equation}\label{2.2.14}
            S_{m_0}F_n(x) > \frac{n}{2}.
        \end{equation}
        }

        Sea $x \in \bigcup_{j=0}^{2n}I_j'$, y sea $l \in \{0,1,\mathellipsis,2n\}$ tal que $x \in I_l'$. Veamos primero que para todo $i \in \{1,2,\mathellipsis,n\}$ se tiene
        \[|x-x_{2i}| \geq \frac{\pi |l -2i|}{2n+1}.\]
        Si $l -2i = 0$, la desigualdad se verifica trivialmente. Si $l - 2i > 0$, entonces $l - 2i \geq 1$, luego
        \begin{align*}
            |x-x_{2i}| &\geq x-x_{2i} > x_l-\delta-x_{2i} = \frac{2\pi l}{2n+1}-\delta-\frac{4\pi i}{2n+1} \\ &> \frac{2\pi l}{2n+1}-\frac{\pi}{2n+1}-\frac{4\pi i}{2n+1}
            = \frac{\pi(2l-1-4i)}{2n+1} \geq \frac{\pi(l-2i)}{2n+1} = \frac{\pi|l-2i|}{2n+1}.
        \end{align*}
        Si $l - 2i < 0$, entonces $2i - l \geq 1$, luego
        \begin{align*}
            |x-x_{2i}| &\geq x_{2i}-x > x_{2i} - (x_l+\delta) = \frac{4\pi i}{2n+1} - \frac{2\pi l}{2n+1} - \delta \\ &>\frac{4\pi i}{2n+1} - \frac{2\pi l}{2n+1} -\frac{\pi}{2n+1}
            = \frac{\pi(4i-2l-1)}{2n+1} \geq \frac{\pi(2i-l)}{2n+1} = \frac{\pi|l-2i|}{2n+1}.
        \end{align*}
        En cualquier caso,
        \[|x-x_{2i}| \geq \frac{\pi |l -2i|}{2n+1}.\]
        Poniendo $k = m_0$ y $j = 0$ en la desigualdad \hyperref[2.2.12]{\color{blue}(2.2.12)} (es claro que $m_j \leq k < m_{j+1}$ y por tanto dicha desigualdad es válida), tenemos
        \[S_{m_0}f_n(x) \geq \frac{1}{2n}\sum_{i=1}^n \frac{m_i - m_0}{m_i + 1}D_{m_0}(x-x_{2i}).\]
        Si $l$ es impar, entonces $x-x_{2i} \neq 0$ (pues $|x-x_{2i}| \geq \frac{\pi |l-2i|}{2n+1} > 0$), luego
        \begin{align*}
            S_{m_0}f_n(x) &\geq \frac{1}{2n}\sum_{i=1}^n \frac{m_i-m_0}{m_i+1}\frac{\sen((m_0+\frac{1}{2})(x-x_{2i}))}{\sen(\frac{1}{2}(x-x_{2i}))}.
        \end{align*}
        Usando que $\sen(\alpha) \geq -1$ y $\sen(\alpha)\leq |\alpha|$ para todo $\alpha \in \R$,
        \begin{align*}
            S_{m_0}f_n(x) &\geq -\frac{1}{2n}\sum_{i=1}^n \frac{m_i-m_0}{m_i+1}\frac{1}{\frac{1}{2}|x-x_{2i}|} = -\frac{1}{2n}\sum_{i=1}^n \frac{m_i-m_0}{m_i+1}\frac{2}{|x-x_{2i}|} .
        \end{align*}
        Para todo $i \in \{1,2,\mathellipsis,n\}$ se tiene $m_0 \leq \frac{1}{2}m_i$, luego \[\frac{m_i-m_0}{m_i+1} \geq \frac{m_i-m_0}{2m_i} = \frac{1}{2}-\frac{m_0}{2m_i} \geq \frac{1}{2}-\frac{1}{4} = \frac{1}{4},\]
        así que
        \begin{align*}
            S_{m_0}f_n(x) &\geq -\frac{1}{2n}\sum_{i=1}^n \frac{1}{4}\frac{2}{|x-x_{2i}|} = -\frac{1}{4n}\sum_{i=1}^n \frac{1}{|x-x_{2i}|}.
        \end{align*}
        Como se ha probado que $|x-x_{2i}| \geq \frac{\pi |l -2i|}{2n+1}$, entonces $-\frac{1}{|x-x_{2i}|} \geq -\frac{2n+1}{\pi|l-2i|}$, luego
        \begin{align*}
            S_{m_0}f_n(x) &\geq -\frac{2n+1}{4\pi n}\sum_{i=1}^n \frac{1}{|l-2i|}.
        \end{align*}
        Si $i \in \{1,2,\mathellipsis,n\}$, como $l \neq 2i$, se tiene que $l - 2i > 0$ si y solo si $i \leq \frac{l-1}{2}$, mientras que $l - 2i < 0$ si y solo si $i \geq \frac{l+1}{2}$. Por tanto,
        \begin{align*}
            S_{m_0}f_n(x) &\geq -\frac{2n+1}{4\pi n}\left(\sum_{i=1}^{(l-1)/2} \frac{1}{l-2i} +\sum_{i=(l+1)/2}^n \frac{1}{2i-l}\right).
        \end{align*}
        Acotemos cada una de las sumas. Por un lado,
        \begin{align*}
            \sum_{i=1}^{(l-1)/2}\frac{1}{l-2i} &= \sum_{i=1}^{(l-1)/2} \frac{1}{l-i-i} \leq \sum_{i=1}^{(l-1)/2} \frac{1}{l-i-\frac{l-1}{2}} = \sum_{i=1}^{(l-1)/2} \frac{1}{\frac{l+1}{2}-i} = \sum_{k=1}^{(l-1)/2} \frac{1}{k} \\ &\leq \sum_{k=1}^{n}\frac{1}{k} = H_n,
        \end{align*}
        utilizándose en la última desigualdad que $\frac{l-1}{2} \leq n$ por ser $l < 2n$. Por otro lado,
        \begin{align*}
            \sum_{i=(l+1)/2}^n \frac{1}{2i-l} &= \sum_{i=(l+1)/2}^n \frac{1}{i+i-l} \leq \sum_{i=(l+1)/2}^n \frac{1}{\frac{l+1}{2}+i-l} = \sum_{i=(l+1)/2}^n \frac{1}{i-\frac{l-1}{2}} \\ &= \sum_{k=1}^{n-(l-1)/2}\frac{1}{k} \leq \sum_{k=1}^n \frac{1}{k} = H_n.
        \end{align*}
        En consecuencia,
        \begin{align*}
            S_{m_0}f_n(x) &\geq -\frac{2n+1}{4\pi n}(H_n+H_n) = -\frac{2n+1}{2\pi n}H_n =-\frac{2n+1}{2\pi n}\log(n)\frac{H_n}{\log(n)} .
        \end{align*}
        Usando de nuevo el \hyperref[2.2.6]{\color{blue}Lema 2.2.6},
        \[\lim_{n\to\infty} \frac{2n+1}{2\pi n}\frac{H_n}{\log(n)} = \frac{1}{\pi} > 0,\]
        luego existe una constante $C'>0$ de manera que para $n$ suficientemente grande, se verifica $C' > \frac{2n+1}{2\pi n}\frac{H_n}{\log(n)}$ y por tanto
        \[S_{m_0}f_n(x) \geq -C'\log(n).\]
    
        Supongamos ahora que $l$ es par, es decir, que existe $i_0 \in \{0,1,\mathellipsis,n\}$ tal que $l = 2i_0$. En tal caso, volviendo a usar \hyperref[2.2.12]{\color{blue}(2.2.12)} con $k = m_0$ y $j = 0$,
        \begin{align*}
            S_{m_0}f_n(x) &\geq \frac{1}{2n}\sum_{i=1}^n \frac{m_i-m_0}{m_i+1}D_{m_0}(x-x_{2i}) \\
            &=\frac{1}{2n}\sum_{\substack{1 \leq i \leq n \\ i \neq i_0}} \frac{m_i-m_0}{m_i+1}D_{m_0}(x-x_{2i})+\frac{1}{2n}\frac{m_{i_0}-m_0}{m_{i_0}+1}D_{m_0}(x-x_l).
        \end{align*}
        Utilizando \hyperref[2.2.9]{\color{blue}(2.2.9)} y que $x-x_l \in I_0'$ por ser $x \in I_l'$, se obtiene $D_{m_0}(x-x_l) \geq 0$, y en consecuencia,
        \[S_{m_0}f_n(x) \geq \frac{1}{2n}\sum_{\substack{1 \leq i \leq n \\ i \neq i_0}} \frac{m_i-m_0}{m_i+1}D_{m_0}(x-x_{2i}).\]
        Como $x-x_{2i} \neq 0$ para todo $i \in \{1,2,\mathellipsis,n\}$ con $i \neq i_0$, pueden repetirse los razonamientos anteriores para obtener que existe una constante $C'' > 0$ tal que
        \begin{align*}
            S_{m_0}f_n(x) &\geq -\frac{2n+1}{4\pi n}\sum_{\substack{1 \leq i \leq n \\ i \neq i_0}} \frac{1}{|l-2i|} > -C''\log(n),
        \end{align*}
        siempre que se tome $n$ suficientemente grande.

        Definiendo $C''' = \max\{C',C''\}$, se tiene que 
        \[S_{m_0}f_n(x) > -C'''\log(n)\] tanto si $l$ es par como impar. Como $\lim_{n \to \infty} \frac{n}{\log(n)} = \infty$, para $n$ suficientemente grande se verifica $\frac{n}{\log(n)}>2C'''$,
        luego $\frac{n}{2} > C'''\log(n)$ y por tanto 
        \begin{align*}
            S_{m_0}F_n(x) &= S_{m_0}f_n(x)+S_{m_0}\phi_n(x) = S_{m_0}f_n(x)+\phi_n(x) \geq S_{m_0}f_n(x)+n \\
            &> -C'''\log(n)+n > -\frac{n}{2}+n = \frac{n}{2},
        \end{align*}
        obteniéndose así \hyperref[2.2.14]{\color{blue}(2.2.14)}. En la segunda igualdad se ha usado que $\phi_n$ es un polinomio trigonométrico de grado $m_0$, y en la primera desigualdad se ha usado \hyperref[2.2.10]{\color{blue}(2.2.10)}.
    \end{itemize}

    Con todo lo anterior finaliza la definición de la sucesión $\{F_n\}_{n=n_0}^\infty$ para cierto $n_0\in\N$. Para cada $n \geq n_0$, sean
    \[E_n = \Bigl[0, \frac{4\pi E(n-\sqrt{n})}{2n+1}\Bigr], \qquad A_n = C\log(n), \qquad \lambda_n = m_0,\]
    donde $C$ es la constante que aparece en \hyperref[2.2.11]{\color{blue}(2.2.11)}. Es claro que $A_n > 0$ (podemos suponer $n>1$) y que $\lim_{n \to \infty} A_n = \infty$.
    
    Obsérvese que $m_0$ depende de $n$, y como se razonó que $\lambda_n = m_0 > \frac{2n-1}{2}$, entonces $\lim_{n\to\infty}\lambda_n = \infty$. 
    
    Por otro lado, la función $f \colon [1,\infty) \to \R$ dada por $f(x) = \frac{x-\sqrt{x}}{2x+1}$ es estrictamente creciente, pues es derivable y para todo $x \geq 1$ se tiene que
    \begin{align*}
        f'(x) &= \frac{(1-\frac{1}{2\sqrt{x}})(2x+1)-2(x-\sqrt{x})}{(2x+1)^2} = \frac{2x-\sqrt{x}+1-\frac{1}{2\sqrt{x}}-2x+2\sqrt{x}}{(2x+1)^2} \\
        &= \frac{\sqrt{x}+1-\frac{1}{2\sqrt{x}}}{(2x+1)^2} = \frac{2x+2\sqrt{x}-1}{2\sqrt{x}(2x+1)^2} \geq \frac{3}{2\sqrt{x}(2x+1)^2} > 0,
    \end{align*}
    utilizándose en la penúltima desigualdad que $x \geq 1$ y que $\sqrt{x}\geq 1$. En consecuencia, la sucesión $\{\frac{4\pi E(n-\sqrt{n})}{2n+1}\}_{n=1}^\infty$ es estrictamente creciente, de donde se obtiene que $E_n \subset E_{n+1}$ para todo $n \geq n_0$. Como además
    \[\frac{4\pi (n-\sqrt{n}-1)}{2n+1} \leq \frac{4\pi E(n-\sqrt{n})}{2n+1} \leq \frac{4\pi (n-\sqrt{n})}{2n+1} \]
    y los extremos de la desigualdad tienden a $\frac{4\pi}{2}$, entonces \[\lim_{n \to \infty}\frac{4\pi E(n-\sqrt{n})}{2n+1} = \frac{4\pi}{2} = 2\pi,\]
    de donde se obtiene que $E_n \subset [0,2\pi]$ para todo $n \geq n_0$ y $\bigcup_{n=1}^\infty E_n = [0,2\pi)$. 
    
    Solo queda por probar el apartado (g). Sea $n \in \N$ y sea $x \in E_n=[x_0,x_{2E(n-\sqrt{n})}]$. Se distinguen los siguientes casos:
    \begin{itemize}
        \item Supongamos que $x\in\bigcup_{j=0}^{2n}I_j'$. Llamando $k = m_0$, se tiene $\lambda_n = m_0 = k < m_n = \nu_n$. Por \hyperref[2.2.14]{\color{blue}(2.2.14)}, también se verifica 
        \[S_{k}F_n(x) = S_{m_0}F_n(x) > \frac{n}{2}.\]
        Como $\lim_{n\to\infty} \frac{n}{\log(n)} = 1$, tomando $n$ suficientemente grande, se cumple $\frac{n}{\log(n)} > 2C$, obteniéndose $\frac{n}{2} > C\log(n) = A_n$ y por tanto $S_kF_n(x) > C\log(n) = A_n$.
        \item Supongamos que $x \in \bigcup_{j=0}^{n-1}(I_{2j}\cup I_{2j+1})$. Sea $j \in \{0,1,\mathellipsis,n-1\}$ con $x \in I_{2j} \cup I_{2j+1}$. Como $x \leq x_{2E(n-\sqrt{n})}$ (porque $x \in E_n$) y $x > x_{2j}$ (porque $x \in I_{2j} \cup I_{2j+1}$), tiene que ser $2j < 2(n-\sqrt{n})$, así que $j < n-\sqrt{n}$ y puede usarse \hyperref[2.2.11]{\color{blue}(2.2.11)} para obtener que existe $k \in \N$ con 
        \[m_0 \leq m_j \leq k < \frac{1}{2}m_{j+1} < \frac{1}{2}m_n < m_n\]
        y tal que $S_{k}F_n(x) >  C\log(n)$. Como $\nu_n = m_n$, $m_0 = \lambda_n$ y $C\log(n) = A_n$, se tiene que $\lambda_n \leq k < \nu_n$ y que $S_{k}F_n(x) > A_n$.
    \end{itemize}
    Nótese que es imposible que se tenga $x\not\in \bigcup_{j=0}^{2n}I_j'$ y $x \not\in \bigcup_{j=0}^{n-1}(I_{2j}\cup I_{2j+1})$, pues en ese caso,
    \[x\not\in \Bigl(\bigcup_{j=0}^{2n}I_j'\Bigr) \cup\Bigl(\bigcup_{j=0}^{2n-1}I_j\Bigr) = [x_0-\delta,x_{2n}+\delta],\]
    y como $[x_0-\delta,x_{2n}+\delta] \supset [x_0,x_{2E(n-\sqrt{n})}] = E_n$, se tendría $x \not\in E_n$.

    De cualquier modo, existe $k \in \N$ con $S_kF_n(x) > A_n$ y tal que $\lambda_n \leq k \leq \nu_n$, concluyendo así la prueba de (g).
\end{proof}

\begin{theorem}
    Existe una función de $L^1(\T)$ cuya serie de Fourier diverge en todo punto.
\end{theorem}

\begin{proof}
    Sean $\{F_n\}_{n=1}^\infty$, $\{A_n\}_{n=1}^\infty$, $\{E_n\}_{n=1}^\infty$ y $\{\lambda_n\}_{n=1}^\infty$ las cuatro sucesiones del lema anterior. Consideramos la sucesión $\{n_i\}_{i=1}^\infty$ definida por inducción de la siguiente manera:
    \begin{itemize}
        \item $n_1 = 1$.
        \item Sea $i \in \N$ tal que $n_i$ está definido. Escogemos $n_{i+1} \in \N$ tal que:
        \begin{enumerate}
            \item $n_{i+1} > n_{i}$.
            \item $\lambda_{n_{i+1}} > \nu_{n_{i}}$.
            \item $A_{n_{i+1}} > 4A_{n_{i}}$.
            \item $\sqrt{A_{n_{i+1}}} > \nu_{n_{i}}$.
        \end{enumerate}
        Esto es posible porque $\lim_{n \to \infty} A_n = \infty$ y $\lim_{n \to \infty} \lambda_n = \infty$.
    \end{itemize}
    Ahora consideramos la función $f \colon \R \to \overline{\R}$ dada por
    \[f(t) = \sum_{k=1}^\infty \frac{1}{\sqrt{A_{n_k}}}F_{n_k}(t).\]
    Veamos en primer lugar que $f \in L^1(\T)$. Es claro que $f$ es medible y $2\pi$-periódica, pues las funciones $F_{n_k}$ lo son. Además,
    \begin{align*}
        \integral{-\pi}{\pi}{|f(t)|} &= \integral{-\pi}{\pi}{\sum_{k=1}^\infty \frac{1}{\sqrt{A_{n_k}}}F_{n_k}(t)}
        \overset{(\asts{1})}{=}\sum_{k=1}^\infty  \integral{-\pi}{\pi}{\frac{1}{\sqrt{A_{n_k}}}F_{n_k}(t)} \\
        &=\sum_{k=1}^\infty \frac{1}{\sqrt{A_{n_k}}} \integral{-\pi}{\pi}{F_{n_k}(t)}
        \overset{(\asts{2})}{=} \sum_{k=1}^\infty \frac{2\pi}{\sqrt{A_{n_k}}},
    \end{align*}
    así que basta ver que $\sum_{k=1}^\infty \frac{1}{\sqrt{A_{n_k}}} < \infty$ para probar que $f \in L^1(\T)$. Se aclara lo siguiente:
    \begin{itemize}
        \item[(\asts{1})] Puede intercambiarse la integral con la suma por el teorema de la convergencia monótona (las funciones $\frac{1}{\sqrt{A_{n_k}}}F_{n_k}$, $k \in \N$, son no negativas).
        \item[(\asts{2})] Se ha usado que
        \begin{align*}
            \integral{-\pi}{\pi}{F_{n_k}(t)} &= \integral{-\pi}{\pi}{\Bigl(1+\sum_{j=1}^{\nu_{n_k}}(a_j(n_k)\cos(jt)+b_j(n_k)\sen(jt))\Bigr)} \\
            &= \integral{-\pi}{\pi}{1} + \sum_{j=1}^{\nu_{n_k}}\Bigl(a_j(n_k)\integral{-\pi}{\pi}{\cos(jt)}+b_j(n_k)\integral{-\pi}{\pi}{\sen(jt)}\Bigr) \\
            &=2\pi +\sum_{j=1}^{\nu_{n_k}}\Bigl(a_j(n_k)\cdot 0+b_j(n_k) \cdot 0\Bigr) =2\pi.
        \end{align*}
    \end{itemize}
    Veamos entonces que $\sum_{k=1}^\infty \frac{1}{\sqrt{A_{n_k}}} < \infty$. Para todo $k \in \N$ se tiene que $A_{n_{k+1}} > 4A_{n_{k}}$ y por tanto $\sqrt{A_{n_{k+1}}} > 2\sqrt{A_{n_{k}}}$, luego
    \[\frac{\sqrt{A_{n_k}}}{\sqrt{A_{n_{k+1}}}} < \frac{\sqrt{A_{n_k}}}{2\sqrt{A_{n_k}}} = \frac{1}{2} \kconv \frac{1}{2} < 1.\]
    Por el criterio del cociente, $\sum_{k=1}^\infty \frac{1}{\sqrt{A_{n_k}}} < \infty$, lo que prueba que $f \in L^1(\T)$. 
    
    Como la serie de Fourier de $f$ es $2\pi$-periódica, basta probar que diverge en todos los puntos de $[0,2\pi)$. Sea $x \in [0,2\pi)$ y veamos que la sucesión de sumas parciales $\{S_nf(x)\}_{n=1}^\infty$ no converge. Como $\bigcup_{n=1}^\infty E_n = [0,2\pi)$, existe $i_0 \in \N$ tal que $x \in E_{n_{i_0}}$. Y como $E_n \subset E_{n+1}$ para todo $n \in \N$, entonces $x \in E_{n_i}$ para todo $i \geq i_0$. Sea $i \in \N$ con $i \geq i_0$. Se tiene que $f = u+v+w$, donde
    \[v = \frac{1}{\sqrt{A_{n_i}}}F_{n_i}, \qquad u = \sum_{j=1}^{i-1}\frac{1}{\sqrt{A_{n_j}}}F_{n_j}, \qquad w = \sum_{j=i+1}^{\infty}\frac{1}{\sqrt{A_{n_j}}}F_{n_j}.\]
    Además, para cada $n \in \N$, 
    \[S_nf(x) = S_nu(x)+S_nv(x)+S_nw(x).\]
    Por el lema anterior, existe $k \in \N$ (que depende de $i$ y de $x$) con $\lambda_{n_i} \leq k \leq \nu_{n_i}$ y tal que $S_kF_{n_i}(x) > A_{n_i}$. Acotemos inferiormente $S_kf(x)$. En primer lugar,
    \[S_kv(x) = \frac{1}{\sqrt{A_{n_i}}}S_kF_{n_i}(x) > \frac{A_{n_i}}{\sqrt{A_{n_i}}} = \sqrt{A_{n_i}}.\]
    Además,
    \[S_ku(x) = \sum_{j=1}^{i-1}\frac{1}{\sqrt{A_{n_j}}}S_kF_{n_j}(x) > \sum_{j=1}^{i-1}\frac{A_{n_j}}{\sqrt{A_{n_j}}} = \sum_{j=1}^{i-1} \sqrt{A_{n_j}} \geq 0.\]
    Por otro lado, usando el \hyperref[2.2.1]{\color{blue}Lema 2.2.1},
    \begin{align*}
        |S_kw(x)| &\leq (4k+1)\|w\|_{1}
        = \frac{4k+1}{\pi}\integral{-\pi}{\pi}{\Bigl|\sum_{j=i+1}^\infty \frac{1}{\sqrt{A_{n_j}}}F_{n_j}(t)\Bigr|} \\
        &= \frac{4k+1}{\pi}\integral{-\pi}{\pi}{\sum_{j=i+1}^\infty\frac{1}{\sqrt{A_{n_j}}}F_{n_j}(t)}
        \overset{(\asts{1})}{=} \frac{4k+1}{\pi}{\sum_{j=i+1}^\infty}\frac{1}{\sqrt{A_{n_j}}}\integral{-\pi}{\pi}{F_{n_j}(t)} \\
        &= \frac{4k+1}{\pi}\sum_{j=i+1}^\infty\frac{2\pi}{\sqrt{A_{n_j}}} 
        = 2(4k+1)\sum_{j=i+1}^\infty \frac{1}{\sqrt{A_{n_j}}}
        \overset{(\asts{2})}{\leq} \frac{2(4k+1)}{\sqrt{A_{n_{i+1}}}}\sum_{j=0}^\infty \frac{1}{2^j} \\
        &=\frac{16k+4}{\sqrt{A_{n_{i+1}}}} \leq \frac{16\nu_{n_i}+4}{\sqrt{A_{n_{i+1}}}} \overset{(\textup{c})}{\leq} \frac{16\nu_{n_i}+4}{\nu_{n_i}} \leq \frac{16\nu_{n_i}+4\nu_{n_i}}{\nu_{n_i}} = 20,
    \end{align*}
    de donde se obtiene que $S_kw(x) > -20$.
    Algunas alcaraciones:
    \begin{itemize}
        \item[(\asts{1})] Se intercambia la integral con la suma por el teorema de la convergencia monótona, teniendo en cuenta que las funciones $\frac{1}{\sqrt{A_{n_j}}}F_{n_j}$, $j \in \N$, son no negativas.
        \item[(\asts{2})] Se usa que, por (c), para todo $j \in \N$ con $j \geq i+1$ se tiene que $A_{n_i} > 4A_{n_{i-1}}$ y por tanto
        \[\sqrt{A_{n_j}} > 2\sqrt{A_{n_{j-1}}} > 2^2 \sqrt{A_{n_{j-2}}} > \mathellipsis > 2^{j-i-1}\sqrt{A_{n_{i+1}}},\]
        de donde se obtiene que
        \[\frac{1}{\sqrt{A_{n_j}}} < \frac{1}{2^{j-i-1}\sqrt{A_{n_{i+1}}}},\]
        y en consecuencia,
        \[\sum_{j=i+1}^\infty \frac{1}{\sqrt{A_{n_j}}} \leq \sum_{j=i+1}^\infty \frac{1}{2^{j-i-1}\sqrt{A_{n_{i+1}}}} = \frac{1}{\sqrt{A_{n_{i+1}}}}\sum_{j=0}^\infty \frac{1}{2^j}.\]
    \end{itemize}
    Reuniendo todo lo anterior,
    \[S_kf(x) = S_ku(x)+S_kv(x)+S_kw(x)> \sqrt{A_{n_i}}-20.\]
    Como $k \leq \nu_{n_i}$, entonces
    \[S_{\nu_{n_i}}f(x) \geq S_kf(x) > \sqrt{A_{n_i}}-20,\]
    y esto es válido para todo $i \in \N$ con $i \geq i_0$. Como $\lim_{i \to \infty} n_i = \infty$ (pues $\{n_i\}_{i=1}^\infty$ es una sucesión estrictamente creciente de números naturales) y $\lim_{n \to \infty} A_n = \infty$, entonces \[\lim_{i \to \infty} \sqrt{A_{n_i}} = \infty.\]
    Llevando esto a la desigualdad anterior, se obtiene
    \[\lim_{i \to \infty} S_{\nu_{n_i}}f(x) = \infty.\]
    Como $\{S_{\nu_{n_i}}f(x)\}_{i=1}^\infty$ es una subsucesión de $\{S_nf(x)\}_{n=1}^\infty$ que no converge, entonces la serie de Fourier de $f$ en $x$ no converge.
\end{proof}

\chapter{Convergencia en \texorpdfstring{\textit{L\textsuperscript{p}}}{\textit{Lp}}}

Un resultado importante que se estudia en Análisis Real y Análisis Funcional es que para toda $f \in L^2(\T)$ se cumple que $\{S_nf\}_{n=1}^\infty$ converge a $f$ en $L^2(\T)$. El objetivo aquí es probar que esto se verifica para todo $p$ con $1 < p < \infty$, y también probaremos que no se cumple para $p = 1$. 

Para $p = \infty$ es claro que no se cumple: en el capítulo anterior se probó que existe $f\in\mathcal{C}(\T)$ tal que $\{S_nf(0)\}_{n=1}^\infty$ no converge puntualmente, así que $\{S_nf\}_{n=1}^\infty$ no converge uniformemente.

Antes de estudiar los casos $p=1$ y $1<p<\infty$, probamos dos resultados auxiliares que serán fundamentales en ambos casos.

\begin{lemma}
    El conjunto de los polinomios trigonométricos, $\mathcal{P}$, es denso en $(\mathcal{C}(\T),\|\cdot\|_\infty)$.
\end{lemma}

\begin{proof}
    Sea $f \in \mathcal{C}(\T)$ y sea $\varepsilon > 0$. Hay que probar que existe $F \in \mathcal{P}$ tal que $\|f-F\|_\infty < \varepsilon$. Por el \hyperref[1.2.6]{\color{blue}Teorema 1.2.6},
    \[\lim_{n \to \infty} \|\sigma_nf-f\|_\infty = 0,\]
    luego existe $n_0 \in \N$ tal que
    \[\|\sigma_{n_0}f-f\|_\infty < \varepsilon.\]
    Basta tomar $F = \sigma_{n_0}f$, que es un polinomio trigonométrico por ser suma de polinomios trigonométricos.
\end{proof}

Si $1 \leq p \leq \infty$, se sabe que $\mathcal{C}(\T)$ es denso en $L^p(\T)$, así que $\mathcal{P}$ también es denso en $L^p(\T)$.

\begin{lemma}\label{3.0.2}
    Si $1 \leq p < \infty$, son equivalentes:
    \begin{enumerate}
        \item $\{S_nf\}_{n=1}^\infty$ converge a $f$ en $L^p(\T)$ para toda $f \in L^p(\T)$.
        \item Existe $C_p > 0$ tal que
        \[\|S_nf\|_{p} \leq C_p \|f\|_{p}\]
        para todo $n \in \N$ y toda $f \in L^p(\T)$.
    \end{enumerate}
\end{lemma}

\begin{proof}
    Supongamos que se cumple (a). Para cada $n \in \N$, sea $T_n \colon L^p(\T) \to L^p(\T)$ la aplicación dada por $T_n(f) = S_nf$. Es claro que $T_n$ es lineal, y es continua porque para toda $f \in L^p(\T)$ se verifica
    \[\|T_n(f)\|_{p} = \|S_nf\|_p = \|D_n \ast f\|_p \leq \|D_n\|_1 \|f\|_p.\]
    Por otra parte, si $f \in L^p(\T)$ y $n \in \N$, entonces
    \[ 0 \leq |\|S_nf\|_{p} - \|f\|_{p}| \leq \|S_nf-f\|_{p}.\]
    Y como $\lim_{n \to \infty} \|S_nf - f\|_p = 0$, entonces $\lim_{n \to \infty} \|S_nf\|_{p} = \|f\|_{p} < \infty$. Por tanto, la sucesión $\{\|S_nf\|_{p}\}_{n=1}^\infty$ es acotada. Así, se tiene que
    \begin{enumerate}
        \item $\{T_n\}_{n\in\N}$ es una familia de aplicaciones lineales y continuas de $L^p(\T)$ en $L^p(\T)$.
        \item $(L^p(\T),\|\cdot\|_p)$ es un espacio de Banach.
        \item Para toda $f \in L^p(\T)$, el conjunto $\{T_n(f) \colon n \in \N\}$ es acotado en $L^p(\T)$.
    \end{enumerate}
    Por el \hyperref[1.2.2]{\color{blue}teorema de la acotación uniforme}, el conjunto $\{\|T_n\| \colon n \in \N\}$ es acotado, es decir, existe $C_p > 0$ tal que
    \[\|T_n\| = \sup_{f \neq 0} \frac{\|T_n(f)\|_p}{\|f\|_p} =  \sup_{f \neq 0} \frac{\|S_nf\|_p}{\|f\|_p} \leq C_p\]
    para todo $n \in \N$. En consecuencia, si $n \in \N$ y $f \in L^p(\T)$,
    \[\|S_nf\|_p\leq C_p \|f\|_p,\]
    así que se verifica (b).

    Supongamos ahora que se cumple (b) y veamos que para toda $f \in L^p(\T)$ se tiene que $\lim_{n \to \infty} \|S_nf-f\|_p = 0$. 
    
    En primer lugar, (a) se verifica para los polinomios trigonométricos, pues si $F$ es un polinomio trigonométrico de grado $N$, entonces $S_nF = F$ para todo $n \geq N$ y por tanto $\lim_{n \to \infty} \|S_nF - F\|_p = 0$. 
    
    Pasamos a probar el caso general. Sea $f \in L^p(\T)$ y sea $\varepsilon > 0$. Como $\mathcal{P}$ es denso en $L^p(\T)$, existe $F \in \mathcal{P}$ tal que
    \[\|f-F\|_p < \frac{\varepsilon}{C_p+1}.\]
    Por tanto, para todo $n\in\N$ mayor que el grado de $F$,
    \[\begin{aligned}[t]
        \|S_nf - f\|_p &\leq \|S_nf-S_nF\|_p+\|S_nF -f\|_p = \|S_n(f-F)\|_p+\|F -f\|_p \\
        &\leq C_p\|f-F\|_p+\|f-F\|_p = \|f-F\|_p(C_p+1)\\
        &<  \frac{\varepsilon}{C_p+1}(C_p+1) = \varepsilon.
    \end{aligned}\]
    Se concluye que $\lim_{n \to \infty} \|S_nf-f\|_p = 0$.
\end{proof}

\section[Convergencia en \texorpdfstring{$L^1$}{L1}]{Convergencia en \texorpdfstring{\boldmath$L^1$}{L1}}

En esta sección se probará que existe $f \in L^1(\T)$ tal que la serie de Fourier de $f$ no converge a $f$ en $L^1(\T)$. Esto será inmediato a partir del \hyperref[3.0.2]{\color{blue}Lema 3.0.2} y el lema siguiente, que puede encontrarse en \cite{katznelson}.

\begin{lemma}
    Sea $n \in \N$ y sea $T_n \colon L^1(\T) \to L^1(\T)$ la aplicación dada por $T_n(f) = S_nf$. Entonces
    \[\|T_n\| = \|D_n\|_1.\]
\end{lemma}

\begin{proof}
    En la demostración del lema anterior se probó que $T_n$ es lineal y continua, así que $\|T_n\|$ tiene sentido. Como también se razonó que $\|T_n(f)\|_1 \leq \|D_n\|_1\|f\|_1$ para toda $f \in L^1(\T)$, entonces $\|T_n\| \leq \|D_n\|_1$. 
    
    Solo queda por demostrar que $\|T_n\|\geq \|D_n\|_1$. Dado $N \in \N$, el núcleo de Fejér $K_N$ verifica
    \[\|T_n(K_N)\|_1 = \|S_nK_N\|_1 = \|K_N \ast D_n\|_1 = \|\sigma_ND_n\|_1.\]
    Por el \hyperref[1.2.6]{\color{blue}Teorema 1.2.6},
    \[\lim_{N\to\infty} \|\sigma_ND_n-D_n\|_1 = 0,\]
    luego
    \[\lim_{N\to\infty} \|T_n(K_N)\|_1 = \lim_{N\to\infty} \|\sigma_ND_n\|_1 = \|D_n\|_1.\]
    Por tanto, para todo $\varepsilon > 0$ existe $N \in \N$ tal que 
    \[\|T_n(K_{N})\|_1 \geq \|D_n\|_1-\varepsilon.\]
    Como $\|K_N\|_1=1$, obtenemos
    \[\|T_n\| = \sup_{\|f\|_1=1} \|T_n(f)\|_1 \geq \|T_n(K_N)\|_1 \geq \|D_n\|_1-\varepsilon,\]
    y como esto es válido para todo $\varepsilon>0$, se concluye que $\|T_n\| \geq \|D_n\|_1$.
\end{proof}

\begin{theorem}
    Existe $f \in L^1(\T)$ tal que $\{S_nf\}_{n=1}^\infty$ no converge a $f$ en $L^1(\T)$.
\end{theorem}

\begin{proof}
    Por el \hyperref[3.0.2]{\color{blue}Lema 3.0.2}, hay que probar que para todo $C > 0$ existen $n \in \N$ y $f \in L^p(\T)$ tales que
    \[\|S_nf\|_1 > C\|f\|_1.\]
    Sea $C>0$. Usando el lema anterior y el \hyperref[2.1.1]{\color{blue}Lema 2.1.1}, obtenemos
    \[\lim_{n\to\infty} \|T_n\| = \lim_{n\to\infty}\|D_n\|_1 = \infty,\]
    siendo $T_n \colon L^1(\T) \to L^1(\T)$ la aplicación dada por $T_n(f) = S_nf$. Por el contrarrecíproco del \hyperref[1.2.2]{\color{blue}teorema de la acotación uniforme}, existe $f \in L^1(\T)$ tal que el conjunto $\{T_n(f) \colon n \in \N\}$ no es acotado en $L^1(\T)$. En consecuencia,
    \[\|T_n(f)\|_1 = \|S_nf\|_1 > C\|f\|_1. \qedhere\]
\end{proof}

\section[Convergencia en \texorpdfstring{$L^p$}{Lp} para \texorpdfstring{$1<p<\infty$}{1<p<oo}]{Convergencia en \texorpdfstring{\boldmath$L^p$}{Lp} para \texorpdfstring{\boldmath$1<p<\infty$}{1<p<oo}}

La referencia principal de esta sección es \cite{miao}. Comenzamos con algunos resultados auxiliares que serán fundamentales en la demostración del teorema sobre la convergencia de series de Fourier en $L^p(\T)$.

Estudiamos en primer lugar un resultado relacionado con el espacio dual de $L^p(\T)$ para $1<p<\infty$.

\begin{lemma}
    Si $1 < p < \infty$ y $p'$ es el exponente conjugado de $p$, entonces la aplicación
    \[\begin{alignedat}{3}
        \Phi \colon L^{p'}(\T) &\longrightarrow L^p(\T)^* & & \\
        g &\longmapsto \Phi(g) \colon & L^p(\T) &\longrightarrow \C \\
        & & f &\longmapsto \Phi(g)(f)=\frac{1}{2\pi}\integral{-\pi}{\pi}{f(t)g(t)}
    \end{alignedat}\]
    es lineal y continua. Además, para toda $g \in L^{p'}(\T)$ se verifica
    \[\|g\|_{p'} = \|\Phi(g)\|.\]
\end{lemma}

\begin{proof}
    Veamos que $\Phi$ está bien definida. Sea $g \in L^{p'}(\T)$ y veamos que la aplicación $\Phi(g)\colon L^p(\T)\to \C$ dada por
    \[\Phi(g)(f) = \frac{1}{2\pi}\integral{-\pi}{\pi}{f(t)g(t)}\]
    es lineal y continua. La linealidad es consecuencia directa de la linealidad de la integral. La continuidad se deduce fácilmente de la desigualdad de Hölder, pues para toda $f \in L^p(\T)$ se tiene
    \[|\Phi(g)(f)| = \frac{1}{2\pi}\Bigl|\integral{-\pi}{\pi}{f(t)g(t)}\Bigr| \leq \frac{1}{2\pi}\integral{-\pi}{\pi}{|f(t){g(t)}|} = \|fg\|_1 \leq \|f\|_p\|g\|_{p'}.\]
    Esto también prueba que $\|\Phi(g)\| \leq \|g\|_{p'}$.
    
    De nuevo, la linealidad de $\Phi$ es consecuencia directa de la linealidad de la integral: si $g,h \in L^{p'}(\T)$ y $\alpha,\beta \in \C$, entonces para toda $f \in L^p(\T)$ se tiene
    \begin{align*}
        \Phi(\alpha g + \beta h)(f) &= \frac{1}{2\pi}\integral{-\pi}{\pi}{f(t)(\alpha g(t)+\beta h(t))} \\ &= \frac{\alpha}{2\pi}\integral{-\pi}{\pi}{f(t)g(t)}+\frac{\beta}{2\pi}\integral{-\pi}{\pi}{f(t)h(t)} = \alpha \Phi(g)(f)+\beta \Phi(h)(f),
    \end{align*}
    luego $\Phi(\alpha g + \beta h) = \alpha \Phi(g) + \beta \Phi(h)$.

    Probemos por último que $\|g\|_{p'} = \|\Phi(g)\|$ para toda $g \in L^{p'}(\T)$. Sea $g \in L^{p'}(\T)$ y veamos que $\|g\|_{p'} \leq \|\Phi(g)\|$. Si $g = 0$, la desigualdad es trivial. Supongamos entonces que $g \neq 0$, y sea
    \[f = \frac{|g|^{p'-1}\overline{\sgn(g)}}{\|g\|^{p'-1}_{p'}},\]
    donde, para cada $z \in \C$, definimos
    \[\sgn(z) = \begin{cases}
        \displaystyle \frac{z}{|z|} & $ si $ z \neq 0, \\[10pt]
        0 & $ si $ z = 0.
    \end{cases}\]
    Observamos que $|\sgn(z)| = 1$ para casi todo $z\in\C$. Usando esto y que $p = \frac{p'}{p'-1}$ por ser $p$ y $p'$ exponentes conjugados, se obtiene
    \[\begin{aligned}[t]
        \|f\|_p^p &= \frac{1}{2\pi}\integral{-\pi}{\pi}{\frac{|g(t)|^{(p'-1)p}\bigl|\overline{\sgn (g(t))}\bigr|^p}{\|g\|_{p'}^{(p'-1)p}}} = \frac{1}{\|g\|_{p'}^{p'}}\frac{1}{2\pi}\integral{-\pi}{\pi}{|g(t)|^{p'}} = 1,
    \end{aligned}\]
    luego
    \[\begin{aligned}[t]
        \|\Phi(g)\| &\geq |\Phi(g)(f)| = \frac{1}{2\pi}\Bigl|\integral{-\pi}{\pi}{f(t)g(t)}\Bigr| = \frac{1}{2\pi}\Bigl|\integral{-\pi}{\pi}{\frac{|g(t)|^{p'-1}\overline{\sgn(g(t))}g(t)}{\|g\|_{p'}^{p'-1}}}\Bigr| \\
        &= \frac{1}{\|g\|_{p'}^{p'-1}} \frac{1}{2\pi}\Bigl|\integral{-\pi}{\pi}{\frac{|g(t)|^{p'-1}\overline{g(t)}g(t)}{|g(t)|}}\Bigr| = \frac{1}{\|g\|_{p'}^{p'-1}} \frac{1}{2\pi}\integral{-\pi}{\pi}{\frac{|g(t)|^{p'-1}|g(t)|^2}{|g(t)|}}
        \\ &= \frac{1}{\|g\|_{p'}^{p'-1}} \frac{1}{2\pi}\integral{-\pi}{\pi}{|g(t)|^{p'}} = \|g\|_{p'}^{p'-(p'-1)} = \|g\|_{p'}.
    \end{aligned}\]

    Como la desigualdad $\|g\|_{p'} \geq \|\Phi(g)\|$ se demostró junto con la continuidad de $\Phi(g)$, tenemos que $\|g\|_{p'} = \|\Phi(g)\|$.
\end{proof}

En \cite{folland} también se demuestra que $\Phi$ es biyectiva, y por tanto se trata de un isomorfismo isométrico. Aquí no usaremos la biyectividad de $\Phi$; lo importante es que si $g \in L^{p'}(\T)$, entonces
\[
    \|g\|_{p'} = \|\Phi(g)\| = \sup_{\|f\|_p = 1} |\Phi(g)(f)| = \sup_{\|f\|_p=1}\frac{1}{2\pi}\Bigl|\integral{-\pi}{\pi}{f(t)g(t)}\Bigr|.
\]
De la misma manera, si $g\in L^p(\T)$, entonces
\begin{equation}\label{1.3.1}
    \|g\|_{p} = \sup_{\|f\|_{p'}=1}\frac{1}{2\pi}\Bigl|\integral{-\pi}{\pi}{f(t)g(t)}\Bigr|.
\end{equation}

El siguiente resultado es una versión simplificada del teorema de interpolación de Riesz-Thorin, suficiente para los fines de este trabajo.

\begin{theorem}[Teorema de interpolación de Riesz-Thorin]\label{3.2.2}
    Sean $p,q,r\in\overline{\R}$ con $1 \leq p \leq r \leq q \leq \infty$. Sea $T \colon L^{p}(\T) \to L^{p}(\T)$ una aplicación lineal. Supongamos que existen $M_p,M_q>0$ tales que
    \begin{itemize}
        \item $\|T(f)\|_{p} \leq M_p\|f\|_{p}$ para toda $f \in L^{p}(\T)$.
        \item $\|T(f)\|_{q} \leq M_q\|f\|_{q}$ para toda $f \in L^{q}(\T)$.
    \end{itemize}
    Entonces existe $M_r>0$ tal que
    \[\|T(f)\|_r \leq M_r\|f\|_r\]
    para toda $f \in L^r(\T)$.
\end{theorem}

\begin{proof}
    
\end{proof}

Antes de enunciar y probar el próximo lema, introducimos las series formales siguientes: si $f\in L^1(\T)$, denotamos
\begin{itemize}
    \item $\displaystyle f^+(t) = \sum_{k=1}^\infty c_k(f)e^{ikt}$,
    \item $\displaystyle \widetilde{f}(t) = -i\sum_{k\in\Z} \sgn(k)c_k(f)e^{ikt}$,
\end{itemize}
donde
\[\sgn(k) = \begin{cases}
    \phantom{-}1 & $ si $ k > 0, \\
    -1 & $ si $ k < 0, \\
    \phantom{-}0 & $ si $ k = 0.
\end{cases}\]

En general, no se sabe para qué puntos estas dos series tienen sentido. Ahora bien, si $F \in \mathcal{P}$ y $n$ es el grado de $F$, entonces $c_k(F) = 0$ para todo $k\in\Z$ con $|k|>n$, así que las series anteriores tienen un número finito de términos no nulos. Es claro que $F^+, \widetilde{F}\in \mathcal{C}(\T)$.

\begin{lemma}
    Si $1 < p < \infty$, entonces existe $C_p > 0$ tal que
    \[\|\widetilde{F}\|_p \leq C_p \|F\|_{p}\]
    para todo $F \in \mathcal{P}$.
\end{lemma}

\begin{proof}
    Sea $F \in \mathcal{P}$. Como $F$ es de la forma
    \[F(t) = \sum_{k=-n}^n c_k(F)e^{ikt},\]
    entonces
    \begin{align*}
        \real(F(t)) &= \frac{F(t)+\overline{F(t)}}{2} = \frac{1}{2}\Bigl(\sum_{k=-n}^n c_k(F)e^{ikt}+\sum_{k=-n}^n\overline{c_k(F)}e^{-ikt}\Bigr) \\
        &= \frac{1}{2}\Bigl(\sum_{k=-n}^n c_k(F)e^{ikt}+\sum_{k=-n}^n\overline{c_{-k}(F)}e^{ikt}\Bigr) = \sum_{k=-n}^n \frac{c_k(F)+\overline{c_{-k}(F)}}{2}e^{ikt}.
    \end{align*}
    Razonando análogamente,
    \begin{align*}
        \imag(F(t)) &= \frac{F(t)-\overline{F(t)}}{2i} = \sum_{k=-n}^n \frac{c_k(F)-\overline{c_{-k}(F)}}{2i}e^{ikt}.
    \end{align*}
    Por tanto, $\real(F)$ e $\imag(F)$ también son polinomios trigonométricos, y sus coeficientes de Fourier son
    \[c_k(\real(F)) = \frac{c_k(F)+\overline{c_{-k}(F)}}{2}, \qquad \qquad c_k(\imag(F)) = \frac{c_k(F)-\overline{c_{-k}(F)}}{2i}.\]
    Usando esto y la linealidad de los coeficientes de Fourier,
    \begin{equation}\label{3.2.3}
    \begin{aligned}[b]
        \widetilde{F}(t) &= -i\sum_{k=-n}^n \sgn(k)c_k(F)e^{ikt} = -i\sum_{k=-n}^n \sgn(k)c_k(\real(F)+i\imag(F))e^{ikt} \\
        &= \sum_{k=-n}^n \sgn(k)c_k(\imag(F))e^{ikt} -i\sum_{k=-n}^n \sgn(k)c_k(\real(F))e^{ikt} \\
        &=  \sum_{k=-n}^n \sgn(k)\frac{c_k(F)-\overline{c_{-k}(F)}}{2i}e^{ikt} -i\sum_{k=-n}^n \sgn(k)\frac{c_k(F)+\overline{c_{-k}(F)}}{2}e^{ikt}.
     \end{aligned}
    \end{equation}
    
    Primero demostraremos el lema para $p = 2m$, con $m \in \N$. Lo hacemos distinguiendo tres casos, cada uno más general que el anterior.
    \begin{itemize}
        \item Supongamos primero que $F$ toma valores reales y que $c_0(F) = 0$. Entonces
        \[\overline{c_{-k}(F)} = \overline{\frac{1}{2\pi}\integral{-\pi}{\pi}{F(t)e^{ikt}}} = \frac{1}{2\pi}\integral{-\pi}{\pi}{\overline{F(t)}e^{-ikt}} =\frac{1}{2\pi}\integral{-\pi}{\pi}{{F(t)}e^{-ikt}} = c_k(F)\]
        para todo $k\in\Z$. Sustituyendo en \hyperref[3.2.3]{\color{blue}(3.2.3)},
        \begin{align*}
            \widetilde{F}(t) &= -i\sum_{k=-n}^n \sgn(k)c_k(F)e^{ikt} = -i\sum_{k=1}^n c_k(F)e^{ikt}+i\sum_{k=1}^{n}c_{-k}(F)e^{-ikt} \\
            &=-i\sum_{k=1}^n c_k(F)e^{ikt}+i\sum_{k=1}^{n}\overline{c_k(F)e^{ikt}}.
        \end{align*}
        Vemos que $\widetilde{F}(t) \in \R$ por ser suma de un número complejo y su conjugado. Por otra parte,
        \begin{align*}
            F(t)+i\widetilde{F}(t) &= \sum_{k=-n}^n c_k(F)e^{ikt}-i^2\sum_{k=-n}^n \sgn(k)c_k(F)e^{ikt} \\ 
            &= \sum_{k=-n}^n c_k(F)e^{ikt}+\sum_{k=1}^n c_k(F)e^{ikt} - \sum_{k=-n}^{-1} c_k(F)e^{ikt} \\
            &=c_0(F)+2\sum_{k=1}^n c_k(F)e^{ikt} = 2\sum_{k=1}^n c_k(F)e^{ikt}.
        \end{align*}
        Por tanto, $c_k(F+i\widetilde{F}) = 0$ para todo $k \leq 0$. Probemos que
        \[\integral{-\pi}{\pi}{(F(t)+i\widetilde{F}(t))^{2m}} = 0\]
        para todo $m \in \N$. La función $G = (F+i\widetilde{F})^{2m}$ es suma y producto de polinomios trigonométricos, así que es un polinomio trigonométrico. Y como $c_k(F+i\widetilde{F}) = 0$ para todo $k \leq 0$, entonces $c_k(G) = 0$ para todo $k \leq 0$. En particular, $c_0(G) = 0$, luego
        \[ \integral{-\pi}{\pi}{(F(t)+i\widetilde{F}(t))^{2m}} = \integral{-\pi}{\pi}{G(t)} = 2\pi c_0(G) = 0.\]
        Por otro lado,
        \[(F(t)+i\widetilde{F}(t))^{2m} = \sum_{k=0}^{2m}\binom{2m}{k}F(t)^ki^{2m-k}\widetilde{F}(t)^{2m-k}.\]Como $F$ y $\widetilde{F}$ toman valores reales, tenemos que
        \[\binom{2m}{k}F(t)^ki^{2m-k}\widetilde{F}(t)^{2m-k} \in \R \iff i^{2m-k} \in \{-1,1\} \iff k \textup{ es par}.\]
        Por tanto,
        \begin{align*}
            \real((F(t)+i\widetilde{F}(t))^{2m}) &= \sum_{k=0}^{m} \binom{2m}{2k}F(t)^{2k}i^{2m-2k}\widetilde{F}(t)^{2m-2k} \\
            &= \sum_{k=0}^{m}(-1)^{m-k} \binom{2m}{2k}F(t)^{2k}\widetilde{F}(t)^{2m-2k}.
        \end{align*}
        Como
        \[\integral{-\pi}{\pi}{(F(t)+i\widetilde{F}(t))^{2m}} = 0,\]
        entonces
        \begin{align*}
            \real\Bigl(\integral{-\pi}{\pi}{(F(t)+i\widetilde{F}(t))^{2m}}\Bigr) &= \integral{-\pi}{\pi}{\real((F(t)+i\widetilde{F}(t))^{2m})} \\ 
            &= \sum_{k=0}^{m}(-1)^{m-k}\binom{2m}{2k}\integral{-\pi}{\pi}{F(t)^{2k}\widetilde{F}(t)^{2m-2k}} = 0.
        \end{align*}
        Separando el sumando $k = 0$ del resto,
        \[(-1)^m\integral{-\pi}{\pi}{\widetilde{F}(t)^{2m}} = -\sum_{k=1}^{m}(-1)^{m-k}\binom{2m}{2k}\integral{-\pi}{\pi}{F(t)^{2k}\widetilde{F}(t)^{2m-2k}}.\]
        Por tanto,
        \begin{align*}
            \integral{-\pi}{\pi}{\widetilde{F}(t)^{2m}} &= (-1)^{m+1}\sum_{k=1}^{m}(-1)^{m-k}\binom{2m}{2k}\integral{-\pi}{\pi}{F(t)^{2k}\widetilde{F}(t)^{2m-2k}} \\
            &= \sum_{k=1}^{m}(-1)^{2m-k+1}\binom{2m}{2k}\integral{-\pi}{\pi}{F(t)^{2k}\widetilde{F}(t)^{2m-2k}},
        \end{align*}
        Utilizando esto y que $\widetilde{F}^{2m}$, $F^{2k}$ y $\widetilde{F}^{2m-2k}$ toman valores reales no negativos,
        \begin{align*}
            \|\widetilde{F}\|_{2m}^{2m} &= \frac{1}{2\pi}\integral{-\pi}{\pi}{|\widetilde{F}(t)|^{2m}} = \frac{1}{2\pi}\integral{-\pi}{\pi}{\widetilde{F}(t)^{2m}} \\
            &= \frac{1}{2\pi}\sum_{k=1}^{m}(-1)^{2m-k+1}\binom{2m}{2k}\integral{-\pi}{\pi}{F(t)^{2k}\widetilde{F}(t)^{2m-2k}} \\
            &= \frac{1}{2\pi}\sum_{k=1}^{m}(-1)^{2m-k+1}\binom{2m}{2k}\integral{-\pi}{\pi}{|F(t)|^{2k}|\widetilde{F}(t)|^{2m-2k}} \\
            &=\sum_{k=1}^{m}(-1)^{2m-k+1}\binom{2m}{2k}\|F^{2k}\widetilde{F}^{2m-2k}\|_1 \\
            &\leq \sum_{k=1}^{m}\binom{2m}{2k}\|F^{2k}\widetilde{F}^{2m-2k}\|_1.
        \end{align*}
        Para $k\in\{1,2,\mathellipsis,m-1\}$, aplicamos la desigualdad de Hölder con exponentes conjugados $\frac{2m}{2m-2k}$ y $\frac{2m}{2k}$, obteniendo
        \begin{align*}
            \|F^{2k}\widetilde{F}^{2m-2k}\|_1 &\leq \|F^{2k}\|_{\frac{2m}{2k}}\|\widetilde{F}^{2m-2k}\|_{\frac{2m}{2m-2k}} \\
            &=\Bigl( \frac{1}{2\pi}\integral{-\pi}{\pi}{F(t)^{2k\frac{2m}{2k}}}\Bigr)^{\frac{2k}{2m}}\Bigl( \frac{1}{2\pi}\integral{-\pi}{\pi}{\widetilde{F}(t)^{{2m-2k}\frac{2m}{2m-2k}}}\Bigr)^{\frac{2m-2k}{2m}} \\
            &= \|F\|_{2m}^{2k}\|\widetilde{F}\|_{2m}^{2m-2k}.
        \end{align*}
        Si $k = m$, sigue siendo cierta la desigualdad $\|F^{2k}\widetilde{F}^{2m-2k}\|_1 \leq \|F\|_{2m}^{2k}\|\widetilde{F}\|_{2m}^{2m-2k}$.  Tenemos entonces
        \[\|\widetilde{F}\|_{2m}^{2m} \leq \sum_{k=1}^m \binom{2m}{2k}\|F\|_{2m}^{2k}\|\widetilde{F}\|_{2m}^{2m-2k}.\]
        Dividiendo por $\|\widetilde{F}\|_{2m}^{2m}$ (si fuese $\widetilde{F} = 0$ no hay nada que probar),
        \[1 \leq \sum_{k=1}^m \binom{2m}{2k}\Bigl(\frac{\|F\|_{2m}}{\|\widetilde{F}\|_{2m}}\Bigr)^{2k}.\]
        Sea $R = \frac{\|F\|_{2m}}{\|\widetilde{F}\|_{2m}}$ y consideremos la función $\varphi \colon [0,\infty)\to\R$ dada por
        \[\varphi(t) = \sum_{k=1}^m \binom{2m}{2k}t^{2k}.\]
        La última desigualdad nos dice que $\varphi(R) \geq 1$. También tenemos que $\varphi$ es continua, estrictamente creciente y tal que
        \[\lim_{t \to \infty} \varphi(t) = \infty, \qquad \varphi(0) = 0.\]
        Por el teorema de los valores intermedios, existe $C > 0$ (que solo depende de $m$) tal que $\varphi(C) = 1$. Como $\varphi(R) \geq 1$ y $\varphi$ es estrictamente creciente, tiene que ser $R \geq C$, luego
        \[\|F\|_{2m} \geq C\|\widetilde{F}\|_{2m}.\]
        Llamando $C_{2m} = \frac{1}{C}$, obtenemos que
        \[\|\widetilde{F}\|_{2m} \leq C_{2m}\|F\|_{2m}.\]
        \item Supongamos ahora que $F$ toma valores reales. Aplicando lo que se acaba de probar a $G = F-c_0(F)$, que es un polinomio trigonométrico que toma valores reales y que verifica $c_0(G)=0$, obtenemos
        \[\|\widetilde{G}\|_{2m}\leq C_{2m}\|G\|_{2m}.\]
        Se tiene que
        \begin{align*}
            \widetilde{F}(t) &= -i\sum_{k\in\Z}\sgn(k)c_k(F)e^{ikt} = -i\sum_{k\in\Z}\sgn(k)c_k(G+c_0(F))e^{ikt} \\
            &= -i\sum_{k\in\Z}\sgn(k)c_k(G)e^{ikt}-i\sum_{k\in\Z}\sgn(k)c_k(c_0(F))e^{ikt} \\
            &= -i\sum_{k\in\Z}\sgn(k)c_k(G)e^{ikt} = \widetilde{G}(t),
        \end{align*}
        utilizándose la linealidad de los coeficientes de Fourier y que $c_k(c_0(F)) = 0$ para $k\neq 0$. Por tanto,
        \begin{align*}
            \|\widetilde{F}\|_{2m} = \|\widetilde{G}\|_{2m} \leq C_{2m}\|G\|_{2m} = C_{2m}\|F-c_0(F)\|_{2m} \leq C_{2m}(\|F\|_{2m}+\|c_0(F)\|_{2m}).
        \end{align*}
        Ahora bien,
        \[\|c_0(F)\|_{2m} = |c_0(F)| = \frac{1}{2\pi}\Bigl|\integral{-\pi}{\pi}{F(t)}\Bigr| \leq \|F\|_1 \leq \|F\|_{2m}, \]
        aplicándose en la última desigualdad la desigualdad de Hölder a la función $F$ y la función constante $1$. Por tanto,
        \[\|\widetilde{F}\|_{2m} \leq C_{2m}(\|F\|_{2m}+\|F\|_{2m}) = 2C_{2m}\|F_{2m}\|.\]
        \item Veamos el caso general. Como $\real(F)$ e $\imag(F)$ son polinomios trigonométricos que toman valores reales, por lo probado anteriormente,
        \[\|\widetilde{\real(F)}\|_{2m} \leq 2C_{2m} \|\real(F)\|_{2m}, \qquad \qquad \|\widetilde{\imag(F)}\|_{2m} \leq 2C_{2m} \|\imag(F)\|_{2m}.\]
        Usando de nuevo la linealidad de los coeficientes de Fourier,
        \begin{align*}
            \widetilde{\real(F)}(t)+i\widetilde{\imag(F)}(t) &= -i\sum_{k\in\Z}\sgn(k)c_k(\real(F))e^{ikt}- i^2\sum_{k\in\Z}\sgn(k)c_k(\imag(F))e^{ikt} \\
            &=-i\sum_{k\in\Z}\sgn(k)c_k(\real(F)+i\imag(F))e^{ikt} \\
            &=-i\sum_{k\in\Z}\sgn(k)c_k(F)e^{ikt}= \widetilde{F}(t).
        \end{align*}
        Por tanto,
        \begin{align*}
            \|\widetilde{F}\|_{2m} &= \|\widetilde{\real(F)}+i\widetilde{\imag(F)}\|_{2m} \leq \|\widetilde{\real(F)}\|_{2m}+\|\widetilde{\imag(F)}\|_{2m} \\ 
            &\leq 2C_{2m}(\|\real(F)\|_{2m}+\|\imag(F)\|_{2m}) \leq 2C_{2m}(\|F\|_{2m}+\|F\|_{2m}) = 4C_{2m}\|F\|_{2m},
        \end{align*}
        utilizando en la última desigualdad que $|\real(F(t))|\leq |F(t)|$ y $|\imag(F(t))| \leq |F(t)|$ para todo $t \in \R$, lo que implica $\|\real(F)\|_{2m} \leq \|F\|_{2m}$ y $\|\imag(F)\|_{2m}\leq\|F\|_{2m}$.
    \end{itemize}

    Con esto concluye la demostración del lema para $p \in 2\N$. Probémoslo ahora para $p > 2$. Sea $p>2$ y sea $m\in\N$ tal que $2m\leq p \leq 2m+2$. Entonces se tiene $L^{2m+2}(\T) \subset L^p(\T) \subset L^{2m}(\T)$. Las aplicaciones
    \[
    \begin{aligned}[t]
        T_{1} \colon \mathcal{P} &\longrightarrow L^{2m}(\T) \\
        F &\longmapsto \widetilde{F} 
    \end{aligned}
    \qquad \qquad
    \begin{aligned}[t]
        T_{2} \colon \mathcal{P} &\longrightarrow L^{2m+2}(\T) \\
        F &\longmapsto \widetilde{F} 
    \end{aligned}
    \]
    son lineales (consecuencia inmediata de la linealidad de los coeficientes de Fourier) y continuas (por lo probado anteriormente). Como $\mathcal{P}$ es denso en $L^{2m}(\T)$ y en $L^{2m+2}(\T)$, por el \hyperref[1.3.3]{\color{blue}Teorema 1.3.3}, $T_{1}$ y $T_{2}$ pueden extenderse de forma lineal y continua a $L^{2m}(\T)$ y $L^{2m+2}(\T)$, respectivamente. Seguimos llamando $T_{1}$ y $T_{2}$ a estas extensiones. Nótese que si $f \in L^{2m+2}(\T)$, entonces $T_2(f) = T_1(f)$, pues si $\{F_n\}_{n=1}^\infty$ es una sucesión en $\mathcal{P}$ que converge a $f$ en $L^{2m+2}(\T)$, entonces también converge a $f$ en $L^{2m}(\T)$, así que, por definición de $T_1$ y $T_2$,
    \[T_2(f)=\lim_{n\to\infty} T_2(F_n) = \lim_{n\to\infty} \widetilde{F_n} =  \lim_{n\to\infty} T_1(F_n) = T_1(f).\] 
    Sea $T = T_{1}$, que es una aplicación lineal que verifica
    \begin{itemize}
        \item Para toda $f \in L^{2m}(\T)$, \[\|T(f)\|_{2m} = \|T_{1}(f)\|_{2m} \leq \|T_{1}\| \|f\|_{2m}.\]
        \item Para toda $f \in L^{2m+2}(\T)$, \[\|T(f)\|_{2m+2} = \|T_1(f)\|_{2m+2} = \|T_2(f)\|_{2m+2} \leq \|T_2\|\|f\|_{2m+2}.\]
    \end{itemize}
    Como $2m<p<2m+2$, por el \hyperref[3.2.2]{\color{blue}teorema de interpolación de Riesz-Thorin}, existe $M_p>0$ tal que
    \[\|T(f)\|_p  \leq M_p\|f\|_p\]
    para toda $f \in L^p(\T)$. En particular,
    \[\|T(F)\|_{p} = \|\widetilde{F}\|_{p} \leq M_p \|F\|_p\]
    para todo $F \in \mathcal{P}$. 
    
    Antes de estudiar el caso restante, resultará útil demostrar que para todo $k \in \Z$ y toda $f \in L^p(\T)$ se tiene
    \[c_k(T(f)) = -i\sgn(k)c_k(f).\]
    Sea $k\in\Z$. La aplicación
    \begin{align*}
        L^1(\T) &\longrightarrow \C \\
        f &\longmapsto c_k(f)
    \end{align*}
    es lineal (por la linealidad de los coeficientes de Fourier) y es continua, pues para toda $f\in L^{1}(\T)$ se tiene
    \[|c_k(f)| = \frac{1}{2\pi}\Bigl|\integral{-\pi}{\pi}{f(t)}\Bigr| \leq \frac{1}{2\pi}\integral{-\pi}{\pi}{|f(t)|} = \|f\|_1.\]
    Sea $f \in L^{p}(\T)$ y sea $\{F_n\}_{n=1}^\infty$ una sucesión en $\mathcal{P}$ que converge a $f$ en $L^{p}(\T)$. Como $L^p(\T) \subset L^{2m}(T)$, entonces
    \[T(f) = T_1(f) = \lim_{n\to\infty} T_1(F_n) = \lim_{n\to\infty} \widetilde{F_n}.\]
    En consecuencia,
    \begin{align*}
        c_k(T(f)) &= c_k(\lim_{n\to\infty} \widetilde{F_n}) =\lim_{n\to\infty} c_k(\widetilde{F_n}) = \lim_{n\to\infty} -i\sgn(k)c_k(F_n) =-i\sgn(k)\lim_{n\to\infty} c_k(F_n) \\ 
        &= -i\sgn(k)c_k(\lim_{n\to\infty} F_n) = -i\sgn(k)c_k(f).
    \end{align*}

    Ahora sí, pasamos a probar el lema para $1 < p < 2$. Supongamos que $1<p<2$. Sea $p'$ el exponente conjugado de $p$. Como $p' > 2$, por lo probado anteriormente, existe una aplicación lineal y continua $T \colon L^{p'}(\T)\to L^{p'}(\T)$ tal que $T(F) = \widetilde{F}$ para todo $F \in \mathcal{P}$. Esta aplicación verifica
    \begin{itemize}
        \item $\|T(f)\|_{p'}\leq \|T\|\|f\|_{p'}$ para toda $f \in L^{p'}(\T)$.
        \item $c_k(T(f)) = -i\sgn(k)c_k(f)$ para todo $k\in\Z$ y toda $f \in L^{p'}(\T)$.
    \end{itemize}
    Sea $F \in \mathcal{P}$. Por \hyperref[1.3.1]{\color{blue}(1.3.1)},
    \[\begin{aligned}[t]
        \|\widetilde{F}\|_p &=  \sup_{\|g\|_{p'} = 1} \frac{1}{2\pi}\Bigl|\integral{-\pi}{\pi}{\widetilde{F}(t)g(t)}\Bigr| = \sup_{\|\overline{g}\|_{p'} = 1} \frac{1}{2\pi}\Bigl|\integral{-\pi}{\pi}{\widetilde{F}(t)\overline{g(t)}}\Bigr| \\
        &= \sup_{\|g\|_{p'} = 1} \frac{1}{2\pi}\Bigl|\integral{-\pi}{\pi}{\widetilde{F}(t)\overline{g(t)}}\Bigr|.
    \end{aligned}\]
    Si $g \in L^{p'}(\T) \subset L^2(\T)$ es tal que $\|g\|_{p'} = 1$, usando \hyperref[1.2.9]{\color{blue}(1.2.9)},
    \begin{align*}
        \frac{1}{2\pi}\integral{-\pi}{\pi}{\widetilde{F}(t)\overline{g(t)}} &= \sum_{k\in\Z} c_k(\widetilde{F})\overline{c_k(g)} = \sum_{k\in\Z} -i\sgn(k)c_k(F)\overline{c_k(g)} \\
        &= \sum_{k\in\Z}c_k(F)\overline{i\sgn(k)c_k(g)} = \sum_{k\in\Z}c_k(F)\overline{-c_k(T(g))} \\
        &= -\sum_{k\in\Z}c_k(F)\overline{c_k(T(g))} =-\frac{1}{2\pi}\integral{-\pi}{\pi}{F(t)\overline{T(g)(t)}}.
    \end{align*}
    Tomando módulos,
    \begin{align*}
        \frac{1}{2\pi}\Bigl|\integral{-\pi}{\pi}{\widetilde{F}(t)\overline{g(t)}}\Bigr| &= \frac{1}{2\pi}\Bigl|\integral{-\pi}{\pi}{F(t)\overline{T(g)(t)}}\Bigr| \leq \frac{1}{2\pi}\integral{-\pi}{\pi}{|F(t)||T(g)(t)|} = \|F \cdot T(g)\|_1 \\ 
        &\leq \|F\|_p\|T(g)\|_{p'} \leq \|T\|\|F\|_p\|g\|_{p'} = \|T\|\|F\|_p,
    \end{align*} 
    utilizándose la desigualdad de Hölder en la segunda desigualdad. Concluimos que
    \[\|\widetilde{F}\|_p = \sup_{\|g\|_{p'}=1}\frac{1}{2\pi}\Bigl|\integral{-\pi}{\pi}{\widetilde{F}(t)\overline{g(t)}}\Bigr| \leq \|T\|\|F\|_p. \qedhere\]
\end{proof}

\begin{lemma}
    Si $1 < p < \infty$, entonces existe $C_p > 0$ tal que
    \[\|F^+\|_p \leq C_p \|F\|_{p}.\]
    para todo $F \in \mathcal{P}$.
\end{lemma}

\begin{proof}
    Por el lema anterior, existe $C_p'>0$ tal que
    \[\|\widetilde{F}\|_p\leq C_p' \|F\|_p\]
    para todo $F \in \mathcal{P}$. Veamos también que para todo $F \in \mathcal{P}$ se tiene
    \[F^+ = \frac{1}{2}(F+i\widetilde{F})-\frac{1}{2}c_0(F).\] 
    Sea $F\in\mathcal{P}$ y sea $n$ el grado de $F$. En la demostración del lema anterior se probó que para todo $t\in\R$,
    \[F(t)+i\widetilde{F}(t) = c_0(F)+2\sum_{k=1}^n c_k(F)e^{ikt}.\]
    En consecuencia,
    \begin{align*}
        \frac{1}{2}(F(t)+i\widetilde{F}(t))-\frac{1}{2}c_0(F) = \frac{1}{2}c_0(F)+\sum_{k=1}^n c_k(F)e^{ikt}-\frac{1}{2}c_0(F) = \sum_{k=1}^n c_k(F)e^{ikt} = F^+(t),
    \end{align*}
    lo que prueba que $F^+=\frac{1}{2}(F+i\widetilde{F})-\frac{1}{2}c_0(F)$. Por otra parte, para todo $F\in\mathcal{P}$,
    \[\|c_0(F)\|_p = |c_0(F)| = \frac{1}{2\pi}\Bigl|\integral{-\pi}{\pi}{F(t)}\Bigr| \leq \|F\|_1\leq \|F\|_p,\]
    aplicando en la última desigualdad la desigualdad de Hölder a la función $F$ y la función constante 1. Reuniendo todo lo anterior,
    \begin{align*}
        \|F^+\|_p &= \Bigl\|\frac{1}{2}(F+i\widetilde{F})-\frac{1}{2}c_0(F)\Bigr\|_p \leq \frac{1}{2}(\|F\|_p+\|\widetilde{F}\|_p+\|c_0(F)\|_p) \\ 
        &\leq \frac{1}{2}(\|F\|_p+C_p\|F\|_p+\|F\|_p) = \Bigl(\frac{C_p}{2}+1\Bigr)\|F\|_p
    \end{align*}
    para todo $F\in\mathcal{P}$.
\end{proof}

\begin{theorem}
    Si $1 < p < \infty$, entonces $\{S_nf\}_{n=1}^\infty$ converge a $f$ en $L^p(\T)$ para toda $f \in L^p(\T)$.
\end{theorem}

\begin{proof}
    Basta probar que se verifica el apartado (b) del \hyperref[3.0.2]{\color{blue}Lema 3.0.2}. 
    
    Lo probamos primero para los polinomios trigonométricos. Sea $C_p$ la constante positiva proporcionada por el teorema anterior. Sea $n \in \N$ y sea  $F \in \mathcal{P}$. Entonces
    \[S_nF(x) = \sum_{k=-n}^n c_k(F)e^{ikx} = \sum_{k=0}^{2n}c_{k-n}(F)e^{i(k-n)x} = e^{-inx}\sum_{k=0}^{2n}c_{k-n}(F)e^{ikx}.\]
    Por definición de los coeficientes de Fourier,
    \[c_{k-n}(F) = \frac{1}{2\pi}\integral{-\pi}{\pi}{F(t)e^{-i(k-n)t}} = \frac{1}{2\pi}\integral{-\pi}{\pi}{e^{int}F(t)e^{-ikt}} = c_k(G),\]
    donde $G \colon \R \to \C$ es la función dada por $G(t)=e^{int}F(t)$. En consecuencia,
    \begin{equation}\label{3.2.6}
    |S_nF(x)| = \Bigl|e^{-inx}\sum_{k=0}^{2n}c_k(G)e^{ikx}\Bigr|= \Bigl|\sum_{k=0}^{2n}c_k(G)e^{ikx}\Bigr|.
    \end{equation}
    Se tiene
    \begin{align*}
        \sum_{k=0}^{2n}c_k(G)e^{ikx} &= c_0(G)+ \sum_{k=1}^{2n} c_k(G)e^{ikx} = c_0(G)+\sum_{k=1}^{\infty} c_k(G)e^{ikx} - \sum_{k=2n+1}^{\infty} c_k(G)e^{ikx} \\
        &= c_0(G) + \sum_{k=1}^{\infty} c_k(G)e^{ikx} - \sum_{k=1}^{\infty} c_{2n+k}(G)e^{i(2n+k)x} \\
        &= c_0(G) + \sum_{k=1}^{\infty} c_k(G)e^{ikx} - e^{2inx}\sum_{k=1}^{\infty} c_{2n+k}(G)e^{ikx}. \\
    \end{align*}
    Ahora bien,
    \begin{align*}
        c_{2n+k}(G) &= \frac{1}{2\pi}\integral{-\pi}{\pi}{G(t)e^{-i(2n+k)t}} = \frac{1}{2\pi}\integral{-\pi}{\pi}{e^{-2int}G(t)e^{-ikt}} \\
        &=  \frac{1}{2\pi}\integral{-\pi}{\pi}{e^{-2int}e^{int}F(t)e^{-ikt}} = \frac{1}{2\pi}\integral{-\pi}{\pi}{e^{-int}F(t)e^{-ikt}} = c_k(H),
    \end{align*}
    donde $H \colon \R \to \C$ es la función dada por $H(t) = e^{-int}F(t)$. Llevando esto a las igualdades anteriores,
    \begin{align*}
        \sum_{k=0}^{2n}c_k(G)e^{ikx} &= c_0(G) + \sum_{k=1}^{\infty} c_k(G)e^{ikx} - e^{2inx}\sum_{k=1}^{\infty} c_k(H)e^{ikx} \\
        &= c_0(G) + G^+(x) - e^{2inx}H^+(x).
    \end{align*}
    Nótese que $G,H\in\mathcal{P}$ y por tanto $G^+$ y $H^+$ tienen sentido. Usando esto junto con \hyperref[3.2.6]{\color{blue}(3.2.6)},
    \begin{equation}\label{3.2.7}
    \begin{aligned}[b]
        \|S_nF\|_p^p &= \frac{1}{2\pi}\integral[x]{-\pi}{\pi}{|S_nF(x)|^p} = \frac{1}{2\pi}\integral[x]{-\pi}{\pi}{\Bigl|\sum_{k=0}^{2n}c_k(G)e^{ikx}\Bigr|^p} \\ 
        &=\frac{1}{2\pi}\integral[x]{-\pi}{\pi}{|c_0(G)+G^+(x)-e^{2inx}H^+(x)|^p} \\
        &\leq \frac{1}{2\pi}\integral[x]{-\pi}{\pi}{\Bigl(|c_0(G)|^p+|G^+(x)|^p+|e^{2inx}H^+(x)|^p\Bigr)} \\
        &= |c_0(G)|^p + \|G^+\|_p^p + \|H^+\|_p^p
    \end{aligned}
    \end{equation}
    Acotemos cada uno de los sumandos.
    \begin{itemize}
        \item Como $G,H \in \mathcal{P}$, entonces $\|G^+\|_p \leq C_p \|G\|_p$ y $\|H^+\|_p \leq C_p\|H\|_p$. Y como $|F(t)| = |G(t)| = |H(t)|$ para todo $t \in \R$, entonces $\|F\|_p=\|G\|_p=\|H\|_p$. Por tanto,
        \[\|G^+\|_p^p + \|H^+\|_p^p \leq C_p^p(\|G\|_p^p+\|H\|_p^p) = 2C_p^p\|F\|_p^p.\]
        \item Se tiene que
        \[|c_0(G)| = \Bigl|\frac{1}{2\pi}\integral{-\pi}{\pi}{G(t)}\Bigr| \leq \frac{1}{2\pi}\integral{-\pi}{\pi}{|G(t)|} = \frac{1}{2\pi}\integral{-\pi}{\pi}{|F(t)|}.\]
        Por la desigualdad de Hölder,
        \[\frac{1}{2\pi}\integral{-\pi}{\pi}{|F(t)|} \leq \Bigl(\frac{1}{2\pi}\integral{-\pi}{\pi}{|F(t)|^p}\Bigr)^{\frac{1}{p}}\Bigl(\frac{1}{2\pi}\integral{-\pi}{\pi}{1}\Bigr)^{\frac{1}{p'}} = \|F\|_p,\]
        luego $|c_0(G)|^p \leq \|F\|_p^p$.
    \end{itemize}
    Llevando esto a \hyperref[3.2.7]{\color{blue}(3.2.7)},
    \[\|S_nF\|_p^p \leq \|F\|_p^p + 2C_p^p\|F\|_p^p = (2C_p^p+1)\|F\|_p^p.\]
    Llamando $C_p' = (2C_p^p+1)^{\frac{1}{p}}$, se tiene
    \[\|S_nF\|_p \leq C_p'\|F\|_p.\]
    Esta desigualdad es válida para todo $F \in \mathcal{P}$ y todo $n \in \N$.

    Sea $n \in \N$ y sea $f \in L^p(\T)$. Por la densidad de $\mathcal{P}$ en $L^p(\T)$, existe una sucesión $\{F_k\}_{k=1}^\infty$ en $\mathcal{P}$ tal que
    \[\|f-F_k\|_p \kconv 0.\]
    Por un lado, por la continuidad de la norma,
    \[\|F_k\|_p \kconv \|f\|_p.\]
    Por otro lado, como la aplicación
    \begin{align*}
        T_n \colon L^p(\T) &\longrightarrow L^p(\T) \\
        f &\longmapsto S_nf
    \end{align*}
    es continua (se razonó en la demostración del \hyperref[3.0.2]{\color{blue}Lema 3.0.2}), entonces 
    \[\|S_n(f-F_k)\|_p \kconv 0.\]
    Por tanto, para todo $k \in \N$,
    \[\|S_nf\|_p = \|S_nf-S_nF_k+S_nF_k\|_p = \|S_n(f-F_k)+S_nF_k\|_p  \leq \|S_n(f-F_k)\|_p + \|S_nF_k\|_p.\]
    Por lo probado anteriormente,
    \[\|S_nF_k\|_p \leq C_p'\|F_k\|_p,\]
    luego
    \[\|S_nf\|_p  \leq \|S_n(f-F_k)\|_p + C_p'\|F_k\|_p.\]
    Tomando límites cuando $k \to\infty$, concluimos que
    \[\|S_nf\|_p \leq C_p'\|f\|_p. \qedhere\]
\end{proof}

\chapter{Provisional}


Por ser $\Phi$ una isometría, también es inyectiva: si $g,h\in L^{p'}(\T)$ verifican $\Phi(g) = \Phi(h)$, entonces
\[\|g-h\|_{p'} = \|\Phi(g-h)\| = \|\Phi(g)-\Phi(h)\| = 0,\]
así que $g = h$ en $L^{p'}(\T)$.

Para terminar, probemos que $\Phi$ es sobreyectiva. Sea $\Psi \in L^p(\T)^*$ y veamos que existe $g \in L^{p'}(\T)$ tal que $\Phi(g) = \Psi$. Consideremos el espacio de medida de Lebesgue en $[-\pi,\pi]$, $([-\pi,\pi],\mathcal{L},m)$, y definamos $\nu \colon \mathcal{L} \to \C$ por $\nu(E) = \Psi(\chi_E)$. Se tiene que
\begin{itemize}
    \item $\nu(\emptyset) = \Psi(\chi_\emptyset) = \Psi(0) = 0$, pues $\chi_\emptyset = 0$ como función de $L^p(\T)$ y $\Psi$ es lineal.
    \item Sea $\{E_n\}_{n=1}^\infty$ es una sucesión de elementos de $\mathcal{L}$ disjuntos dos a dos. Entonces
    \[
    \begin{aligned}[t]
        \nu\Bigl(\, \bigcup_{n=1}^\infty E_n\Bigr) &=\Psi(\chi_{\bigcup_{n=1}^\infty E_n}) \overset{(\asts{1})}{=} \Psi\Bigl(\, \sum_{n=1}^\infty \chi_{E_n}\Bigr) \overset{(\asts{2})}{=} \Psi\Bigl(\, \lim_{n \to \infty} \sum_{k=1}^n \chi_{E_k}\Bigr) \\ \overset{(\asts{3})}&{=} \lim_{n\to\infty}\Psi\Bigl(\, \sum_{k=1}^n \chi_{E_k}\Bigr) =  \lim_{n\to\infty} \sum_{k=1}^n \Psi(\chi_{E_k}) = \sum_{n=1}^\infty \Psi(\chi_{E_n}) = \sum_{n=1}^\infty \nu(E_n).
    \end{aligned}
    \]
    Se aclara que
    \begin{itemize}
        \item[(\asts{1})] Para todo $t \in [-\pi,\pi]$ se verifica
        \[\chi_{\bigcup_{n=1}^\infty E_n}(t) = \sum_{n=1}^\infty \chi_{E_n}(t).\]
        En efecto, si $\chi_{\bigcup_{n=1}^\infty E_n}(t) = 0$, entonces $\chi_{E_n}(t) = 0$ para todo $n \in \N$ y se cumple la igualdad. Y si $\chi_{\bigcup_{n=1}^\infty E_n}(t) = 1$, entonces existe $n_0 \in \N$ tal que $t \in E_{n_0}$, y como los conjuntos son disjuntos dos a dos, se tiene que $t \not\in E_n$ para todo $n \neq n_0$, luego $\sum_{n=1}^\infty \chi_{E_n}(t) = 1$ y también se verifica la igualdad.
        \item[(\asts{2})] $\{\sum_{k=1}^n \chi_{E_k}\}_{n=1}^\infty$ converge a $\sum_{n=1}^\infty \chi_{E_n}$ en $L^p(\T)$. Veámoslo. Para todo $n \in \N$ se tiene
        \[
        \begin{aligned}[t]
            \Bigl\|\sum_{k=1}^\infty \chi_{E_k} - \sum_{k=1}^n \chi_{E_k}\Bigr\|_p^p &= \Bigl\|\sum_{k=n+1}^\infty \chi_{E_k}\Bigr\|_p^p = \frac{1}{2\pi}\integral{-\pi}{\pi}{\Bigl(\,\sum_{k=n+1}^\infty\chi_{E_k}(t)\Bigr)^p}.
        \end{aligned}
        \]
        Razonando como antes se prueba que $\sum_{k=n+1}^\infty \chi_{E_k}(t) = \chi_{\bigcup_{k=n+1}^\infty E_k}(t)$ para todo $t \in [-\pi,\pi]$. En consecuencia,
        \[
        \begin{aligned}[t]
            \Bigl\|\sum_{k=1}^\infty \chi_{E_k} - \sum_{k=1}^n \chi_{E_k}\Bigr\|_p^p &= \frac{1}{2\pi}\integral{-\pi}{\pi}{\chi_{\bigcup_{k=n+1}^\infty E_k}(t)} = m\Bigl(\, \bigcup_{k=n+1}^\infty E_k\Bigr) \\
            &= m\Bigl(\, \bigcup_{k=1}^\infty E_k \setminus \bigcup_{k=1}^n E_k\Bigr) = m\Bigl(\,\bigcup_{k=1}^\infty E_k\Bigr) - m\Bigl(\,\bigcup_{k=1}^n E_k\Bigr) \\ &= \sum_{k=1}^\infty m(E_k) - \sum_{k=1}^n m(E_k) \nconv 0,
        \end{aligned}
        \]
        utilizándose en la última igualdad que los conjuntos de la sucesión $\{E_n\}_{n=1}^\infty$ son disjuntos dos a dos.
        \item[(\asts{3})] $\Psi$ es continua. 
    \end{itemize}
\end{itemize}

Tenemos entonces que $\nu$ es una medida compleja. Además, si $m(E) = 0$, entonces $\chi_E = 0$ como función de $L^p(\T)$, y como $\Psi$ es lineal, $\nu(E) = \Psi(\chi_E) = \Psi(0)= 0$. Por tanto, $\nu \ll m$. Por el teorema de Radon-Nikodym, existe $g \in L^1(\T)$ tal que para todo $E\in\mathcal{L}$ se verifica $\nu(E) =\int_E g\, dm$, es decir,
\[\Psi(\chi_E) = \frac{1}{2\pi}\integral{-\pi}{\pi}{\chi_E(t)g(t)}.\]
Veamos que $g \in L^{p'}(\T)$ y que $\Psi = \Phi(g)$. 

%-----------------------------------------------------------------------------------------------------%

% BIBLIOGRAFÍA

\addcontentsline{toc}{chapter}{Bibliografía}
\bibliographystyle{babplain}
\bibliography{ref} % crear archivo .bib con referencias en bibtex

\end{document}





